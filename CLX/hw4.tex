\documentclass{article}
\usepackage[margin=1in]{geometry}
\usepackage[linesnumbered,ruled,vlined]{algorithm2e}
\usepackage{amsfonts}
\usepackage{amsmath}
\usepackage{amssymb}
\usepackage{amsthm}
\usepackage{chemformula}
\usepackage{enumitem}
\usepackage{fancyhdr}
\usepackage{graphicx}
\usepackage{hyperref}
\usepackage{listings}
\usepackage{minted}
\usepackage{multicol}
\usepackage{pdfpages}
\usepackage{siunitx}
\usepackage{standalone}
\usepackage{svg}
\usepackage[many]{tcolorbox}
\usepackage{tikz-cd}
\usepackage{transparent}
\usepackage{xcolor}
% \tcbuselibrary{minted}

\author{Nathan Solomon}

\newcommand{\fig}[1]{
    \begin{center}
        \includegraphics[width=\textwidth]{#1}
    \end{center}
}

% Math commands
\renewcommand{\d}{\mathrm{d}}
\DeclareMathOperator{\id}{id}
\DeclareMathOperator{\im}{im}
\DeclareMathOperator{\proj}{proj}
\DeclareMathOperator{\Span}{span}
\DeclareMathOperator{\Tr}{Tr}
\DeclareMathOperator{\tr}{tr}
\DeclareMathOperator{\ad}{ad}
\DeclareMathOperator{\ord}{ord}
%%%%%%%%%%%%%%% \DeclareMathOperator{\sgn}{sgn}
\DeclareMathOperator{\Aut}{Aut}
\DeclareMathOperator{\Inn}{Inn}
\DeclareMathOperator{\Out}{Out}
\DeclareMathOperator{\stab}{stab}

\newcommand{\N}{\ensuremath{\mathbb{N}}}
\newcommand{\Z}{\ensuremath{\mathbb{Z}}}
\newcommand{\Q}{\ensuremath{\mathbb{Q}}}
\newcommand{\R}{\ensuremath{\mathbb{R}}}
\newcommand{\C}{\ensuremath{\mathbb{C}}}
\renewcommand{\H}{\ensuremath{\mathbb{H}}}
\newcommand{\F}{\ensuremath{\mathbb{F}}}

\newcommand{\E}{\ensuremath{\mathbb{E}}}
\renewcommand{\P}{\ensuremath{\mathbb{P}}}

\newcommand{\es}{\ensuremath{\varnothing}}
\newcommand{\inv}{\ensuremath{^{-1}}}
\newcommand{\eps}{\ensuremath{\varepsilon}}
\newcommand{\del}{\ensuremath{\partial}}
\renewcommand{\a}{\ensuremath{\alpha}}

\newcommand{\abs}[1]{\ensuremath{\left\lvert #1 \right\rvert}}
\newcommand{\norm}[1]{\ensuremath{\left\lVert #1\right\rVert}}
\newcommand{\mean}[1]{\ensuremath{\left\langle #1 \right\rangle}}
\newcommand{\floor}[1]{\ensuremath{\left\lfloor #1 \right\rfloor}}
\newcommand{\ceil}[1]{\ensuremath{\left\lceil #1 \right\rceil}}
\newcommand{\bra}[1]{\ensuremath{\left\langle #1 \right\rvert}}
\newcommand{\ket}[1]{\ensuremath{\left\lvert #1 \right\rangle}}
\newcommand{\braket}[2]{\ensuremath{\left.\left\langle #1\right\vert #2 \right\rangle}}

\newcommand{\catname}[1]{{\normalfont\textbf{#1}}}

\newcommand{\up}{\ensuremath{\uparrow}}
\newcommand{\down}{\ensuremath{\downarrow}}

% Custom environments
\newtheorem{thm}{Theorem}[section]

\definecolor{probBackgroundColor}{RGB}{250,240,240}
\definecolor{probAccentColor}{RGB}{140,40,0}
\newenvironment{prob}{
    \stepcounter{thm}
    \begin{tcolorbox}[
        boxrule=1pt,
        sharp corners,
        colback=probBackgroundColor,
        colframe=probAccentColor,
        borderline west={4pt}{0pt}{probAccentColor},
        breakable
    ]
    \color{probAccentColor}\textbf{Problem \thethm.} \color{black}
} {
    \end{tcolorbox}
}

\definecolor{exampleBackgroundColor}{RGB}{212,232,246}
\newenvironment{example}{
    \stepcounter{thm}
    \begin{tcolorbox}[
      boxrule=1pt,
      sharp corners,
      colback=exampleBackgroundColor,
      breakable
    ]
    \textbf{Example \thethm.}
} {
    \end{tcolorbox}
}

\definecolor{propBackgroundColor}{RGB}{255,245,220}
\definecolor{propAccentColor}{RGB}{150,100,0}
\newenvironment{prop}{
    \stepcounter{thm}
    \begin{tcolorbox}[
        boxrule=1pt,
        sharp corners,
        colback=propBackgroundColor,
        colframe=propAccentColor,
        breakable
    ]
    \color{propAccentColor}\textbf{Proposition \thethm. }\color{black}
} {
    \end{tcolorbox}
}

\definecolor{thmBackgroundColor}{RGB}{235,225,245}
\definecolor{thmAccentColor}{RGB}{50,0,100}
\renewenvironment{thm}{
    \stepcounter{thm}
    \begin{tcolorbox}[
        boxrule=1pt,
        sharp corners,
        colback=thmBackgroundColor,
        colframe=thmAccentColor,
        breakable
    ]
    \color{thmAccentColor}\textbf{Theorem \thethm. }\color{black}
} {
    \end{tcolorbox}
}

\definecolor{corBackgroundColor}{RGB}{240,250,250}
\definecolor{corAccentColor}{RGB}{50,100,100}
\newenvironment{cor}{
    \stepcounter{thm}
    \begin{tcolorbox}[
        enhanced,
        boxrule=0pt,
        frame hidden,
        sharp corners,
        colback=corBackgroundColor,
        borderline west={4pt}{0pt}{corAccentColor},
        breakable
    ]
    \color{corAccentColor}\textbf{Corollary \thethm. }\color{black}
} {
    \end{tcolorbox}
}

\definecolor{lemBackgroundColor}{RGB}{255,245,235}
\definecolor{lemAccentColor}{RGB}{250,125,0}
\newenvironment{lem}{
    \stepcounter{thm}
    \begin{tcolorbox}[
        enhanced,
        boxrule=0pt,
        frame hidden,
        sharp corners,
        colback=lemBackgroundColor,
        borderline west={4pt}{0pt}{lemAccentColor},
        breakable
    ]
    \color{lemAccentColor}\textbf{Lemma \thethm. }\color{black}
} {
    \end{tcolorbox}
}

\definecolor{proofBackgroundColor}{RGB}{255,255,255}
\definecolor{proofAccentColor}{RGB}{80,80,80}
\renewenvironment{proof}{
    \begin{tcolorbox}[
        enhanced,
        boxrule=1pt,
        sharp corners,
        colback=proofBackgroundColor,
        colframe=proofAccentColor,
        borderline west={4pt}{0pt}{proofAccentColor},
        breakable
    ]
    \color{proofAccentColor}\emph{\textbf{Proof. }}\color{black}
} {
    \qed \end{tcolorbox}
}

\definecolor{noteBackgroundColor}{RGB}{240,250,240}
\definecolor{noteAccentColor}{RGB}{30,130,30}
\newenvironment{note}{
    \begin{tcolorbox}[
        enhanced,
        boxrule=0pt,
        frame hidden,
        sharp corners,
        colback=noteBackgroundColor,
        borderline west={4pt}{0pt}{noteAccentColor},
        breakable
    ]
    \color{noteAccentColor}\textbf{Note. }\color{black}
} {
    \end{tcolorbox}
}


\fancyhf{}
\setlength{\headheight}{24pt}

\date{\today}
\title{Complex Analysis Homework \#4}

\begin{document}
\maketitle

\begin{prob}
    prob 1
\end{prob}
(a)
\[ \sin(i) = \frac{e^{i*i}-e^{-i*i}}{2i} = \frac{1/e-e}{2i} = \frac{ie}{2}- \frac{i}{2e}. \]
Alternatively, we could use the identity $\sin(ix)=i \sinh(x)$ to get
\[ \sin(i) = i \sinh(1) = i \left( \frac{e}{2} - \frac{1}{2e} \right).  \]
(b) The following equations are equivalent:
\begin{align*}
    \sin(z) &= 0 \\
    \frac{e^{iz}-e^{-iz}}{2i} &= 0 \\
    e^{iz} &= e^{-iz}
\end{align*}
Let $a$ and $b$ be the real and imaginary components of $z$, respectively. Then
\[ e^{iz}=e^{-b}e^{ia}. \]
This is the polar form of $e^{iz}$ -- the magnitude is $e^{-b}$ and the argument is $a$. Similarly, $e^{-iz}$ has magnitude $e^b$ and argument $-a$. For $e^{iz}$ and $e^{-iz}$ to have the same magnitude, we must have $e^{-b}=e^b$ (which means $b=0$), and for them to have the same angle, we must have $e^{ia}=e^{-ia}$. That second criterion means $e^{2ia}=1$, so $a$ is an integer multiple of $\pi$.
\par
We have shown that $z=a+bi$ satisfies $\sin(z)=0$ iff $z$ is a (real) integer multiple of $\pi$.

\begin{prob}
    prob 2
\end{prob}
(a) The coefficients of this power series are
\[ \left\{ a_n \right\}_{n \in \N_0} = \left\{ 1,0,1,0,1,\dots \right\}, \]
so we can use the formula for radius of convergence, which gives
\[ R = \frac{1}{\limsup |a_n|^{1/n}} = \frac{1}{1} = 1. \]
That means that within the open disc of radius 1 about the origin, $D_1(0)$, the sum converges absolutely. Because it converges absolutely, we can rearrange it as much as we want.
\[ (1+z^2+z^4+\cdots)(1-z^2)=(1+z^2+z^4+\cdots)-(z^2+z^4+z^6+\cdots)=1 \]
\[ 1+z^2+z^4+\cdots= \frac{1}{1-z^2} \]
That sum does not converge anywhere on the boundary of the disc of convergence, because if $|z|=1$, then each term in that power series has magnitude 1, so the partial sums are not a Cauchy sequence.
\bigskip
\par
(b) The coefficients of this new power series are
\[ \left\{ a_n \right\}_{n \in \N_0} = \left\{ \frac{1}{4}, \frac{1}{16}, \frac{1}{64}, \cdots \right\}. \]
so the radius of convergence is
\[ R = \frac{1}{\limsup \sqrt[n]{|a_n|}} = \frac{1}{\limsup \left\{ \sqrt[n]{4^{-n-1} } \right\}} = \frac{1}{\limsup \left\{ \frac{1}{4\sqrt[n]{ \frac{1}{4}}}  \right\} } = \frac{1}{ \frac{1}{4}} = 4.\]
Within that radius of convergence, we have
\[ \sum_{n=0}^\infty \frac{z^n}{4^{n+1}} = \frac{1}{4} \sum_{n=0}^\infty \left( \frac{z}{4} \right)^n = \frac{1}{4} \cdot \frac{1}{1- \left( \frac{z}{4} \right) }  = \frac{1}{4-z}. \]
On the boundary of the disc of convergence, every term of the power series will have magnitude $ \frac{1}{4}$, so the partial sums are not a Cauchy sequence, meaning there is no point on that boundary where the power series converges.

\begin{prob}
    prob 3
\end{prob}

\begin{prob}
    prob 4
\end{prob}

\begin{prob}
    prob 5
\end{prob}
Let $b_0$ be any complex number, and for every positive integer $n$, let $b_n= \frac{a_{n-1}}{n}$. Then define another power series
\[ g(z) := \sum_{n=0}^\infty b_n(z-a)^n. \]

\end{document}
