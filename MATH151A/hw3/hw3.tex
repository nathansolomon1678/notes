\documentclass{article}
\usepackage[margin=1in]{geometry}
\usepackage[linesnumbered,ruled,vlined]{algorithm2e}
\usepackage{amsfonts}
\usepackage{amsmath}
\usepackage{amssymb}
\usepackage{amsthm}
\usepackage{chemformula}
\usepackage{enumitem}
\usepackage{fancyhdr}
\usepackage{graphicx}
\usepackage{hyperref}
\usepackage{listings}
\usepackage{minted}
\usepackage{multicol}
\usepackage{pdfpages}
\usepackage{siunitx}
\usepackage{standalone}
\usepackage{svg}
\usepackage[many]{tcolorbox}
\usepackage{tikz-cd}
\usepackage{transparent}
\usepackage{xcolor}
% \tcbuselibrary{minted}

\author{Nathan Solomon}

\newcommand{\fig}[1]{
    \begin{center}
        \includegraphics[width=\textwidth]{#1}
    \end{center}
}

% Math commands
\renewcommand{\d}{\mathrm{d}}
\DeclareMathOperator{\id}{id}
\DeclareMathOperator{\im}{im}
\DeclareMathOperator{\proj}{proj}
\DeclareMathOperator{\Span}{span}
\DeclareMathOperator{\Tr}{Tr}
\DeclareMathOperator{\tr}{tr}
\DeclareMathOperator{\ad}{ad}
\DeclareMathOperator{\ord}{ord}
%%%%%%%%%%%%%%% \DeclareMathOperator{\sgn}{sgn}
\DeclareMathOperator{\Aut}{Aut}
\DeclareMathOperator{\Inn}{Inn}
\DeclareMathOperator{\Out}{Out}
\DeclareMathOperator{\stab}{stab}

\newcommand{\N}{\ensuremath{\mathbb{N}}}
\newcommand{\Z}{\ensuremath{\mathbb{Z}}}
\newcommand{\Q}{\ensuremath{\mathbb{Q}}}
\newcommand{\R}{\ensuremath{\mathbb{R}}}
\newcommand{\C}{\ensuremath{\mathbb{C}}}
\renewcommand{\H}{\ensuremath{\mathbb{H}}}
\newcommand{\F}{\ensuremath{\mathbb{F}}}

\newcommand{\E}{\ensuremath{\mathbb{E}}}
\renewcommand{\P}{\ensuremath{\mathbb{P}}}

\newcommand{\es}{\ensuremath{\varnothing}}
\newcommand{\inv}{\ensuremath{^{-1}}}
\newcommand{\eps}{\ensuremath{\varepsilon}}
\newcommand{\del}{\ensuremath{\partial}}
\renewcommand{\a}{\ensuremath{\alpha}}

\newcommand{\abs}[1]{\ensuremath{\left\lvert #1 \right\rvert}}
\newcommand{\norm}[1]{\ensuremath{\left\lVert #1\right\rVert}}
\newcommand{\mean}[1]{\ensuremath{\left\langle #1 \right\rangle}}
\newcommand{\floor}[1]{\ensuremath{\left\lfloor #1 \right\rfloor}}
\newcommand{\ceil}[1]{\ensuremath{\left\lceil #1 \right\rceil}}
\newcommand{\bra}[1]{\ensuremath{\left\langle #1 \right\rvert}}
\newcommand{\ket}[1]{\ensuremath{\left\lvert #1 \right\rangle}}
\newcommand{\braket}[2]{\ensuremath{\left.\left\langle #1\right\vert #2 \right\rangle}}

\newcommand{\catname}[1]{{\normalfont\textbf{#1}}}

\newcommand{\up}{\ensuremath{\uparrow}}
\newcommand{\down}{\ensuremath{\downarrow}}

% Custom environments
\newtheorem{thm}{Theorem}[section]

\definecolor{probBackgroundColor}{RGB}{250,240,240}
\definecolor{probAccentColor}{RGB}{140,40,0}
\newenvironment{prob}{
    \stepcounter{thm}
    \begin{tcolorbox}[
        boxrule=1pt,
        sharp corners,
        colback=probBackgroundColor,
        colframe=probAccentColor,
        borderline west={4pt}{0pt}{probAccentColor},
        breakable
    ]
    \color{probAccentColor}\textbf{Problem \thethm.} \color{black}
} {
    \end{tcolorbox}
}

\definecolor{exampleBackgroundColor}{RGB}{212,232,246}
\newenvironment{example}{
    \stepcounter{thm}
    \begin{tcolorbox}[
      boxrule=1pt,
      sharp corners,
      colback=exampleBackgroundColor,
      breakable
    ]
    \textbf{Example \thethm.}
} {
    \end{tcolorbox}
}

\definecolor{propBackgroundColor}{RGB}{255,245,220}
\definecolor{propAccentColor}{RGB}{150,100,0}
\newenvironment{prop}{
    \stepcounter{thm}
    \begin{tcolorbox}[
        boxrule=1pt,
        sharp corners,
        colback=propBackgroundColor,
        colframe=propAccentColor,
        breakable
    ]
    \color{propAccentColor}\textbf{Proposition \thethm. }\color{black}
} {
    \end{tcolorbox}
}

\definecolor{thmBackgroundColor}{RGB}{235,225,245}
\definecolor{thmAccentColor}{RGB}{50,0,100}
\renewenvironment{thm}{
    \stepcounter{thm}
    \begin{tcolorbox}[
        boxrule=1pt,
        sharp corners,
        colback=thmBackgroundColor,
        colframe=thmAccentColor,
        breakable
    ]
    \color{thmAccentColor}\textbf{Theorem \thethm. }\color{black}
} {
    \end{tcolorbox}
}

\definecolor{corBackgroundColor}{RGB}{240,250,250}
\definecolor{corAccentColor}{RGB}{50,100,100}
\newenvironment{cor}{
    \stepcounter{thm}
    \begin{tcolorbox}[
        enhanced,
        boxrule=0pt,
        frame hidden,
        sharp corners,
        colback=corBackgroundColor,
        borderline west={4pt}{0pt}{corAccentColor},
        breakable
    ]
    \color{corAccentColor}\textbf{Corollary \thethm. }\color{black}
} {
    \end{tcolorbox}
}

\definecolor{lemBackgroundColor}{RGB}{255,245,235}
\definecolor{lemAccentColor}{RGB}{250,125,0}
\newenvironment{lem}{
    \stepcounter{thm}
    \begin{tcolorbox}[
        enhanced,
        boxrule=0pt,
        frame hidden,
        sharp corners,
        colback=lemBackgroundColor,
        borderline west={4pt}{0pt}{lemAccentColor},
        breakable
    ]
    \color{lemAccentColor}\textbf{Lemma \thethm. }\color{black}
} {
    \end{tcolorbox}
}

\definecolor{proofBackgroundColor}{RGB}{255,255,255}
\definecolor{proofAccentColor}{RGB}{80,80,80}
\renewenvironment{proof}{
    \begin{tcolorbox}[
        enhanced,
        boxrule=1pt,
        sharp corners,
        colback=proofBackgroundColor,
        colframe=proofAccentColor,
        borderline west={4pt}{0pt}{proofAccentColor},
        breakable
    ]
    \color{proofAccentColor}\emph{\textbf{Proof. }}\color{black}
} {
    \qed \end{tcolorbox}
}

\definecolor{noteBackgroundColor}{RGB}{240,250,240}
\definecolor{noteAccentColor}{RGB}{30,130,30}
\newenvironment{note}{
    \begin{tcolorbox}[
        enhanced,
        boxrule=0pt,
        frame hidden,
        sharp corners,
        colback=noteBackgroundColor,
        borderline west={4pt}{0pt}{noteAccentColor},
        breakable
    ]
    \color{noteAccentColor}\textbf{Note. }\color{black}
} {
    \end{tcolorbox}
}


\fancyhf{}
\setlength{\headheight}{24pt}

\date{\today}
\title{Math 151A Homework \#3}

\begin{document}
\maketitle

\bigskip
\begin{prob}
\end{prob}
\begin{enumerate}[label=(\alph*)]
    \item \begin{align*}
            g(x) &= \frac{1}{2} \left( x + \frac{a}{x} \right) \\
            g( \sqrt{a}) &= \frac{1}{2} \left( \sqrt{a} + \frac{a}{\sqrt{a}} \right) = \frac{1}{2} \left( \sqrt{a} + \sqrt{a} \right) = \sqrt{a} \\
    \end{align*}
\item We just showed $g(p)=p$, and since $g'(x) = \frac{1}{2} \left( 1 - \frac{a}{x^2} \right)$, we also have $g'(p)=0$. But since $g''(x) = \frac{a}{4x^3}$, $g''(p) = 1/(4 \sqrt{a}) \neq 0$. $g \in C^2([a,b])$ is a function such that $g(p)=p$, $g'(p)=0$, and $g''(p) \neq 0$, so by the theorem from lecture 7 on FPI convergence rate, $p_n$ will converge quadratically (that is, with order $\alpha=2$) to $p$ for $p_0$ sufficiently close to $p$.
\end{enumerate}


\bigskip
\begin{prob}
\end{prob}
\begin{enumerate}[label=(\alph*)]
    \item $f(x) = e^x - 1 - x - x^2/2 = x^3/6 + x^4/24 + x^5/120 + \cdots$, so $f(0)=f'(0)=f''(0)=0$, but $f'''(0)=1\neq 0$. Therefore $x=0$ is a zero of $f$ with multiplicity 3.
    \item Since the multiplicity of $x=0$ is not 1, we are not guaranteed even linear convergence with Newton's method, but we are still guaranteed quadratic convergence with the modified version of Newton's method. That's why the modified version converges so much faster:

\begin{lstlisting}[language=Python]
import math

def f(x):
    return math.exp(x) - 1 - x - x**2/2
def f_prime(x):
    return math.exp(x) - 1 - x
def f_prime_prime(x):
    return math.exp(x) - 1
def mu(x):
    return f(x) / f_prime(x)
def mu_prime(x):
    return 1 - f(x) * f_prime_prime(x) / f_prime(x)**2

def newtons_method(x_0, tolerance, max_iterations):
    print(f"\nNewton's method with {x_0=}, {tolerance=}, {max_iterations=}")
    x_n = x_0
    n = 0
    residual = abs(f(x_n))
    print(f"{n=:02}  {x_n=:+1.8f}  {residual=:1.17f}")
    while abs(x_n) > tolerance and n < max_iterations:
        x_n -= f(x_n) / f_prime(x_n)
        n += 1
        residual = abs(f(x_n))
        print(f"{n=:02}  {x_n=:+1.8f}  {residual=:1.17f}")

def modified_newtons_method(x_0, tolerance, max_iterations):
    print(f"\nModified Newton's method with {x_0=}, {tolerance=}, {max_iterations=}")
    x_n = x_0
    n = 0
    residual = abs(f(x_n))
    print(f"{n=:02}  {x_n=:+1.8f}  {residual=:1.17f}")
    while abs(x_n) > tolerance and n < max_iterations:
        x_n -= mu(x_n) / mu_prime(x_n)
        n += 1
        residual = abs(f(x_n))
        print(f"{n=:02}  {x_n=:+1.8f}  {residual=:1.17f}")

newtons_method(1, 1e-6, 1000)
modified_newtons_method(1, 1e-6, 1000)


Newton's method with x_0=1, tolerance=1e-06, max_iterations=1000
n=00  x_n=+1.00000000  residual=0.21828182845904509
n=01  x_n=+0.69610560  residual=0.06753849522061861
n=02  x_n=+0.47811290  residual=0.02061871165303474
n=03  x_n=+0.32528512  residual=0.00623499263339255
n=04  x_n=+0.21985780  residual=0.00187302505944356
n=05  x_n=+0.14793388  residual=0.00056013544459508
n=06  x_n=+0.09923640  residual=0.00016700016443702
n=07  x_n=+0.06643295  residual=0.00004968763000134
n=08  x_n=+0.04441177  residual=0.00001476321366354
n=09  x_n=+0.02966279  residual=0.00000438240689480
n=10  x_n=+0.01979969  residual=0.00000130009935675
n=11  x_n=+0.01321069  residual=0.00000038553289180
n=12  x_n=+0.00881198  residual=0.00000011429490871
n=13  x_n=+0.00587681  residual=0.00000003387760031
n=14  x_n=+0.00391883  residual=0.00000001004026650
n=15  x_n=+0.00261298  residual=0.00000000297537972
n=16  x_n=+0.00174218  residual=0.00000000088169007
n=17  x_n=+0.00116154  residual=0.00000000026126051
n=18  x_n=+0.00077440  residual=0.00000000007741430
n=19  x_n=+0.00051628  residual=0.00000000002293816
n=20  x_n=+0.00034419  residual=0.00000000000679660
n=21  x_n=+0.00022947  residual=0.00000000000201381
n=22  x_n=+0.00015298  residual=0.00000000000059670
n=23  x_n=+0.00010200  residual=0.00000000000017688
n=24  x_n=+0.00006799  residual=0.00000000000005248
n=25  x_n=+0.00004529  residual=0.00000000000001558
n=26  x_n=+0.00003009  residual=0.00000000000000465
n=27  x_n=+0.00001982  residual=0.00000000000000123
n=28  x_n=+0.00001356  residual=0.00000000000000033
n=29  x_n=+0.00000996  residual=0.00000000000000021
n=30  x_n=+0.00000569  residual=0.00000000000000005
n=31  x_n=+0.00000869  residual=0.00000000000000007
n=32  x_n=+0.00000693  residual=0.00000000000000006
n=33  x_n=+0.00000458  residual=0.00000000000000002
n=34  x_n=+0.00000261  residual=0.00000000000000007
n=35  x_n=-0.00001870  residual=0.00000000000000113
n=36  x_n=-0.00001223  residual=0.00000000000000031
n=37  x_n=-0.00000805  residual=0.00000000000000008
n=38  x_n=-0.00000566  residual=0.00000000000000005
n=39  x_n=-0.00000227  residual=0.00000000000000001
n=40  x_n=-0.00000506  residual=0.00000000000000003
n=41  x_n=-0.00000760  residual=0.00000000000000011
n=42  x_n=-0.00000381  residual=0.00000000000000001
n=43  x_n=-0.00000305  residual=0.00000000000000001
n=44  x_n=-0.00000423  residual=0.00000000000000005
n=45  x_n=+0.00000168  residual=0.00000000000000001
n=46  x_n=-0.00000534  residual=0.00000000000000008
n=47  x_n=+0.00000014  residual=0.00000000000000000

Modified Newton's method with x_0=1, tolerance=1e-06, max_iterations=1000
n=00  x_n=+1.00000000  residual=0.21828182845904509
n=01  x_n=-0.11308312  residual=0.00023435150458198
n=02  x_n=-0.00103017  residual=0.00000000018216402
n=03  x_n=-0.00000009  residual=0.00000000000000003
\end{lstlisting}

    \item If we change the tolerance from $10^{-6}$ to $10^{-10}$, Newton's method doesn't converge, even in 1000 iterations. This is not surprising -- what did surprise me was to see that the modified version of Newton's method also did not converge, even in 1000 iterations. Although that modified method should converge quadratically in theory, round-off errors in the denominator of the iterate got in the way of that.
\end{enumerate}


\bigskip
\begin{prob}
\end{prob}
$\hat{p}_n$ converges to $p=1$ significantly faster than $p_n$ does.
\begin{lstlisting}[language=Python]
p = 1
def p(n):
    return 1 + 1 / n
def p_hat(n):
    return p(n) - (p(n+1) - p(n))**2 / (p(n+2) - 2 * p(n+1) + p(n))

for n in range(1, 8):
    print(f"{n=} {p(n)=:1.5f} {p_hat(n)=:1.5f}")


n=1 p(n)=2.00000 p_hat(n)=1.25000
n=2 p(n)=1.50000 p_hat(n)=1.16667
n=3 p(n)=1.33333 p_hat(n)=1.12500
n=4 p(n)=1.25000 p_hat(n)=1.10000
n=5 p(n)=1.20000 p_hat(n)=1.08333
n=6 p(n)=1.16667 p_hat(n)=1.07143
n=7 p(n)=1.14286 p_hat(n)=1.06250
\end{lstlisting}

\bigskip
\begin{prob}
\end{prob}
\begin{enumerate}[label=(\alph*)]
    \item
\[ P(x) := \sum_{k=0}^n f(x_k)L_{n,k}(x), \hspace{1cm} L_{n,k}(x) := \prod_{i \in [0,n]\cap \Z - \{k\}} \frac{x-x_i}{x_k-x_i}. \]
Let $n=2, x_0=1, x_1=2, x_2=3$. Then
\begin{align*}
    L_{2,0} &= \frac{(x-2)(x-3)}{(1-2)(1-3)} = \frac{x^2-5x+6}{2}\\
    L_{2,1} &= \frac{(x-1)(x-3)}{(2-1)(2-3)} = -x^2+4x-3 \\
    L_{2,2} &= \frac{(x-1)(x-2)}{(3-1)(3-2)} = \frac{x^2-3x+2}{2} \\
    P(x) &= \ln (2) (-x^2+4x-3) + \ln (3) \frac{x^2-3x+2}{2}.
\end{align*}
    \item
\begin{lstlisting}[language=Python]
import math
def P(x):
    return math.log(2) * (-x**2+4*x-3) + math.log(3) * (x**2-3*x+2)/2
print(f"{P(1)=}")
print(f"{P(2)=}")
print(f"{P(3)=}")
print(f"{P(1.5)=}")
print(f"{P(2.4)=}")

P(1)=0.0
P(2)=0.6931471805599453
P(3)=1.0986122886681098
P(1.5)=0.3825338493364452
P(2.4)=0.889855072497425
\end{lstlisting}
When $x=1.5$, the absolute error is $0.0229312587717192$ and the relative error is $0.05655544290534098$. When $x=2.4$, the absolute error is $0.014386335143525164$ and the relative error is $0.016432722871416054$.
    \item This is a pretty bad method, but it shows that the error is maximized when $x=1.367$, and that the error at that point is $0.024817$.
\begin{lstlisting}[language=Python]
def error(x):
    return abs(P(x) - math.log(x))
import numpy as np
for x in np.linspace(1,3,100):
    print(f"{x=:1.5f} {error(x)=:1.5f}")
print('\n')
# x=1.34343 error(x)=0.02474
# x=1.36364 error(x)=0.02482
# x=1.38384 error(x)=0.02479
for x in np.linspace(1.34,1.39,100):
    print(f"{x=:1.5f} {error(x)=:1.10f}")
# x=1.36677 error(x)=0.0248176530
# x=1.36727 error(x)=0.0248177110
# x=1.36778 error(x)=0.0248177061
\end{lstlisting}
\end{enumerate}

\bigskip
\begin{prob}
\end{prob}

\begin{enumerate}[label=(\alph*)]
    \item
        \[ P(x) = -0.00252225 x^5 +0.2866292 x^4 -10.793792 x^3 +157.31208 x^2 +1642.7517 +179323 \]
\begin{lstlisting}[language=Python]
import numpy as np
# x is years since 1960, y is United States population (in thousands)
x = [0, 10, 20, 30, 40, 50]
y = [179_323, 203_302, 226_542, 249_633, 281_422, 308_746]
N = 5
assert len(x) == N + 1 and len(y) == N + 1
V = np.vander(x, N+1)
coeffs = np.linalg.inv(V) @ np.matrix([y]).T
print("P(x) = " + " ".join([f"{coeffs[i, 0]:+} x^{N-i}" for i in range(N+1)]))
def P(x):
    return round((np.matrix([[x**(N-i) for i in range(N+1)]]) @ coeffs)[0, 0])
for x in [0, 10, 20, 30, 40, 50, 60]:
    print(f"year {1960+x}: {P(x)=:6}")
actual_value = 329_500
print(f"Relative error: {abs(P(60) - actual_value) / actual_value}")
\end{lstlisting}
    \item This method significantly underestimates the US population in 2020. This is an example of the Runge phenomenon.
\begin{lstlisting}[language=Python]
year 1960: P(x)=179323
year 1970: P(x)=203302
year 1980: P(x)=226542
year 1990: P(x)=249633
year 2000: P(x)=281422
year 2010: P(x)=308746
year 2020: P(x)=266165
Relative error: 0.1922154779969651
\end{lstlisting}
\end{enumerate}


\end{document}
