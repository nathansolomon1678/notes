\documentclass{article}
\usepackage[margin=1in]{geometry}
\usepackage[linesnumbered,ruled,vlined]{algorithm2e}
\usepackage{amsfonts}
\usepackage{amsmath}
\usepackage{amssymb}
\usepackage{amsthm}
\usepackage{chemformula}
\usepackage{enumitem}
\usepackage{fancyhdr}
\usepackage{graphicx}
\usepackage{hyperref}
\usepackage{listings}
\usepackage{minted}
\usepackage{multicol}
\usepackage{pdfpages}
\usepackage{siunitx}
\usepackage{standalone}
\usepackage{svg}
\usepackage[many]{tcolorbox}
\usepackage{tikz-cd}
\usepackage{transparent}
\usepackage{xcolor}
% \tcbuselibrary{minted}

\author{Nathan Solomon}

\newcommand{\fig}[1]{
    \begin{center}
        \includegraphics[width=\textwidth]{#1}
    \end{center}
}

% Math commands
\renewcommand{\d}{\mathrm{d}}
\DeclareMathOperator{\id}{id}
\DeclareMathOperator{\im}{im}
\DeclareMathOperator{\proj}{proj}
\DeclareMathOperator{\Span}{span}
\DeclareMathOperator{\Tr}{Tr}
\DeclareMathOperator{\tr}{tr}
\DeclareMathOperator{\ad}{ad}
\DeclareMathOperator{\ord}{ord}
%%%%%%%%%%%%%%% \DeclareMathOperator{\sgn}{sgn}
\DeclareMathOperator{\Aut}{Aut}
\DeclareMathOperator{\Inn}{Inn}
\DeclareMathOperator{\Out}{Out}
\DeclareMathOperator{\stab}{stab}

\newcommand{\N}{\ensuremath{\mathbb{N}}}
\newcommand{\Z}{\ensuremath{\mathbb{Z}}}
\newcommand{\Q}{\ensuremath{\mathbb{Q}}}
\newcommand{\R}{\ensuremath{\mathbb{R}}}
\newcommand{\C}{\ensuremath{\mathbb{C}}}
\renewcommand{\H}{\ensuremath{\mathbb{H}}}
\newcommand{\F}{\ensuremath{\mathbb{F}}}

\newcommand{\E}{\ensuremath{\mathbb{E}}}
\renewcommand{\P}{\ensuremath{\mathbb{P}}}

\newcommand{\es}{\ensuremath{\varnothing}}
\newcommand{\inv}{\ensuremath{^{-1}}}
\newcommand{\eps}{\ensuremath{\varepsilon}}
\newcommand{\del}{\ensuremath{\partial}}
\renewcommand{\a}{\ensuremath{\alpha}}

\newcommand{\abs}[1]{\ensuremath{\left\lvert #1 \right\rvert}}
\newcommand{\norm}[1]{\ensuremath{\left\lVert #1\right\rVert}}
\newcommand{\mean}[1]{\ensuremath{\left\langle #1 \right\rangle}}
\newcommand{\floor}[1]{\ensuremath{\left\lfloor #1 \right\rfloor}}
\newcommand{\ceil}[1]{\ensuremath{\left\lceil #1 \right\rceil}}
\newcommand{\bra}[1]{\ensuremath{\left\langle #1 \right\rvert}}
\newcommand{\ket}[1]{\ensuremath{\left\lvert #1 \right\rangle}}
\newcommand{\braket}[2]{\ensuremath{\left.\left\langle #1\right\vert #2 \right\rangle}}

\newcommand{\catname}[1]{{\normalfont\textbf{#1}}}

\newcommand{\up}{\ensuremath{\uparrow}}
\newcommand{\down}{\ensuremath{\downarrow}}

% Custom environments
\newtheorem{thm}{Theorem}[section]

\definecolor{probBackgroundColor}{RGB}{250,240,240}
\definecolor{probAccentColor}{RGB}{140,40,0}
\newenvironment{prob}{
    \stepcounter{thm}
    \begin{tcolorbox}[
        boxrule=1pt,
        sharp corners,
        colback=probBackgroundColor,
        colframe=probAccentColor,
        borderline west={4pt}{0pt}{probAccentColor},
        breakable
    ]
    \color{probAccentColor}\textbf{Problem \thethm.} \color{black}
} {
    \end{tcolorbox}
}

\definecolor{exampleBackgroundColor}{RGB}{212,232,246}
\newenvironment{example}{
    \stepcounter{thm}
    \begin{tcolorbox}[
      boxrule=1pt,
      sharp corners,
      colback=exampleBackgroundColor,
      breakable
    ]
    \textbf{Example \thethm.}
} {
    \end{tcolorbox}
}

\definecolor{propBackgroundColor}{RGB}{255,245,220}
\definecolor{propAccentColor}{RGB}{150,100,0}
\newenvironment{prop}{
    \stepcounter{thm}
    \begin{tcolorbox}[
        boxrule=1pt,
        sharp corners,
        colback=propBackgroundColor,
        colframe=propAccentColor,
        breakable
    ]
    \color{propAccentColor}\textbf{Proposition \thethm. }\color{black}
} {
    \end{tcolorbox}
}

\definecolor{thmBackgroundColor}{RGB}{235,225,245}
\definecolor{thmAccentColor}{RGB}{50,0,100}
\renewenvironment{thm}{
    \stepcounter{thm}
    \begin{tcolorbox}[
        boxrule=1pt,
        sharp corners,
        colback=thmBackgroundColor,
        colframe=thmAccentColor,
        breakable
    ]
    \color{thmAccentColor}\textbf{Theorem \thethm. }\color{black}
} {
    \end{tcolorbox}
}

\definecolor{corBackgroundColor}{RGB}{240,250,250}
\definecolor{corAccentColor}{RGB}{50,100,100}
\newenvironment{cor}{
    \stepcounter{thm}
    \begin{tcolorbox}[
        enhanced,
        boxrule=0pt,
        frame hidden,
        sharp corners,
        colback=corBackgroundColor,
        borderline west={4pt}{0pt}{corAccentColor},
        breakable
    ]
    \color{corAccentColor}\textbf{Corollary \thethm. }\color{black}
} {
    \end{tcolorbox}
}

\definecolor{lemBackgroundColor}{RGB}{255,245,235}
\definecolor{lemAccentColor}{RGB}{250,125,0}
\newenvironment{lem}{
    \stepcounter{thm}
    \begin{tcolorbox}[
        enhanced,
        boxrule=0pt,
        frame hidden,
        sharp corners,
        colback=lemBackgroundColor,
        borderline west={4pt}{0pt}{lemAccentColor},
        breakable
    ]
    \color{lemAccentColor}\textbf{Lemma \thethm. }\color{black}
} {
    \end{tcolorbox}
}

\definecolor{proofBackgroundColor}{RGB}{255,255,255}
\definecolor{proofAccentColor}{RGB}{80,80,80}
\renewenvironment{proof}{
    \begin{tcolorbox}[
        enhanced,
        boxrule=1pt,
        sharp corners,
        colback=proofBackgroundColor,
        colframe=proofAccentColor,
        borderline west={4pt}{0pt}{proofAccentColor},
        breakable
    ]
    \color{proofAccentColor}\emph{\textbf{Proof. }}\color{black}
} {
    \qed \end{tcolorbox}
}

\definecolor{noteBackgroundColor}{RGB}{240,250,240}
\definecolor{noteAccentColor}{RGB}{30,130,30}
\newenvironment{note}{
    \begin{tcolorbox}[
        enhanced,
        boxrule=0pt,
        frame hidden,
        sharp corners,
        colback=noteBackgroundColor,
        borderline west={4pt}{0pt}{noteAccentColor},
        breakable
    ]
    \color{noteAccentColor}\textbf{Note. }\color{black}
} {
    \end{tcolorbox}
}


\fancyhf{}
\setlength{\headheight}{24pt}

\date{\today}
\title{Math 151A Homework \#5}

\begin{document}
\maketitle


\bigskip
\begin{prob}
\end{prob}
The formulas for $P_{1,2}, P_{1,4}$ are given by this Python code:

\begin{lstlisting}[language=Python]
import numpy as np
import math
import matplotlib.pyplot as plt

def f(x):
    return math.sin(2 * math.pi * x)

def P(n, x):
    m = math.floor(n * x)
    fract = n * x - m
    return f(m / n) + fract * (f((m+1) / n) - f(m / n))

x_data = np.linspace(0, 1, 400)
plt.plot(x_data, [f(x) for x in x_data])
for n in [2, 4, 8]:
    y_data = [P(n, x) for x in x_data]
    plt.plot(x_data, y_data)
plt.show()
\end{lstlisting}
\fig{Figure_1.png}
Even in the supremum norm, $\lim_{n \rightarrow \infty} \abs{f(x) - P_{1,n}} = 0$, because $P_{1, n}$ converges uniformly to $f$ (in the sup norm).

\bigskip
\begin{prob}
\end{prob}
A degree 3 polynomial has the form $f(x) = a x^3 + b x^2 + c x + d$, where $a,b,c,d$ are real numbers and $a \neq 0$ (because if $a=0$, the degree of the polynomial would be less than 3). Let $ \left\{ x_0, x_1, x_2, x_3 \right\}$ be four points in the domain of $f$, satisfying $x_0 < x_1 < x_2 < x_3$. $f$ is its own natural clamped cubic spline interpolant, because $f(x_i) = f(x_i)$ and $f'(x_i)=f'(x_i)$ for any $i \in \left\{ 0,1,2,3 \right\}$. However, $f$ is not its own natural cubic spline, because $a \neq 0$ implies $f''$ is either linearly increasing or linearly decreasing, which means $f''(x_0)$ and $f''(x_n)$ cannot both be zero.

\bigskip
\begin{prob}
\end{prob}
\[ s(x) := \begin{cases}
    s_0(x) := a_0 x^3 + b_0 x^2 + c_0 x + d_0 & 0.1 \leq x < 0.2 \\
    s_1(x) := a_1 x^3 + b_1 x^2 + c_1 x + d_1 & 0.2 \leq x \leq 0.3
\end{cases} \]
Now to find these 8 constants, we need 8 equations:
\begin{align*}
    -0.29004996 &= 0.001 a_0 + 0.01 b_0 + 0.1 c_0 + d_0 \\
    -0.56079734 &= 0.008 a_0 + 0.04 b_0 + 0.2 c_0 + d_0 \\
    -0.56079734 &= 0.008 a_1 + 0.04 b_1 + 0.2 c_1 + d_1 \\
    -0.81401972 &= 0.027 a_1 + 0.09 b_1 + 0.3 c_1 + d_1 \\
    s''(0.1)=0 \Rightarrow 0 &=  0.6 a_0 + 2 b_0 \\
    s''(0.3)=0 \Rightarrow 0 &=  1.8 a_1 + 2 b_1 \\
    s_0'(0.2)=s_1'(0.2) \Rightarrow 0 &= 0.12 a_0 + 0.4 b_0 + c_0 - 0.12 a_1 - 0.4 b_1 - c_1 \\
    s_0''(0.2)=s_1''(0.2) \Rightarrow 0 &= 1.2 a_0 + 2 b_0 - 1.2 a_1 - 2 b_1
\end{align*}

Here is the code and output for the rest of this question. Note that $f'(0.2)=s'(0.2)$, which is a result of the definition of a cubic spline. Osculating polynomial interpolation would also ensure this is true.

\begin{lstlisting}[language=Python]
import scipy.integrate as integrate
import numpy as np

A = np.matrix([
    [.001, .01, .1, 1,    0,   0,  0, 0],
    [.008, .04, .2, 1,    0,   0,  0, 0],
    [   0,   0,  0, 0, .008, .04, .2, 1],
    [   0,   0,  0, 0, .027, .09, .3, 1],
    [  .6,   2,  0, 0,    0,   0,  0, 0],
    [   0,   0,  0, 0,  1.8,   2,  0, 0],
    [ .12,  .4,  1, 0, -.12, -.4, -1, 0],
    [ 1.2,   2,  0, 0, -1.2,  -2,  0, 0]
])
B = np.matrix([[-0.29004996, -0.56079734, -0.56079734, -0.81401972, 0, 0, 0, 0]]).T

X = np.linalg.inv(A) @ B
a_0 = X[0, 0]
b_0 = X[1, 0]
c_0 = X[2, 0]
d_0 = X[3, 0]
a_1 = X[4, 0]
b_1 = X[5, 0]
c_1 = X[6, 0]
d_1 = X[7, 0]
print(f"{a_0=}\n{b_0=}\n{c_0=}\n{d_0=}\n\n{a_1=}\n{b_1=}\n{c_1=}\n{d_1=}")

def f(x):
    return x**2 * np.cos(x) - 3 * x
def f_prime(x):
    return 2 * x * np.cos(x) - x**2 * np.sin(x) - 3
def s(x):
    if x < 0.2:
        return a_0 * x**3 + b_0 * x**2 + c_0 * x + d_0
    return     a_1 * x**3 + b_1 * x**2 + c_1 * x + d_1
def s_prime(x):
    if x < 0.2:
        return 6 * a_0 * x**2 + 2 * b_0 * x + c_0
    return     6 * a_1 * x**2 + 2 * b_1 * x + c_1

print(f"\n{f(0.18)=}")
print(f"{s(0.18)=}")
print(f"Relative error = {abs((f(0.18) - s(0.18)) / f(0.18))}")
print(f"\n{f_prime(0.18)=}")
print(f"{s_prime(0.18)=}")
print(f"Relative error = {abs((f_prime(0.18) - s_prime(0.18)) / f_prime(0.18))}")

print(f"\n{f(0.2)=}\n{s(0.2)=}")

exact_integral  = integrate.quad(f, 0.1, 0.3)[0]
approx_integral = integrate.quad(s, 0.1, 0.3)[0]
print(f"\n{exact_integral=}")
print(f"{approx_integral=}")
print(f"Relative error = {abs((exact_integral - approx_integral) / approx_integral)}")



a_0=4.38124999999998
b_0=-1.3143749999999947
c_0=-2.6198488000000006
d_0=-0.01930258000000029

a_1=-4.381249999999994
b_1=3.9431249999999807
c_1=-3.671348799999997
d_1=0.05079741999999965

f(0.18)=-0.508123464353665
s(0.18)=-0.5079096640000003
Relative error = 0.0004207645752723696

f_prime(0.18)=-2.651616828775273
s_prime(0.18)=-2.2413088000000028
Relative error = 0.1547388085347094

f(0.2)=-0.5607973368863504
s(0.2)=-0.5607973400000005

exact_integral=-0.11157403517621209
approx_integral=-0.1115022805000001
Relative error = 0.0006435265349750621
\end{lstlisting}

\bigskip
\begin{prob}
\end{prob}
\begin{lstlisting}[language=Python]
import matplotlib.pyplot as plt
import numpy as np
from scipy.interpolate import CubicSpline

def f(a, x):
    return np.cos(a * x) * x**2 + 10 * x

xvals = np.linspace(0, 10, 15)
x_points = np.linspace(0, 10, 500)

for a in [1, 3, 5]:
    fvals = [f(a, x) for x in xvals]
    plt.scatter(xvals, fvals)
    spln = CubicSpline(xvals, fvals)
    y_points = [f(a, x) for x in x_points]
    plt.plot(x_points, y_points)
    y_points = [spln(x) for x in x_points]
    plt.plot(x_points, y_points)
    plt.show()
\end{lstlisting}
\fig{Figure_2.png}
\fig{Figure_3.png}
\fig{Figure_4.png}
As $a$ increases, the spline approximations get worse. That's because we're pushing the Nyquist-Shannon sampling limit.

\end{document}
