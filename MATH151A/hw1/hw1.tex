\documentclass{article}
\usepackage[margin=1in]{geometry}
\usepackage[linesnumbered,ruled,vlined]{algorithm2e}
\usepackage{amsfonts}
\usepackage{amsmath}
\usepackage{amssymb}
\usepackage{amsthm}
\usepackage{enumitem}
\usepackage{fancyhdr}
\usepackage{hyperref}
\usepackage{minted}
\usepackage{multicol}
\usepackage{pdfpages}
\usepackage{standalone}
\usepackage[many]{tcolorbox}
\usepackage{tikz-cd}
\usepackage{transparent}
\usepackage{xcolor}
% \tcbuselibrary{minted}

\author{Nathan Solomon}

\newcommand{\fig}[1]{
    \begin{center}
        \includegraphics[width=\textwidth]{#1}
    \end{center}
}

% Math commands
\renewcommand{\d}{\mathrm{d}}
\DeclareMathOperator{\id}{id}
\DeclareMathOperator{\im}{im}
\DeclareMathOperator{\proj}{proj}
\DeclareMathOperator{\Span}{span}
\DeclareMathOperator{\Tr}{Tr}
\DeclareMathOperator{\tr}{tr}
\DeclareMathOperator{\ad}{ad}
\DeclareMathOperator{\ord}{ord}
%%%%%%%%%%%%%%% \DeclareMathOperator{\sgn}{sgn}
\DeclareMathOperator{\Aut}{Aut}
\DeclareMathOperator{\Inn}{Inn}
\DeclareMathOperator{\Out}{Out}
\DeclareMathOperator{\stab}{stab}

\newcommand{\N}{\ensuremath{\mathbb{N}}}
\newcommand{\Z}{\ensuremath{\mathbb{Z}}}
\newcommand{\Q}{\ensuremath{\mathbb{Q}}}
\newcommand{\R}{\ensuremath{\mathbb{R}}}
\newcommand{\C}{\ensuremath{\mathbb{C}}}
\renewcommand{\H}{\ensuremath{\mathbb{H}}}
\newcommand{\F}{\ensuremath{\mathbb{F}}}

\newcommand{\E}{\ensuremath{\mathbb{E}}}
\renewcommand{\P}{\ensuremath{\mathbb{P}}}

\newcommand{\es}{\ensuremath{\varnothing}}
\newcommand{\inv}{\ensuremath{^{-1}}}
\newcommand{\eps}{\ensuremath{\varepsilon}}
\newcommand{\del}{\ensuremath{\partial}}
\renewcommand{\a}{\ensuremath{\alpha}}

\newcommand{\abs}[1]{\ensuremath{\left\lvert #1 \right\rvert}}
\newcommand{\norm}[1]{\ensuremath{\left\lVert #1\right\rVert}}
\newcommand{\mean}[1]{\ensuremath{\left\langle #1 \right\rangle}}
\newcommand{\floor}[1]{\ensuremath{\left\lfloor #1 \right\rfloor}}
\newcommand{\ceil}[1]{\ensuremath{\left\lceil #1 \right\rceil}}
\newcommand{\bra}[1]{\ensuremath{\left\langle #1 \right\rvert}}
\newcommand{\ket}[1]{\ensuremath{\left\lvert #1 \right\rangle}}
\newcommand{\braket}[2]{\ensuremath{\left.\left\langle #1\right\vert #2 \right\rangle}}

\newcommand{\catname}[1]{{\normalfont\textbf{#1}}}

\newcommand{\up}{\ensuremath{\uparrow}}
\newcommand{\down}{\ensuremath{\downarrow}}

% Custom environments
\newtheorem{thm}{Theorem}[section]

\definecolor{probBackgroundColor}{RGB}{250,240,240}
\definecolor{probAccentColor}{RGB}{140,40,0}
\newenvironment{prob}{
    \stepcounter{thm}
    \begin{tcolorbox}[
        boxrule=1pt,
        sharp corners,
        colback=probBackgroundColor,
        colframe=probAccentColor,
        borderline west={4pt}{0pt}{probAccentColor},
        breakable
    ]
    \color{probAccentColor}\textbf{Problem \thethm.} \color{black}
} {
    \end{tcolorbox}
}

\definecolor{exampleBackgroundColor}{RGB}{212,232,246}
\newenvironment{example}{
    \stepcounter{thm}
    \begin{tcolorbox}[
      boxrule=1pt,
      sharp corners,
      colback=exampleBackgroundColor,
      breakable
    ]
    \textbf{Example \thethm.}
} {
    \end{tcolorbox}
}

\definecolor{propBackgroundColor}{RGB}{255,245,220}
\definecolor{propAccentColor}{RGB}{150,100,0}
\newenvironment{prop}{
    \stepcounter{thm}
    \begin{tcolorbox}[
        boxrule=1pt,
        sharp corners,
        colback=propBackgroundColor,
        colframe=propAccentColor,
        breakable
    ]
    \color{propAccentColor}\textbf{Proposition \thethm. }\color{black}
} {
    \end{tcolorbox}
}

\definecolor{thmBackgroundColor}{RGB}{235,225,245}
\definecolor{thmAccentColor}{RGB}{50,0,100}
\renewenvironment{thm}{
    \stepcounter{thm}
    \begin{tcolorbox}[
        boxrule=1pt,
        sharp corners,
        colback=thmBackgroundColor,
        colframe=thmAccentColor,
        breakable
    ]
    \color{thmAccentColor}\textbf{Theorem \thethm. }\color{black}
} {
    \end{tcolorbox}
}

\definecolor{corBackgroundColor}{RGB}{240,250,250}
\definecolor{corAccentColor}{RGB}{50,100,100}
\newenvironment{cor}{
    \stepcounter{thm}
    \begin{tcolorbox}[
        enhanced,
        boxrule=0pt,
        frame hidden,
        sharp corners,
        colback=corBackgroundColor,
        borderline west={4pt}{0pt}{corAccentColor},
        breakable
    ]
    \color{corAccentColor}\textbf{Corollary \thethm. }\color{black}
} {
    \end{tcolorbox}
}

\definecolor{lemBackgroundColor}{RGB}{255,245,235}
\definecolor{lemAccentColor}{RGB}{250,125,0}
\newenvironment{lem}{
    \stepcounter{thm}
    \begin{tcolorbox}[
        enhanced,
        boxrule=0pt,
        frame hidden,
        sharp corners,
        colback=lemBackgroundColor,
        borderline west={4pt}{0pt}{lemAccentColor},
        breakable
    ]
    \color{lemAccentColor}\textbf{Lemma \thethm. }\color{black}
} {
    \end{tcolorbox}
}

\definecolor{proofBackgroundColor}{RGB}{255,255,255}
\definecolor{proofAccentColor}{RGB}{80,80,80}
\renewenvironment{proof}{
    \begin{tcolorbox}[
        enhanced,
        boxrule=1pt,
        sharp corners,
        colback=proofBackgroundColor,
        colframe=proofAccentColor,
        borderline west={4pt}{0pt}{proofAccentColor},
        breakable
    ]
    \color{proofAccentColor}\emph{\textbf{Proof. }}\color{black}
} {
    \qed \end{tcolorbox}
}

\definecolor{noteBackgroundColor}{RGB}{240,250,240}
\definecolor{noteAccentColor}{RGB}{30,130,30}
\newenvironment{note}{
    \begin{tcolorbox}[
        enhanced,
        boxrule=0pt,
        frame hidden,
        sharp corners,
        colback=noteBackgroundColor,
        borderline west={4pt}{0pt}{noteAccentColor},
        breakable
    ]
    \color{noteAccentColor}\textbf{Note. }\color{black}
} {
    \end{tcolorbox}
}


\fancyhf{}
\setlength{\headheight}{24pt}

\date{\today}
\title{Math 151A Homework \#1}

\begin{document}
\maketitle

\begin{prob}
\end{prob}
(a) The intermediate value theorem says that if a real function (real domain and codomain) is less than $a$ somewhere and greater than $a$ somewhere else, and continuous everywhere between those points, it must have gone through $a$ at some point, meaning there is a point where the function is exactly equal to $a$.

\bigskip
(b) Let $a=0.2$ and $b=0.3$, and let $f(x)=x \cos x - 2x^2 +3x - 1$. Then $f(a)=0.2 \cos (0.2) - 0.08 + 0.6 - 1= 0.2 * \cos(0.2) -.48$ which is negative because $\cos(0.2)$ can't be more than 1, and $f(b)=0.3 \cos(0.3) - 0.18 + 0.9 - 1 = 0.3 \cos(0.3) -0.28$, which is positive because I plugged it into a calculator and got 0.0066. $f: \R \rightarrow \R$ is continuous everywhere, so by IVT, there must be some $x \in (a, b)$ such that $f(x)=0$.

\bigskip
\begin{prob}
\end{prob}
\begin{align*}
    f(x) &= \sqrt{x+1} \\
    f'(x) &= \frac{1}{2} (x+1)^{-1/2} \\
    f''(x) &= -\frac{1}{4} (x+1)^{-3/2} \\
    f'''(x) &= \frac{3}{8} (x+1)^{-5/2} \\
\end{align*}
Then $f(0)=1, f'(0)= \frac{1}{2}, f''(0)= - \frac{1}{4}, f'''(0)= \frac{3}{8}$. Plugging that into the formula for Taylor series, we get $P_3(x) = 1 + \frac{1}{2} x - \frac{1}{8} x^2 + \frac{1}{16} x^3$.
\begin{center}
    \begin{tabular}{|c|c|c|c|}
        \hline
        $x$ & $P_3(x - 1)$ & $\sqrt{x}$ & Absolute error $ \abs{P_3(x)-\sqrt{x}}$ \\
        \hline
        \hline
        0.50 & 0.71094 & 0.70711 & 0.00383 \\
        \hline
        0.75 & 0.86621 & 0.86603 & 0.00019 \\
        \hline
        1.25 & 1.11816 & 1.11803 & 0.00013 \\
        \hline
        1.50 & 1.22656 & 1.22474 & 0.00182 \\
        \hline
    \end{tabular}
\end{center}
Here is the code used to generate that:

\begin{lstlisting}[language=Python]
>>> from math import sqrt
>>> def P_3(x):
...     return 1 + x/2 - x**2/8 + x**3/16
... 
>>> for x in [.5, .75, 1.25, 1.5]:
...     print(f"{x:.2f} & {P_3(x-1):.5f} & {sqrt(x):.5f} & " +
              f"{abs(P_3(x-1) - sqrt(x)):.5f} \\\\\n\\hline")
\end{lstlisting}

\bigskip
\begin{prob}
\end{prob}
\begin{enumerate}[label=(\alph*)]
    \item (i) $\frac{12}{15} + \frac{5}{15} = \frac{17}{15} = 1.13333\dots$ \par (ii) $ \left( \frac{11}{33} + \frac{9}{33} \right) - \frac{3}{20} = \frac{20}{33} - \frac{3}{20} = \frac{400 - 99}{660} = \frac{301}{660} = .456060606\dots$.
    \item (i) $0.800 + 0.333 = 1.13$ \par (ii) $ (0.333 + 0.272) - 0.150 = 0.605 - 0.150 = 0.455$
    \item (i) $0.800 + 0.333 = 1.13$ \par (ii) $ (0.333 + 0.273) - 0.150 = 0.606 - 0.150 = 0.456$
    \item (i) Chopping and rounding both give relative error of $\frac{\abs{1.13-17/15}}{17/15} = 0.0029411764$ \par (ii) With chopping, relative error is $\frac{\abs{0.455-301/660}}{301/660} = 0.00232558139$. With rounding, relative error is $\frac{\abs{0.456-301/660}}{301/660} = 0.00013289036$.
\end{enumerate}

\bigskip
\begin{prob}
\end{prob}

\begin{lstlisting}[language=Python]
>>> def f(x): return x**2 - 3
... 
>>> a = 0
>>> b = 2
>>> max_possible_error = b - a
>>> iteration = 0
>>> while max_possible_error > 1e-4:
...     iteration += 1
...     midpoint = (a + b) / 2
...     max_possible_error = midpoint - a
...     print(f"{iteration=}\t{midpoint=:.7f}\t{max_possible_error=:.7f}")
...     if f(midpoint) < 0: a = midpoint
...     else: b = midpoint
... 
iteration=1	midpoint=1.0000000	max_possible_error=1.0000000
iteration=2	midpoint=1.5000000	max_possible_error=0.5000000
iteration=3	midpoint=1.7500000	max_possible_error=0.2500000
iteration=4	midpoint=1.6250000	max_possible_error=0.1250000
iteration=5	midpoint=1.6875000	max_possible_error=0.0625000
iteration=6	midpoint=1.7187500	max_possible_error=0.0312500
iteration=7	midpoint=1.7343750	max_possible_error=0.0156250
iteration=8	midpoint=1.7265625	max_possible_error=0.0078125
iteration=9	midpoint=1.7304688	max_possible_error=0.0039062
iteration=10	midpoint=1.7324219	max_possible_error=0.0019531
iteration=11	midpoint=1.7314453	max_possible_error=0.0009766
iteration=12	midpoint=1.7319336	max_possible_error=0.0004883
iteration=13	midpoint=1.7321777	max_possible_error=0.0002441
iteration=14	midpoint=1.7320557	max_possible_error=0.0001221
iteration=15	midpoint=1.7319946	max_possible_error=0.0000610
\end{lstlisting}
So after 15 iterations, we get that $\sqrt{3} \approx 1.7319946$.

\bigskip
\begin{prob}
\end{prob}
\begin{enumerate}[label=(\alph*)]
    \item Let $\alpha = 1$, $p=0$, and $\lambda=0.1$. Then
        \[ \lim_{n \rightarrow \infty} \frac{\abs{p_{n+1}-p}}{\abs{p_n-p}^\alpha} = \lim_{n \rightarrow \infty} \frac{10^{-(n+1)} - 0}{(10^{-n}-0)^1} = \lim_{n \rightarrow \infty} \frac{10^{-n-1}}{10^{-n}} = \lambda. \]
        So $p_n$ converges to $p=0$ with order $\alpha = 1$.
    \item Let $\alpha = 2$, $p=0$, and $\lambda=1$. Then
        \[ \lim_{n \rightarrow \infty} \frac{\abs{p_{n+1}-p}}{\abs{p_n-p}^\alpha} = \lim_{n \rightarrow \infty} \frac{10^{-2^{n+1}} - 0}{(10^{-2^n}-0)^2} = \lim_{n \rightarrow \infty} \frac{10^{2(-2^n)}}{10^{2(-2^n)}} = \lambda. \]
        So $p_n$ converges to $p=0$ with order $\alpha = 2$.
\end{enumerate}

\bigskip
\begin{prob}
\end{prob}
Suppose $f: [a,b] \rightarrow \R$ is a continuous function such that $f(a)-c$ and $f(b)-c$ have opposite signs, and you are using the bisection method to find a solution to $f(x)=c$ (a root of $f(x)-c$) on $[a,b]$. After $n$ iterations, the bisection method will give you an approximation which is off by at most $2^{-n}(b-a)$, so define $\varepsilon_n := 2^{-n}(b-a)$. Then $\varepsilon_n$ converges to $\varepsilon := 0$ with order $\alpha := 1$, because 
\[ \lim_{n \rightarrow \infty} \frac{\abs{\varepsilon_{n+1}-\varepsilon}}{\abs{\varepsilon_n-\varepsilon}^\alpha} = \lim_{n \rightarrow \infty} \frac{2^{-(n+1)}(b-a)}{2^{-n}(b-a)} = \frac{1}{2}, \]
which is positive.
\par
Call the root we're going to find $p$, so $f(p)=c$, and call the midpoint after $n$ steps of the bisection method $p_n$. Since $\abs{p_n-p} \leq \varepsilon_n$ for all $n \in \N$, and we know $\varepsilon_n$ converges linearly to $\varepsilon=0$, $p_n$ converges to $p$ with at least order 1.

\bigskip
\begin{prob}
\end{prob}
The slope of the line is $(f(a_k) - f(b_k)) / (a_k - b_k)$ and the line goes through $(a_k, f(a_k))$, so the formula for the line is
\[ y = f(a_k) + \frac{f(a_k)-f(b_k)}{a_k-b_k} \cdot (x - a_k). \]
The root $x_k$ is the $x$-value such that $y=0$:
\begin{align*}
    0 &= f(a_k) + \frac{f(a_k)-f(b_k)}{a_k-b_k} \cdot (x_k - a_k) \\
    f(a_k) (b_k-a_k) &= (f(a_k)-f(b_k)) (x_k - a_k) \\
    f(a_k) b_k - f(b_k) a_k &= (f(a_k)-f(b_k)) x_k \\
    x_k &= \frac{f(a_k) b_k - f(b_k) a_k}{f(a_k)-f(b_k)}
\end{align*}

\bigskip
\begin{prob}
\end{prob}
\begin{enumerate}[label=(\alph*)]
    \item Here is the code I used:

\begin{lstlisting}[language=Matlab]
clear all
close all
a=1.75; %%%%%%%%%%%
a=1; %%%%%%%%%%%
b=2.95; %%%%%%%%%%%
b=1.2; %%%%%%%%%%%
%b=2; %%%%%%%%%%%
tol = 1e-6;
Nmax = 100;
%f=@(x) (x-1)*(x-2)*(x-3); %%%%%%%%%%%
f=@(x) x^6 - x - 1; %%%%%%%%%%%

i = 1;
success = 0;
while (i<=Nmax) && (success==0)
    midpoint = (a+b)/2; %%%%%%%%%%%
    %midpoint=(f(a)*b - f(b)*a) / (f(a) - f(b)); %%%%%%%%%%%
    if (abs(f(midpoint)) < tol)
        success = 1;
    else
        i = i + 1;
        midpoint = (a+b)/2; %%%%%%%%%%%
        %midpoint=(f(a)*b - f(b)*a) / (f(a) - f(b)); %%%%%%%%%%%
        % disp([a midpoint b])
        if (sign(f(midpoint)) == sign(f(a)))
            a = midpoint;
        else
            b = midpoint;
        end
    end
end
if (success == 1)
    disp('success!');
else
    disp('did not converge');
end
i
format long
display(midpoint)
format long
numericalsolution = fzero(f, midpoint)
\end{lstlisting}

    \item The bisection method gives an answer of 2.000000762939453 after 19 iterations, and the Reguli-Falsi method gives an answer of 1.999999999762428 after 5 iterations. The actual solution is 2.
    \item The actual solution is 1.134724138401519.
        \par
        For the interval $[1.0, 1.2]$, the bisection method gives 1.134724044799805 after 19 iterations, and the Reguli-Falsi method gives 1.134724105897137 after 8 iterations.
        \par
        For the interval $[1.0, 2.0]$, the bisection method gives 1.134724140167236 after 21 iterations and the Reguli-Falsi method gives 1.134724053091091 after 92 iterations.
        \par
        Since the Reguli-Falsi method took way more steps to converge in that last case, and since one iteration of the Reguli-Falsi method requires more FLOPs than one iteration of the bisection method, we can conclude that the Reguli-Falsi method is not always better than the bisection method.
\end{enumerate}



\end{document}
