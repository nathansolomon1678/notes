\documentclass{article}
\usepackage[margin=1in]{geometry}
\usepackage[linesnumbered,ruled,vlined]{algorithm2e}
\usepackage{amsfonts}
\usepackage{amsmath}
\usepackage{amssymb}
\usepackage{amsthm}
\usepackage{enumitem}
\usepackage{fancyhdr}
\usepackage{hyperref}
\usepackage{minted}
\usepackage{multicol}
\usepackage{pdfpages}
\usepackage{standalone}
\usepackage[many]{tcolorbox}
\usepackage{tikz-cd}
\usepackage{transparent}
\usepackage{xcolor}
% \tcbuselibrary{minted}

\author{Nathan Solomon}

\newcommand{\fig}[1]{
    \begin{center}
        \includegraphics[width=\textwidth]{#1}
    \end{center}
}

% Math commands
\renewcommand{\d}{\mathrm{d}}
\DeclareMathOperator{\id}{id}
\DeclareMathOperator{\im}{im}
\DeclareMathOperator{\proj}{proj}
\DeclareMathOperator{\Span}{span}
\DeclareMathOperator{\Tr}{Tr}
\DeclareMathOperator{\tr}{tr}
\DeclareMathOperator{\ad}{ad}
\DeclareMathOperator{\ord}{ord}
%%%%%%%%%%%%%%% \DeclareMathOperator{\sgn}{sgn}
\DeclareMathOperator{\Aut}{Aut}
\DeclareMathOperator{\Inn}{Inn}
\DeclareMathOperator{\Out}{Out}
\DeclareMathOperator{\stab}{stab}

\newcommand{\N}{\ensuremath{\mathbb{N}}}
\newcommand{\Z}{\ensuremath{\mathbb{Z}}}
\newcommand{\Q}{\ensuremath{\mathbb{Q}}}
\newcommand{\R}{\ensuremath{\mathbb{R}}}
\newcommand{\C}{\ensuremath{\mathbb{C}}}
\renewcommand{\H}{\ensuremath{\mathbb{H}}}
\newcommand{\F}{\ensuremath{\mathbb{F}}}

\newcommand{\E}{\ensuremath{\mathbb{E}}}
\renewcommand{\P}{\ensuremath{\mathbb{P}}}

\newcommand{\es}{\ensuremath{\varnothing}}
\newcommand{\inv}{\ensuremath{^{-1}}}
\newcommand{\eps}{\ensuremath{\varepsilon}}
\newcommand{\del}{\ensuremath{\partial}}
\renewcommand{\a}{\ensuremath{\alpha}}

\newcommand{\abs}[1]{\ensuremath{\left\lvert #1 \right\rvert}}
\newcommand{\norm}[1]{\ensuremath{\left\lVert #1\right\rVert}}
\newcommand{\mean}[1]{\ensuremath{\left\langle #1 \right\rangle}}
\newcommand{\floor}[1]{\ensuremath{\left\lfloor #1 \right\rfloor}}
\newcommand{\ceil}[1]{\ensuremath{\left\lceil #1 \right\rceil}}
\newcommand{\bra}[1]{\ensuremath{\left\langle #1 \right\rvert}}
\newcommand{\ket}[1]{\ensuremath{\left\lvert #1 \right\rangle}}
\newcommand{\braket}[2]{\ensuremath{\left.\left\langle #1\right\vert #2 \right\rangle}}

\newcommand{\catname}[1]{{\normalfont\textbf{#1}}}

\newcommand{\up}{\ensuremath{\uparrow}}
\newcommand{\down}{\ensuremath{\downarrow}}

% Custom environments
\newtheorem{thm}{Theorem}[section]

\definecolor{probBackgroundColor}{RGB}{250,240,240}
\definecolor{probAccentColor}{RGB}{140,40,0}
\newenvironment{prob}{
    \stepcounter{thm}
    \begin{tcolorbox}[
        boxrule=1pt,
        sharp corners,
        colback=probBackgroundColor,
        colframe=probAccentColor,
        borderline west={4pt}{0pt}{probAccentColor},
        breakable
    ]
    \color{probAccentColor}\textbf{Problem \thethm.} \color{black}
} {
    \end{tcolorbox}
}

\definecolor{exampleBackgroundColor}{RGB}{212,232,246}
\newenvironment{example}{
    \stepcounter{thm}
    \begin{tcolorbox}[
      boxrule=1pt,
      sharp corners,
      colback=exampleBackgroundColor,
      breakable
    ]
    \textbf{Example \thethm.}
} {
    \end{tcolorbox}
}

\definecolor{propBackgroundColor}{RGB}{255,245,220}
\definecolor{propAccentColor}{RGB}{150,100,0}
\newenvironment{prop}{
    \stepcounter{thm}
    \begin{tcolorbox}[
        boxrule=1pt,
        sharp corners,
        colback=propBackgroundColor,
        colframe=propAccentColor,
        breakable
    ]
    \color{propAccentColor}\textbf{Proposition \thethm. }\color{black}
} {
    \end{tcolorbox}
}

\definecolor{thmBackgroundColor}{RGB}{235,225,245}
\definecolor{thmAccentColor}{RGB}{50,0,100}
\renewenvironment{thm}{
    \stepcounter{thm}
    \begin{tcolorbox}[
        boxrule=1pt,
        sharp corners,
        colback=thmBackgroundColor,
        colframe=thmAccentColor,
        breakable
    ]
    \color{thmAccentColor}\textbf{Theorem \thethm. }\color{black}
} {
    \end{tcolorbox}
}

\definecolor{corBackgroundColor}{RGB}{240,250,250}
\definecolor{corAccentColor}{RGB}{50,100,100}
\newenvironment{cor}{
    \stepcounter{thm}
    \begin{tcolorbox}[
        enhanced,
        boxrule=0pt,
        frame hidden,
        sharp corners,
        colback=corBackgroundColor,
        borderline west={4pt}{0pt}{corAccentColor},
        breakable
    ]
    \color{corAccentColor}\textbf{Corollary \thethm. }\color{black}
} {
    \end{tcolorbox}
}

\definecolor{lemBackgroundColor}{RGB}{255,245,235}
\definecolor{lemAccentColor}{RGB}{250,125,0}
\newenvironment{lem}{
    \stepcounter{thm}
    \begin{tcolorbox}[
        enhanced,
        boxrule=0pt,
        frame hidden,
        sharp corners,
        colback=lemBackgroundColor,
        borderline west={4pt}{0pt}{lemAccentColor},
        breakable
    ]
    \color{lemAccentColor}\textbf{Lemma \thethm. }\color{black}
} {
    \end{tcolorbox}
}

\definecolor{proofBackgroundColor}{RGB}{255,255,255}
\definecolor{proofAccentColor}{RGB}{80,80,80}
\renewenvironment{proof}{
    \begin{tcolorbox}[
        enhanced,
        boxrule=1pt,
        sharp corners,
        colback=proofBackgroundColor,
        colframe=proofAccentColor,
        borderline west={4pt}{0pt}{proofAccentColor},
        breakable
    ]
    \color{proofAccentColor}\emph{\textbf{Proof. }}\color{black}
} {
    \qed \end{tcolorbox}
}

\definecolor{noteBackgroundColor}{RGB}{240,250,240}
\definecolor{noteAccentColor}{RGB}{30,130,30}
\newenvironment{note}{
    \begin{tcolorbox}[
        enhanced,
        boxrule=0pt,
        frame hidden,
        sharp corners,
        colback=noteBackgroundColor,
        borderline west={4pt}{0pt}{noteAccentColor},
        breakable
    ]
    \color{noteAccentColor}\textbf{Note. }\color{black}
} {
    \end{tcolorbox}
}


\fancyhf{}
\setlength{\headheight}{24pt}

\date{\today}
\title{Math 151A Homework \#4}

\begin{document}
\maketitle

\bigskip
\begin{prob}
\end{prob}
First, we can use $P_{0,1}$ and $P_{1,2}$ to find $P_{0,2}$:
\[ P_{0,2}(x) = \frac{(x-0)P_{1,2}(x) - (x-2)P_{0,1}(x)}{2-0} = \frac{x(3x-1) - (x-2)(x+1)}{2} = \frac{2x^2 +2}{2} = x^2+1. \]
Using the same recursive formula, we get
\[ P_{0,3}(1.5) = \frac{(1.5-0)P_{1,3}(1.5) - (1.5-3)P_{0,2}(1.5)}{3} = \frac{P_{1,3}(1.5)+P_{0,2}(1.5)}{2} = \frac{4+3.25}{2} = 3.625 \]

\bigskip
\begin{prob}
\end{prob}
\[ f(x) = a_0 + a_1 (x - x_0) + a_2 (x - x_0) (x - x_1) \]
Since $f(x_0) = 0$, we have $a_0 = 0$. Then $f(x_1) = \sin(\pi/4) = 1/\sqrt{2}$, so $a_1 = f(x_1) / (x_1 - x_2) = (1/\sqrt{2}) / (\pi/4) = 2\sqrt{2}/\pi$.
\par
Therefore $f(x_2) = 1 = (2\sqrt{2}/\pi)\cdot(\pi/2) + a_2 (\pi^2/8)$, which simplifies to
\[ a_2 = \frac{8}{\pi^2} \left( 1 - \sqrt{2} \right), \]
so the interpolating polynomial is
\[ f(x) = \frac{2\sqrt{2}}{\pi} x + \frac{8-8\sqrt{2}}{\pi^2} \left( x- \frac{\pi}{4} \right) x. \]

\bigskip
\begin{prob}
\end{prob}
\[ f[x_0,x_1,x_2] = - \frac{3}{2} = \frac{f[x_1,x_2] - f[x_0, x_1]}{x_2-x_0} = \frac{f[x_1,x_2]-5}{2-0}, \]
so $f[x_1,x_2]=2$.
\[ f[x_0,x_1] = 5 = \frac{f[x_1]-f[x_0]}{x_1-x_0} = \frac{f[x_1]+1}{1}, \]
so $f[x_1]=4$.
\[ f[x_1,x_2] = 2 = \frac{f[x_2]-f[x_1]}{x_2-x_1} = \frac{f[x_2]-4}{1}, \]
so $f[x_2]=6$.

\bigskip
\begin{prob}
\end{prob}
For the error bound to be less than $10^{-6}$, we need at least $n+1=7$ nodes.
\begin{lstlisting}[language=Python]
>>> from math import sin
>>> def err_bound(n):
...     M = 1 if n%2==0 else sin(1)
...     return M * n**(-1-n) / 4 / (n+1)
... 
>>> for i in range(1, 10):
...     print(i, err_bound(i))
... 
1 0.10518387310098706
2 0.010416666666666666
3 0.0006492831672900435
4 4.8828125e-05
5 2.2439226261543903e-06
6 1.2758018159252726e-07
7 4.5614702528754705e-09
8 2.06960572136773e-10
9 6.033288038733948e-12
\end{lstlisting}

\bigskip
\begin{prob}
\end{prob}
The points $x_0,x_1$ are equally spaced in the interval $[x_0, x_1]$, so we can use the theorem for the error bound when nodes are equally spaced. Let $n=1$, and let $M$ be the maximum value of $\abs{f^{(n+1)}(x)} = \abs{f''(x)}$ on the interval $[x_0, x_1]$. Then on that same interval,
\[ \abs{f(x)-P(x)} \leq \frac{1}{4} \cdot \frac{M}{n+1} \cdot h^{n+1} = \frac{Mh^2}{8} = \frac{h^2}{8} \max_{x\in[x_0,x_1]} \abs{f''(x)}. \]

\bigskip
\begin{prob}
\end{prob}
Here's $N=7$. The spline is the best fit, although the Chebyshev interpolator isn't bad either.
\fig{graph7.png}
Below is the graph for $N=11$. Once again, the spline is the best fit. The Lagrange polynomial has gotten worse, and has pretty bad Runge phenomena.
\fig{graph11.png}
Below is the graph for $N=20$. The Chebyshev interpolator and spline fits are both very good -- the spline looks slightly better, but I can't really tell. The Lagrange polynomial is good for $x$ near zero, but very bad near $x=\pm1$.
\fig{graph20.png}
\begin{lstlisting}[language=matlab]
%%%%%%%%%%%%%%%%%%%%%%%%%%%%%%%%%%%%%%%%%
%%%
%%% Note: If the plot below does not contain all four functions,
%%%       the command window will spit out a warning: "Ignoring
%%%       extra legend entries". 
%%%
%%%       The code will still run, but the legend label might not 
%%%       correspond to the curve that's actually shown. 
%%%
%%%       It's recommended
%%%       you comment out the 'legend' command below unless plotting
%%%       all four curves at once. 
%%%
%%%%%%%%%%%%%%%%%%%%%%%%%%%%%%%%%%%%%%%%%

N = 11;      % number of interpolation points
a = -1;      % left endpoint of interval
b = 1;       % right endpoint of interval


f = @(x) 1./(1.+25.*x.^2);      % create function 'handle'
                                % note the '.' after x in the quadratic
                                % term. If this is removed, code will break

xinterp = linspace(-1,1,N);     % equispaced interpolation points
yinterp = f(xinterp);           % values of f(x) at these points


Nplot = 1000;                   % number of evaluation/plotting points
xplot = linspace(-1,1,Nplot);   % xvals at which each function will be 
                                %    evaluated

ytrue = f(xplot);               % values of the true f(x) to be plotted

%%%%% get lagrange interpolant
ylagrange = lagrange(xplot,xinterp,yinterp);

%%%%% get chebyshev interpolant
c = chebyshev_coefficients(a, b, N, f);
ycheb = chebyshev_interpolant(a, b, N, c, Nplot, xplot);

%%%% get cubic spline interpolant
yspline = spline(xinterp, yinterp, xplot);
%                                   [this function is native to matlab and
%                                        doesn't require an extra .m file]


%%%% plot each function from -1 to 1
plot(xplot,ytrue,'g-o','MarkerIndices', 100:500:length(xplot),'MarkerSize',8); hold on; 
plot(xplot,ylagrange,'r-s', 'MarkerIndices', 400:500:length(xplot),'MarkerSize',8); hold on; 
plot(xplot,ycheb, 'm-x','MarkerIndices', 300:500:length(xplot),'MarkerSize',8); hold on; 
plot(xplot,yspline,'b-p','MarkerIndices',200:500:length(xplot),'MarkerSize',8); hold on; 
axis([-1,1,-2,2])
legend('1/(1+x^2)', 'Lagrange', 'Chebyshev', 'Spline') % only call me if all four curves are plotted
\end{lstlisting}

\end{document}
