\documentclass[12pt]{article}
\usepackage[margin=1in]{geometry}
\usepackage{amsmath}
\usepackage{amssymb}
\usepackage{amsfonts}
\usepackage{amsthm}
\newtheorem{thm}{Theorem}[section]
\newtheorem{cor}[thm]{Corollary}
\newtheorem{lem}[thm]{Lemma}
\newtheorem{prop}[thm]{Proposition}
\usepackage{tikz-cd}
\renewcommand{\d}{\mathrm{d}}

\begin{document}

\title{Math 110BH Homework 6}
\author{Nathan Solomon}
\maketitle

\section{}
\noindent\fbox{\fbox{\parbox{6.5in}{
            Let $M$ be a (left) $R$-module generated by one element. Prove that $M$ is isomorphic to the factor module $R/I$ where $I$ is a (left) ideal of $R$.
}}}\bigskip\par

\section{}
\noindent\fbox{\fbox{\parbox{6.5in}{
            Let $R$ be a commutative ring. Show that for every two $R$ modules $M$ and $N$, the group $ \operatorname{Hom}_R(M,N)$ has a structure of an $R$-module.
}}}\bigskip\par

\section{}
\noindent\fbox{\fbox{\parbox{6.5in}{
            Let $M$ be a (left) $R$-module, $N \subset M$ a submodule. Prove that if $N$ and $M/N$ are finitely generated, then so is $M$.
}}}\bigskip\par

\section{}
\noindent\fbox{\fbox{\parbox{6.5in}{
            Prove that for any (left) $R$-module $M$, the groups $ \operatorname{Hom}_R(R,M)$ and $M$ are isomorphic.
}}}\bigskip\par

\section{}
\noindent\fbox{\fbox{\parbox{6.5in}{
            Let $f: R^n \rightarrow R^m$ be a ring $R$-module homomorphism. Show that there is a unique $m \times n$ matrix $A$ such that $F(x)=A \cdot x$ for any $x \in R^n$.
}}}\bigskip\par

\section{}
\noindent\fbox{\fbox{\parbox{6.5in}{
    Let $R$ be a commutative ring and $I \subset R$ an ideal. Prove that if $I$ is a free $R$-module, then $I$ is a principal ideal.
}}}\bigskip\par

\section{}
\noindent\fbox{\fbox{\parbox{6.5in}{
            Show that $\mathbb{Q}$ is not a free abelian group ($\mathbb{Z}$-module).
}}}\bigskip\par
Suppose it has a basis and then try to find a contradiction?

\section{}
\noindent\fbox{\fbox{\parbox{6.5in}{
            Prove that a free finitely generated (left) $R$-module has a finite basis.
}}}\bigskip\par

\section{}
\noindent\fbox{\fbox{\parbox{6.5in}{
            Let $M$ be a (left) $R$-module, $I \subset R$ an ideal. Denote by $IM$ the submodule of $M$ generated by the products $am$ for all $a \in I$ and $m \in M$.
            \begin{itemize}
                \item (a) Assume that $IM=0$. Show that $M$ admits a structure of a (left) module over the factor ring $R/I$.
                \item (b) Show that $M/IM$ admits a structure of a (left) module over the factor ring $R/I$.
                \item (c) Prove that if $M$ is a free $R$-module, then $M/IM$ is a free $R/I$-module. Hint: Show that if $S$ is a basis for $M$< then the set of cosets $ \left\{ s+IM, s\in S \right\}$ is a basis for $M/IM$.
                \item (d) Let $R$ be a nonzero commutative ring. Prove that if (left) $R$-module $R^n$ and $R^m$ are isomorphic, then $n=m$. Deduce that every two bases for a free finitely generated $R$-module have the same number of elements. Hint: Consider modules over the factor ring $R/I$ where $I$ is a maximal ideal of $R$.
            \end{itemize}
}}}\bigskip\par
\begin{itemize}
    \item (a)
    \item (b)
    \item (c)
    \item (d)
\end{itemize}

\section{}
\noindent\fbox{\fbox{\parbox{6.5in}{
            Let $A$ be an abelian group, $f \in \operatorname{End}(A)$. Show that $A$ admits a $\mathbb{Z}[x]$-module structure such that $x \cdot a = f(a)$ for all $a \in A$.
}}}\bigskip\par


\end{document}
