\documentclass[12pt]{article}
\usepackage[margin=1in]{geometry}
\usepackage{amsmath}
\usepackage{amsfonts}
\usepackage{amsthm}
\newtheorem{thm}{Theorem}[section]
\newtheorem{cor}[thm]{Corollary}
\newtheorem{lem}[thm]{Lemma}
\usepackage{tikz-cd}
\renewcommand{\d}{\mathrm{d}}

\begin{document}

\title{Math 110BH homework 2}
\author{Nathan Solomon}
\maketitle

\textbf{Due January 23rd}

\section{}
\noindent\fbox{\fbox{\parbox{6.5in}{
            Prove that every (left) ideal of the product $R \times S$ of two rings is a product $I \times J$, where $I \subset R$ and $J \subset S$ are (left) ideals.
}}}\bigskip\par

\section{}
\noindent\fbox{\fbox{\parbox{6.5in}{
            \begin{itemize}
                \item (a) Find all idempotents in $ \mathbb{Z}/105\mathbb{Z}$.
                \item (b) Prove that $ \mathbb{Z}/p^n\mathbb{Z}$, $p$ a prime, has no nontrivial idempotents.
    \end{itemize}
}}}\bigskip\par
\begin{itemize}
    \item (a)
        \[ \{0, 1, 15, 21, 36, 70, 85, 91 \} \]
    \item (b)
\end{itemize}

\section{}
\noindent\fbox{\fbox{\parbox{6.5in}{
            Suppose a commutative ring has finitely many idempotents. Prove that the number of idempotents is a power of 2.
}}}\bigskip\par

\section{}
\noindent\fbox{\fbox{\parbox{6.5in}{
            Show that the ring $M_2( \mathbb{R})$ has infinitely many idempotents.
}}}\bigskip\par
Consider projection matrices

\section{}
\noindent\fbox{\fbox{\parbox{6.5in}{
    Describe all homomorphisms from $ \mathbb{Z} \times \mathbb{Z}$ to $ \mathbb{Z}$. In each case determine the kernel and the image.
}}}\bigskip\par

\section{}
\noindent\fbox{\fbox{\parbox{6.5in}{
    Prove that an element $a$ of a commutative ring $R$ is invertible if and only if $a$ does not belong to any maximal ideal of $R$.
}}}\bigskip\par

\section{}
\noindent\fbox{\fbox{\parbox{6.5in}{
    Determine all maximal and prime ideals of $ \mathbb{Z}/n\mathbb{Z}$.
}}}\bigskip\par

\section{}
\noindent\fbox{\fbox{\parbox{6.5in}{
            Let $R$ be a commutative ring. The \textit{radical} $\operatorname{Rad}(R)$ of $R$ is the intersection of all maximal ideals in $R$.
            \begin{itemize}
                \item (a) Determine $\operatorname{Rad}( \mathbb{Z})$ and $ \operatorname{Rad}( \mathbb{Z}/12\mathbb{Z})$.

                \item (b) Prove that $ \operatorname{Rad}(R)$ consists of all elments $a \in R$ such that $1+ab$ is invertible for all $b \in R$.
            \end{itemize}
}}}\bigskip\par

\section{}
\noindent\fbox{\fbox{\parbox{6.5in}{
            \begin{itemize}
                \item (a) Prove that every nilradical $\operatorname{Nil}(R)$ of a commutative ring $R$ is contained in every prime ideal of $R$.
                \item (b) Prove that $\operatorname{Nil}(R) \subset \operatorname{Rad}(R)$.
    \end{itemize}
}}}\bigskip\par

\section{}
\noindent\fbox{\fbox{\parbox{6.5in}{
            Let $A$ be an abelian group (written additively). Define a product on the (additive) group $R = \mathbb{Z} \oplus A$ by $(n,a) \cdot (m,b) = (nm, nb_ma)$.
            \begin{itemize}
                \item (a) Prove that $R$ is a ring.
                \item (b) Determine all prime and maximal ideals of $R$.
            \end{itemize}
}}}\bigskip\par

\end{document}
