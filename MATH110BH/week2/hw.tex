\documentclass[12pt]{article}
\usepackage[margin=1in]{geometry}
\usepackage{amsmath}
\usepackage{amsfonts}
\usepackage{amsthm}
\newtheorem{thm}{Theorem}[section]
\newtheorem{cor}[thm]{Corollary}
\newtheorem{lem}[thm]{Lemma}
\usepackage{tikz-cd}
\renewcommand{\d}{\mathrm{d}}

\begin{document}

\title{Math 110BH homework 2}
\author{Nathan Solomon}
\maketitle

\textbf{Due January 23rd}

\section{}
\noindent\fbox{\fbox{\parbox{6.5in}{
            Prove that every (left) ideal of the product $R \times S$ of two rings is a product $I \times J$, where $I \subset R$ and $J \subset S$ are (left) ideals.
}}}\bigskip\par
Let $K$ be any left ideal of $R \times S$. Then for any $(a,b) \in K$ and any $(x,y) \in R \times S$, $(x,y)\cdot(a,b)=(xa,yb)$ is also in $K$. Let $\pi_1: R \times S \rightarrow R$ and $\pi_2: R \times S \rightarrow S$ be the projection homomorphisms which take any $(x,y)$ to $x$ and to $y$, respectively. Define $I$ to be $\pi_1(K)$ and $J$ to be $\pi_2(K)$. $I$ is a left ideal of $R$ because:
\begin{itemize}
    \item It contains zero -- since $(0,0) \in R \times S$ and $\pi_1((0,0)) = 0$, $I$ also contains 0.
    \item It is closed under addition -- if $a_1, a_2 \in I$, then because $\pi_1$ is surjective, there exist elements $b_1, b_2 \in J$ such that $(a_1, b_1)$ and $(a_2, b_2) \in K$, which implies $(a_1 + a_2, b_1 + b_2) \in K$, so $a_1 + b_2$ is in $\pi_1(K)$.
    \item It is closed under left multiplication by any element of $I$ -- if $a \in I$, then by the same logic, there exists some $(a,b) \in K$, so for any $(x,y) \in R \times S$, $(xa, yb)$ is also in $K$, which implies $xa$ is in $I$.
\end{itemize}
So $I$ is a left ideal of $R$, and by the same reasoning, $J$ is a left ideal of $S$, and we already stated that $K = I \times J$.

\section{}
\noindent\fbox{\fbox{\parbox{6.5in}{
            \begin{itemize}
                \item (a) Find all idempotents in $ \mathbb{Z}/105\mathbb{Z}$.
                \item (b) Prove that $ \mathbb{Z}/p^n\mathbb{Z}$, $p$ a prime, has no nontrivial idempotents.
    \end{itemize}
}}}\bigskip\par
\begin{itemize}
    \item (a) The python code ``print([x for x in range(105) if x**2\%105==x])" shows that the answer is
        \[ \{0, 1, 15, 21, 36, 70, 85, 91 \}. \]
        Alternatively, we can use the Chinese Remainder Theorem to say that $ \mathbb{Z}/105\mathbb{Z}$ is ring-isomorphic to $ \mathbb{Z}/3\mathbb{Z} \times \mathbb{Z}/5\mathbb{Z} \times \mathbb{Z}/7\mathbb{Z}$. An element of $([a], [b], [c]) \in \mathbb{Z}/3\mathbb{Z} \times \mathbb{Z}/5\mathbb{Z} \times \mathbb{Z}/7\mathbb{Z}$ is idempotent if and only $[a]_3$, $[b]_5$, and $[c]_7$ are all idempotent.
        \par
        If $a$ is an idempotent element in a field $\mathbb{F}$, then $a^2=a$, so $a$ can be zero. If $a$ is nonzero, then $a$ is invertible, so $a^{-1}a^2=a^{-1}a$, meaning is the identity. We know that $ \mathbb{Z}/p\mathbb{Z}$ is a field when $p$ is prime, so the only idempotent elements of $ \mathbb{Z}/p\mathbb{Z}$ are $[0]$ and $[1]$.
        \par
        Therefore in $\mathbb{Z}/3\mathbb{Z} \times \mathbb{Z}/5\mathbb{Z} \times \mathbb{Z}/7\mathbb{Z}$, the idempotent elements are precisely the 8 elements for which each component is either $[0]$ or $[1]$. That set is generated by $(1,0,0)$, $(0,1,0)$, and $(0,0,1)$. To find the element of $ \mathbb{Z}/105\mathbb{Z}$ which corresponds to $(1,0,0)$, we need to find a number which is congurent to 1 (modulo 3) and is a multiple of both 5 and 7, so we test all multiples of 35 between 0 and 104 until we find that it's 70. By the same method, we find $(0,1,0)$ corresponds to 21 and $(0,0,1)$ corresponds to 15, and we can then take the sum (modulo 105) of all subsets of $\{70,21,15\}$ to get the full list of idempotents:
        \[ \{0, 1, 15, 21, 36, 70, 85, 91 \}. \]
        \par
    \item (b) Suppose there exists an idempotent element $a \in \mathbb{Z}/p^n\mathbb{Z}$. Then $a(a-1)$ is a multiple of $p^n$. If $a$ is a multiple of $p$, then $a-1$ is not, and if $a-1$ is a multiple of $p$, then $a$ is not. Therefore either $a$ or $a-1$ divides $p^n$, which is true if and only if $a$ is equal to $[0]$ or $[1]$ in $ \mathbb{Z}/p^n\mathbb{Z}$.
\end{itemize}

\section{}
\noindent\fbox{\fbox{\parbox{6.5in}{
            Suppose a commutative ring has finitely many idempotents. Prove that the number of idempotents is a power of 2.
}}}\bigskip\par
Lemma: if $x$ is idempotent and $x \neq 1$, then $x$ is not invertible. Proof: if $x$ is invertible and idempotent, then $x=x^{-1}x^2=x^{-1}x=1$.
\par
Let $R$ be a ring with finitely many idempotents. If $R$ does not contain any nontrivial idempotents, than it has either 1 or 2 idempotents, so we're done.
\par
If $R$ contains a nontrivial idempotent $a$, then $R=aR+(1-a)R$, so by the Chinese Remainder Theorem, $R$ is isomorphic to $R/aR \times R/(1-a)R$, which implies the number of idempotents in $R$ is the number of idempotents in $R/aR$ times the number of idempotents in $R/(1-a)R$. By the lemma above, $a$ and $1-a$ are not invertible, so neither $aR$ nor $(1-a)R$ are unit ideals. Therefore $R/aR$ and $R/(1-a)R$ are both nonzero rings, meaning they contain at least two distinct idempotents (zero and one).
\par
If we let $n$ be the number of idempotents in a ring $R$, the paragraph above proves that if $n$ is greater than two, $n$ is the product of two natural numbers which are each at least two, and which each represent the number of idempotents in some other ring. Since $n$ is finite, this means we can repeatedly decompose $R$ as a product of rings until $R$ is expressed as a product of rings which each have exactly one or two elements, which means the number of idempotents in $R$ is a power of 2.

\section{}
\noindent\fbox{\fbox{\parbox{6.5in}{
            Show that the ring $M_2( \mathbb{R})$ has infinitely many idempotents.
}}}\bigskip\par
For any real number $a$, the matrix
\[ A := \begin{bmatrix}
    0 & 0 \\
    a & 1
\end{bmatrix} \]
satisfies the equation $A^2=A$, so there are infinitely many idempotents in $M_2( \mathbb{R})$. More generally, a real matrix is idempotent if and only if it represents a projection.

\section{}
\noindent\fbox{\fbox{\parbox{6.5in}{
    Describe all homomorphisms from $ \mathbb{Z} \times \mathbb{Z}$ to $ \mathbb{Z}$. In each case, determine the kernel and the image.
}}}\bigskip\par
Let $f$ be a ring homomorphism from $ \mathbb{Z} \times \mathbb{Z}$ to $ \mathbb{Z}$. Since the multiplicative identity in $ \mathbb{Z} \times \mathbb{Z}$ is $(1,1)$, we know that $f((1,1)) = 1$.
\par
Now let $x=f((1,0))$. Since $1=f((1,1))=f((1,0)+(0,1))=x+f((0,1))$, we can say that $f((0,1))=1-x$, and so for any $a,b \in \mathbb{Z}$, $f((a,b))=xa+(1-x)b$. Therefore $f$ is fully defined by what $x$ is, and $x$ can be any integer, so every ring homomorphism $f: \mathbb{Z} \times \mathbb{Z} \rightarrow \mathbb{Z}$ can be defined by
\[ f((a,b))=ax+(1-x)b \text{ for some integer $x$.} \]
For any $a \in \mathbb{Z}$, $f((a,a))=a$, so $f$ is surjective, so $\operatorname{Im}(f)=0$ no matter what $x$ is. The kernel of $f$ is the set of pairs $(a,b)$ for which $ax=(x-1)b$. Below are some examples.
\begin{center}
    \begin{tabular}{|c|c|}
        \hline
        $x$ & $\operatorname{Ker}(f)$ \\
        \hline
        \hline
        0 & $\{\dots, (-1,0), (0,0), (1,0), \dots\}$ \\
        \hline
        1 & $\{\dots, (0,-1), (0,0), (0,1), \dots \}$ \\
        \hline
        2 & $\{\dots, (-1,-2), (0,0), (1,2), \dots \}$ \\
        \hline
        3 & $\{\dots, (-2,-3), (0,0), (2,3), \dots \}$ \\
        \hline
    \end{tabular}
\end{center}

\section{}
\noindent\fbox{\fbox{\parbox{6.5in}{
    Prove that an element $a$ of a commutative ring $R$ is invertible if and only if $a$ does not belong to any maximal ideal of $R$.
}}}\bigskip\par
Let $a$ be an invertible element of $R$, and let $I$ be an ideal of $R$ which contains $a$. Then for any element $x \in R$, $I$ also contains $(xa^{-1})a=x$, so $I=R$, therefore any ideal $I$ which contains an invertible element $a$ is not maximal.
\par
By that same logic, if $a$ is an element of $R$ which is not invertible and $I$ is an ideal of $R$ which contains $a$, then $I \neq R$. Then in the poset of ideals of $R$, there exists a chain of ideals which includes $I$, and if Zorn's lemma is true, then that chain terminates in a maximal ideal, which would have to contain $a$.
\par
So assuming Zorn's lemma, an element of a commutative ring is invertible if and only if it does not belong to any maximal ideal.

\section{}
\noindent\fbox{\fbox{\parbox{6.5in}{
    Determine all maximal and prime ideals of $ \mathbb{Z}/n\mathbb{Z}$.
}}}\bigskip\par

\section{}
\noindent\fbox{\fbox{\parbox{6.5in}{
            Let $R$ be a commutative ring. The \textit{radical} $\operatorname{Rad}(R)$ of $R$ is the intersection of all maximal ideals in $R$.
            \begin{itemize}
                \item (a) Determine $\operatorname{Rad}( \mathbb{Z})$ and $ \operatorname{Rad}( \mathbb{Z}/12\mathbb{Z})$.

                \item (b) Prove that $ \operatorname{Rad}(R)$ consists of all elments $a \in R$ such that $1+ab$ is invertible for all $b \in R$.
            \end{itemize}
}}}\bigskip\par
\begin{itemize}
    \item (a) We proved in class that the set of maximal ideals of $ \mathbb{Z}$ is the set of ideals generated by prime numbers, so a number $x$ is only in the intersection of all ideals if it is a multiple of every prime number. Therefore $\operatorname{Rad}( \mathbb{Z})=0$.
        \par
        In the previous question, we showed that every maximal ideal of $ \mathbb{Z}/12\mathbb{Z}$ has the form
    \item (b)
\end{itemize}

\section{}
\noindent\fbox{\fbox{\parbox{6.5in}{
            \begin{itemize}
                \item (a) Prove that every nilradical $\operatorname{Nil}(R)$ of a commutative ring $R$ is contained in every prime ideal of $R$.
                \item (b) Prove that $\operatorname{Nil}(R) \subset \operatorname{Rad}(R)$.
    \end{itemize}
}}}\bigskip\par
\begin{itemize}
    \item (a) Let $x$ be some element of the nilradical of $R$. Then there exists a positive integer $m$ such that $x^m=0$. Let $P$ be a prime ideal of $R$.
        \par
        Base case: $x^n$ is in $P$ when $n=m$, because $x^m=0\in P$.
        \par
        Inductive step: if $x^n$ is in $P$, then since $x^n=x^{n-1}x$, either $x$ or $x^{n-1}$ is in $P$.
        \par
        Since $m$ is finite, induction is valid here, so $x^n$ is in $P$ for any positive integer $n$ less than or equal to $m$. Therefore $x \in P$.
    \item (b) Every maximal ideal is prime, and we showed that if $x$ is nilpotent, every prime ideal contains $x$. Therefore every maximal ideal contains every nilpotent element, so
        \[ \operatorname{Nil}(R) \subset \operatorname{Rad}(R). \]
\end{itemize}

\section{}
\noindent\fbox{\fbox{\parbox{6.5in}{
            Let $A$ be an abelian group (written additively). Define a product on the (additive) group $R = \mathbb{Z} \oplus A$ by $(n,a) \cdot (m,b) = (nm, nb+ma)$.
            \begin{itemize}
                \item (a) Prove that $R$ is a ring.
                \item (b) Determine all prime and maximal ideals of $R$.
            \end{itemize}
}}}\bigskip\par
\begin{itemize}
    \item (a) $R$ is an abelian group under addition. $R$ contains a multiplicative identity, which is $(1,0)$. From the definition of the product ($\cdot$), we see that $R$ is also associative, left-distributive, and right-distributive.
    \item (b)
\end{itemize}

\end{document}
