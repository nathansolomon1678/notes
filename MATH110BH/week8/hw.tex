\documentclass[12pt]{article}
\usepackage[margin=1in]{geometry}
\usepackage{amsmath}
\usepackage{amssymb}
\usepackage{amsfonts}
\usepackage{amsthm}
\newtheorem{thm}{Theorem}[section]
\newtheorem{cor}[thm]{Corollary}
\newtheorem{lem}[thm]{Lemma}
\newtheorem{prop}[thm]{Proposition}
\usepackage{tikz-cd}
\renewcommand{\d}{\mathrm{d}}

\begin{document}

\title{Math 110BH Homework 8}
\author{Nathan Solomon}
\maketitle

\section{}
\noindent\fbox{\fbox{\parbox{6.5in}{
            Let $F$ be a free (left) $R$-module with basis $ \left\{ x_1, x_2, \dots, x_n \right\}$ and let $M$ be an $R$-module. Prove that for any elements $m_1,m_2,\dots, m_n \in M$, there is a unique $R$-module homomorphism $f: F \rightarrow M$ such tat $f(x_i)=m_i$ for all $i$.
}}}\bigskip\par

\section{}
\noindent\fbox{\fbox{\parbox{6.5in}{
            Let $f: M\rightarrow N$ be a surjective homomorphism of (left) $R$-modules. Prove that if $N$ is free, there is a homomorphism of (left) $R$-modules $g: N\rightarrow M$ such that $f \circ g$ is the identity of $N$.
}}}\bigskip\par

\section{}
\noindent\fbox{\fbox{\parbox{6.5in}{
            Let $f$ be a linear operator in a vector space $V$ over $\mathbb{R}$ such that $f(f(v))=-v$ for all $v \in V$. Prove that $V$ has the structure of a vector space over $\mathbb{C}$ such that $iv=f(v)$ for all $v \in V$.
}}}\bigskip\par

\section{}
\noindent\fbox{\fbox{\parbox{6.5in}{
    Show that a submodule of a cyclic module over a PID is also cyclic.
}}}\bigskip\par

\section{}
\noindent\fbox{\fbox{\parbox{6.5in}{
    Let $a$ and $b$ be nonzero elements of a PID $R$. Prove that $R/aR \oplus R/bR \cong R/cR \oplus R/dR$, where $c$ is a least common multiple and $d$ is a greatest common divisor of $a$ and $b$.
}}}\bigskip\par

\section{}
\noindent\fbox{\fbox{\parbox{6.5in}{
            Let $M$ be a finitely generated torsion module over a PID $R$ and let $n = | \operatorname{IF}(M)|$. Prove that $M$ can be generated by $n$ elements and cannot be generated by fewer than $n$ elements.
}}}\bigskip\par

\section{}
\noindent\fbox{\fbox{\parbox{6.5in}{
            A module is colled \textit{indexomposable} if it is not equal to the direct sum of its nonzero submodules. Prove that a finitely generated module $M$ over a PID $R$ is indexomposable if and only if $M \cong R$ or $M \cong R/P^n$, where $P$ is a prime ideal of $R$ and $n \geq 0$.
}}}\bigskip\par

\section{}
\noindent\fbox{\fbox{\parbox{6.5in}{
    Let $n$ be an integer. Prove that every abelian group $A$ with $nA=0$ has the structure of a $\mathbb{Z}n\mathbb{Z}$-module.
}}}\bigskip\par

\section{}
\noindent\fbox{\fbox{\parbox{6.5in}{
            Classify all finite $\mathbb{Z}/n\mathbb{Z}$-modules up to isomorphism. (Hint: Use the classification of finite abelian groups.)
}}}\bigskip\par

\section{}
\noindent\fbox{\fbox{\parbox{6.5in}{
            Let $M$ be a subgroup of a free abelian group $F$ of finite rank. Suppose that $M \cap pF=pM$ for all prime integers $p$. Prove that the quotient group $F/M$ is free.
}}}\bigskip\par

\end{document}
