\documentclass[12pt]{article}
\usepackage[margin=1in]{geometry}
\usepackage{amsmath}
\usepackage{amssymb}
\usepackage{amsfonts}
\usepackage{amsthm}
\newtheorem{thm}{Theorem}[section]
\newtheorem{cor}[thm]{Corollary}
\newtheorem{lem}[thm]{Lemma}
\usepackage{tikz-cd}
\renewcommand{\d}{\mathrm{d}}

\begin{document}

\title{Math 110BH Homework 4}
\author{Nathan Solomon}
\maketitle

\section{}
\noindent\fbox{\fbox{\parbox{6.5in}{
            Prove that the ideal in $\mathbb{Z}[\sqrt{-5}]$ generated by 2 and $1+\sqrt{-5}$ is not principal.
}}}\bigskip\par

\section{}
\noindent\fbox{\fbox{\parbox{6.5in}{
            Determine whether the ring $\mathbb{Z}[\sqrt{5}]$ is a PID.
}}}\bigskip\par
We can prove that it's not a UFD

\section{}
\noindent\fbox{\fbox{\parbox{6.5in}{
            Let $R=\mathbb{Z}[i]$ be the ring of Gauss integers and let $p$ be a prime integer such that $p \equiv 3 \pmod{4}$. Prove that $p$ is prime in $R$.
}}}\bigskip\par
Suppose there exist $a+bi, c+di \in R$ such that $p=(a+bi)(c+di)$. Then
\[ p=(ac-bd)+(bc+ad)i \]

\section{}
\noindent\fbox{\fbox{\parbox{6.5in}{
            Let $R = \mathbb{Z}[i]$ and let $p$ be a prime integer such that $p \equiv 1 \pmod{4}$.
            \begin{itemize}
                \item (a) Prove that $p$ is not prime in $R$. (Hint: use HW 5, Problem 9 in 110AH).
                \item (b) Prove that there are integers $a$ and $b$ such that $p=a^2+b^2$.
            \end{itemize}
}}}\bigskip\par

\section{}
\noindent\fbox{\fbox{\parbox{6.5in}{
    Let $R$ be a PID and let $a$ be a prime element in $R$. Prove that the ideal $pR$ is maximal.
}}}\bigskip\par

\section{}
\noindent\fbox{\fbox{\parbox{6.5in}{
    Prove that the product of two Noetherian rings is also Noetherian.
}}}\bigskip\par

\section{}
\noindent\fbox{\fbox{\parbox{6.5in}{
            An integral domain in which every ideal generated by two elements is principal is called a \textit{Bezout domain}. Prove that a ring $R$ is a PID if and only if $R$ is a Noetherian Bezout domain.
}}}\bigskip\par

\section{}
\noindent\fbox{\fbox{\parbox{6.5in}{
            Let $R_1 \subset R_2 \subset R_3 \subset \dots$ be a chain of countably many subrings of a ring $R$ such that $R=\cup R_i$. Suppose that all the $R_i$ are UFD and any prime element in $R_i$ is prime in $R_{i+1}$. Prove that $R$ is a UFD.
}}}\bigskip\par

\section{}
\noindent\fbox{\fbox{\parbox{6.5in}{
            Prove that the polynomial ring $\mathbb{Z}[x_1,x_2,x_3,\dots]$ in countably many variables is a UFD but not a Noetherian ring.
}}}\bigskip\par
It's not a Noetherian ring because we have a chain of infinitely many ideals:
\[ \mathbb{Z} \subset \mathbb{Z}[x_1] \subset \mathbb{Z}[x_1,x_2] \subset \cdots \]
but it's a UFD because every polynomial in that ring belongs to an ideal of the form
\[ \mathbb{Z}[x_1,x_2,\dots,x_n] \]
and we can use induction to prove that all ideals of that form are UFDs.

\section{}
\noindent\fbox{\fbox{\parbox{6.5in}{
            Let $R=\mathbb{Z}[\sqrt{-5}]$. Prove that the product of the two ideals $2R+(1+\sqrt{-5})R$ and $3R+(1+\sqrt{-5})$ in $R$ is the principal ideal $(1+\sqrt{-5})R$.
}}}\bigskip\par

\end{document}
