\documentclass[12pt]{article}
\usepackage[margin=1in]{geometry}
\usepackage{amsmath}
\usepackage{amssymb}
\usepackage{amsfonts}
\usepackage{amsthm}
\newtheorem{thm}{Theorem}[section]
\newtheorem{cor}[thm]{Corollary}
\newtheorem{lem}[thm]{Lemma}
\newtheorem{prop}[thm]{Proposition}
\usepackage{tikz-cd}
\renewcommand{\d}{\mathrm{d}}

\begin{document}

\title{Math 110BH Homework 9}
\author{Nathan Solomon}
\maketitle

\section{}
\noindent\fbox{\fbox{\parbox{6.5in}{
            Find the invariant factors of the quotient group $\mathbb{Z}^3/N$, where $N$ is generated by $(-4,4,2)$, $(16,-4,-8)$, $(12,0,-6)$, and $(8,4,2)$.
}}}\bigskip\par
The element $(12,0,-6)$ is generated by the other elements, since $(-4,4,2)+(16,-4,-8)=(12,0,-6)$. That means we can ignore the generator $(12,0,-6)$, and the other 3 elements will still generate $N$.
\par
The coefficient matrix is then
\[ A = \begin{pmatrix}
    -4 & 4 & 2 \\
    16 & -4 & -8 \\
    8 & 4 & 2
\end{pmatrix} \rightarrow \begin{pmatrix}
    2 & 4 & -4 \\
    -8 & -4 & 16 \\
    2 & 4 & 8
\end{pmatrix} \rightarrow \begin{pmatrix}
    2 & -4 & 4 \\
    0 & 12 & 0 \\
    0 & 0 & 12
\end{pmatrix} \rightarrow \begin{pmatrix}
    2 & 0 & 0 \\
    0 & 12 & 0 \\
    0 & 0 & 12
\end{pmatrix} \]
so the invariant factors are $ \left\{ 2\mathbb{Z}, 12\mathbb{Z}, 12\mathbb{Z} \right\}$.

\section{}
\noindent\fbox{\fbox{\parbox{6.5in}{
    Find the rational canonical form over $\mathbb{Q}$ of the matrix 
    \[\begin{pmatrix}
        -2 & 0 & 0 \\
        -1 & -4 & -1 \\
        2 & 4 & 0
    \end{pmatrix}\]
}}}\bigskip\par
Row reducing ($xI$ minus that matrix) gives
\begin{align*}
    \begin{bmatrix}
        x+2 & 0 & 0 \\
        -1 & x+4 & -1 \\
        2 & 4 & x
        \end{bmatrix} &\rightarrow \begin{bmatrix}
        1 & -x-4 & 1 \\
        x+2 & 0 & 0 \\
        2 & 4 & x
    \end{bmatrix} \rightarrow \begin{bmatrix}
        1 & -x-4 & 1 \\
        0 & x^2+6x+8 & -x-2 \\
        0 & 2x+12 & x-2
    \end{bmatrix}\\
    & \rightarrow \begin{bmatrix}
        1 & 0 & 0 \\
        0 & 2x+12 & x-2 \\
        0 & x^2+6x+8 & -x-2
    \end{bmatrix} \rightarrow \begin{bmatrix}
        1 & 0 & 0 \\
        0 & 16 & x-2 \\
        0 & x^2 + 8x + 12 & -x-2
    \end{bmatrix} \\
    & \rightarrow \begin{bmatrix}
        1 & 0 & 0 \\
        0 & 16 & x-2 \\
        0 & x^2+8x+12 & -4
    \end{bmatrix} \rightarrow \begin{bmatrix}
        1 & 0 & 0 \\
        0 & x-2 & 16 \\
        0 & -4 & x^2+8x+12
    \end{bmatrix}
\end{align*}
Since the only invariant factor is $(x+2)^3=x^3+6x^2+12x+8$, the RCF is
\[ \begin{bmatrix}
    0 & 0 & -8 \\
    1 & 0 & -12 \\
    0 & 1 & -6
\end{bmatrix} \]
NOTE: THIS IS WRONG, THERE SHOULD BE TWO INVARIANT FACTORS

\section{}
\noindent\fbox{\fbox{\parbox{6.5in}{
    Find the rational canonical form over $\mathbb{Z}/2\mathbb{Z}$ of the matrix
    \[ \begin{pmatrix}
        1 & 1 & 0 \\
        0 & 1 & 1 \\
        0 & 0 & 1
    \end{pmatrix} \]
}}}\bigskip\par
Call that matrix $A$. Then by row-reducing $xI-A$, we get
\begin{align*}
    xI-A & = \begin{bmatrix}
        x-1 & 1 & 0 \\
        0 & x-1 & 1 \\
        0 & 0 & x-1
    \end{bmatrix} \rightarrow \begin{bmatrix}
        1 & x-1 & 0 \\
        x-1 & 0 & 1 \\
        0 & 0 & x-1
    \end{bmatrix} \\
         & \rightarrow \begin{bmatrix}
             1 & 0 & 0 \\
             0 & -(x-1)^2 & 1 \\
             0 & 0 & x-1
         \end{bmatrix} \rightarrow \begin{bmatrix}
             1 & 0 & 0 \\
             0 & x^2 + 1 & 1 \\
             0 & 0 & x+1
         \end{bmatrix} \\
         & \rightarrow \begin{bmatrix}
             1 & 0 & 0 \\
             0 & 1 & 0 \\
             0 & 0 & -(x+1)(x^2+1)
         \end{bmatrix}.
\end{align*}
Therefore the only invariant factor is $x^3+x^2+x+1$, so the RCF is
\[ \begin{bmatrix}
    0 & 0 & -1 \\
    1 & 0 & -1 \\
    0 & 1 & -1
\end{bmatrix} \]
but in $\mathbb{Z}/2\mathbb{Z}$, that's the same as
\[ \begin{bmatrix}
    0 & 0 & 1 \\
    1 & 0 & 1 \\
    0 & 1 & 1
\end{bmatrix} \]

\section{}
\noindent\fbox{\fbox{\parbox{6.5in}{
            Let $V \subset \mathbb{R}[x,y]$ be the subspace of all polynomials of the form $ax+by+c$, where $a,b,c \in \mathbb{R}$. Let $\mathcal{A}$ be a linear operator in $V$ defined by
            \[ \mathcal{A}(ax+by+c)=a(x+1)+b(y-1)+c. \]
            Find the elementary divisors and the canonical form of $\mathcal{A}$.
}}}\bigskip\par

\section{}
\noindent\fbox{\fbox{\parbox{6.5in}{
    Find the Jordan canonical form over $\mathbb{C}$ of the matrix
    \[ \begin{pmatrix}
        2i & 1 \\
        1 & 0
    \end{pmatrix} \]
}}}\bigskip\par

\section{}
\noindent\fbox{\fbox{\parbox{6.5in}{
            Prove that two $2 \times 2$ matrices over a field that are not scalar matrices are similar if and only if they have the same characteristic polynomials.
}}}\bigskip\par

\section{}
\noindent\fbox{\fbox{\parbox{6.5in}{
    Prove that two $3 \times 3$ matrices are similar if and only if they have the same characteristic and the same minimal polynomials.
}}}\bigskip\par

\section{}
\noindent\fbox{\fbox{\parbox{6.5in}{
    Show that the minimal polynomial of an $n \times n$ matrix $A$ has the same irreducible divisors as the characteristic polynomial of $A$.
}}}\bigskip\par

\section{}
\noindent\fbox{\fbox{\parbox{6.5in}{
            Let $A$ be a nilpotent $n \times n$ matrix (that is, $A^N=0$ for some $N > 0$). Show that the invariant factors of $A$ are powers of $X$. Prove that $A^n=0$.
}}}\bigskip\par

\section{}
\noindent\fbox{\fbox{\parbox{6.5in}{
    Prove that any $n \times n$ matrix $A$ is similar to its transpose $A^t$.
}}}\bigskip\par


\end{document}
