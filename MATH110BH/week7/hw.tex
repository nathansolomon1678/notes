\documentclass[12pt]{article}
\usepackage[margin=1in]{geometry}
\usepackage{amsmath}
\usepackage{amssymb}
\usepackage{amsfonts}
\usepackage{amsthm}
\newtheorem{thm}{Theorem}[section]
\newtheorem{cor}[thm]{Corollary}
\newtheorem{lem}[thm]{Lemma}
\newtheorem{prop}[thm]{Proposition}
\usepackage{tikz-cd}
\renewcommand{\d}{\mathrm{d}}

\begin{document}

\title{Math 110BH Homework 7}
\author{Nathan Solomon}
\maketitle

\section{}
\noindent\fbox{\fbox{\parbox{6.5in}{
    Prove that the intersection of two principal ideals in a UFD is a principal ideal.
}}}\bigskip\par

\section{}
\noindent\fbox{\fbox{\parbox{6.5in}{
    Find an example of a non-free submodule $N \subset M$ of a free module $M$ over some domain $R$.
}}}\bigskip\par

\section{}
\noindent\fbox{\fbox{\parbox{6.5in}{
    Show that a submodule $N$ of a module $M$ generated by $n$ elements over a PID also can be generated by $n$ elements.
}}}\bigskip\par

\section{}
\noindent\fbox{\fbox{\parbox{6.5in}{
    Prove that the group $\mathbb{Z}^n$ cannot be generated by $n-1$ elements.
}}}\bigskip\par

\section{}
\noindent\fbox{\fbox{\parbox{6.5in}{
    Find two non-free modules $M$ and $N$ over $\mathbb{Z}/6\mathbb{Z}$ such that $M \oplus N$ is free.
}}}\bigskip\par
Let $M=\mathbb{Z}/2\mathbb{Z}$ and let $N=\mathbb{Z}/3\mathbb{Z}$.

\section{}
\noindent\fbox{\fbox{\parbox{6.5in}{
            Let $R$ be a PID and let $M$ be a torsion finitely generated $R$-module with the invariant factors $d_1|d_2|\cdots|d_k$. Set
            \[ I = \left\{ \text{$a \in R$ such that $aM=0$} \right\}. \]
            Prove that $I = d_kR$.
}}}\bigskip\par

\section{}
\noindent\fbox{\fbox{\parbox{6.5in}{
    Classify all abelian groups of order 300.
}}}\bigskip\par
Any abelian group of order 300 is isomorphic to one of the four following groups:
\begin{align*}
    (\mathbb{Z}/2\mathbb{Z}) \oplus (\mathbb{Z}/2\mathbb{Z}) &\oplus (\mathbb{Z}/3\mathbb{Z}) \oplus (\mathbb{Z}/5\mathbb{Z}) \oplus (\mathbb{Z}/5\mathbb{Z}) \\
    (\mathbb{Z}/2^2\mathbb{Z}) &\oplus (\mathbb{Z}/3\mathbb{Z}) \oplus (\mathbb{Z}/5\mathbb{Z}) \oplus (\mathbb{Z}/5\mathbb{Z}) \\
    (\mathbb{Z}/2\mathbb{Z}) \oplus (\mathbb{Z}/2\mathbb{Z}) &\oplus (\mathbb{Z}/3\mathbb{Z}) \oplus (\mathbb{Z}/5^2\mathbb{Z}) \\
    (\mathbb{Z}/2^2\mathbb{Z}) &\oplus (\mathbb{Z}/3\mathbb{Z}) \oplus (\mathbb{Z}/5^5\mathbb{Z}).
\end{align*}

\section{}
\noindent\fbox{\fbox{\parbox{6.5in}{
            Find the rank of the subgroup in $\mathbb{Z}^3$ generated by $(2,-2,0)$, $(0,4,-4)$, and $(5,0,-5)$.
}}}\bigskip\par

\section{}
\noindent\fbox{\fbox{\parbox{6.5in}{
            Determine the invariant factors of the factor group $\mathbb{Z}^3/N$, where $N$ is generated by $(3,-3,3)$, $(0,6,12)$, and $(9,0,-9)$.
}}}\bigskip\par
The group $N$ is generated by the columns of the following matrix:
\[ \begin{bmatrix}
    3 & 0 & 9 \\
    -3 & 6 & 0 \\
    3 & 12 & -9
\end{bmatrix}. \]
\par
Applying elementary transformations doesn't change 

\section{}
\noindent\fbox{\fbox{\parbox{6.5in}{
            Let $M$ be a finitely generated torsion module over a PID $R$. Prove that $M$ is cyclic if and only if every two elementary divisors of $M$ are relatively prime.
}}}\bigskip\par


\end{document}
