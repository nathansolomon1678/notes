\documentclass[12pt]{article}
\usepackage[margin=1in]{geometry}
\usepackage{amsmath}
\usepackage{amssymb}
\usepackage{amsfonts}
\usepackage{amsthm}
\newtheorem{thm}{Theorem}[section]
\newtheorem{cor}[thm]{Corollary}
\newtheorem{lem}[thm]{Lemma}
\usepackage{tikz-cd}
\renewcommand{\d}{\mathrm{d}}

\begin{document}

\title{Math 110BH homework 5}
\author{Nathan Solomon}
\maketitle

\section{}
\noindent\fbox{\fbox{\parbox{6.5in}{
            Show that over any field there exist infinitely many non-associate irreducible polynomials.
}}}\bigskip\par
This is pretty much the same method we use to prove there are infinitely many prime numbers.
\par
Let $\mathbb{F}$ be a field and suppose there is a finite set of all irreducible elements in $\mathbb{F}[x]$, excluding elements which are associate to an element in that set. Call that set $p= \left\{ p_1, p_2, \dots, p_n \right\}$. Note that $p$ is nonempty, because it contains the irreducible polynomial $p_1=x$.
\par
Let $p_{n+1}=1+\prod_{i=1}^n p_i$. Then $p_{n+1}$ is irreducible, since it is not divisible by any of the irreducible elements in $p$ (and so $p_{n+1}$ is also not associate to any of the other elements of $p$).
\par
This is a contradiction, so there must be infinitely many non-associate irreducible elements in $\mathbb{F}[x]$.

\section{}
\noindent\fbox{\fbox{\parbox{6.5in}{
            Prove that the factor ring $\mathbb{Z}[i]/(1+i)\mathbb{Z}[i]$ is a field of two elements.
}}}\bigskip\par
Let $\mathbb{F}_2$ be the field whose only elements are 0 and 1, and let $f: \mathbb{Z}[i] \rightarrow \mathbb{F}_2$ be the function defined by
\[ f(a+bi) = \begin{cases}
    0 & \text{if $a$ and $b$ have the same parity (both even or both odd)} \\
    1 & \text{if $a$ and $b$ have different parity (one even and one odd)}
\end{cases} \]
for any integers $a, b$. Alternatively, we could define a parity function $p: \mathbb{Z} \rightarrow \mathbb{F}_2$ by $p(x)= \frac{1-(-1)^x)}{2}$, so then $f$ can be defined by $f(a+bi)=p(a)+p(b)$.
\par
For any Gauss integers $a+bi$ and $c+di$,
\begin{itemize}
    \item $f(1)=1$
    \item $f((a+bi)+(c+di))=p(a)+p(c)+p(b)+p(d)=f(a+bi)+f(c+di)$.
    \item $f((a+bi) \cdot (c+di)) = f(ac-bd+(ad+bc)i)=p(ac)+p(bd)+p(ad)+p(bc)=(p(a+b))(p(c+d))=f(a+bi)f(c+di)$.
\end{itemize}
Therefore $f$ is a ring homomorphism, and $f$ is clearly surjective.
\par
For any element $a+bi \in \mathbb{Z}[i]$ for which $f(a+bi)=0$, $\frac{(a+bi)(1-i)}{2} = \frac{a-b-ai+bi}{2}$ is a Gauss integer, since $a-b$ and $b-a$ are even. Also, for any $(1+i)(a+bi) \in (1+i) \mathbb{Z}[i]$, $f((1+i)(a+bi))= f(a-b+ai-bi)=p(a-b)+p(a-b)=0$, so the kernel of $f$ is $(a+i)\mathbb{Z}[i]$.
\par
By the first isomorphism theorem (for rings),
\[ \mathbb{Z}[i]/(1+i)\mathbb{Z}[i] \cong \mathbb{F}_2. \]

\section{}
\noindent\fbox{\fbox{\parbox{6.5in}{
            Let $f,g \in \mathbb{Q}[X]$ with $fg \in \mathbb{Z}[X]$. Prove that there is $a \in \mathbb{Q}^\times$ such that $af \in \mathbb{Z}[X]$ and $a^{-1}g \in \mathbb{Z}[X]$.
}}}\bigskip\par
See the proof we did in class of Gauss' lemma.

\section{}
\noindent\fbox{\fbox{\parbox{6.5in}{
            Let $F$ be a field. Prove that the set $R$ of all polynomials in $F[X]$ whose $X$-coefficient is equal to 0 is a subring of $F[X]$ and that $R$ is not a UFD. (Hint: Use $X^6=(X^2)^3=(X^3)^2$.)
}}}\bigskip\par
The identity in $R$ is the constant monic polynomial, which is the same as the identity in $F[x]$, and for any polynomials $a,b \in R$, $a+b$ and $ab$ and $-a$ are also polynomials whose $X$-coefficient is 0. Therefore $R$ is a subring of $F[x]$.
\par
Next, we want to show that $R$ is not a UFD, by showing that there are two distinct ways to write $X^6$ as a product of irreducible elements:
\[ X^2 \cdot X^2 \cdot X^2 = X^6 = X^3 \cdot X^3. \]
In $F[x]$, if $X^2$ is written as a product of $a$ and $b$, then either $a$ and $b$ both have degree 1, or one of them has degree 0. Similarly, if $ab=X^3$, then either one of them has degree 1 (and the other has degree two) or one of them has degree 0 (and the other has degree 3). That means if $a,b \in R$ and $ab$ is either $X^2$ or $X^3$, then either $a$ or $b$ is a (nonzero) constant polynomial, which is a unit in $R$.
\par
Since $X^2$ and $X^3$ are both irreducible, we have found distinct ways to write $X^6$, which is an element of $R$, as a product of irreducibles. Therefore $R$ is not a UFD.

\section{}
\noindent\fbox{\fbox{\parbox{6.5in}{
            Find all irreducible polynomials of degree $\leq 4$ in $(\mathbb{Z}/2\mathbb{Z})[X]$.
}}}\bigskip\par
There are no irreducible polynomials of degree 0, and the only irreducible polynomials of degree 1 in $\mathbb{Z}/2\mathbb{Z}$ are $x$ and $x+1$. In degree 2 or 3, a polynomial is irreducible if and only if it is not divisible by any degree 1 polynomial -- the only such polynomials are $x^2+x+1$, $x^3+x^2+1$, and $x^3+x+1$. A polynomial of degree 4 is irreducible if and only if it is is not divisible by any degree 1 or 2 polynomial. There are 16 degree 4 polynomials we need to consider, but we can ignore the ones whose constant term is zero, because those are divisible by $x$. Going through the remaining 8 cases individually, we see that the the only degree 4 polynomials (in $\mathbb{Z}/2\mathbb{Z}$) are $x^4+x+1$, $x^4+x^2+1$, $x^4+x^3+1$, and $x^4+x^3+x^2+x+1$.

\section{}
\noindent\fbox{\fbox{\parbox{6.5in}{
            Let $f \in \mathbb{Z}[X], a,b \in \mathbb{Z}, a\neq b$. Prove that $a-b$ divides $f(a)-f(b)$. (Hint: $a-b$ divides $a^n-b^n$.)
}}}\bigskip\par
\begin{lem}\label{divides}
    $a-b$ divides $a^n-b^n$.
\end{lem}
\begin{proof}
    \[ a^n-b^n = (a-b) \left( a^{n-1}+ a^{n-2}b + \cdots + b^{n-1} \right) \]
\end{proof}
Let $g$ be the function which is the same as $f$ but without the highest order term. Then $f(a)-f(b)$ is equal to $g(a)-g(b)$ plus some multiple of $a^n-b^n$, so $a-b$ divides $f(a)-f(b)$ if and only if $a-b$ divides $g(a)-g(b)$. If $f$ has degree zero, then it is clearly divisible by $a-b$, so by induction on the degree of $f$, $a-b$ must always divide $f(a)-f(b)$.

\section{}
\noindent\fbox{\fbox{\parbox{6.5in}{
            Prove that $X^n+Y^n-1$ is irreducible in $\mathbb{Z}[X,Y]$ for every $n> 0$. (Hint: Use Eisenstein's Criterion.)
}}}\bigskip\par

\section{}
\noindent\fbox{\fbox{\parbox{6.5in}{
            Let $f$ be a monic polynomial in $\mathbb{Z}[X]$. Prove that if $a \in \mathbb{Q}$ is a root of $f$ then $a \in \mathbb{Z}$.
}}}\bigskip\par
Suppose $a$ is a root of $f$ which is rational but not an integer.
\par
Then let $b,c \in \mathbb{Z}$ be nonzero coprime integers such that $\frac{b}{c}=a$ and $c$ is not a unit. Also let $n$ be the degree of $f$, and let $g=f-X^n$.
\par
Since $g$ is a degree $n-1$ polynomial with integer coefficients, $g(a)$ is the sum of terms which can all be written as fractions with denominator $c^{n-1}$, so
\[ g(a) = \frac{\text{some integer}}{c^{n-1}}. \]
Because $a$ is a root of $f$, $f(a)=\frac{b^n}{c^n}+g(a)$ has to be zero, which implies $b^n$ is equal to some integer times $-c$. However, $b^n$ and $c^n$ are coprime, so $b^n$ cannot be divisible by $c$. Since we have reached a contradiction, every root of a monic polynomial in $\mathbb{Z}[X]$ must either be an integer or be irrational.

\section{}
\noindent\fbox{\fbox{\parbox{6.5in}{
            Find all roots of $f = X^p-X$ in $(\mathbb{Z}/p\mathbb{Z})[X]$ ($p$ prime) and factor $f$ into a product of irreducible polynomials. (Hint: Use Fermat's Little Theorem.)
}}}\bigskip\par
By Fermat's Little Theorem, if $a$ is an integer and $p$ is a prime integer, then $a^p-a \equiv 0 \pmod{p}$, so every $a \in \frac{Z}{p\mathbb{Z}}$ is a root of $X^p-p$. That means $X^p-X$ must be divisible by $X-a$ for every $a \in \frac{Z}{p\mathbb{Z}}$, so
\[ X^p-X=X(X-1)(X-2)\cdots (X-(p-1)). \]

\section{}
\noindent\fbox{\fbox{\parbox{6.5in}{
            Determine whether $X^4+4$ is irreducible in $\mathbb{Z}[X]$.
}}}\bigskip\par
This is reducible because
\[ (X^2+2X+2) \cdot (X^2-2X+2) = X^4+4 \]
and $X^2 \pm 2X + 2$ is not a unit in $\mathbb{Z}[X]$.

\end{document}
