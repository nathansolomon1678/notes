\documentclass{article}
\usepackage[margin=1in]{geometry}
\usepackage[linesnumbered,ruled,vlined]{algorithm2e}
\usepackage{amsfonts}
\usepackage{amsmath}
\usepackage{amssymb}
\usepackage{amsthm}
\usepackage{chemformula}
\usepackage{enumitem}
\usepackage{fancyhdr}
\usepackage{graphicx}
\usepackage{hyperref}
\usepackage{listings}
\usepackage{minted}
\usepackage{multicol}
\usepackage{pdfpages}
\usepackage{siunitx}
\usepackage{standalone}
\usepackage{svg}
\usepackage[many]{tcolorbox}
\usepackage{tikz-cd}
\usepackage{transparent}
\usepackage{xcolor}
% \tcbuselibrary{minted}

\author{Nathan Solomon}

\newcommand{\fig}[1]{
    \begin{center}
        \includegraphics[width=\textwidth]{#1}
    \end{center}
}

% Math commands
\renewcommand{\d}{\mathrm{d}}
\DeclareMathOperator{\id}{id}
\DeclareMathOperator{\im}{im}
\DeclareMathOperator{\proj}{proj}
\DeclareMathOperator{\Span}{span}
\DeclareMathOperator{\Tr}{Tr}
\DeclareMathOperator{\tr}{tr}
\DeclareMathOperator{\ad}{ad}
\DeclareMathOperator{\ord}{ord}
%%%%%%%%%%%%%%% \DeclareMathOperator{\sgn}{sgn}
\DeclareMathOperator{\Aut}{Aut}
\DeclareMathOperator{\Inn}{Inn}
\DeclareMathOperator{\Out}{Out}
\DeclareMathOperator{\stab}{stab}

\newcommand{\N}{\ensuremath{\mathbb{N}}}
\newcommand{\Z}{\ensuremath{\mathbb{Z}}}
\newcommand{\Q}{\ensuremath{\mathbb{Q}}}
\newcommand{\R}{\ensuremath{\mathbb{R}}}
\newcommand{\C}{\ensuremath{\mathbb{C}}}
\renewcommand{\H}{\ensuremath{\mathbb{H}}}
\newcommand{\F}{\ensuremath{\mathbb{F}}}

\newcommand{\E}{\ensuremath{\mathbb{E}}}
\renewcommand{\P}{\ensuremath{\mathbb{P}}}

\newcommand{\es}{\ensuremath{\varnothing}}
\newcommand{\inv}{\ensuremath{^{-1}}}
\newcommand{\eps}{\ensuremath{\varepsilon}}
\newcommand{\del}{\ensuremath{\partial}}
\renewcommand{\a}{\ensuremath{\alpha}}

\newcommand{\abs}[1]{\ensuremath{\left\lvert #1 \right\rvert}}
\newcommand{\norm}[1]{\ensuremath{\left\lVert #1\right\rVert}}
\newcommand{\mean}[1]{\ensuremath{\left\langle #1 \right\rangle}}
\newcommand{\floor}[1]{\ensuremath{\left\lfloor #1 \right\rfloor}}
\newcommand{\ceil}[1]{\ensuremath{\left\lceil #1 \right\rceil}}
\newcommand{\bra}[1]{\ensuremath{\left\langle #1 \right\rvert}}
\newcommand{\ket}[1]{\ensuremath{\left\lvert #1 \right\rangle}}
\newcommand{\braket}[2]{\ensuremath{\left.\left\langle #1\right\vert #2 \right\rangle}}

\newcommand{\catname}[1]{{\normalfont\textbf{#1}}}

\newcommand{\up}{\ensuremath{\uparrow}}
\newcommand{\down}{\ensuremath{\downarrow}}

% Custom environments
\newtheorem{thm}{Theorem}[section]

\definecolor{probBackgroundColor}{RGB}{250,240,240}
\definecolor{probAccentColor}{RGB}{140,40,0}
\newenvironment{prob}{
    \stepcounter{thm}
    \begin{tcolorbox}[
        boxrule=1pt,
        sharp corners,
        colback=probBackgroundColor,
        colframe=probAccentColor,
        borderline west={4pt}{0pt}{probAccentColor},
        breakable
    ]
    \color{probAccentColor}\textbf{Problem \thethm.} \color{black}
} {
    \end{tcolorbox}
}

\definecolor{exampleBackgroundColor}{RGB}{212,232,246}
\newenvironment{example}{
    \stepcounter{thm}
    \begin{tcolorbox}[
      boxrule=1pt,
      sharp corners,
      colback=exampleBackgroundColor,
      breakable
    ]
    \textbf{Example \thethm.}
} {
    \end{tcolorbox}
}

\definecolor{propBackgroundColor}{RGB}{255,245,220}
\definecolor{propAccentColor}{RGB}{150,100,0}
\newenvironment{prop}{
    \stepcounter{thm}
    \begin{tcolorbox}[
        boxrule=1pt,
        sharp corners,
        colback=propBackgroundColor,
        colframe=propAccentColor,
        breakable
    ]
    \color{propAccentColor}\textbf{Proposition \thethm. }\color{black}
} {
    \end{tcolorbox}
}

\definecolor{thmBackgroundColor}{RGB}{235,225,245}
\definecolor{thmAccentColor}{RGB}{50,0,100}
\renewenvironment{thm}{
    \stepcounter{thm}
    \begin{tcolorbox}[
        boxrule=1pt,
        sharp corners,
        colback=thmBackgroundColor,
        colframe=thmAccentColor,
        breakable
    ]
    \color{thmAccentColor}\textbf{Theorem \thethm. }\color{black}
} {
    \end{tcolorbox}
}

\definecolor{corBackgroundColor}{RGB}{240,250,250}
\definecolor{corAccentColor}{RGB}{50,100,100}
\newenvironment{cor}{
    \stepcounter{thm}
    \begin{tcolorbox}[
        enhanced,
        boxrule=0pt,
        frame hidden,
        sharp corners,
        colback=corBackgroundColor,
        borderline west={4pt}{0pt}{corAccentColor},
        breakable
    ]
    \color{corAccentColor}\textbf{Corollary \thethm. }\color{black}
} {
    \end{tcolorbox}
}

\definecolor{lemBackgroundColor}{RGB}{255,245,235}
\definecolor{lemAccentColor}{RGB}{250,125,0}
\newenvironment{lem}{
    \stepcounter{thm}
    \begin{tcolorbox}[
        enhanced,
        boxrule=0pt,
        frame hidden,
        sharp corners,
        colback=lemBackgroundColor,
        borderline west={4pt}{0pt}{lemAccentColor},
        breakable
    ]
    \color{lemAccentColor}\textbf{Lemma \thethm. }\color{black}
} {
    \end{tcolorbox}
}

\definecolor{proofBackgroundColor}{RGB}{255,255,255}
\definecolor{proofAccentColor}{RGB}{80,80,80}
\renewenvironment{proof}{
    \begin{tcolorbox}[
        enhanced,
        boxrule=1pt,
        sharp corners,
        colback=proofBackgroundColor,
        colframe=proofAccentColor,
        borderline west={4pt}{0pt}{proofAccentColor},
        breakable
    ]
    \color{proofAccentColor}\emph{\textbf{Proof. }}\color{black}
} {
    \qed \end{tcolorbox}
}

\definecolor{noteBackgroundColor}{RGB}{240,250,240}
\definecolor{noteAccentColor}{RGB}{30,130,30}
\newenvironment{note}{
    \begin{tcolorbox}[
        enhanced,
        boxrule=0pt,
        frame hidden,
        sharp corners,
        colback=noteBackgroundColor,
        borderline west={4pt}{0pt}{noteAccentColor},
        breakable
    ]
    \color{noteAccentColor}\textbf{Note. }\color{black}
} {
    \end{tcolorbox}
}


\fancyhf{}
\setlength{\headheight}{24pt}

\date{\today}
\title{MATH 131B Homework \#7}

\begin{document}
\maketitle

\begin{prob}
    Exercise 4.7.1: prove theorem 4.7.2.
\end{prob}
\begin{enumerate}[label=(\alph*)]
    \item \begin{align*}
            \sin(x)^2+\cos(x)^2 &= \left( \frac{e^{ix}-e^{-ix}}{2i} \right)^2 + \left( \frac{e^{ix}+e^{-ix}}{2} \right)^2 \\
                                &= \left( \frac{e^{2ix}-2e^0+e^{-2ix}}{-4} \right) + \left( \frac{e^{2ix}+2e^0+e^{-2ix}}{4} \right) \\
                                &= \frac{-(e^{2ix}-2+e^{-2ix})+(e^{2ix}+2+e^{-2ix})}{4} \\
                                &= 1.
    \end{align*}
    Since $x \in \R$, $\sin(x)$ and $\cos(x)$ are also real numbers, so $\sin(x)^2$ and $\cos(x)^2$ are both positive. That means $\sin(x)^2$ and $\cos(x)^2$ are both less than or equal to one, so $\sin(x), \cos(x) \in [-1, 1]$.
\item \begin{align*}
        \sin'(x) &= \frac{d}{dx} \left( \frac{e^{ix}-e^{-ix}}{2i} \right) \\
                 &= \frac{ie^{ix}+ie^{-ix}}{2i} \\
                 &= \cos(x). \\
        \cos'(x) &= \frac{d}{dx} \left( \frac{e^{ix}+e^{-ix}}{2} \right) \\
                 &= \frac{ie^{ix}-ie^{-ix}}{2} \\
                 &= - \sin(x).
\end{align*}
\item \begin{align*}
        \sin(-x) &= \frac{e^{i(-x)}-e^{-i(-x)}}{2i} \\
                 &= \frac{e^{-ix}-e^{ix}}{2i} \\
                 &= - \sin(x). \\
        \cos(-x) &= \frac{e^{i(-x)}+e^{-i(-x)}}{2} \\
                 &= \cos(x).
\end{align*}
\item \begin{align*}
        \cos(x+y) &= \frac{e^{i(x+y)}+e^{-i(x+y)}}{2} \\
                  &= \frac{(e^{ix}+e^{-ix})(e^{iy}+e^{-iy})}{4} + \frac{(e^{ix}-e^{-ix})(e^{iy}-e^{-iy})}{4} \\
                  &= \cos(x)\cos(y) - \sin(x)\sin(y). \\
        \sin(x+y) &= \frac{e^{i(x+y)}-e^{-i(x+y)}}{2i} \\
                  &= \frac{(e^{ix}+e^{-ix})(e^{iy}-e^{-iy})}{4i} + \frac{(e^{ix}-e^{-ix})(e^{iy}+e^{-iy})}{4i} \\
                  &= \cos(x)\sin(y) + \sin(x)\cos(y).
\end{align*}
\item \begin{align*}
        \sin(0) &= \frac{e^{0i}-e^{-0i}}{2i} \\
                &= \frac{1-1}{2i} \\
                &= 0. \\
        \cos(0) &= \frac{e^{0i}+e^{-0i}}{2} \\
                &= \frac{1+1}{2} \\
                &= 1.
\end{align*}
\item \begin{align*}
        \cos(x) + i \sin(x) &= \frac{e^{ix}+e^{-ix}}{2} + i \cdot \frac{e^{ix}-e^{-ix}}{2i} \\
                            &= \frac{(e^{ix}+e^{-ix}) + (e^{ix}-e^{-ix})}{2} \\
                            &= e^{ix}. \\
        e^{-ix} &= \overline{e^{ix}} \\
                &= \overline{\cos(x)+i\sin(x)} \\
                &= \cos(x) - i \sin(x). \\
        \cos(x) &= \Re(e^{ix}). \\
        \sin(x) &= \Im(e^{ix}).
\end{align*}
\end{enumerate}


\bigskip
\begin{prob}
    Exercise 4.7.3: prove theorem 4.7.5.
\end{prob}
\begin{enumerate}[label=(\alph*)]
    \item From the problem above, we have
\[ \cos(x+y) = \cos(x)\cos(y) - \sin(x)\sin(y) \]
and
\[ \sin(x+y) = \cos(x)\sin(y) + \sin(x)\cos(y). \]
Plugging in $y=\pi$, this becomes
\begin{align*}
    \cos(x+\pi) &= \cos(x)(-1) - \sin(x)(0) &= -\cos(x) \\
    \sin(x+\pi) &= \cos(x)(0) + \sin(x)(-1) &= -\sin(x).
\end{align*}
Applying that first formula twice will give you $\cos(x+2\pi)=\cos(x)$, and applying that second formula twice will give you $\sin(x + 2\pi) = \sin(x)$.
\item I am going to use the lemma that if $n \in \Z$, then $\sin(n \pi) = 0$.
\begin{proof}
    \textbf{Base case:} If $n=0$, then $\sin(n\pi)=\sin(0)=0$.
    \par
    \textbf{Inductive step $P(n) \Rightarrow P(n+1)$:} If $\sin(n\pi)=0$, then $\sin((n+1)\pi) = \sin(n\pi + \pi) = -\sin(n \pi) = -0=0$.
    \bigskip
    \par
    So by induction, this statement is true for any nonnegative integer $n$. If $n$ is a negative integer, then $\sin(n \pi) = \sin(-(-n)\pi) = -\sin((-n)\pi)=-0=0$, so this statement is actually true for any $n \in \Z$.
\end{proof}
Let $x$ be any real number. Then we can write $x/\pi=n+a$, for some $n \in \Z, a \in [0, 1)$. If $a=0$, then $x=n\pi$, so $\sin(x)=\sin(n\pi)=0$. If $a \neq 0$, then $\sin(x)=\sin(n\pi+a\pi)=\pm \sin(a\pi)$ (also by induction), which is nonzero because $a \in (0, \pi)$. Therefore $\sin(x)=0$ iff $x/\pi$ is an integer.
\item Using those same angle-addition identities with $x=y=\pi/2$, we get
    \begin{align*}
        -1 = \cos(\pi) &= \cos(\pi/2)^2 - \sin(\pi/2)^2 \\
        0 = \sin(\pi) &= 2 \cos(\pi/2) \sin(\pi/2).
    \end{align*}
    We also know that for any real number $\sin(x)^2$ and $\cos(x)^2$ are both in $[0, 1]$, so that first equation can only be true if $\cos(\pi/2)=0$ and $\sin(\pi/2)=\pm 1$. But we already proved that $\sin(x)$ is positive when $x \in (0, \pi)$, so $\sin(\pi/2)=1$. Now we can use the angle-addition identity again with $y=\pi/2$ and any $x \in \R$:
\[ \sin(x + \pi/2) = \cos(x)\sin(\pi/2) - \sin(x)\cos(\pi/2) = \cos(x). \]
So the cosine of $x$ is zero iff $\sin(x+\pi/2)=0$ which we just showed occurs iff $(x+\pi/2)/\pi$ is an integer. Therefore $\cos(x)=0$ iff $x/\pi$ is an integer plus $1/2$.
\end{enumerate}

\bigskip
\begin{prob}
    Exercise 4.7.5
\end{prob}
If $re^{i \theta}=se^{i \alpha}$, then $\norm{re^{i \theta}}=\norm{se^{i \alpha}}$, which simplifies to $r = s$.
\par
Now that we know $r=s$, we can divide both sides by $r$ to get $e^{i \theta}=e^{i \alpha}$, which is equivalent to $e^{i(\theta - \alpha)}=1$. That becomes
\[ \cos(\theta - \alpha) + i \sin(\theta - \alpha) = 1, \]
which is true iff $\sin(\theta - \alpha)=0$ and $\cos(\theta - \alpha)= 1$. We showed that $\sin(x)=0$ iff $x$ is an integer multiple of $\pi$, so $\theta - \alpha$ is an integer multiple of $\pi$. However, if $\theta - \alpha$ is an odd multiple of $\pi$, then $\cos(\theta-\alpha)=\cos(2 \pi n + \pi)$ for some $n \in \Z$, and $\cos(2 \pi n + \pi) = \cos(\pi) = -1 \neq 1$. Therefore $\theta - \alpha$ can only be an even multiple of $2 \pi$. That is, $\theta = \alpha + 2 \pi k $ for some $k \in \Z$.

\bigskip
\begin{prob}
    Exercise 4.7.6
\end{prob}
First, I will find one value of $(r, \theta)$ which works, then I will prove it is unique.
\par
Let $r = \norm{z}$, so then $\Re(z/r)^2+\Im(z/r)^2=1$. By the result from exercise 4.7.4, there is a unique $\theta \in (-\pi, \pi]$ such that $\sin(\theta)=\Im(z/r)$ and $\cos(\theta)=\Re(z/r)$. Then
\[ z = r \left( \cos(\theta) + i \sin(\theta) \right) = re^{i \theta}. \]
If there was some other $(r', \theta')$ such that $z = r' e^{i \theta'}$, then $\norm{z} = \norm{r'} \cdot \norm{e^{i \theta'}} = \norm{r'} = r'$, so $r=r'$. Also, by the result from exercise 4.7.5, $\theta$ and $\theta'$ must differ by an integer multiple of $2 \pi$, but since they are both in $(-\pi, \pi]$, $\theta=\theta'$. Therefore $(r, \theta)$ is the only pair that works.

\includepdf[pages=-]{assignment.pdf}

\end{document}
