\documentclass{article}
\usepackage[margin=1in]{geometry}
\usepackage[linesnumbered,ruled,vlined]{algorithm2e}
\usepackage{amsfonts}
\usepackage{amsmath}
\usepackage{amssymb}
\usepackage{amsthm}
\usepackage{chemformula}
\usepackage{enumitem}
\usepackage{fancyhdr}
\usepackage{graphicx}
\usepackage{hyperref}
\usepackage{listings}
\usepackage{minted}
\usepackage{multicol}
\usepackage{pdfpages}
\usepackage{siunitx}
\usepackage{standalone}
\usepackage{svg}
\usepackage[many]{tcolorbox}
\usepackage{tikz-cd}
\usepackage{transparent}
\usepackage{xcolor}
% \tcbuselibrary{minted}

\author{Nathan Solomon}

\newcommand{\fig}[1]{
    \begin{center}
        \includegraphics[width=\textwidth]{#1}
    \end{center}
}

% Math commands
\renewcommand{\d}{\mathrm{d}}
\DeclareMathOperator{\id}{id}
\DeclareMathOperator{\im}{im}
\DeclareMathOperator{\proj}{proj}
\DeclareMathOperator{\Span}{span}
\DeclareMathOperator{\Tr}{Tr}
\DeclareMathOperator{\tr}{tr}
\DeclareMathOperator{\ad}{ad}
\DeclareMathOperator{\ord}{ord}
%%%%%%%%%%%%%%% \DeclareMathOperator{\sgn}{sgn}
\DeclareMathOperator{\Aut}{Aut}
\DeclareMathOperator{\Inn}{Inn}
\DeclareMathOperator{\Out}{Out}
\DeclareMathOperator{\stab}{stab}

\newcommand{\N}{\ensuremath{\mathbb{N}}}
\newcommand{\Z}{\ensuremath{\mathbb{Z}}}
\newcommand{\Q}{\ensuremath{\mathbb{Q}}}
\newcommand{\R}{\ensuremath{\mathbb{R}}}
\newcommand{\C}{\ensuremath{\mathbb{C}}}
\renewcommand{\H}{\ensuremath{\mathbb{H}}}
\newcommand{\F}{\ensuremath{\mathbb{F}}}

\newcommand{\E}{\ensuremath{\mathbb{E}}}
\renewcommand{\P}{\ensuremath{\mathbb{P}}}

\newcommand{\es}{\ensuremath{\varnothing}}
\newcommand{\inv}{\ensuremath{^{-1}}}
\newcommand{\eps}{\ensuremath{\varepsilon}}
\newcommand{\del}{\ensuremath{\partial}}
\renewcommand{\a}{\ensuremath{\alpha}}

\newcommand{\abs}[1]{\ensuremath{\left\lvert #1 \right\rvert}}
\newcommand{\norm}[1]{\ensuremath{\left\lVert #1\right\rVert}}
\newcommand{\mean}[1]{\ensuremath{\left\langle #1 \right\rangle}}
\newcommand{\floor}[1]{\ensuremath{\left\lfloor #1 \right\rfloor}}
\newcommand{\ceil}[1]{\ensuremath{\left\lceil #1 \right\rceil}}
\newcommand{\bra}[1]{\ensuremath{\left\langle #1 \right\rvert}}
\newcommand{\ket}[1]{\ensuremath{\left\lvert #1 \right\rangle}}
\newcommand{\braket}[2]{\ensuremath{\left.\left\langle #1\right\vert #2 \right\rangle}}

\newcommand{\catname}[1]{{\normalfont\textbf{#1}}}

\newcommand{\up}{\ensuremath{\uparrow}}
\newcommand{\down}{\ensuremath{\downarrow}}

% Custom environments
\newtheorem{thm}{Theorem}[section]

\definecolor{probBackgroundColor}{RGB}{250,240,240}
\definecolor{probAccentColor}{RGB}{140,40,0}
\newenvironment{prob}{
    \stepcounter{thm}
    \begin{tcolorbox}[
        boxrule=1pt,
        sharp corners,
        colback=probBackgroundColor,
        colframe=probAccentColor,
        borderline west={4pt}{0pt}{probAccentColor},
        breakable
    ]
    \color{probAccentColor}\textbf{Problem \thethm.} \color{black}
} {
    \end{tcolorbox}
}

\definecolor{exampleBackgroundColor}{RGB}{212,232,246}
\newenvironment{example}{
    \stepcounter{thm}
    \begin{tcolorbox}[
      boxrule=1pt,
      sharp corners,
      colback=exampleBackgroundColor,
      breakable
    ]
    \textbf{Example \thethm.}
} {
    \end{tcolorbox}
}

\definecolor{propBackgroundColor}{RGB}{255,245,220}
\definecolor{propAccentColor}{RGB}{150,100,0}
\newenvironment{prop}{
    \stepcounter{thm}
    \begin{tcolorbox}[
        boxrule=1pt,
        sharp corners,
        colback=propBackgroundColor,
        colframe=propAccentColor,
        breakable
    ]
    \color{propAccentColor}\textbf{Proposition \thethm. }\color{black}
} {
    \end{tcolorbox}
}

\definecolor{thmBackgroundColor}{RGB}{235,225,245}
\definecolor{thmAccentColor}{RGB}{50,0,100}
\renewenvironment{thm}{
    \stepcounter{thm}
    \begin{tcolorbox}[
        boxrule=1pt,
        sharp corners,
        colback=thmBackgroundColor,
        colframe=thmAccentColor,
        breakable
    ]
    \color{thmAccentColor}\textbf{Theorem \thethm. }\color{black}
} {
    \end{tcolorbox}
}

\definecolor{corBackgroundColor}{RGB}{240,250,250}
\definecolor{corAccentColor}{RGB}{50,100,100}
\newenvironment{cor}{
    \stepcounter{thm}
    \begin{tcolorbox}[
        enhanced,
        boxrule=0pt,
        frame hidden,
        sharp corners,
        colback=corBackgroundColor,
        borderline west={4pt}{0pt}{corAccentColor},
        breakable
    ]
    \color{corAccentColor}\textbf{Corollary \thethm. }\color{black}
} {
    \end{tcolorbox}
}

\definecolor{lemBackgroundColor}{RGB}{255,245,235}
\definecolor{lemAccentColor}{RGB}{250,125,0}
\newenvironment{lem}{
    \stepcounter{thm}
    \begin{tcolorbox}[
        enhanced,
        boxrule=0pt,
        frame hidden,
        sharp corners,
        colback=lemBackgroundColor,
        borderline west={4pt}{0pt}{lemAccentColor},
        breakable
    ]
    \color{lemAccentColor}\textbf{Lemma \thethm. }\color{black}
} {
    \end{tcolorbox}
}

\definecolor{proofBackgroundColor}{RGB}{255,255,255}
\definecolor{proofAccentColor}{RGB}{80,80,80}
\renewenvironment{proof}{
    \begin{tcolorbox}[
        enhanced,
        boxrule=1pt,
        sharp corners,
        colback=proofBackgroundColor,
        colframe=proofAccentColor,
        borderline west={4pt}{0pt}{proofAccentColor},
        breakable
    ]
    \color{proofAccentColor}\emph{\textbf{Proof. }}\color{black}
} {
    \qed \end{tcolorbox}
}

\definecolor{noteBackgroundColor}{RGB}{240,250,240}
\definecolor{noteAccentColor}{RGB}{30,130,30}
\newenvironment{note}{
    \begin{tcolorbox}[
        enhanced,
        boxrule=0pt,
        frame hidden,
        sharp corners,
        colback=noteBackgroundColor,
        borderline west={4pt}{0pt}{noteAccentColor},
        breakable
    ]
    \color{noteAccentColor}\textbf{Note. }\color{black}
} {
    \end{tcolorbox}
}


\fancyhf{}
\setlength{\headheight}{24pt}

\date{\today}
\title{MATH 131B practice final}

\begin{document}
\maketitle

\begin{prob}
    \begin{enumerate}[label=(\alph*)]
        \item Let $E$ be a subset of a compact metric space $(X, d_X)$. Show that if $E$ is open in $X$, then $E^c := X \backslash E$ is compact.
        \item Let $f: X \rightarrow Y$, where $(X,d_X)$ and $(Y, d_Y)$ are metric spaces. Suppose $f$ is bijective and continuous, and $(X,d_X)$ is compact. Show that $f$ is an open map (that is, for any $E \subset X$ open, $f(E) \subset Y$ is open).
    \end{enumerate}
\end{prob}
\begin{enumerate}[label=(\alph*)]
    \item Since $E$ is open, its complement is closed. Any closed subset of a compact space is compact, so $E^c$ is compact.
    \item Since $E$ is open, $E^c$ is compact. Because $f$ is continuous, that also means $f(E^c)$ is compact, and since $f$ is bijective, $f(E^c)=f(E)^c$. Every compact space is closed, so $f(E)^c$ is closed, which means $f(E)$ is open.
\end{enumerate}


\bigskip
\begin{prob}
    Consider $(\R^2, d_{\ell^2})$.
    \begin{enumerate}[label=(\alph*)]
        \item What is $B_{(\R^2, d_{\ell^2})}((0,0), 1)$ as a set?
        \item Denote $B = B_{(\R^2, d_{\ell^2})}((0,0),1)$. Let $f: \R^2 \rightarrow \R, f(x) = \inf \left\{ d(x,y) | y \in B \right\}$. Show that $f$ is continuous (with respect to the $\ell^2$ metric on $\R^2$ and the standard metric on $\R$).
    \end{enumerate}
\end{prob}
\begin{enumerate}[label=(\alph*)]
    \item The set of points distance less than one from the origin can be written as
        \[ B := \left\{ (x,y) \in \R^2 : x^2+y^2 < 1 \right\}. \]
    \item $f$ can be written as $g \circ h$, where $h((x,y)) = \norm{(x,y)} = \sqrt{x^2+y^2}$ and $g(x)=\max(0, x-1)$. Since $g$ and $h$ are both continuous, so is $f$.\par
        This is not a rigorous proof, a better way to do this would be to use the triangle inequality to show that for any $\varepsilon > 0, (x,y) \in \R^2$, there exists some $\delta > 0$ such that $f(B((x,y),\delta))\subset B(f((x,y)), \varepsilon)$. But I won't write that whole thing out because it's annoying to do.
\end{enumerate}


\bigskip
\begin{prob}
    Let $(X, d_X)$ be a metric space. Suppose for any collection of closed subsets $ \left\{ S_\alpha \right\}_{\alpha \in A}$ (i.e., for every $\alpha \in A$, $S_\alpha$ is a closed subset of $X$) such that $\cap_{\alpha \in F} S_\alpha \neq \emptyset$ for any finite subset $F$ of $A$, we have that $\cap_{\alpha \in A} S_\alpha \neq \emptyset$. Show that $X$ is compact.
\end{prob}
Suppose $X$ is not compact. Let $ \left\{ T_\alpha \right\}_{\alpha \in A}$ be any open cover of $X$. There is no finite subcover of that, so for any finite subset $F \subset A$, $ \cup_{\alpha \in F} T_\alpha \neq X$, which means $\cap_{\alpha \in F} T_\alpha^c \neq \emptyset$. Then $\cap_{\alpha \in A} T_\alpha^c \neq 0$, which means $ \cup_{\alpha \in A} T_\alpha \neq X$, so $ \left\{ T_\alpha \right\}_{\alpha \in A}$ does not cover $X$. This is a contradiction, so $X$ must be compact.

\bigskip
\begin{prob}
    Let $(f_n)_{n=1}^\infty$ be a sequence of Riemann-integrable functions from $[a,b]$ to $\R$, and $f: [a,b] \rightarrow \R$ another Riemann-integrable function. Suppose $(f_n)_{n=1}^\infty$ converges uniformly to $f$ on $[a,b]$.
    \begin{enumerate}[label=(\alph*)]
        \item Show that $\lim_{n \rightarrow \infty} \int_{[a,b]} f_n = \int_{[a,b]} f$.
        \item For every $n$, define $F_n : [a,b] \rightarrow \R, F_n(x) = \int_{[a,x]} f_n$ and $F: [a,b] \rightarrow \R, F(x) = \int_{[a,b]} f$. Show that $F_n$ converges uniformly to $F$ on $[a,b]$.
    \end{enumerate}
\end{prob}
\begin{enumerate}[label=(\alph*)]
    \item For any $\varepsilon > 0$, there exists an $N$ such that $\abs{f(x)-f_n(x)}< \varepsilon/(b-a)$ for any $n\geq N, x \in [a,b]$. Then \begin{align*}
            d \left( \int_{[a,b]} f_n, \int_{[a,b]} f \right) &\leq \abs{\int_{[a,b]} (f_n-f)(x) \d x} \\
                                                              &= \int_{[a,b]} \abs{f_n(x)-f(x)} \d x \\
                                                              &\leq \int_{[a,b]} \frac{\varepsilon}{b-a} \d x \\
                                                              &= \varepsilon.
    \end{align*}
    Therefore $\int_{[a,b]} f_n$ converges to $\int_{[a,b]} f$ as $n$ goes to $\infty$.
\item Define $N$ to be the same as in part (a). For any $x \in [a,b]$, \begin{align*}
        d(F_N(x),F(x)) &\leq \abs{\int_{[a,x]} \left( f_n(x)-f(x) \right) \d x} \\
                       &\leq \int_{[a,x]} \abs{f_n(x)-f(x)} \d x \\
                       &\leq \int_{[a,x]} \frac{\varepsilon}{b-a} \d x \\
                       &= \varepsilon \cdot \frac{x-a}{b-a} \\
                       &\leq \varepsilon.
\end{align*}
\end{enumerate}


\bigskip
\begin{prob}
    \begin{enumerate}[label=(\alph*)]
        \item Let $f: E \rightarrow \R$ where $E \subset \R$ and $a \in \operatorname{int}(E)$ (interior of $E$). State the definition of $f$ being real analytic at $x=a$.
        \item Now let $f(x) = \operatorname{arccot}(x)$ (inverse cotangent, where $\cot(x) := \cos(x)/\sin(x)$). Show that $f'(x)$ is real analytic at $x=0$ using the geometric series formula
            \[ \sum_{k=0}^\infty cr^k = \frac{c}{1-r}. \]
            Find the radius of convergence $R$ for the power series expansion for $f'(x)$ at $x=0$.
        \item Show that $f(x)$ is real analytic at $x=0$ (you can use $f(x) = \pi/2 + \int_{[0,x]} f'$ for all $x \in \R$).
    \end{enumerate}
\end{prob}
\begin{enumerate}[label=(\alph*)]
    \item $f$ is real analytic at $x=a$ iff there exists some $\delta > 0$ such that $f$ is equal to its power series expansion in $(a-\delta,a+\delta)$.
    \item First, we want to find the derivative of $f$ using implicit differentiation. Let $y=f(x)$.
        \begin{align*}
            x &= \cot(y) \\
            \frac{\partial x}{\partial y} &= \frac{1}{\sin^2(y)} = 1 + \cot^2(y) = 1+x^2 \\
            f'(x) &= \left( \frac{\partial x}{\partial y} \right)^{-1} = \frac{1}{1+x^2} = \sum_{k=0}^\infty (-x^2)^k.
        \end{align*}
        Therefore the radius of convergence for $f'$ is
        \[ R := \frac{1}{\limsup_{n \rightarrow \infty} \sqrt[n]{\abs{c_n}}}, \]
        where $c_n = 0, 0, -1, 0, 1, 0, -1, 0, 1, \dots$, so the radius of convergence is 1.
    \item The function
        \[ x \mapsto \int_{[0,x]} f'(y) \d y \]
        is real analytic with the same radius of convergence, so $f(x)$ is also real analytic at $x=0$.
\end{enumerate}


\bigskip
\begin{prob}
    \begin{enumerate}[label=(\alph*)]
        \item State the Fourier theorem for $f \in C(\R/\Z,\C)$.
        \item Let $f$ be a function in $C(\R/\Z,\C)$. For $n \in \Z_{\geq 0}$, let \begin{align*}
                a_n &= 2 \int_0^1 f(x) \cos(2\pi nx) \d x, \\
                b_n &= 2 \int_0^1 f(x) \sin(2\pi nx) \d x.
        \end{align*}
        Use the Fourier series of $f$ to show that the series
    \[ \frac{a_0}{2} + \sum_{n=1}^\infty \left[ a_n \cos(2\pi nx) + b_n \sin(2\pi nx) \right] \]
        converges to $f$ in the $L^2$ metric.
    \end{enumerate}
\end{prob}
\begin{enumerate}[label=(\alph*)]
    \item For any $f \in C(\R/\Z,\C)$, $\sum_{n=-\infty}^\infty \hat{f}(n)e_n$ converges to $f$ in the $L^2$ metric.
    \item \begin{align*}
            f &= \sum_{n=-\infty}^\infty \hat{f}(n)e_n \\
              &= \sum_{n=0}^\infty \left( \hat{f}(n)e_n+\hat{f}(-n)e_{-n} \right) \\
              &= \sum_{n=0}^\infty \left( \left( \int_{[0,1]} f(x)e_{-n}(x) \d x \right) e_n+ \left( \int_{[0,1]} f(x) e_n(x) \d x \right) e_{-n} \right) \\
              &= \sum_{n=0}^\infty \left( \frac{a_n-ib_n}{2}\cdot e_n + \frac{a_n+ib_n}{2}\cdot e_{-n} \right) \\
              &= \sum_{n=0}^\infty \left( a_n \cdot \frac{e_n+e_{-n}}{2} + b_n \cdot \frac{e_n-e_{-n}}{2i} \right) \\
            f(x) &= \sum_{n=0}^\infty \left( a_n \cos(2\pi nx) + b_n \sin(2\pi nx) \right) \\
            &= \frac{a_0}{2} + \sum_{n=1}^\infty \left( a_n \cos(2\pi nx) + b_n \sin(2\pi nx) \right).
    \end{align*}
\end{enumerate}

\bigskip
\begin{prob}
    \begin{enumerate}[label=(\alph*)]
        \item Let $(f_n)_n \subset C(\R/\Z, \C)$, and $f,g \in C(\R/\Z,\C)$ and $f_n \rightarrow f$ uniformly on $\R$. Show that $f_n * g \rightarrow f*g$ pointwise and uniformly.
        \item Let $(f_n)_n \subset C(\R/\Z, \C)$, and $f \in C(\R/\Z,\C)$. Suppose for each $n$, $f_n$ is a periodic $ \left( \frac{1}{n}, \frac{1}{2n} \right) $ approximation to the identity. Show that $f_n * f \rightarrow f$ uniformly.
    \end{enumerate}
\end{prob}
\begin{enumerate}[label=(\alph*)]
    \item $g$ is bounded by $\norm{g}_\infty$, and for any $\varepsilon > 0$, there exists $N$ such that for any $x \in \R$ and any $n \geq N$, $\abs{f_n(x)-f(x)} < \frac{\varepsilon}{\norm{g}_\infty}$. Then \begin{align*}
            \norm{f_n*g-f*g} &= \abs{\int_{[0,1]} f_n(y)-f(y))g(x-y) \d y} \\
                             &\leq \int_{[0,1]} \abs{f_n(y)-f(y)} g(x-y) \d y \\
                             &\leq \frac{\varepsilon}{\norm{g}_\infty} \int_{[0,1]} g(x-y) \d y \\
                             &\leq \varepsilon.
    \end{align*}
    Therefore $f_n*g$ converges uniformly (and therefore, also pointwise) to $f*g$.
    \item \begin{align*}
            (f_n*f)(x) &= \int_{[-1/2,1/2]} f_n(y)f(x-y) \d y \\
                       &= \left( \int_{[-1/2,-1/2n]\cup[1/2n,1/2]} f_n(y)f(x-y) \d y \right) + \left( \int_{[-1/2n,1/2n]} f_n(y)f(x-y) \d y \right).
    \end{align*}
    Now, use the facts that $0 \leq f_n(y) < \frac{1}{n}$ and that $f$ is bounded by $\norm{f}_\infty$ (the magnitude of $f$ in the supremum norm), so for any $\varepsilon > 0$, there exists $N_1$ such that $f_n(y) < \frac{\varepsilon}{3 \norm{f}_\infty}$ whenever $n \geq N_1$. Therefore that first term can be bounded as follows:
    \begin{align*}
        \int_{[-1/2,-1/2n]\cup[1/2n,1/2]} f_n(y)f(x-y) \d y &\leq \int_{[-1/2,1/2]} f_n(y) f(x-y) \d y \\
                                                            &\leq \int_{[-1/2,1/2]} \left( \frac{\varepsilon}{3 \norm{f}_\infty} \right) \left( \norm{f}_\infty \right) \d y \\
                                                            &\leq \frac{\varepsilon}{3}.
    \end{align*}
    \par
    Also, $f$ is uniformly continuous, so there exists $N_2 \in \N$ such that $\abs{f(x-y)-f(x)} < \varepsilon / 3$ for any $y \in (- \frac{1}{2n}, \frac{1}{2n})$. Therefore \begin{align*}
        \int_{[-1/2n,1/2n]} f_n(y)\abs{f(x-y)-f(x)} \d y &\leq \frac{\varepsilon}{3} \int_{[-1/2n,1/2n]} f_n(y) \d y \\
                                                         &\leq \frac{\varepsilon}{3}.
    \end{align*}
    Lastly, there exists $N_3 \in \N$ such that whenever $n \geq N_3$,
    \[ \abs{1 - \int_{[-1/2n,1/2n]} f_n(y) \d y} < \frac{\varepsilon}{3 \norm{f}_\infty}. \]
    This implies \begin{align*}
        \abs{f(x)-\int_{[-1/2n,1/2n]} f_n(y)f(x-y) \d y} &\leq \abs{f(x) \left( 1- \int_{[-1/2n,1/2n]} f_n(y) \d y \right) } + \abs{\int_{[-1/2n,1/2n]} f_n(y)(f(x-y)-f(x)) \d y} \\
                                                         &\leq \norm{f}_\infty \cdot \frac{\varepsilon}{3 \norm{f}_\infty} + \int_{[-1/2n,1/2n]} f_n(y)\abs{f(x-y)-f(x)} \d y \\
                                                         &\leq \frac{\varepsilon}{3} + \frac{\varepsilon}{3}.
    \end{align*}
    If we let
    \[ N := \max (N_1, N_2, N_3), \]
    then by the triangle inequality, \begin{align*}
        \abs{f(x)-(f_n*f)(x)} &\leq \varepsilon,
    \end{align*}
    so $f_n*f$ converges to $f$, and since the choice of $N$ did not depend on $x$, that convergence is uniform.
\end{enumerate}


\bigskip
\begin{prob}
    Let $f \in C(\R/\Z,\C)$ and $g$ be a trigonometric polynomial. Show that $\widehat{f * g}(n) = \widehat{f}(n)\widehat{g}(n)$ for all $n \in \Z$.
\end{prob}
\begin{align*}
    \widehat{f*g}(n) &= \mean{f*g,e_n} \\
                     &= \int_{[0,1]} (f*g)(x) e_n^*(x) \d x \\
                     &= \int_{[0,1]} \left( \int_{[0,1]} f(y)g(x-y) \d y \right) e_n^*(x) \d x \\
                     &= \int_{[0,1]} \left( \int_{[0,1]} f(y)\sum_{k=-\infty}^\infty \hat{g}(k) e_k(x-y) \d y \right) e_n^*(x) \d x \\
                     &= \int_{[0,1]} \sum_{k=-\infty}^\infty \hat{g}(k)e_k(x) \left( \int_{[0,1]} f(y) e_k^*(y) \d y \right) e_n^*(x) \d x \\
                     &= \int_{[0,1]} \sum_{k=-\infty}^\infty \hat{g}(k)e_k(x) \hat{f}(k) e_n^*(x) \d x \\
                     &= \sum_{k=-\infty}^\infty \hat{g}(k)e_k(x) \hat{f}(k) \delta_{k,n} \\
                     &=\hat{f}(n) \hat{g}(n).
\end{align*}

\bigskip
\begin{prob}
    
    Let $f \in C(\R/\Z, \C)$. For each $N \in \N$, let $F_N = \sum_{n=-N}^N \hat{f}(n) e_n$, and $S_N$ be the collection of trigonometric polynomial $p = \sum_{n=-N}^N c_n e_n$ where $\sum_{n=-N}^N \abs{c_n}^2 \leq 1$ (recall $e_n(x) := e^{2\pi i n x}$ is the character with frequency $n$). For each $N$, show that $\abs{\mean{f,p}} \leq \norm{F_N}$ for all $p \in S_N$. Find an element $p \in S_N$ such that equality holds.
\end{prob}
\begin{align*}
    \mean{f,p} &= \int_{[0,1]} f(x) p^*(x) \d x \\
               &= \int_{[0,1]} \left( f(x) \sum_{n=-N}^N c_n^* e_n^*(x) \right) \d x \\
               &= \sum_{n=-N}^N \mean{f, c_n e_n} \\
               &= \sum_{n=-N}^N c_n^* \hat{f}(n) \\
               &= \sum_{n=-N}^N \hat{f}(n)\mean{e_n,c_ne_n} \\
               &= \mean{F_N,p}.
\end{align*}
Now that we know $\mean{f,p}=\mean{F_N,p}$, we can use the Cauchy-Shwarz inequality and the fact that $\norm{p} \leq 1$:
\begin{align*}
    \abs{\mean{f,p}} &= \abs{\mean{F_N,p}} \\
                     &\leq \norm{F_N} \cdot \norm{p} \\
                     &\leq \norm{F_N}.
\end{align*}
Now we want to find some $p$ such that $\abs{\mean{f,p}}=\norm{F_N}$. If $F_N=0$, then $p$ can be anything, otherwise, let
\[ p= \frac{F_N}{\norm{F_N}}. \]

\end{document}
