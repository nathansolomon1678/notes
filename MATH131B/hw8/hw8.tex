\documentclass{article}
\usepackage[margin=1in]{geometry}
\usepackage[linesnumbered,ruled,vlined]{algorithm2e}
\usepackage{amsfonts}
\usepackage{amsmath}
\usepackage{amssymb}
\usepackage{amsthm}
\usepackage{chemformula}
\usepackage{enumitem}
\usepackage{fancyhdr}
\usepackage{graphicx}
\usepackage{hyperref}
\usepackage{listings}
\usepackage{minted}
\usepackage{multicol}
\usepackage{pdfpages}
\usepackage{siunitx}
\usepackage{standalone}
\usepackage{svg}
\usepackage[many]{tcolorbox}
\usepackage{tikz-cd}
\usepackage{transparent}
\usepackage{xcolor}
% \tcbuselibrary{minted}

\author{Nathan Solomon}

\newcommand{\fig}[1]{
    \begin{center}
        \includegraphics[width=\textwidth]{#1}
    \end{center}
}

% Math commands
\renewcommand{\d}{\mathrm{d}}
\DeclareMathOperator{\id}{id}
\DeclareMathOperator{\im}{im}
\DeclareMathOperator{\proj}{proj}
\DeclareMathOperator{\Span}{span}
\DeclareMathOperator{\Tr}{Tr}
\DeclareMathOperator{\tr}{tr}
\DeclareMathOperator{\ad}{ad}
\DeclareMathOperator{\ord}{ord}
%%%%%%%%%%%%%%% \DeclareMathOperator{\sgn}{sgn}
\DeclareMathOperator{\Aut}{Aut}
\DeclareMathOperator{\Inn}{Inn}
\DeclareMathOperator{\Out}{Out}
\DeclareMathOperator{\stab}{stab}

\newcommand{\N}{\ensuremath{\mathbb{N}}}
\newcommand{\Z}{\ensuremath{\mathbb{Z}}}
\newcommand{\Q}{\ensuremath{\mathbb{Q}}}
\newcommand{\R}{\ensuremath{\mathbb{R}}}
\newcommand{\C}{\ensuremath{\mathbb{C}}}
\renewcommand{\H}{\ensuremath{\mathbb{H}}}
\newcommand{\F}{\ensuremath{\mathbb{F}}}

\newcommand{\E}{\ensuremath{\mathbb{E}}}
\renewcommand{\P}{\ensuremath{\mathbb{P}}}

\newcommand{\es}{\ensuremath{\varnothing}}
\newcommand{\inv}{\ensuremath{^{-1}}}
\newcommand{\eps}{\ensuremath{\varepsilon}}
\newcommand{\del}{\ensuremath{\partial}}
\renewcommand{\a}{\ensuremath{\alpha}}

\newcommand{\abs}[1]{\ensuremath{\left\lvert #1 \right\rvert}}
\newcommand{\norm}[1]{\ensuremath{\left\lVert #1\right\rVert}}
\newcommand{\mean}[1]{\ensuremath{\left\langle #1 \right\rangle}}
\newcommand{\floor}[1]{\ensuremath{\left\lfloor #1 \right\rfloor}}
\newcommand{\ceil}[1]{\ensuremath{\left\lceil #1 \right\rceil}}
\newcommand{\bra}[1]{\ensuremath{\left\langle #1 \right\rvert}}
\newcommand{\ket}[1]{\ensuremath{\left\lvert #1 \right\rangle}}
\newcommand{\braket}[2]{\ensuremath{\left.\left\langle #1\right\vert #2 \right\rangle}}

\newcommand{\catname}[1]{{\normalfont\textbf{#1}}}

\newcommand{\up}{\ensuremath{\uparrow}}
\newcommand{\down}{\ensuremath{\downarrow}}

% Custom environments
\newtheorem{thm}{Theorem}[section]

\definecolor{probBackgroundColor}{RGB}{250,240,240}
\definecolor{probAccentColor}{RGB}{140,40,0}
\newenvironment{prob}{
    \stepcounter{thm}
    \begin{tcolorbox}[
        boxrule=1pt,
        sharp corners,
        colback=probBackgroundColor,
        colframe=probAccentColor,
        borderline west={4pt}{0pt}{probAccentColor},
        breakable
    ]
    \color{probAccentColor}\textbf{Problem \thethm.} \color{black}
} {
    \end{tcolorbox}
}

\definecolor{exampleBackgroundColor}{RGB}{212,232,246}
\newenvironment{example}{
    \stepcounter{thm}
    \begin{tcolorbox}[
      boxrule=1pt,
      sharp corners,
      colback=exampleBackgroundColor,
      breakable
    ]
    \textbf{Example \thethm.}
} {
    \end{tcolorbox}
}

\definecolor{propBackgroundColor}{RGB}{255,245,220}
\definecolor{propAccentColor}{RGB}{150,100,0}
\newenvironment{prop}{
    \stepcounter{thm}
    \begin{tcolorbox}[
        boxrule=1pt,
        sharp corners,
        colback=propBackgroundColor,
        colframe=propAccentColor,
        breakable
    ]
    \color{propAccentColor}\textbf{Proposition \thethm. }\color{black}
} {
    \end{tcolorbox}
}

\definecolor{thmBackgroundColor}{RGB}{235,225,245}
\definecolor{thmAccentColor}{RGB}{50,0,100}
\renewenvironment{thm}{
    \stepcounter{thm}
    \begin{tcolorbox}[
        boxrule=1pt,
        sharp corners,
        colback=thmBackgroundColor,
        colframe=thmAccentColor,
        breakable
    ]
    \color{thmAccentColor}\textbf{Theorem \thethm. }\color{black}
} {
    \end{tcolorbox}
}

\definecolor{corBackgroundColor}{RGB}{240,250,250}
\definecolor{corAccentColor}{RGB}{50,100,100}
\newenvironment{cor}{
    \stepcounter{thm}
    \begin{tcolorbox}[
        enhanced,
        boxrule=0pt,
        frame hidden,
        sharp corners,
        colback=corBackgroundColor,
        borderline west={4pt}{0pt}{corAccentColor},
        breakable
    ]
    \color{corAccentColor}\textbf{Corollary \thethm. }\color{black}
} {
    \end{tcolorbox}
}

\definecolor{lemBackgroundColor}{RGB}{255,245,235}
\definecolor{lemAccentColor}{RGB}{250,125,0}
\newenvironment{lem}{
    \stepcounter{thm}
    \begin{tcolorbox}[
        enhanced,
        boxrule=0pt,
        frame hidden,
        sharp corners,
        colback=lemBackgroundColor,
        borderline west={4pt}{0pt}{lemAccentColor},
        breakable
    ]
    \color{lemAccentColor}\textbf{Lemma \thethm. }\color{black}
} {
    \end{tcolorbox}
}

\definecolor{proofBackgroundColor}{RGB}{255,255,255}
\definecolor{proofAccentColor}{RGB}{80,80,80}
\renewenvironment{proof}{
    \begin{tcolorbox}[
        enhanced,
        boxrule=1pt,
        sharp corners,
        colback=proofBackgroundColor,
        colframe=proofAccentColor,
        borderline west={4pt}{0pt}{proofAccentColor},
        breakable
    ]
    \color{proofAccentColor}\emph{\textbf{Proof. }}\color{black}
} {
    \qed \end{tcolorbox}
}

\definecolor{noteBackgroundColor}{RGB}{240,250,240}
\definecolor{noteAccentColor}{RGB}{30,130,30}
\newenvironment{note}{
    \begin{tcolorbox}[
        enhanced,
        boxrule=0pt,
        frame hidden,
        sharp corners,
        colback=noteBackgroundColor,
        borderline west={4pt}{0pt}{noteAccentColor},
        breakable
    ]
    \color{noteAccentColor}\textbf{Note. }\color{black}
} {
    \end{tcolorbox}
}


\fancyhf{}
\setlength{\headheight}{24pt}

\date{\today}
\title{MATH 131B Homework \#8}

\begin{document}
\maketitle

\begin{prob}
    Exercise 5.1.2: Prove lemma 5.1.5.
\end{prob}
\begin{enumerate}[label=(\alph*)]
    \item If $f \in C(\R/\Z, \C)$, define a function $g: [0, 1] \rightarrow \C$ such that $g(x) = f(x + \Z)$ for any $x \in \R$. Since $g$ is a continuous function on the compact interval $[0,1]$, it must be bounded, because continuous functions map compact sets to compact sets, and since $\C$ is isometric to $\R^2$, if the image of $g$ is compact, it is also bounded. Every element of the domain of $f$ can be written as $x + \Z$ for some $x \in [0, 1]$, so the image of $f$ is the same as the image of $g$, meaning $f$ is also bounded.
    \item Ignore the notation I used above -- for the rest of this homework, I will treat functions $f,g \in C(\R/\Z, \C)$ as continuous, $\Z$-periodic functions from $\R$ to $\C$, instead of treating them as continuous functions from $\R/\Z$ to $\C$.
        \par
        If $f,g \in C(\R/\Z, \C)$, then $f$ and $g$ are continuous, so $f+g$, $f-g$, and $fg$ are also continuous. Also, for any $x \in \R$, the following are all true:
        \begin{align*}
            (f+g)(x+1) &= f(x+1)+g(x+1)=f(x)+g(x)=(f+g)(x) \\
            (f-g)(x+1) &= f(x+1)-g(x+1)=f(x)-g(x)=(f-g)(x) \\
            (fg)(x+1) &= f(x+1)g(x+1)=f(x)g(x)=(fg)(x).
        \end{align*}
        Therefore $f+x,f-g,fg \in C(\R/\Z, \C)$.
        \par
        For any $f \in C(\R/\Z,\C), c \in \C$, $cf$ is continuous and for any $x \in \R$,
        \[ (cf)(x+1)=c \cdot f(x+1) = c \cdot f(x) = (cf)(x), \]
        so $cf \in C(\R/\Z, \C)$.
    \item We have already shown that if a sequence of continuous functions converges uniformly, that sequence converges to another continuous function, so I only need to show that the sequence converges to another $\Z$-periodic function.
        \par
        For any $\varepsilon > 0$, there is some $N \in \N$ such that whenever $n \geq N$, $d(f_n(x), f(x)) < \varepsilon / 2$ for every $x \in \R$. Since $f_n$ is $\Z$-periodic, $f_n(x+1)=f_n(x)$ for any $x \in \R$. By the triangle inequality, for any $x \in \R$,
        \begin{align*}
            d(f(x), f(x+1)) &\leq d(f(x), f_n(x)) + d(f_n(x+1), f(x+1)) \\
                            &\leq \frac{\varepsilon}{2} + \frac{\varepsilon}{2} = \varepsilon.
        \end{align*}
        $f$ is continuous, and the only way $d(f(x),f(x+1))$ can be less than $\varepsilon$ for any positive $\varepsilon$ is if $f(x)=f(x+1)$, so $f \in C(\R/\Z, \C)$.
\end{enumerate}


\bigskip
\begin{prob}
    Exercise 5.2.1: Prove lemma 5.2.5.
\end{prob}
\begin{enumerate}[label=(\alph*)]
    \item \begin{align*}
            \mean{g,f} &= \int_{[0,1]} f(x) \overline{g(x)} \d x \\
                       &= \overline{ \int_{[0,1]} g(x) \overline{f(x)} \d x } \\
                       &= \overline{\mean{f,g}}.
    \end{align*}
\item The function $g(x) := \norm{f(x)}^2$ is continuous and nonnegative, and
    \[ \mean{f,f} = \int_{[0,1]} g(x) \d x. \]
    We can partition the interval $[0,1]$ into the disjoint subset of Lebesgue-measurable sets $A, B, C$, where
    \begin{align*}
        A &:= \left\{ x \in [0,1] : g(x)<0 \right\} \\
        B &:= \left\{ x \in [0,1] : g(x)=0 \right\} \\
        C &:= \left\{ x \in [0,1] : g(x)>0 \right\}, \\
        \therefore \mean{f,f} &= \left( \int_A g(x) \d x \right) + \left( \int_B g(x) \d x \right) + \left( \int_C g(x) \d x \right).
    \end{align*}
    Clearly $A = \emptyset$ though, so I will ignore that first integral. The second integral is also obviously zero, so we only need to consider the integral over $C$.
    \par
    If $f \neq 0$, then $C$ is nonempty, so there is some $x_0 \in C$ such that $g(x_0)>0$, and since $g$ is continuous, that implies there is some interval contained in $C$ for which $g(x)\geq g(x_0)/2 > 0$. The integral of $g$ over that interval is positive, and the integral of $g$ over the remainder of $C$ is nonnegative, so $\mean{f,f}>0$.
    \par
    On the other hand, if $f=0$, then $\mean{f,f}=0$.
\item \begin{align*}
        \mean{f+g,h} &= \int_{[0,1]} (f+g)(x) \overline{h(x)} \d x \\
                     &= \left( \int_{[0,1]} f(x) \overline{h(x)} \d x \right) + \left( \int_{[0,1]} g(x) \overline{h(x)} \d x \right) \\
                     &= \mean{f,h} + \mean{g,h}. \\
        \mean{cf,g} &= \int_{[0,1]} (cf)(x) \overline{h(x)} \d x \\
                    &= c \cdot \int_{[0,1]} f(x) \overline{h(x)} \d x \\
                    &= c \mean{f,g}.
\end{align*}
\item \begin{align*}
        \mean{f,g+h} &= \int_{[0,1]} f(x) \overline{(g+h)(x)} \d x \\
                     &= \left( \int_{[0,1]} f(x) \overline{g(x)} \d x \right) + \left( \int_{[0,1]} f(x) \overline{h(x)} \d x \right) \\
                     &= \mean{f,g} + \mean{f,h}. \\
        \mean{f,cg} &= \overline{\mean{cg,f}} = \overline{c\mean{g,f}} = \overline{c} \cdot \overline{\mean{g,f}} = \overline{c} \mean{f,g}.
\end{align*}
\end{enumerate}


\bigskip
\begin{prob}
    Exercise 5.2.2
\end{prob}
\fbox{$d(f,f)=0$ for any $f \in C(\R/\Z, \C)$,} because
\[ d(f,f)=\sqrt{\int_{[0,1]} (f-f)(x) \d x} = \sqrt{\int_{[0,1]} 0 \d x} = 0. \]
\bigskip
\par
\fbox{$d(f,g)=d(g,f)$ for any $f,g \in C(\R/\Z, \C)$,} because
\begin{align*}
    d(f,g) &= \sqrt{\int_{[0,1]} \abs{f(x)-g(x)}^2 \d x} \\
           &= \sqrt{ \int_{[0,1]} \abs{g(x)-f(x)}^2 \d x} \\
           &= d(g,f).
\end{align*}
\bigskip
\par
\fbox{$d(f,g)>0$ whenever $f \neq g$,} because
\[ d(f,g) = \sqrt{\int_{[0,1]} \abs{f(x)-g(x)}^2 \d x}, \]
and the function $h(x)=\abs{f(x)-g(x)}^2$ is continuous, positive somewhere, and nonnegative everywhere. I showed in the previous problem that the integral of such a function must be positive, so $d(f,g)>0$.
\bigskip
\par
\fbox{$d(f,g)+d(g,h)\geq d(f,h)$ for any $f,g,h \in C(\R/\Z, \C)$,} because
\begin{align*}
    d(f,h)^2 &= \mean{f-h,f-h} \\
             &= \mean{(f-g)+(g-h),(f-g)+(g-h)} \\
             &= \mean{f-g,f-g}+\mean{g-h,g-h}+\mean{f-g,g-h}+\overline{\mean{f-g,g-h}} \\
             &= \norm{f-g}^2 + \norm{g-h}^2 + 2 \Re\left( \mean{f-g,g-h} \right) \\
             &\leq \norm{f-g}^2 + \norm{g-h}^2 + 2 \abs{ \mean{f-g,g-h} } \\
             &\leq \norm{f-g}^2 + \norm{g-h}^2 + 2 \cdot \norm{f-g} \cdot \norm{g-h} \\
             &= (\norm{f-g} + \norm{g-h})^2 \\
             &= \left( d(f,g)+d(g,h) \right)^2,
\end{align*}
which implies $d(f,h) \leq d(f,g)+d(g,h)$. Note that I used the Cauchy-Schwarz inequality, which says that
\[ \abs{\mean{u,v}}^2\leq \norm{u}\cdot \norm{v} \]
for any $u,v$ in some inner product space. In this case, $u=f-g$ and $v=g-h$.
\bigskip
\par
\noindent\fbox{\fbox{\parbox{6.5in}{
Here is a proof of the Cauchy-Schwarz inequality, which I copied from Wikipedia:
\par
If $v=0$, then we have $0 \leq \mean{u,u}+0$, which we already know is true, so we only need to consider the case where $v \neq 0$. Let
\[ z := u - \frac{\mean{u,v}}{\mean{v,v}} v. \]
Note that $v$ is orthogonal to $z$, so
\[ \norm{u}^2 = \abs{ \frac{\mean{u,v}}{\mean{v,v}}}^2 + \norm{z}^2 = \frac{\abs{\mean{u,v}}^2}{\norm{v}^2} + \norm{z}^2 \geq \frac{\abs{\mean{u,v}}^2}{\norm{v}^2}. \]
Therefore $\norm{v}^2 \cdot \norm{u}^2 \geq \abs{\mean{u,v}}^2$, so
\[ \abs{\mean{u,v}} \leq \norm{u} \cdot \norm{v}. \]
}}}\bigskip\par


\bigskip
\begin{prob}
    Exercise 5.2.6
\end{prob}
\begin{enumerate}[label=(\alph*)]
    \item If $f_n$ converges uniformly to $f$, then for every $\varepsilon > 0$, there exists some $N \in \N$ such that for any $n \geq N, x \in \R$, $\abs{f_n(x)-f(x)} < \sqrt{\varepsilon}$. The $L^2$ distance between $f_n$ and $f$ can be written as
        \[ d(f_n, f) = \int_{[0,1]} g(x) \d x, \]
        where $g(x) := \abs{f_n(x)-f(x)}^2$.
        \par
        For any $x \in \R$, $0 \leq g(x) < \varepsilon$, so
        \[ 0 \leq d(f_n,f) < \varepsilon. \]
        Therefore $f_n$ converges to $f$ in the $L^2$ metric.
    \item Let $f=0$, and let $f_n=\cos(\pi x)^n$. Then
        \begin{align*}
            d(f_n, f) &= \sqrt{\int_{[0,1]} \cos(\pi x)^{2n} \d x} \\
                      &= \sqrt{\int_{[0,1]} \left( \frac{e^{i\pi x}+e^{-i \pi x}}{2} \right)^{2n} \d x} \\
                      &= \sqrt{\int_{[0,1]} \left( \frac{1}{4^n} \sum_{k=0}^{2n} \binom{2n}{k} e^{2i\pi (k-n) x} \right) \d x} \\
                      &= \sqrt{\int_{[0,1]} \frac{1}{4^n} \binom{2n}{n}  \d x} \\
                      &= \sqrt{ \frac{(2n)!}{4^n(n!)^2} } \\
                      &= \frac{\sqrt{(2n)!}}{2^n n!} \\
                      &= \prod_{k=1}^n \frac{\sqrt{(2k-1)(2k)}}{2k} \\
                      &= \prod_{k=1}^n \sqrt{ \frac{2k-1}{2k}} \\
                      &= \exp\left( \frac{1}{2} \sum_{k=1}^n \left( \log(2k-1)-\log(2k) \right)  \right).
        \end{align*}
        This is useful because that summation goes to $-\infty$ (which you can prove by bounding the sum with an integral, evaluating the integral, and then taking the limit as $n \rightarrow \infty$), so as $n \rightarrow \infty$, $d(f_n, f) \rightarrow 0$.
        \par
        Therefore $f_n$ converges to $f$ in the $L^2$ metric, but it does ot converge uniformly, because $f_n(0)$ is always $\cos(0)^n=1$, and $f(0)$ is always zero.
    \item My example from part (b) works here too.
    \item Define $f_n$ by
        \[ f_n(x) = \begin{cases}
            n \sin(n \pi x) & 0 \leq x \leq \frac{1}{n} \\
            0 & \frac{1}{n} \leq x \leq 1 \\
            f(x-\floor{x}) & x \not\in [0,1]
        \end{cases} \]
        and let $f=0$. In the $L^2$ norm,
        \begin{align*}
        d(f_n, f) &= \sqrt{ \int_{[0,1]} \abs{f_n(x)-f(x)}^2 \d x } \\
                  &= \sqrt{ \int_{[0,1/n]} n^2 \sin(n \pi x)^2 \d x } \\
                  &= \sqrt{ \int_{u=0}^1 n \sin(\pi u)^2 \d u } \\
                  &= \sqrt{ \frac{n}{2}},
        \end{align*}
        so $f_n$ does not converge to $f$. But $f_n$ does converge pointwise to $f$, because for any $x \in [0,1]$, either $x=0$, in which case $f_n(x)$ is always zero, or $f_n(x)=0$ for any $n \geq 1/x$.
        
\end{enumerate}

\includepdf[pages=-]{assignment.pdf}

\end{document}
