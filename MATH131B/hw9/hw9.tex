\documentclass{article}
\usepackage[margin=1in]{geometry}
\usepackage[linesnumbered,ruled,vlined]{algorithm2e}
\usepackage{amsfonts}
\usepackage{amsmath}
\usepackage{amssymb}
\usepackage{amsthm}
\usepackage{chemformula}
\usepackage{enumitem}
\usepackage{fancyhdr}
\usepackage{graphicx}
\usepackage{hyperref}
\usepackage{listings}
\usepackage{minted}
\usepackage{multicol}
\usepackage{pdfpages}
\usepackage{siunitx}
\usepackage{standalone}
\usepackage{svg}
\usepackage[many]{tcolorbox}
\usepackage{tikz-cd}
\usepackage{transparent}
\usepackage{xcolor}
% \tcbuselibrary{minted}

\author{Nathan Solomon}

\newcommand{\fig}[1]{
    \begin{center}
        \includegraphics[width=\textwidth]{#1}
    \end{center}
}

% Math commands
\renewcommand{\d}{\mathrm{d}}
\DeclareMathOperator{\id}{id}
\DeclareMathOperator{\im}{im}
\DeclareMathOperator{\proj}{proj}
\DeclareMathOperator{\Span}{span}
\DeclareMathOperator{\Tr}{Tr}
\DeclareMathOperator{\tr}{tr}
\DeclareMathOperator{\ad}{ad}
\DeclareMathOperator{\ord}{ord}
%%%%%%%%%%%%%%% \DeclareMathOperator{\sgn}{sgn}
\DeclareMathOperator{\Aut}{Aut}
\DeclareMathOperator{\Inn}{Inn}
\DeclareMathOperator{\Out}{Out}
\DeclareMathOperator{\stab}{stab}

\newcommand{\N}{\ensuremath{\mathbb{N}}}
\newcommand{\Z}{\ensuremath{\mathbb{Z}}}
\newcommand{\Q}{\ensuremath{\mathbb{Q}}}
\newcommand{\R}{\ensuremath{\mathbb{R}}}
\newcommand{\C}{\ensuremath{\mathbb{C}}}
\renewcommand{\H}{\ensuremath{\mathbb{H}}}
\newcommand{\F}{\ensuremath{\mathbb{F}}}

\newcommand{\E}{\ensuremath{\mathbb{E}}}
\renewcommand{\P}{\ensuremath{\mathbb{P}}}

\newcommand{\es}{\ensuremath{\varnothing}}
\newcommand{\inv}{\ensuremath{^{-1}}}
\newcommand{\eps}{\ensuremath{\varepsilon}}
\newcommand{\del}{\ensuremath{\partial}}
\renewcommand{\a}{\ensuremath{\alpha}}

\newcommand{\abs}[1]{\ensuremath{\left\lvert #1 \right\rvert}}
\newcommand{\norm}[1]{\ensuremath{\left\lVert #1\right\rVert}}
\newcommand{\mean}[1]{\ensuremath{\left\langle #1 \right\rangle}}
\newcommand{\floor}[1]{\ensuremath{\left\lfloor #1 \right\rfloor}}
\newcommand{\ceil}[1]{\ensuremath{\left\lceil #1 \right\rceil}}
\newcommand{\bra}[1]{\ensuremath{\left\langle #1 \right\rvert}}
\newcommand{\ket}[1]{\ensuremath{\left\lvert #1 \right\rangle}}
\newcommand{\braket}[2]{\ensuremath{\left.\left\langle #1\right\vert #2 \right\rangle}}

\newcommand{\catname}[1]{{\normalfont\textbf{#1}}}

\newcommand{\up}{\ensuremath{\uparrow}}
\newcommand{\down}{\ensuremath{\downarrow}}

% Custom environments
\newtheorem{thm}{Theorem}[section]

\definecolor{probBackgroundColor}{RGB}{250,240,240}
\definecolor{probAccentColor}{RGB}{140,40,0}
\newenvironment{prob}{
    \stepcounter{thm}
    \begin{tcolorbox}[
        boxrule=1pt,
        sharp corners,
        colback=probBackgroundColor,
        colframe=probAccentColor,
        borderline west={4pt}{0pt}{probAccentColor},
        breakable
    ]
    \color{probAccentColor}\textbf{Problem \thethm.} \color{black}
} {
    \end{tcolorbox}
}

\definecolor{exampleBackgroundColor}{RGB}{212,232,246}
\newenvironment{example}{
    \stepcounter{thm}
    \begin{tcolorbox}[
      boxrule=1pt,
      sharp corners,
      colback=exampleBackgroundColor,
      breakable
    ]
    \textbf{Example \thethm.}
} {
    \end{tcolorbox}
}

\definecolor{propBackgroundColor}{RGB}{255,245,220}
\definecolor{propAccentColor}{RGB}{150,100,0}
\newenvironment{prop}{
    \stepcounter{thm}
    \begin{tcolorbox}[
        boxrule=1pt,
        sharp corners,
        colback=propBackgroundColor,
        colframe=propAccentColor,
        breakable
    ]
    \color{propAccentColor}\textbf{Proposition \thethm. }\color{black}
} {
    \end{tcolorbox}
}

\definecolor{thmBackgroundColor}{RGB}{235,225,245}
\definecolor{thmAccentColor}{RGB}{50,0,100}
\renewenvironment{thm}{
    \stepcounter{thm}
    \begin{tcolorbox}[
        boxrule=1pt,
        sharp corners,
        colback=thmBackgroundColor,
        colframe=thmAccentColor,
        breakable
    ]
    \color{thmAccentColor}\textbf{Theorem \thethm. }\color{black}
} {
    \end{tcolorbox}
}

\definecolor{corBackgroundColor}{RGB}{240,250,250}
\definecolor{corAccentColor}{RGB}{50,100,100}
\newenvironment{cor}{
    \stepcounter{thm}
    \begin{tcolorbox}[
        enhanced,
        boxrule=0pt,
        frame hidden,
        sharp corners,
        colback=corBackgroundColor,
        borderline west={4pt}{0pt}{corAccentColor},
        breakable
    ]
    \color{corAccentColor}\textbf{Corollary \thethm. }\color{black}
} {
    \end{tcolorbox}
}

\definecolor{lemBackgroundColor}{RGB}{255,245,235}
\definecolor{lemAccentColor}{RGB}{250,125,0}
\newenvironment{lem}{
    \stepcounter{thm}
    \begin{tcolorbox}[
        enhanced,
        boxrule=0pt,
        frame hidden,
        sharp corners,
        colback=lemBackgroundColor,
        borderline west={4pt}{0pt}{lemAccentColor},
        breakable
    ]
    \color{lemAccentColor}\textbf{Lemma \thethm. }\color{black}
} {
    \end{tcolorbox}
}

\definecolor{proofBackgroundColor}{RGB}{255,255,255}
\definecolor{proofAccentColor}{RGB}{80,80,80}
\renewenvironment{proof}{
    \begin{tcolorbox}[
        enhanced,
        boxrule=1pt,
        sharp corners,
        colback=proofBackgroundColor,
        colframe=proofAccentColor,
        borderline west={4pt}{0pt}{proofAccentColor},
        breakable
    ]
    \color{proofAccentColor}\emph{\textbf{Proof. }}\color{black}
} {
    \qed \end{tcolorbox}
}

\definecolor{noteBackgroundColor}{RGB}{240,250,240}
\definecolor{noteAccentColor}{RGB}{30,130,30}
\newenvironment{note}{
    \begin{tcolorbox}[
        enhanced,
        boxrule=0pt,
        frame hidden,
        sharp corners,
        colback=noteBackgroundColor,
        borderline west={4pt}{0pt}{noteAccentColor},
        breakable
    ]
    \color{noteAccentColor}\textbf{Note. }\color{black}
} {
    \end{tcolorbox}
}


\fancyhf{}
\setlength{\headheight}{24pt}

\date{\today}
\title{MATH 131B Homework \#9}

\begin{document}
\maketitle

\begin{prob}
    Exercise 5.3.3: prove corollary 5.3.6.
\end{prob}
If $-N \leq n \leq N$, then by the linearity (in the first argument) of the inner product,
\[ \mean{f,e_n} = \sum_{m=-N}^N c_m \mean{e_m,e_n}. \]
Lemma 5.3.5 tells us that
\[ \mean{e_m,e_n} = \delta_{m,n} := \begin{cases}
    1 & m=n \\
    0 & m \neq n
\end{cases}, \]
so all of the terms except the one where $m=n$ are zero, and we are left with
\[ \mean{f,e_n}=c_n. \]
If $n<-N$ or $n>N$, then
\[ \mean{f,e_n} = \sum_{m=-N}^N c_m \mean{e_m,e_n} = 0, \]
because $m$ will never be equal to $n$.
\par
Lastly, we have the identity
\begin{align*}
    \norm{f}^2 &= \mean{f,f} \\
               &= \mean{\sum_{n=-N}^N c_n e_n , \sum_{m=-N}^N c_m e_m} \\
               &= \sum_{n=-N}^N c_n \sum_{m=-N}^N \overline{c_m} \mean{e_n, e_m} \\
               &= \sum_{n=-N}^N c_n \sum_{m=-N}^N \overline{c_m} \delta_{n,m} \\
               &= \sum_{n=-N}^N \norm{c_n}^2. \\
\end{align*}

\bigskip
\begin{prob}
    Exercise 5.4.2: prove lemma 5.4.4.
\end{prob}
\begin{enumerate}[label=(\alph*)]
    \item Suppose $n \in \Z$ and $x \in \R$. Then \begin{align*}
            (f*g)(x+n) &= \int_{[0,1]} f(y)g(x+n-y)\d y \\
                       &= \int_{[0,1]} f(y)g(x-y)\d y \\
                       &= (f*g)(x),
    \end{align*}
    so $(f*g)$ is $\Z$-periodic.
    \par
    Next, I want to show that $f*g$ is continuous. Since $f$ is continuous on the closed interval $[0,1]$, $f$ is also bounded on $[0,1]$, so there exists some $M \in \R$ such that $f(x)<M$ for any $x$. Similarly, $g$ is continuous on $[0, 1]$, so $g$ is uniformly continuous on $[0, 1]$, and since $g$ is $\Z$-periodic, that means $g$ in uniformly continuous everywhere.
    \par
    Therefore, for any $\varepsilon > 0$, there exists a $\delta > 0$ such that for any $x, y \in \R$, $d(x,y)<\delta$ implies $\abs{g(x) - g(y)} < \varepsilon / M$. Now consider the difference between $(f*g)(x)$ and $(f*g)(x')$ for some $x' \in (x-\delta,x+\delta)$: \begin{align*}
        \abs{(f*g)(x')-(f*g)(x)} &= \abs{ \int_{[0,1]} \left( f(y)g(x'-y) - f(y)g(x-y) \right) \d y } \\
                                 &\leq M \abs{ \int_{[0,1]} \left( g(x'-y)-g(x-y) \right) \d y }.
    \end{align*}
    But $\abs{g(x'-y)-g(x-y)} < \varepsilon/M$ because $g$ is uniformly continuous everywhere, so the value of that integral is less than $\varepsilon/M$ everywhere, so that entire expression is less than $\varepsilon$, meaning $f*g$ is (uniformly) continuous.
\item The following steps substitute $u=x-y$, then use the facts that $f$ and $g$ are both $\Z$-periodic (so we can shift the interval we're integrating over by any integer amount) and that $x-1 < \floor{x} \leq x$ for any $x \in \R$.
    \begin{align*}
        (f*g)(x) &= \int_{y=0}^1 f(y) g(x-y) \d y \\
                 &= \int_{u=x}^{x-1} f(x-u)g(u) (-\d u) \\
                 &= \int_{[x-1,x]} g(u)f(x-u) \d u \\
                 &= \left( \int_{[x-1,\floor{x}]} g(u)f(x-u) \d u \right) + \left( \int_{[\floor{x},x]} g(u)f(x-u) \d u \right) \\
                 &= \left( \int_{[x-1,\floor{x}]} g(u)f(x-u) \d u \right) + \left( \int_{[\floor{x}-1,x-1]} g(u)f(x-u) \d u \right) \\
                 &= \int_{[\floor{x}-1,\floor{x}]} g(u)f(x-u) \d u \\
                 &= \int_{[0,1]} g(u)f(x-u) \d u \\
                 &= (g*f)(x).
    \end{align*}
\item For any $f, g, h \in C(\R/\Z, \C)$ and any $c \in \C$, \begin{align*}
        (f*(g+h))(x) &= \int_{[0,1]} f(y) (g+h)(x-y) \d y \\
                &= \int_{[0,1]} f(y) \left( g(x-y)+h(x-y) \right) \d y \\
                &= \left( \int_{[0,1]} f(y) g(x-y) \d y \right) \left( \int_{[0,1]} f(y) h(x-y) \d y \right) \\
                &= (f*g+f*h)(x). \\
        ((f+g)*h))(x) &= \int_{[0,1]} (f+g)(y) h(x-y) \d y \\
                      &= \int_{[0,1]} \left( f(y) +g(y) \right) h(x-y) \d y \\
                &= \left( \int_{[0,1]} f(y) h(x-y) \d y \right) \left( \int_{[0,1]} g(y) h(x-y) \d y \right) \\
                &= (f*h+g*h)(x). \\
        (cf)*g &= \int_{[0,1]} (cf)(y)g(x-y) \d y \\
               &= c \int{[0,1]} f(y)g(x-y) \d y \\
               &= c(f*g). \\
        f*(cg) &= \int_{[0,1]} f(y)(cg)(x-y) \d y \\
               &= c \int{[0,1]} f(y)g(x-y) \d y \\
               &= c(f*g).
\end{align*}
\end{enumerate}


\bigskip
\begin{prob}
    Exercise 5.5.2
\end{prob}
\begin{enumerate}[label=(\alph*)]
    \item First, I will compute the $a_n$ and $b_n$ described in exercise 5.5.1:
    \begin{align*}
            b_n &= 2 \int_{[0,1]} (1-2x)^2 \sin(2\pi n x) \d x \\
                &= \left( \int_{[0,1/2]} (1-2x)^2 \sin(2 \pi n x) \d x \right) + \left( \int_{x=1/2}^1 (1-2x)^2 \sin(2 \pi n x) \d x \right) \\
                &= \left( \int_{[0,1/2]} (1-2x)^2 \sin(2 \pi n x) \d x \right) + \left( \int_{u=1/2}^0 (1-2(1-u))^2 \sin(2 \pi n (1-u)) (-\d u) \right) \\
                &= \left( \int_{[0,1/2]} (1-2x)^2 \sin(2 \pi n x) \d x \right) + \left( \int_{[0, 1/2]} (2u-1)^2 \sin(2 \pi n (1-u)) \d u \right) \\
                &= \left( \int_{[0,1/2]} (1-2x)^2 \sin(2 \pi n x) \d x \right) + \left( \int_{[0, 1/2]} (1-2u)^2 \sin(2 \pi n u) \d u \right) \\
                &= 0.
    \end{align*}
    To get that, I substituted $u=1-x$, then used the fact that $x \mapsto \sin(2 \pi n x)$ is odd and $\Z$-periodic. Similarly, to find $a_n$, I will use the fact that $x \mapsto \cos(2 \pi n x)$ is even and $\Z$-periodic. If $n > 0$, then
    \begin{align*}
            a_n &= 2 \int_{[0,1]} (1-2x)^2 \cos(2\pi n x) \d x \\
                &= \int_{[0,1]} \left( 2-8x+8x^2 \right) \cos \left( 2 \pi n x \right) \d x \\
                &= \left( \int_{[0,1]} 2 \cos \left( 2 \pi n x \right) \d x \right) + \left( \int_{[0,1]} 8x \cos \left( 2 \pi n x \right) \d x \right) + \left( \int_{[0,1]} 8x^2 \cos \left( 2 \pi n x \right) \d x \right) \\
                &= \left( \left[ \frac{1}{\pi n} \sin \left( 2 \pi n x \right)  \right]_{x=0}^1 \right) + \left( \int_{[0,1]} 8x \cos \left( 2 \pi n x \right) \d x \right) + \left( \int_{[0,1]} 8x^2 \cos \left( 2 \pi n x \right) \d x \right) \\
                &= \left( \int_{[0,1]} 8x \cos \left( 2 \pi n x \right) \d x \right) + \left( \int_{[0,1]} 8x^2 \cos \left( 2 \pi n x \right) \d x \right) \\
                &= \left( \left[ (8x) \left( \frac{1}{2 \pi n} \sin(2\pi n x) \right) \right]_{x=0}^1 - \int_{[0,1]} (8) \left( \frac{1}{2 \pi n} \sin(2\pi n x) \right) \d x \right) + \left( \int_{[0,1]} 8x^2 \cos \left( 2 \pi n x \right) \d x \right) \\
                &= \left( 0 - \int_{[0,1]} \frac{4}{\pi n} \sin(2\pi n x) \d x \right) + \left( \int_{[0,1]} 8x^2 \cos \left( 2 \pi n x \right) \d x \right) \\
                &= 0 + \int_{[0,1]} 8x^2 \cos \left( 2 \pi n x \right) \d x \\
                &= \left[ (8x^2) \left( \frac{\sin(2\pi n x)}{2 \pi n} \right)  \right]_{x=0}^1 - \int_{[0,1]} (16x) \frac{\sin(2\pi n x)}{2 \pi n} \d x \\
                &= 0 - \int_{[0,1]} \frac{8x}{\pi n} \sin(2\pi n x) \d x \\
                &= -\left[ \left( \frac{8x}{\pi n} \right) \left( \frac{-\cos(2\pi n x)}{2 \pi n} \right) \right]_{x=0}^1 - \int_{[0,1]} \frac{8}{\pi n} \left( \frac{-\cos(2\pi n x)}{2 \pi n} \right) \d x \\
                &= \left[ \frac{4x\cos(2\pi n x)}{\pi^2 n^2} \right]_{x=0}^1 + \int_{[0,1]} \frac{4}{\pi^2 n^2} \cos(2\pi n x) \d x \\
                &= \frac{4}{\pi^2 n^2} + \left[ \left( \frac{4}{\pi^2 n^2} \right) \left( \frac{\sin(2\pi n x)}{2 \pi n} \right) \right]_{x=0}^1 \\
                &= \frac{4}{\pi^2 n^2}.
    \end{align*}
    And in the case where $n=0$, we have \begin{align*}
        b_0 &= 2 \int_{[0,1]} (1-2x)^2 \sin(0) \d x \\
            &= 0. \\
        a_0 &= 2 \int_{[0,1]} (1-2x)^2 \cos(0) \d x \\
            &= \int_{[0,1]} (2-8x+8x^2) \d x \\
            &= \left[ 2x- 4x^2 + \frac{8x^3}{3} \right]_{x=0}^1 \\
            &= 2-4+ \frac{8}{3} \\
            &= \frac{2}{3}.
    \end{align*}
    By the result from exercise 5.5.1, \begin{align*}
        f(x) &= \frac{a_0}{2} + \sum_{n=1}^\infty \left( a_n \cos(2\pi nx) + b_n \sin(2\pi nx) \right) \\
             &= \frac{1}{3} + \sum_{n=1}^\infty \frac{4}{\pi^2n^2} \cos(2\pi nx).
    \end{align*}
    \item At $x=0$, we have \begin{align*}
            1 &= (1-2(0)^2 \\
              &= f(0) \\
              &= \frac{1}{3} + \sum_{n=1}^\infty \frac{4}{\pi^2n^2} \cos(0) \\
            \frac{2}{3} &= \sum_{n=1}^\infty \frac{4}{\pi^2n^2} \\
            \frac{\pi^2}{6} &= \sum_{n=1}^\infty \frac{1}{n^2}.
    \end{align*}
\item \begin{align*}
        \norm{f}^2 &= \sum_{n=-\infty}^\infty \abs{\hat{f}(n)}^2 \\
        \mean{f,f} &= \left( \sum_{n=-\infty}^{-1} \abs{\hat{f}(n)}^2 \right) + \abs{\hat{f}(0)}^2 + \left( \sum_{n=1}^{\infty} \abs{\hat{f}(n)}^2 \right) \\
        \int_{[0,1]} (1-2x)^4 \d x &= \left( \sum_{n=-\infty}^{-1} \abs{\hat{f}(n)}^2 \right) + \frac{1}{9} + \left( \sum_{n=1}^{\infty} \abs{\hat{f}(n)}^2 \right) \\
        \left[ \frac{(1-2x)^5}{-10} \right]_{x=0}^1 &= \left( \sum_{n=-\infty}^{-1} \left( \frac{a_{-n}}{2} \right)^2 \right) + \frac{1}{9} + \left( \sum_{n=1}^{\infty} \left(\frac{a_n}{2}\right)^2 \right) \\
        \frac{1}{10} - \left( - \frac{1}{10} \right) &= \frac{1}{9} + 2 \sum_{n=1}^{\infty} \left( \frac{2}{\pi^2n^2} \right)^2 \\
        \frac{2}{10} - \frac{1}{9} &= \sum_{n=1}^{\infty} \frac{8}{\pi^4n^4} \\
        \frac{8}{90} &= \frac{8}{\pi^4} \sum_{n=1}^{\infty} \frac{1}{n^4} \\
        \sum_{n=1}^{\infty} \frac{1}{n^4} &= \frac{\pi^4}{90}.
\end{align*}
\end{enumerate}


\bigskip
\begin{prob}
    Exercise 5.5.4
\end{prob}
We are given that $f'$ is continuous, and it is $\Z$-periodic because for any $k \in \Z, x \in \R$,
\begin{align*}
    f'(x+k) &= \lim_{\varepsilon \rightarrow 0} \frac{f(x+k+\varepsilon)-f(x+k)}{\varepsilon} \\
            &= \lim_{\varepsilon \rightarrow 0} \frac{f(x+\varepsilon)-f(x)}{\varepsilon} \\
            &= f'(x).
\end{align*}
This means that $f' \in C(\R/\Z, \C)$.
\begin{align*}
    \hat{f'}(n) &= \mean{f', e_n} \\
                &= \int_{[0,1]} \left( \frac{\d}{\d x} f(x) \right) e^{-2i\pi n x} \d x \\
                &= \left[ f(x)e^{-2i \pi nx} \right]_{x=0}^1 - \int_{[0,1]} f(x) (-2i\pi n e^{-2i\pi n x}) \d x \\
                &= 0 + 2 i \pi n \int_{[0,1]} f(x) e^{-2i\pi n x} \d x \\
                &= 2i\pi n \hat{f}(n).
\end{align*}

\includepdf[pages=-]{assignment.pdf}

\end{document}
