\documentclass[12pt]{article}
\usepackage[margin=1in]{geometry}
\usepackage{amsmath}
\usepackage{amsfonts}
\usepackage{amsthm}
\newtheorem{thm}{Theorem}[section]
\newtheorem{cor}[thm]{Corollary}
\newtheorem{lem}[thm]{Lemma}
\usepackage{tikz-cd}
\renewcommand{\d}{\mathrm{d}}

\begin{document}

\title{Math 180 homework 1}
\author{Nathan Solomon}
\maketitle

\textbf{Due January 19th}

\section{}
\noindent\fbox{\fbox{\parbox{6.5in}{
            For a set $X$, let $\mathrm{id}_X: X \rightarrow X$ denote the function defined by $\mathrm{id}_X(x)=x$ for all $x \in X$ (the \textit{identity function}). Let $f: X \rightarrow Y$ be some function. Prove:
            \begin{itemize}
                \item (a) A function $g: Y \rightarrow X$ such that $g \circ f = \mathrm{id}_X$ exists if and only if $f$ is one-to-one.
                \item (b) A function $g: Y \rightarrow X$ such that $f \circ g = \mathrm{id}_Y$ exists if and only if $f$ is onto.
                \item (c) A function $g: Y \rightarrow X$ such that both $f \circ g = \mathrm{id}_Y$ and $g \circ f = \mathrm{id}_X$ exists if and only if $f$ is a bijection.
                \item (d) If $f: X \rightarrow Y$ is a bijection, then the following three conditions are equivalent for a function $g: Y \rightarrow X$:
                    \begin{itemize}
                        \item (i) $g = f^{-1}$
                        \item (ii) $g \circ f = \mathrm{id}_X$
                        \item (iii) $f \circ g = \mathrm{id}_Y$
                    \end{itemize}
            \end{itemize}
}}}\bigskip\par
\begin{itemize}
    \item (a) If $f$ is one-to-one, then the function $f: X \rightarrow \operatorname{Im}(f)$ is a bijection, so we can define $g: \operatorname{Im}(f) \rightarrow X$ to be it's inverse. Then $f \circ f = \mathrm{id}_X$.
        \par
        If $f$ is not one-to-one, there exist elements $a,b \in X$ such that $a \neq b$ but $f(a)=f(b)$. Then $g(f(a))=g(f(b))$, so $g \circ f$ is not one-to-one, and therefore is not $\mathrm{id}_X$.
    \item (b) If $f: X \rightarrow Y$ is surjective, then let $g: Y \rightarrow X$ be any function which takes $y \in Y$ to some element in the preimage $f^{-1}(y)$. Then $f \circ g = \mathrm{id}_Y$.
        \par
        If $f$ is not surjective, then there exists some $y \in Y$ which is not in the image of $f$, and therefore not in the image of $f \circ g$. But since the image of $\mathrm{id}_Y$ is $Y$, that means $f \circ g \neq \mathrm{id}_Y$.
    \item (c) This follows from parts (a) and (b). A function is a bijection if and only if it is both injective and surjective.
    \item (d) If statement (i) is true, then $f^{-1}(f(x))=x$ for any $x \in X$, so (ii) is true. Also, statement (i) is equivalent to saying $g^{-1}=f$, so by the same logic, (i) implies (iii).
        \par
        If (ii) is true, then $f \circ g \circ f = f \circ \mathrm{id}_X = f$, and since bijections are invertible, we can compose with $f^{-1}$ on the right to get $f \circ g = \mathrm{id}_Y$, o (iii). By the exact same logic, (iii) implies (ii), and by the definition of inverse functions, (ii) and (iii) together imply (i). Therefore each of the three statements implies the other two.
\end{itemize}

\section{}
\noindent\fbox{\fbox{\parbox{6.5in}{
            Prove that that following two statements about a function $f: X \rightarrow Y$ are equivalent ($X$ and $Y$ are some arbitrary sets):
            \begin{itemize}
                \item (i) $f$ is one-to-one
                \item (ii) For any set $Z$ and any two distinct functions $g_1: Z \rightarrow X$ and $g_2: Z \rightarrow X$, the composed functions $f \circ g_1$ and $f \circ g_2$ are also distinct.
            \end{itemize}
            (First, make sure you understand what it means that two functions are equal and what it means that they are distinct.)
}}}\bigskip\par
If (i) is true, let $Z$ be a set and let $g_1, g_2$ be distinct functions from $Z$ to $X$. Since $g_1 \neq g_2$, there exists at least one $z \in Z$ such that $g_1(z) \neq g_2(z)$. Then, because $f$ is injective, $f(g_1(z)) \neq f(g_2(z))$, so (ii) is true.
\par
If (ii) is true, then let $x_1, x_2$ be any distinct elements of $X$, let $Z = \{0\}$, and let $g_1, g_2$ be the functions such that $g_1(0)=x_1$ and $g_2(0)=x_2$. Since $f \circ g_1 \neq f \circ g_2$, there has to be at least one $z \in Z$ such that $f(g_1(z)) \neq f(g_2(z))$. But because $z$ has to be zero, that means $f(x_1) \neq f(x_2)$, so (i) is true.

\section{}
\noindent\fbox{\fbox{\parbox{6.5in}{
            Let $f: A \rightarrow B$ be a surjective function. Let use define a relation on $A$ by setting $a_1 \sim a_2$ if $f(a_1)=f(a_2)$.
            \begin{itemize}
                \item (a) Show that $\sim$ is an equivalence relation.
                \item (b) Show that there is a bijective correspondence between the set of equivalence classes in $A$ and the set $B$.
            \end{itemize}
}}}\bigskip\par
\begin{itemize}
    \item (a) Reflexivity: $a \sim a$ for any $a \in A$ because $f(a)=f(a)$. Symmetry: If $f(a_1)=f(a_2)$, then we also know $f(a_2)=f(a_1)$, so $a_1 \sim a_2$ implies $a_2 \sim a_1$. Transitivity: If $a_1 \sim a_2$ and $a_2 \sim a_3$, then $f(a_1) = f(a_2) = f(a_3)$, so $a_1 \sim a_3$.
    \item (b) Let $f'$ be the function which takes the equivalence class $[a]$ to $f(a)$ for any $a \in A$. This map is well-defined because $f(a_1)=f(a_2)$ for any $a_1, a_2 \in [a]$. It's surjective, because for any $b \in B$, there exists $a \in A$ such that $f(a)=b$, which implies $f'([a])=b$. And it's also injective, because if $f'([a_1])=f'([a_2])$, then $f(a_1)=f(a_2)$, which means $a_1$ and $a_2$ are in the same equivalence class.
\end{itemize}

\section{}
\noindent\fbox{\fbox{\parbox{6.5in}{
            Find the number of strictly increasing three-digit strings that use digits from $\{0,1,2,\dots,9\}$. In other words, count strings $abc$ in which $a<b<c$.
}}}\bigskip\par
Any such string can be formed uniquely by choosing 3 distinct digits from 0 to 9, then letting $a$ be the smallest one, $b$ be the middle one, and $c$ be the largest. There are
\[ \binom{10}{3}=120 \]
ways to do that.

\section{}
\noindent\fbox{\fbox{\parbox{6.5in}{
    Determine the number of ways to distribute 10 orange drinks, 1 lemon drink, and 1 lime drink to 4 thirsty students so that each student gets at least 1 drink, and the lemon and lime drinks go to different students.
}}}\bigskip\par
First, distribute the lemon and lime drinks. The lemon drink can go to of the 4 students, then the lime drink can go to any of the 3 remaining students, so there are 12 options for those.
\par
Next, distribute the 10 orange drinks. The students that did not get the lemon or lime drinks must receive at least one orange drink, so there are only 8 left to distribute (and no restrictions on whom those 8 drinks can go to).
\par
To figure out how many ways 8 drinks can be distributed among 4 students, consider a ``stars and bars" diagram, where you write a string starting with a number of stars equal to the number of drinks you give to the first person, then a bar, then a number of stars equal to the number of drinks given to the second person, and so on. For example, giving everyone 2 orange drinks would be denoted as
\[ \star \star | \star \star | \star \star | \star \star. \]
Each possible string corresponds to a unique way to distribute the eight drinks, and the number of ways to choose such a string (with 11 characters, where 3 are bars and 8 are stars) is
\[ \binom{11}{3} = \binom{11}{8} = \frac{11!}{3!8!} = 165. \]
This is independent of our choice of who to give the lemon and lime drinks to, so the total number of options is
\[ 12 \times 165 = 1980. \]

\section{}
\noindent\fbox{\fbox{\parbox{6.5in}{
    Use the binomial theorem to show that
    \[ 3^n=\sum_{k=0}^n \binom{n}{k} 2^k. \]
    Give a combinatorial proof of this identity.
}}}\bigskip\par

\[ 3^n= (1+2)^n=\sum_{k=0}^n \binom{n}{k} 1^k2^k = \sum_{k=0}^n \binom{n}{k} 2^k\]
Combinatorial proof: $3^n$ is the number of ways to put $n$ distinguishable items in 3 distinguishable barrels. That's equivalent to first choosing the number of items ($n-k$) to put into the first barrel, then choosing which $k$ items to not put in the first barrel, and out of those $k$ items, choosing any subset of them to go into the second barrel, then putting the remaining items in the third barrel.

\section{}
\noindent\fbox{\fbox{\parbox{6.5in}{
            Prove that for all natural numbers $n$, the equality
            \[ \sum_{k=1, k \text{ odd}}^n \binom{n}{k} 5^k = \frac{6^n-(-4)^n}{2} \]
            holds.
}}}\bigskip\par
\begin{align*}
    \frac{6^n-(-4)^n}{2} &= \frac{(1+5)^n - (1-5)^n}{2} \\
                         &= \left( \frac{1}{2} \sum_{k=0}^n \binom{n}{k} 5^k \right) - \left( \frac{1}{2} \sum_{k=0}^n \binom{n}{k} (-5)^k \right) \\
                         &= \sum_{k=0}^n \binom{n}{k} 5^k \cdot \frac{1-(-1)^k}{2} \\
                         &= \sum_{k=0}^n \binom{n}{k} 5^k \cdot \left(\text{1 if $k$ is odd, 0 if $k$ is even}\right) \\
                         &= \sum_{k=0,k \text{ odd}}^n \binom{n}{k} 5^k
\end{align*}

\section{}
\noindent\fbox{\fbox{\parbox{6.5in}{
            In how many ways can we choose an ordered pair $(A, B)$ of subsets of $[n]$ so that $A \cap B = \emptyset$?
}}}\bigskip\par
Let $C=[n]-(A \cup B)$. Then every integer from 1 to $n$ is in $A$ or $B$ or $C$, but cannot be in more than one. Therefore each natural number up to $n$ can be put in any one of those three sets, and those choices are all independent, so there are $3^n$ such ordered pairs.
\par
Equivalently, that is the number of $n$-letter words where each letter is `A', `B', or `C'.

\section{}
\noindent\fbox{\fbox{\parbox{6.5in}{
            How many $k$-element subsets of $\{1,2,\dots,n\}$ exist containing no two consecutive numbers? \textit{Hint:} Use the idea used to count compositions.
}}}\bigskip\par

Any subset of $[n]$ can be written as an $n$-bit string, with the $i$th bit being a 1 if $i$ is in the subset of $[n]$, and being 0 otherwise. The only additional restriction here is that every `1' in that $n$-bit string must be followed by a `0'. This restriction can be represented by adding a zero to the end of the string and then considering `10' to be a single letter, so the possible choices are now uniquely represented by strings with $n+1-k$ characters, or which $k$ characters are `10' and the remaining $n+1-2k$ characters are all `0'. Therefore there are
\[ \binom{n+1-k}{k} \]
ways to do this.

\end{document}
