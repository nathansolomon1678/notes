\documentclass[12pt]{article}
\usepackage[margin=1in]{geometry}
\usepackage{amsmath}
\usepackage{amsfonts}
\usepackage{amsthm}
\newtheorem{thm}{Theorem}[section]
\newtheorem{cor}[thm]{Corollary}
\newtheorem{lem}[thm]{Lemma}
\usepackage{tikz-cd}
\renewcommand{\d}{\mathrm{d}}

\begin{document}

\title{Math 180 homework 1}
\author{Nathan Solomon}
\maketitle

\textbf{Due January 19th}

\section{}
\noindent\fbox{\fbox{\parbox{6.5in}{
            For a set $X$, let $\mathrm{id}_X: X \rightarrow X$ denote the function defined by $\mathrm{id}_X(x)=x$ for all $x \in X$ (the \textit{identity function}). Let $f: X \rightarrow Y$ be some function. Prove:
            \begin{itemize}
                \item (a) A function $g: Y \rightarrow X$ such that $g \circ f = \mathrm{id}_X$ exists if and only if $f$ is one-to-one.
                \item (b) A function $g: Y \rightarrow X$ such that $f \circ g = \mathrm{id}_X$ exists if and only if $f$ is onto.
                \item (c) A function $g: Y \rightarrow X$ such that both $f \circ g = \mathrm{id}_X$ and $g \circ f = \mathrm{id}_X$ exists if and only if $f$ is a bijection.
                \item (d) If $f: X \rightarrow Y$ is a bijection, then the following three conditions are equivalent for a function $g: Y \rightarrow X$:
                    \begin{itemize}
                        \item (i) $g = f^{-1}$
                        \item (ii) $g \circ f = \mathrm{id}_X$
                        \item (iii) $f \circ g = \mathrm{id}_Y$
                    \end{itemize}
            \end{itemize}
}}}\bigskip\par

\section{}
\noindent\fbox{\fbox{\parbox{6.5in}{
            Prove that that following two statements about a function $f: X \rightarrow Y$ are equivalent ($X$ and $Y$ are some arbitrary sets):
            \begin{itemize}
                \item (i) $f$ is one-to-one
                \item (ii) For any set $Z$ and any two distinct functions $g_1: Z \rightarrow X$ and $g_2: Z \rightarrow X$, the composed functions $f \circ g_1$ and $f \circ g_2$ are also distinct.
            \end{itemize}
            (First, make sure you understand what it means that two functions are equal and what it means that they are distinct.)
}}}\bigskip\par

\section{}
\noindent\fbox{\fbox{\parbox{6.5in}{
            Let $f: A \rightarrow B$ be a surjective function. Let use define a relation on $A$ by setting $a_1 \sim a_2$ if $f(a_1)=f(a_2)$.
            \begin{itemize}
                \item (a) Show that $\sim$ is an equivalence relation.
                \item (b) Show that there is a bijective correspondence between the set of equivalence classes in $A$ and the set $B$.
            \end{itemize}
}}}\bigskip\par

\section{}
\noindent\fbox{\fbox{\parbox{6.5in}{
            Describe all the relations on a set $X$ that are equivalences and orderings at the same time.
}}}\bigskip\par
Equivalences are reflexive, symmetric, and transitive. Orderings are the same except they're required to be antisymmetric instead of symmetric, so these relations on $X$ must be both.
\par
Let $R$ be a of relation on $X$ which is an equivalence and an ordering. For any two elements $a,b \in X$, if $a \neq b$, then $(a,b) \not\in R$. That's because if $(a,b)$ were in $R$, by symmetry, $(b,a)$ would be in $R$, but by antisymmetry, $(b,a)$ could not be in $R$, which is a contradiction. If $a=b$, then by reflexivity, $(a,b) \in R$.
\par
There is only one relation $R$ which satisfies those criteria:
\[ R=\{(x,x): x\in X\} \]
(that is, the relation where elements of $X$ are only related to themselves)

\section{}
\noindent\fbox{\fbox{\parbox{6.5in}{
    Determine the number of ways to distribute 10 orange drinks, 1 lemon drink, and 1 lime drink to 4 thirsty students so that each student gets at least 1 drink, and the lemon and lime drinks go to different students.
}}}\bigskip\par
First, distribute the lemon and lime drinks. The lemon drink can go to of the 4 students, then the lime drink can go to any of the 3 remaining students, so there are 12 options for those.
\par
Next, distribute the 10 orange drinks. The students that did not get the lemon or lime drinks must receive at least one orange drink, so there are only 8 left to distibute.
\par
To figure out how many ways 8 drinks can be distributed among 4 students, consider a ``stars and bars" diagram, where you write a string starting with a number of stars equal to the number of drinks you give to the first person, then a bar, then a number of stars equal to the number of drinks given to the second person, and so on. For example, giving everyone 2 orange drinks would be denoted as
\[ \star \star | \star \star | \star \star | \star \star. \]
Each possible string corresponds to a unique way to distribute the eight drinks, and the number of ways to choose such a string (with 11 characters, where 3 are bars and 8 are stars) is
\[ \binom{11}{3} = \binom{11}{8} = \frac{11!}{3!8!} = 165. \]
This is independent of our choice of who to give the lemon and lime drinks to, so the total number of options is
\[ 12 \times 165 = 1980. \]

\section{}
\noindent\fbox{\fbox{\parbox{6.5in}{
    Use the binomial theorem to show that
    \[ 3^n=\sum_{k=0}^n \binom{n}{k} 2^k. \]
    Give a combinatorial proof of this identity.
}}}\bigskip\par

\[ 3^n= (1+2)^n=\sum_{k=0}^n \binom{n}{k} 1^k2^k = \sum_{k=0}^n \binom{n}{k} 2^k\]
Combinatorial proof: $3^n$ is the number of ways to put $n$ distinguishable items in 3 distinguishable barrels. That's equivalent to first choosing the number of items ($n-k$) to put into the first barrel, then choosing which $k$ items to not put in the first barrel, and out of those $k$ items, choosing any subset of them to go into the second barrel, then putting the remaining items in the third barrel.

\section{}
\noindent\fbox{\fbox{\parbox{6.5in}{
            Prove the formula
            \[ \binom{r}{r} + \binom{r+1}{r} + \binom{r+2}{r} + \cdots + \binom{n}{r} = \binom{n+1}{r+1} \]
            by induction on $n$ (for $r$ arbitrary but fixed). Note what the formula says for $r=1$.
}}}\bigskip\par
When $r=1$, this is equivalent to the formula
\[ 1+2+3+\cdots +n = \binom{n+1}{2} = \frac{n^2+n}{2}. \]


\section{}
\noindent\fbox{\fbox{\parbox{6.5in}{
            In how many ways can we choose an ordered pair $(A, B)$ of subsets of $[n]$ so that $A \cap B = \emptyset$?
}}}\bigskip\par
Let $C=[n]-(A \cup B)$. Then every integer from 1 to $n$ is in $A$ or $B$ or $C$, but cannot be in more than one. Therefore each natural number up to $n$ can be put in one of those three sets, and those choices are all independent, so there are $3^n$ such ordered pairs.
\par
Equivalently, that is the number of $n$-letter words where each word is A, B, or C.

\section{}
\noindent\fbox{\fbox{\parbox{6.5in}{
            How many $k$-element subsets of $\{1,2,\dots,n\}$ exist containing no two consecutive numbers? \textit{Hint:} Use the idea used to count compositions.
}}}\bigskip\par

\end{document}
