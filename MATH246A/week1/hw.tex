\documentclass[12pt]{article}
\usepackage[margin=1in]{geometry}
\usepackage{amsmath}
\usepackage{amsfonts}
\usepackage{tikz-cd}
\usepackage{hyperref}

\begin{document}

\title{Math 246A HW 1}
\author{Nathan Solomon}
\maketitle

\textbf{Notes 0: Exercises 11, 14 (i)-(iii), 16, 17}
\par
\textbf{Stein-Shakarchi Chapter 1: Exercises 4, 5}
\par
\textbf{Due Friday, October 6th}

\section{Exercises from Notes 0}

\bigskip
\noindent\fbox{\fbox{\parbox{6.5in}{
            \textbf{11.i.} Let $T: \mathbb{R}^2 \rightarrow \mathbb{R}^2$ be an isometry of the Euclidean plane that fixes the origin $(0, 0)$. Show that $T$ is either a a rotation around the origin by some angle $\theta \in \mathbb{R}$, or the reflection around some line through the origin. (\textit{Hint:} try to compose $T$ with rotations or reflections to achieve some normalisation of $T$, e.g. that $T$ fixes $(1, 0)$. Then consider what $T$ must do to other points in the plane, such as $(0, 1)$. Alternatively, one can use various formulae relating distances to angle, such as the sine rule or cosine rule, or the formula $\langle z, w \rangle = |z| |w| \cos \theta$ for the inner product.) For this question, you may use any result you already know from Euclidean geometry or trigonometry.
}}}\bigskip

The standard way to solve this problem is to show that if we know $T(1,0)$ and $T(0, 1)$ then for any point $p \in \mathbb{R}^2$, $T(p)$ is uniquely determined. But I personally think that's tedious and messy, so I'll instead prove that $T$ is linear (which will also be messy), and then since linear functions can be written as matrices, it will be very painless to show that a linear isometry is either a rotation or reflection.
\par
$T$ is a linear function if and only if $T(a) + T(b) = T(a + b)$ and $\lambda T(a) = T(\lambda a)$ for any $a, b \in \mathbb{R}^2$ and any scalar $\lambda \in \mathbb{R}$. However, that second case is the same as the first when $b = (\lambda - 1) a$, so we only need to show the first condition is true.
\par
For any $a, b \in \mathbb{R}^2$, $T$ preserves the magnitude of $a$ and $b$ and $a+b$ as well as the inner product $\langle a, b \rangle$. The cosine rule says that
\[ |a+b| = \sqrt{|a|^2 + |b|^2 + 2 \langle a, b \rangle} \]
so that must also be equal to
\[ \sqrt{|T(a)|^2 + |T(b)|^2 + 2 \langle T(a), T(b) \rangle} \]
which is clearly also equal to

\url{https://math.stackexchange.com/a/3253558}.

\bigskip
\noindent\fbox{\fbox{\parbox{6.5in}{
            \textbf{11.ii.} Show that all isometries $T: \mathbb{C} \rightarrow \mathbb{C}$ of the complex numbers take the form
            \[ T(z) = z_0 + wz \]
            or
            \[ T(z) = z_0 + w \overline{z} \]
            for some complex number $z_0$ and phase $w \in S^1$.
}}}\bigskip

WAIT TO DO THIS PROBLEM UNTIL I HAVE A GOOD ANSWER TO THE FIRST QUESTION

\noindent\fbox{\fbox{\parbox{6.5in}{
            \textbf{14.i.} If $n$ is a positive integer, show that the only complex number solutions to the equation $z^n = 1$ are given by the $n$ complex numbers $e^{2 \pi i k / n}$ for $k=0, \dots, n-1$; these numbers are thus known as the $n^{th}$ roots of unity. Conclude the identity
            \[ z^n - 1 = \prod_{k=0}^{n-1} (z - e^{2 \pi i k / n}) \]
            for any complex number $z$.
}}}\bigskip

Let $s \in \mathbb{C}$ be a solution to $z^n=1$. Since $s$ is clearly nonzero, we can write $s$ in polar form as $r \omega$, where $r := |s|$ is the radius and $\omega := s/|s|$ is the phase. Then the equation becomes $r^n \omega^n = 1$.
\par
In the October 4th lecture, we proved that $S^1$, the set of complex numbers with magnitude one, is a subgroup of $\mathbb{C} \backslash \{0\}$, so $\omega^n$ must have magnitude 1.
\par
In the same lecture, we also proved that the magnitude of the product of two complex numbers is equal to the product of their magnitudes (that is, $\forall a,b \in \mathbb{C}, |ab|=|a|\cdot|b|$). This means $1 = |r^n|\cdot|\omega^n| = |r^n|$. Since $r$ is positive, $r$ must be one, so $s = \omega$.
\par
Every element of $S^1$ can be written as $e^{ix}$ for some real number $x$ CITATION NEEDED. If $x$ is not equal to $2 \pi i k / n$ (plus an integer multiple of $2 \pi$) for some $k \in \{0, 1, \dots, n-1\}$, then $2 \pi i k$ would not be an integer multiple of $2 \pi$, so by Euler's identity, $s^n = e^{2 \pi i k}$ would not be equal to one. By the same reasoning, any $s \in \{e^{2 \pi i k/n} : k = 0, 1, \dots, n-1\}$ is a solution to $z^n = 1$, and multiplying or dividing by $e^{2 \pi i}$ doesn't change a number, so those $n$ solutions are unique.
\par
The polynomial $z^n - 1$ has $n$ solutions, degree $n$, and the coefficient of $z^n$ is one
CITE RESULT THE RESULT THAT FINITE DEGREE POLYNOMIALS ARE UNIQUELY DETERMINED UP TO A CONSTANT MULTIPLE BY THEIR ROOTS (IF THE DEGREE IS THE NUMBER OF ROOTS).

\bigskip
\noindent\fbox{\fbox{\parbox{6.5in}{
            \textbf{14.ii.} Show that the only compact subgroups $G$ of the multiplicative complex numbers $\mathbb{C}^\times$ are the unit circle $S^1$ and the $n^{th}$ roots of unity
            \[ \mathbb{C}_n := \{ e^{2 \pi i k / n} : k = 0, 1, \dots, n-1 \} \]
for $n = 1, 2, \dots$. (Hint: there are two cases, depending on whether $1$ is a limit point of $G$ or not.)
}}}\bigskip

First, I'll show that every compact subgroup of $G$ is also a subgroup of $S^1$, then I'll show that if 1 is a limit point in $G$, $G=S^1$, and if 1 is not a limit point in $G$, then $G$ is a finite group.
\par
Suppose there exists an element $g \in G \backslash S^1$. Just like in the previous question, since $g \neq 0$, there exist numbers $r>0, \omega \in S^1$ such that $g=r\omega$. In the previous question we also showed that multiplying by an element of $S^1$ doesn't change magnitude, so $|g|=r$, which also implies $|g^n|=r^n$ for any integer $n$.
\par
For any $\varepsilon > 0$, let $n$ be an integer greater than $\log_r \varepsilon$ if $r>1$, and let $n$ be an integer less than $\log_r \varepsilon$ if $r<1$. In either case, $|g^n|=r^n>\varepsilon$. Since $G$ contains elements with arbitrarily large magnitudes, $G$ is not bounded and therefore not compact (according to the Heine-Borel theorem as stated in \url{https://en.wikipedia.org/wiki/Heine%E2%80%93Borel_theorem}). That's a contradiction, so $G$ must be a subgroup of $S^1$.
\par
STILL NEED TO DO THE SECOND PART OF THIS QUESTION, WITH THE LIMIT POINT STUFF

\bigskip
\noindent\fbox{\fbox{\parbox{6.5in}{
            \textbf{14.iii.} Give an example of a non-compact subgroup of $S^1$.
}}}\bigskip

Let $G$ be the following set. First, I will prove that $G$ obeys the requirements to be a subgroup of $\mathbb{C} \backslash \{0\}$, then I'll construct a Cauchy sequence in $G$ which converges outside of $G$. Since $G$ is not complete, it cannot be compact.

\[ G := \left\{ e^{q \pi i} : q \in \mathbb{Q} \right\} \]

It is known that $(\mathbb{Q}, +)$ is a subgroup of $(\mathbb{R}, +)$, because (1) the sum of any two rational rumbers is rational, (2) the additive inverse of any rational numbers is rational, and (3) zero is a rational number. Using equation 12 from \url{https://terrytao.wordpress.com/2016/09/18/246a-notes-0-the-complex-numbers/}, we see that (1) the product of any two elements of $G$ is in $G$, (2) the multiplicative inverse of an element in $G$ is in $G$, and (3) $G$ contains the multiplicative identity. This proves $G$ is a subgroup of $S^1$.
\par
Let $q_n$ be a Cauchy sequence of rational numbers that converges to an irrational number $q_\infty \in \mathbb{R} \backslash \mathbb{Q}$. Also, define the following sequence:
\[ z_n := e^{q_n \pi i} \]
\par
For any $n$, the distance from $z_n$ to $z_\infty := e^{q_\infty \pi i}$ in $\mathbb{C}$ is less than or equal to $\pi$ times the distance from $q_n$ to $q_\infty$ in $\mathbb{R}$, because
\begin{align*}
    |z_n - z_\infty| &= |e^{(q_n - q_\infty) \pi i} - 1| \\
                                         &= |\cos((q_n - q_\infty) \pi) + i \sin((q_n - q_\infty) \pi) - 1| \\
                                         &= \sqrt{\left( \cos^2((q_n - q_\infty) \pi) - 2 \cos((q_n - q_\infty) \pi) + 1 \right) + \sin^2((q_n - q_\infty) \pi)} \\
                                         &= \sqrt{2 - 2 \cos((q_n - q_\infty) \pi)} \\
                                         &= 2 \left| \sin \left( \frac{\pi}{2} (q_n - q_\infty)\right) \right| \\
                                         &\leq \pi \left| q_n - q_\infty \right|
\end{align*}
The second-to-last step there used a trig identity from \url{https://math.stackexchange.com/a/668844} (which implicitly assumed $\theta$ is real, but that's fine). The bound in the last step ($\forall x \in \mathbb{R}, |\sin x| \leq |x|$) is a consequence of the mean value theorem, which I took from \url{https://en.wikipedia.org/wiki/Mean_value_theorem}. If that bound were not true, the mean value theorem would imply that there is a real number $x$ where the derivative of $\sin x$ is not in $[-1, 1]$, which is not possible, since the the cosine of a real number is always in $[-1, 1]$.
\par
Another way to say that $q_n$ converges to $q_\infty$ is
\[ \limsup_{n \rightarrow \infty} | q_n - q_\infty | = 0 \]
Then we can use the inequality we just proved ($|z_n - z_\infty| \leq \pi \left| q_n - q_\infty \right|$) to show that
    \[ \limsup_{n \rightarrow \infty} | z_n - z_\infty| \leq \limsup_{n \rightarrow \infty} \pi | q_n - q_\infty | = 0 \]
    meaning $z_n$ converges to $z_\infty$. Since $q_\infty$ is irrational, $z_\infty$ is a limit point of $G$ that is not in $G$, so $G$ is not closed. According to the Heine-Borel theorem as stated in \url{https://en.wikipedia.org/wiki/Heine%E2%80%93Borel_theorem} and the fact that $\mathbb{C}$ has the same metric as $\mathbb{R}^2$, this means $G$ is a non-compact subgroup of $S^1$.
THIS PROOF IS A MESS, NEED TO CLEAN IT UP

\bigskip
\noindent\fbox{\fbox{\parbox{6.5in}{
            \textbf{16.} Let $z_n$ be a sequence of complex numbers. Show that $\sin(z_n)$ is bounded if and only if the imaginary part of $z_n$ is bounded, and similarly with $\sin(z_n)$ replaced by $\cos(z_n)$.
}}}\bigskip

First, I'll show that $e^{i z_n}$ is bounded, then I'll use that to show that both $\sin(z_n)$ and $\cos(z_n)$ are bounded. Using Euler's identity along with equation 12 from \url{https://terrytao.wordpress.com/2016/09/18/246a-notes-0-the-complex-numbers/}, $e^{i z_n}$ can be rewritten as follows for some $\omega \in S^1$:
\begin{align*}
    e^{i z_n} &= e^{i (\operatorname{Re}(z_n) + i \operatorname{Im}(z_n))} \\
              &= e^{i \operatorname{Re}(z_n)} e^{- \operatorname{Im} (z_n)} \\
              &= e^{- \operatorname{Im}(z_n)} \omega \\
    |e^{i z_n}| &= |e^{- \operatorname{Im}(z_n)} \omega| = |e^{- \operatorname{Im}(z_n)}| |\omega| = e^{- \operatorname{Im}(z_n)}
\end{align*}

Let $l = \inf_{n \in \mathbb{N}} \operatorname{Im} (z_n)$ be a lower bound on the imaginary component of $z_n$. Then since $\exp : \mathbb{R} \rightarrow \mathbb{R}$ is positive and strictly increasing, for every $n \in \mathbb{N}$, we have the following bound:
\[ 0 < e^{- \operatorname{Im}(z_n)} \leq e^{-l} \]
Note that when $x$ is real, $\sin x$, $\cos x$, and $\exp x$ are also all real (this is easy to prove from the Taylor series definitions of those functions), so Euler's identity is then equivalent to the statement ``$\operatorname{Re}(e^{ix}) = \cos(x)$ and $\operatorname{Im}(e^{ix}) = \sin(x)$". Since neither the real nor the imaginary component of a complex number can be larger than its magnitude, both $\sin z_n$ and $\cos z_n$ are bounded:
\begin{align*}
    |\sin(z_n)| &= |\operatorname{Im}(e^{i z_n})| \leq |e^{i z_n}| = e^{- \operatorname{Im}(z_n)} \leq e^{-l} \\
    |\cos(z_n)| &= |\operatorname{Re}(e^{i z_n})| \leq |e^{i z_n}| = e^{- \operatorname{Im}(z_n)} \leq e^{-l}
\end{align*}
\par
Applying the same reasoning in the converse direction, if either $\sin x$ or $\cos x$ is not bounded, then $|e^{i z_n}| = e^{- \operatorname{Im}(z_n)}$ is not bounded. If the imaginary component of $z_n$ is bounded, there exists an upper bound on $- \operatorname{Im}(z_n)$, and since $\exp: \mathbb{R} \rightarrow \mathbb{R}$ is strictly increasing, that would imply there is an upper bound on $e^{- \operatorname{Im}(z_n)}$. That is a contradiction, so the imaginary component of $z_n$ cannot be bounded.

\bigskip
\noindent\fbox{\fbox{\parbox{6.5in}{
            \textbf{17.} (This question was drawn from a previous version of this course taught by Rowan Killip.) Let $w_1, w_2$ be distinct complex numbers, and let $\lambda$ be a positive real that is not equal to $1$.
        \begin{itemize}
            \item (i) Show that the set $\left\{ z \in \mathbb{C}: \left| \frac{z - w_1}{z - w_2} \right| = \lambda \right\}$ defines a circle in the complex plane. (Ideally, you should be able to do this without breaking everything up into real and imaginary parts.)
            \item (ii) Conversely, show that every circle in the complex plane arises in such a fashion (for suitable choices of $w_1, w_2, \lambda$, of course).
            \item (iii) What happens if $\lambda = 1$?
            \item (iv) Let $\gamma$ be a circle that does not pass through the origin. Show that the image of $\gamma$ under the inversion map $z \mapsto 1/z$ is a circle. What happens if $\gamma$ is a line? What happens if the $\gamma$ passes through the origin (and one then deletes the origin from $\gamma$ before applying the inversion map)?
        \end{itemize}
}}}\bigskip

\begin{itemize}
    \item (i) Taking the reciprocal of both sides of the equation gives an equivalent equation in the same format, except it swaps $w_1$ with $w_2$ and $\lambda$ with $\lambda^{-1}$. Therefore, we can assume without loss of generality that $\lambda > 1$. Also, we can multiply both sides by $|z-w_2|$, so the equation becomes
        \[ |z-w_1| = \lambda |z-w_2| \]
        \begin{align*}
            |z|^2-z \overline{w_1}-\overline{z} w_1 + |w_1|^2 = \lambda^2 \left( |z|^2-z \overline{w_2}-\overline{z} w_2 + |w_2|^2 \right)
        \end{align*}
    \item (ii)
    \item (iii) If $\lambda=1$ then that set is the set of points that are just as far from $w_1$ as they are from $w_2$, which intuitively is a line.
    \item (iv) The image of $\gamma$ under the inversion map is
        \begin{align*}
            & \left\{ z \in \mathbb{C}: \left| \frac{z^{-1} - w_1}{z^{-1} - w_2} \right| = \lambda \right\} \\
            =& \left\{ z \in \mathbb{C}: \left| \frac{z^{-1} - w_1}{z^{-1} - w_2} \right| = \lambda \right\} \\
        \end{align*}
\end{itemize}

\section{Exercises from Stein \& Shakarchi Chapter 1}

\noindent\fbox{\fbox{\parbox{6.5in}{
            \textbf{4.} Show that it is impossible to define a total ordering on C. In other words, one cannot find a relation $\succ$ between complex numbers so that:
            \begin{itemize}
        \item (i) For any complex numbers $z, w$, one and only one of the following is true: $z \succ w, z \prec w, z = w$.
        \item (ii) For all $z_1, z_2, z_3 \in \mathbb{C}$ the relation $z_1 \succ z_2$ imples $z_1 + z_3 \succ z_2 + z_3$.
        \item (iii) Moreover, for all $z_1, z_2, z_3 \in \mathbb{C}$ with $z_3 \succ 0$, then $z_1 \succ z_2$ implies $z_1 z_3 \succ z_2 z_3$.
            \end{itemize}

            [Hint: first check if $i \succ 0$ is possible.]
}}}\bigskip

Suppose $i \succ 0$. Applying condition (iii) twice in a row with $z_3=i$, we see that $i \cdot i \cdot i \succ 0 \cdot 0 \cdot 0$. By condition (ii), $i + i^3 \succ 0 + 0^3$, but since $i + i^3 = 0 = 0$, that contradicts condition (i).
\par
Suppose $i \prec 0$. Applying condition (ii) with $z_3=-i$, we get that $0 \prec (-i)$. Then repeating the same process as in the previous paragraph, we get that $(-i)+(-i)^3 \succ 0$, which once again contradicts condition (i).
\par
Lastly, $i=0$ is false, so there is no way for condition (i) to be true when $z=i$ and $w=0$. Therefore it is impossible to define a total ordering on $\mathbb{C}$.

\bigskip
\noindent\fbox{\fbox{\parbox{6.5in}{
            \textbf{5.} A set $\Omega$ is called \textbf{pathwise connected} if any two points in $\Omega$ can be joined by a (piecewise-smooth) curve entirely contained in $\Omega$. The purpose of this exercise is to prove that an \textit{open} set $\Omega$ is pathwise connected if and only if $\Omega$ is connected.
            \begin{itemize}
                \item (a) Suppose first that $\Omega$ is open and pathwise connected, and that it can be written as $\Omega = \Omega_1 \cup \Omega_2$ where $\Omega_1$ and $\Omega_2$ are disjoint non-empty open sets. Choose two points $w_1 \in \Omega$ and $w_2 \in \Omega_2$ and let $\gamma$ denote a curve in $\Omega$ joining $w_1$ to $w_2$. Consider a parameterization $z : [0, 1] \rightarrow \Omega$ of this curve with $z(0) = w_1$ and $z(1) = w_2$, and let
                    \[ t^* = \sup_{0 \leq t \leq 1} \left\{ t : z(s) \in \Omega_1 \text{  for all } 0 \leq s < t \right\} \]
                    Arrive at a contradiction by considering a point $z(t^*)$.
                \item (b) Conversely, suppose that $\Omega$ is open and connected. Fix a point $w \in \Omega$ and let $\Omega_1 \subset \Omega$ denote the set of all points that can be joined to $w$ by a curve contained in $\Omega$. Also, let $\Omega_2 \subset \Omega$ denote the set of all points that cannot be joined to $w$ by a curve in $\Omega$. Prove that both $\Omega_1$ and $\Omega_2$ are open, disjoint, and their union is $\Omega$. Finally, since $\Omega_1$ is non-empty (why?) conclude that $\Omega=\Omega_1$ as desired.

The proof actually shows that the regularity and type of curves we used to define pathwise connectedness can be relaxed without changing the equivalence between the two definitions when $\Omega$ is open. For instance, we may take all curves to be continuous, or simply polygonal lines.
            \end{itemize}
}}}\bigskip

For this problem, I'll treat $\Omega$ as a topological space. This is valid since every metric space is a topological space, according to \url{https://math.stackexchange.com/a/3284602}.
\begin{itemize}
    \item (a) Since $z(t^*)$ is in $\Omega$, it must also be in either $\Omega_1$ or $\Omega_2$. But $z$ is a parameterized curve, so it's continuous by definition. Therefore the preimage of $\Omega_1$ under $z$ is open in $[0, 1]$, and the preimage of $\Omega_2$ is as well. Since the preimage of $\Omega$ under $z$ is all of $[0, 1]$, $[0, 1]$ must be the union of those preimages, which are two disjoint open sets. According to \url{https://proofwiki.org/wiki/Subset_of_Real_Numbers_is_Interval_iff_Connected}, every interval is connected, so that's a contradiction.
    \item (b) Since there exists a disc around every point in $\Omega$ and discs are obviously path connected, $\Omega$ is locally path connected. That means $\Omega_1$ is open, because for any point in $\Omega_1$, there exists a disc around that point such that every point in the disc is path connected that point, and therefore path connected to $w$ as well. STILL NEED TO SHOW THAT OMEGA ONE IS CLOSED
        \par
        $\Omega_2$ is defined as the complement of $\Omega_1$, so they're clearly disjoint and their union is $\Omega$. $\Omega_1$ is non-empty because it contains $w$. Since $\Omega$ is connected, the only subsets that are both open and closed are the empty set and $\Omega$, so $\Omega_1 = \Omega$, meaning $\Omega$ is path-connected.
\end{itemize}

\end{document}
