\documentclass[12pt]{article}
\usepackage[margin=1in]{geometry}
\usepackage{amsmath}
\usepackage{amsfonts}
\usepackage{tikz-cd}
\usepackage{hyperref}
\renewcommand*{\d}{\mathrm{d}}

\begin{document}

\title{Math 246A HW 3}
\author{Nathan Solomon}
\maketitle

\textbf{Notes 1, Exercises 23, 26, 27; Notes 2, Exercise 3 (i)-(iv); Stein-Shakarchi Chapter 1, Exercises 9, 25. Due Friday, October 20th.}

\bigskip
\noindent\fbox{\fbox{\parbox{6.5in}{
            \textbf{Notes 1, Exercise 23 (Wirtinger derivatives).} Let $U$ be an open subset of $\mathbb{C}$, and let $f:U \rightarrow \mathbb{C}$ be a Fréchet differentiable function. Define the Wirtinger derivatives $ \frac{\partial f}{\partial z} : U \rightarrow \mathbb{C}, \frac{\partial f}{\partial \overline{z}} : U \rightarrow \mathbb{C}$ by the formulae
            \begin{align*}
                \frac{\partial f}{\partial z} &:= \frac{1}{2} \left( \frac{\partial f}{\partial x} + \frac{1}{i} \frac{\partial f}{\partial y} \right) \\
                \frac{\partial f}{\partial \overline{z}} &:= \frac{1}{2} \left( \frac{\partial f}{\partial x} - \frac{1}{i} \frac{\partial f}{\partial y} \right)
            \end{align*}
            \begin{itemize}
                \item (i) Show that $f$ is holomorphic on $U$ if and only if the Wirtinger derivative $ \frac{\partial f}{\partial \overline{z}} $ vanishes identically on $U$.
                \item (ii) If $f$ is given by a polynomial
                    \[ f(z) = \sum_{n,m \geq 0: n+m \leq d} c_{n,m}z^n \overline{z}^m \]
                    in both $z$ and $\overline{z}$ for some complex coefficients $c_{n,m}$ and some natural number $d$, show that
                    \[ \frac{\partial f}{\partial z} (z) = \sum_{n,m \geq 0: n+m \leq d} c_{n,m}(nz^{n-1}) \overline{z}^m \]
                    and
                    \[ \frac{\partial f}{\partial \overline{z}} (z) = \sum_{n,m \geq 0: n+m \leq d} c_{n,m}z^n (m\overline{z}^{m-1}) \]
                    (\textit{Hint:} first establish a Leibniz rule for Wirtinger derivatives.) Conclude in particular that $f$ is holomorphic if and only if $c_{n,m}$ vanishes whenever $m \geq 1$ (i.e. $f$ does not contain any terms that involve $\overline{z}$).
                \item (iii) If $z_0$ is a point in $U$, show that one has the Taylor expansion
                    \[ f(z) = f(z_0) + \frac{\partial f}{\partial z} (z_0)(z-z_0) + \frac{\partial f}{\partial \overline{z}} (z_0) \overline{(z-z_0)} + o(|z-z_0|) \]
                    as $z \rightarrow z_0$, where $o(|z-z_0|)$ denotes a quantity of the form $|z-z_0| c(z)$, where $c(z)$ goes to zero as $z$ goes to $z_0$ (compare with equation (1) in Notes 1). Conversely, show that this property determines the numbers $ \frac{\partial f}{\partial z} (z_0)$ and $ \frac{\partial f}{\partial \overline{z}} (z_0)$ uniquely (and thus can be used as an alternate definition of the Wirtinger derivatives).
            \end{itemize}
}}}\bigskip

\begin{itemize}
    \item (i) If $\frac{\partial f}{\partial \overline{z}}$ vanishes identically on $U$, then at every point in $U$,
        \[ \frac{1}{2} \frac{\partial f}{\partial x} = \frac{1}{2i} \frac{\partial f}{\partial y}, \]
        so $f$ satisfies the Cauchy-Riemann equations. Since $f$ is also Fréchet differentiable, that means $f$ is holomorphic.
        \par
        If $f$ is holomorphic, then the partial derivatives are defined and satisfy the Cauchy-Riemann equations, which implies
        \[ \frac{\partial f}{\partial \overline{z}} = \frac{1}{2} \left( \frac{\partial f}{\partial x} - \frac{1}{i} \frac{\partial f}{\partial y} \right) = 0 \]
        at every point in $U$, so the Wirtinger derivative $ \frac{\partial f}{\partial \overline{z}} $ vanishes identically on $U$ if an only $f$ is holomorphic.
    \item (ii) Since $x$ and $y$ are real numbers satisfying $z=x+iy$, we can expand the definition of $f$ to
        \[ f(x+iy) = \sum_{n,m \geq 0: n+m \leq d} c_{n,m} (x+iy)^n (x-iy)^m. \]
        For each term in that sum, we can use the power rule and product rule to take the partial derivatives:
        \begin{align*}
            \frac{\partial f}{\partial x} &= \sum_{n,m \geq 0: n+m \leq d} c_{n,m} \left( n (x+iy)^{n-1} (x-iy)^m + m (x+iy)^n (x-iy)^{m-1} \right) \\
            \frac{\partial f}{\partial y} &= \sum_{n,m \geq 0: n+m \leq d} c_{n,m} \left( in (x+iy)^{n-1} (x-iy)^m - im (x+iy)^n (x-iy)^{m-1} \right).
        \end{align*}
        Then applying the definitions of the Wirtinger derivatives, that becomes
        \begin{align*}
            \frac{\partial f}{\partial z} &= \sum_{n,m \geq 0: n+m \leq d} c_{n,m}(nz^{n-1}) \overline{z}^m \\
            \text{and  } \frac{\partial f}{\partial \overline{z}} &= \sum_{n,m \geq 0: n+m \leq d} c_{n,m}z^n (m\overline{z}^{m-1}).
        \end{align*}
        If $c_{n,m}$ vanishes whenever $m \geq 1$, then $f$ is a polynomial in $z$, so $f$ is holomorphic.
        \par
        Let $m_{max}$ be the highest $m$ for which there exists an $n$ such that $n+m \leq d$ and $c_{n,m} \neq 0$. When $m=m_{max}$,
        \[ \frac{\partial^m f}{(\partial \overline{z})^m} = \sum_{n=0}^{d-m} c_{n,m} (m!) z^n \]
        which is a nonzero polynomial in $z$. The statement we want to prove is vacuously true if $U$ is empty, so we assume $U$ is nonempty. Since $U$ is open, it must contain infinitely many points, but because it's a finite degree polynomial, by the fundamental theorem of algebra, it has finitely many roots. This implies
        \[ \frac{\partial f}{\partial \overline{z}} \neq 0 \]
        so $f$ is holomorphic if and only if $c_{n,m} = 0$ whenever $m \geq 1$.
    \item (iii) Notation: let $x, y, x_0, y_0$ be real numbers such that
        \[ z=x+iy \hspace{1cm} \text{and} \hspace{1cm} z_0=x_0+iy_0. \]
        COMPARE WITH DEFINITION OF FRECHET DIFFERENTIABLE
\end{itemize}

\noindent\fbox{\fbox{\parbox{6.5in}{
            \textbf{Notes 1, Exercise 26 (Maximum principle for holomorphic functions).} If $f: U \rightarrow \mathbb{C}$ is a continuously twice differentiable holomorphic function on an open set $U$, and $K$ is a compact subset of $U$, show that
            \[ \sup_{z \in K} |f(z)| = \sup_{z \in \partial K} |f(z)|. \]
            (\textit{Hint:} use Theorem 25 and the fact that $|w| = \sup_{\theta \in \mathbb{R}} \operatorname{Re}(w e^{i \theta})$ for any compex number $w$.) What happens if we replace the suprema on both sides by infima? This result is also known as the maximum modulus principle.
}}}\bigskip

Consider the function $g: \mathbb{C} \rightarrow \mathbb{R}$ defined by $g(x+iy) = \ln{|f(x+iy)|}$. One can verify that
\begin{align*}
    \Delta g(x+iy) &= \frac{\partial^2}{\partial x^2} g(x+iy) + \frac{\partial^2}{\partial y^2} g(x+iy) \\
                   &= \frac{\partial^2}{\partial x^2} \ln{|f(x+iy)|} + \frac{\partial^2}{\partial y^2} \ln{|f(x+iy)|} \\
                   &= \frac{\partial}{\partial x} \frac{f'(x+iy)}{f(x+iy)} + \frac{\partial}{\partial y} \frac{if'(x+iy)}{f(x+iy)} \\
                   &= \frac{f''(x+iy)f(x+iy)-(f'(x+iy))^2}{f(x+iy)^2} + \frac{-f''(x+iy)f(x+iy)-(if'(x+iy))^2}{f(x+iy)^2} \\
                   &= 0.
\end{align*}
Therefore $g$ is a harmonic function. So by theorem 25 from Notes 1,
\[ \sup_{z \in K} g(z) = \sup_{z \in \partial K} g(z). \]
Since $\exp: \mathbb{R} \rightarrow \mathbb{R}$ is strictly increasing, it is valid to apply it to both sides of that equation, so we get
\[ \sup_{z \in K} |f(z)| = \sup_{z \in \partial K} |f(z)|. \]
\par
Note that if we replace the suprema with infima, the statement is no longer true -- for example, if $f$ is the identity map on $\mathbb{C}$ and $K$ is the unit disc, then
\[ \inf_{z \in K} |f(z)| = 0 \]
but since $\partial K$ is the circle of radius 1 around the origin,
\[ \inf_{z \in \partial K} |f(z)| = 1. \]

\bigskip
\noindent\fbox{\fbox{\parbox{6.5in}{
            \textbf{Notes 1, Exercise 27.} Recall the Wirtinger derivatives defined in Exercise 23(i).
            \begin{itemize}
                \item (i) If $f: U \rightarrow \mathbb{C}$ is twice continuously differentiable on an open subset $U$ of $\mathbb{C}$, show that
                    \[ \Delta f = 4 \frac{\partial}{\partial z} \frac{\partial f}{\partial \overline{z}} = 4 \frac{\partial}{\partial \overline{z}} \frac{\partial f}{\partial z}. \]
                    Use this to give an alternate proof that ($C^2$) holomorphic functions are harmonic.
                \item (ii) If $f$ is given by a polynomial
                    \[ f(z) = \sum_{n,m \geq 0:n_m \leq d} c_{n,m} z^n \overline{z}^m \]
                    in both $z$ and $\overline{z}$ for some complex coefficients $c_{n,m}$ and some natural number $d$, show that $f$ is harmonic on $\mathbb{C}$ if and only if $c_{n,m}$ vanishes whenever $n$ and $m$ are both positive (i.e. $f$ only contains terms $c_{n,0}z^n$ or $c_{0,m} \overline{z}^m$ that only involve one of $z$ or $\overline{z}$).
                \item (iii) If $u: U \rightarrow \mathbb{R}$ is a real polynomial
                    \[ u(x+iy) = \sum_{n,m \geq 0:n_m \leq d} a_{n,m} x^n y^m \]
                    in $x$ and $y$ for some real coefficients $a_{n,m}$ and some natural number $d$, show that $u$ is harmonic if and only if it is the real part of a polynomial $f(z) = \sum_{n=0}^d c_n z^n$ of one complex variable $z$.
            \end{itemize}
}}}\bigskip

\noindent\fbox{\fbox{\parbox{6.5in}{
            \textbf{Notes 2, Exercise 3.} Let $\gamma_1, \gamma_2, \gamma_3, \widetilde{\gamma_1}, \widetilde{\gamma_2}$ be continuous curves. Suppose that the terminal point of $\gamma_1$ equals the initial point of $\gamma_2$, and the terminal point of $\gamma_2$ equals the initial point of $\gamma_3$.
            \begin{itemize}
                \item (i) (Concatenation and reversal well defined up to equivalence) If $\gamma_1 \equiv \widetilde{\gamma_1}$ and $\gamma_2 \equiv \widetilde{\gamma_2}$, show that $\gamma_1 + \gamma_2 \equiv \widetilde{\gamma_1} + \widetilde{\gamma_2}$ and $-\gamma_1 \equiv \widetilde{\gamma_1}$.
                \item (ii)
                \item (iii)
                \item (iv)
            \end{itemize}
}}}\bigskip

\noindent\fbox{\fbox{\parbox{6.5in}{
            \textbf{Stein-Shakarchi Chapter 1, Exercise 9.} Show that in polar coordinates, the Cauchy-Riemann equations take the form
            \[ \frac{\partial u}{\partial r} = \frac{1}{r} \frac{\partial v}{\partial \theta} \hspace{1cm} \text{and} \hspace{1cm} \frac{1}{r} \frac{\partial u}{\partial \theta} = - \frac{\partial v}{\partial r}. \]
            Use these equations to show that the logarithm function defined by
            \[ \log{z} = \log{r} + i \theta \hspace{1cm} \text{where } z = re^{i \theta} \text{ with } - \pi < \theta < \pi \]
            is holomorphic in the region $r>0$ and $-\pi < \theta < \pi$.
}}}\bigskip

Begin with the equations
\begin{align*}
    x &= r \cos{\theta} \\
    y &= r \sin{\theta}.
\end{align*}
Differentiating using the chain rule and substituting in the Cartesian form of the Cauchy-Riemann equations, we get
\begin{align*}
    \frac{\partial u}{\partial r} &= (\cos{\theta}) \frac{\partial u}{\partial x} + (\sin{\theta}) \frac{\partial u}{\partial } 
\end{align*}
In the given region, we have $u=\log{r}$ and $v=\theta$, so
\[ \frac{\partial u}{\partial r} = \frac{1}{r}  = \frac{1}{r} \frac{\partial v}{\partial \theta} \]
and
\[ \frac{1}{r} \frac{\partial u}{\partial \theta} = 0 = - \frac{\partial v}{\partial r}. \]
Also, the partial derivatives of $u$ and $v$ with respect to $\theta$ and $r$ are all continuous, which implies $\log$ is Fréchet differentiable. Since $\log$ satisfies the Cauchy-Riemann equations, it is holomorphic (on the region where $r>0$ and $-\pi < \theta < \pi$).

\bigskip
\noindent\fbox{\fbox{\parbox{6.5in}{
            \textbf{Stein-Shakarchi Chapter 1, Exercise 25.} The next three calculations provide some insight into Cauchy's theorem, which we treat in the next chapter.
            \begin{itemize}
                \item (a) Evaluate the integrals
                    \[ \int_\gamma z^n dz \]
                    for all integers $n$. Here $\gamma$ is the circle centered at the origin with the positive (counterclockwise) orientation.
                \item (b) Same question as before, but with $\gamma$ any circle not containing the origin.
                \item (c) Show that if $|a|<r<|b|$, then
                    \[ \int_\gamma \frac{1}{(z-a)(z-b)} dz = \frac{2 \pi i}{a-b}, \]
                    where $\gamma$ denotes the circle centered at the origin, of radius $r$, with the positive orientation.
            \end{itemize}
}}}\bigskip

\begin{itemize}
    \item (a) Suppose $\gamma$ is a circle with radius $r>0$, so it can be parameterized by
        \[ z(t) = re^{it} \]
        for $t$ in $[0, 2 \pi]$. Then according the formula Stein-Shakarchi uses (on page 21) to define the integral along a curve in $\mathbb{C}$,
        \begin{align*}
            \int_\gamma z^n dz &= \int_0^{2 \pi} (z(t))^n z'(t) dt \\
                               &= \int_0^{2 \pi} r^n e^{i n t} (ir e^{it})dt \\
                               &= i r^{n+1} \int_0^{2 \pi} e^{i (n+1) t} dt \\
        \end{align*}
        which evaluates to zero when $n \neq -1$ and to $2 \pi i$ when $n=-1$.
    \item (b) If $\gamma$ does not contain or enclose the origin, then it has the parameterization
        \[ z(t) = R + re^{it} \]
        for some $R > r > 0$. Using the same method as in part (a), we get
        \begin{align*}
            \int_\gamma z^n dz &= \int_0^{2 \pi} (z(t))^n z'(t) dt \\
                               &= \int_0^{2 \pi} (R+re^{it})^n (ir e^{it})dt \\
        \end{align*}
        If $n \geq 0$ then we can use a binomial expansion on the integrand, and see that every term of that expansion is multiplied at least once by $e^{it}$, so the integral evaluates to zero. If $n$ is negative, we instead use the negative binomial series (which is valid when $|re^{it}| < R$, according to \url{https://mathworld.wolfram.com/NegativeBinomialSeries.html}):
        \[ (R+re^{it})^n = \sum_{k=0}^\infty (-1)^k \binom{-n+k-1}{k} R^{n-k} (re^{it})^k. \]
        Every term in the integrand is a constant multiple of $e^{it(k+1)}$, which when integrated from $t=0$ to $t=2 \pi$, becomes zero unless $k=-1$. However, the summation does not include a $k=-1$ term, so that integral is zero no matter what $n$ is.
    \item (c) Using the same method as in parts (a) and (b), suppose $\gamma$ has the parameterization
        \[ z(t) = re^{it} \]
        and then write
        \begin{align*}
            \int_\gamma f(z) dz &= \int_0^{2 \pi} f(z(t)) z'(t) dt \\
                               &= \int_0^{2\pi} \frac{ire^{it}}{(re^{it}-a)(re^{it}-b)} dt \\
                               &= i \int_0^{2\pi} \left[ \frac{a}{(a-b)(re^{it}-a)} + \frac{b}{(a-b)(b-re^{it})} \right] dt \\
                               &= \left[ \frac{ia}{a-b} \int_0^{2\pi} \frac{e^{-it}}{r} \cdot \frac{1}{1 - \frac{a}{r} e^{-it}} dt \right] + \left[ \frac{i}{a-b} \int_0^{2\pi} \frac{1}{1 - \frac{r}{b} e^{it}} dt \right] .
        \end{align*}
        We know that the equation
        \[ \frac{1}{1-x} = \sum_{n=0}^\infty x^n \]
        is true whenever $|x|<1$, so that equation becomes
        \begin{align*}
            \int_\gamma f(z) dz &= \left[ \frac{ia}{a-b} \int_0^{2\pi} \frac{1}{r} \cdot \left( e^{-it} + \frac{a}{r} e^{-2it} + \frac{a^2}{r^2} e^{-3it} + \cdots \right) dt \right] + \left[ \frac{i}{a-b} \int_0^{2\pi} \frac{1}{1 - \frac{r}{b} e^{it}} dt \right] \\
                                &= \frac{i}{a-b} \int_0^{2\pi} \left( 1 + \frac{r}{b} e^{it} + \frac{r^2}{b^2} e^{2it} + \cdots \right) dt \\
                                &= \frac{2 \pi i}{a-b}.
        \end{align*}
\end{itemize}

\end{document}
