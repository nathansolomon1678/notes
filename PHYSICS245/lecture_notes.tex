\documentclass{article}
\usepackage[margin=1in]{geometry}
\usepackage[linesnumbered,ruled,vlined]{algorithm2e}
\usepackage{amsfonts}
\usepackage{amsmath}
\usepackage{amssymb}
\usepackage{amsthm}
\usepackage{enumitem}
\usepackage{fancyhdr}
\usepackage{hyperref}
\usepackage{minted}
\usepackage{multicol}
\usepackage{pdfpages}
\usepackage{standalone}
\usepackage[many]{tcolorbox}
\usepackage{tikz-cd}
\usepackage{transparent}
\usepackage{xcolor}
% \tcbuselibrary{minted}

\author{Nathan Solomon}

\newcommand{\fig}[1]{
    \begin{center}
        \includegraphics[width=\textwidth]{#1}
    \end{center}
}

% Math commands
\renewcommand{\d}{\mathrm{d}}
\DeclareMathOperator{\id}{id}
\DeclareMathOperator{\im}{im}
\DeclareMathOperator{\proj}{proj}
\DeclareMathOperator{\Span}{span}
\DeclareMathOperator{\Tr}{Tr}
\DeclareMathOperator{\tr}{tr}
\DeclareMathOperator{\ad}{ad}
\DeclareMathOperator{\ord}{ord}
%%%%%%%%%%%%%%% \DeclareMathOperator{\sgn}{sgn}
\DeclareMathOperator{\Aut}{Aut}
\DeclareMathOperator{\Inn}{Inn}
\DeclareMathOperator{\Out}{Out}
\DeclareMathOperator{\stab}{stab}

\newcommand{\N}{\ensuremath{\mathbb{N}}}
\newcommand{\Z}{\ensuremath{\mathbb{Z}}}
\newcommand{\Q}{\ensuremath{\mathbb{Q}}}
\newcommand{\R}{\ensuremath{\mathbb{R}}}
\newcommand{\C}{\ensuremath{\mathbb{C}}}
\renewcommand{\H}{\ensuremath{\mathbb{H}}}
\newcommand{\F}{\ensuremath{\mathbb{F}}}

\newcommand{\E}{\ensuremath{\mathbb{E}}}
\renewcommand{\P}{\ensuremath{\mathbb{P}}}

\newcommand{\es}{\ensuremath{\varnothing}}
\newcommand{\inv}{\ensuremath{^{-1}}}
\newcommand{\eps}{\ensuremath{\varepsilon}}
\newcommand{\del}{\ensuremath{\partial}}
\renewcommand{\a}{\ensuremath{\alpha}}

\newcommand{\abs}[1]{\ensuremath{\left\lvert #1 \right\rvert}}
\newcommand{\norm}[1]{\ensuremath{\left\lVert #1\right\rVert}}
\newcommand{\mean}[1]{\ensuremath{\left\langle #1 \right\rangle}}
\newcommand{\floor}[1]{\ensuremath{\left\lfloor #1 \right\rfloor}}
\newcommand{\ceil}[1]{\ensuremath{\left\lceil #1 \right\rceil}}
\newcommand{\bra}[1]{\ensuremath{\left\langle #1 \right\rvert}}
\newcommand{\ket}[1]{\ensuremath{\left\lvert #1 \right\rangle}}
\newcommand{\braket}[2]{\ensuremath{\left.\left\langle #1\right\vert #2 \right\rangle}}

\newcommand{\catname}[1]{{\normalfont\textbf{#1}}}

\newcommand{\up}{\ensuremath{\uparrow}}
\newcommand{\down}{\ensuremath{\downarrow}}

% Custom environments
\newtheorem{thm}{Theorem}[section]

\definecolor{probBackgroundColor}{RGB}{250,240,240}
\definecolor{probAccentColor}{RGB}{140,40,0}
\newenvironment{prob}{
    \stepcounter{thm}
    \begin{tcolorbox}[
        boxrule=1pt,
        sharp corners,
        colback=probBackgroundColor,
        colframe=probAccentColor,
        borderline west={4pt}{0pt}{probAccentColor},
        breakable
    ]
    \color{probAccentColor}\textbf{Problem \thethm.} \color{black}
} {
    \end{tcolorbox}
}

\definecolor{exampleBackgroundColor}{RGB}{212,232,246}
\newenvironment{example}{
    \stepcounter{thm}
    \begin{tcolorbox}[
      boxrule=1pt,
      sharp corners,
      colback=exampleBackgroundColor,
      breakable
    ]
    \textbf{Example \thethm.}
} {
    \end{tcolorbox}
}

\definecolor{propBackgroundColor}{RGB}{255,245,220}
\definecolor{propAccentColor}{RGB}{150,100,0}
\newenvironment{prop}{
    \stepcounter{thm}
    \begin{tcolorbox}[
        boxrule=1pt,
        sharp corners,
        colback=propBackgroundColor,
        colframe=propAccentColor,
        breakable
    ]
    \color{propAccentColor}\textbf{Proposition \thethm. }\color{black}
} {
    \end{tcolorbox}
}

\definecolor{thmBackgroundColor}{RGB}{235,225,245}
\definecolor{thmAccentColor}{RGB}{50,0,100}
\renewenvironment{thm}{
    \stepcounter{thm}
    \begin{tcolorbox}[
        boxrule=1pt,
        sharp corners,
        colback=thmBackgroundColor,
        colframe=thmAccentColor,
        breakable
    ]
    \color{thmAccentColor}\textbf{Theorem \thethm. }\color{black}
} {
    \end{tcolorbox}
}

\definecolor{corBackgroundColor}{RGB}{240,250,250}
\definecolor{corAccentColor}{RGB}{50,100,100}
\newenvironment{cor}{
    \stepcounter{thm}
    \begin{tcolorbox}[
        enhanced,
        boxrule=0pt,
        frame hidden,
        sharp corners,
        colback=corBackgroundColor,
        borderline west={4pt}{0pt}{corAccentColor},
        breakable
    ]
    \color{corAccentColor}\textbf{Corollary \thethm. }\color{black}
} {
    \end{tcolorbox}
}

\definecolor{lemBackgroundColor}{RGB}{255,245,235}
\definecolor{lemAccentColor}{RGB}{250,125,0}
\newenvironment{lem}{
    \stepcounter{thm}
    \begin{tcolorbox}[
        enhanced,
        boxrule=0pt,
        frame hidden,
        sharp corners,
        colback=lemBackgroundColor,
        borderline west={4pt}{0pt}{lemAccentColor},
        breakable
    ]
    \color{lemAccentColor}\textbf{Lemma \thethm. }\color{black}
} {
    \end{tcolorbox}
}

\definecolor{proofBackgroundColor}{RGB}{255,255,255}
\definecolor{proofAccentColor}{RGB}{80,80,80}
\renewenvironment{proof}{
    \begin{tcolorbox}[
        enhanced,
        boxrule=1pt,
        sharp corners,
        colback=proofBackgroundColor,
        colframe=proofAccentColor,
        borderline west={4pt}{0pt}{proofAccentColor},
        breakable
    ]
    \color{proofAccentColor}\emph{\textbf{Proof. }}\color{black}
} {
    \qed \end{tcolorbox}
}

\definecolor{noteBackgroundColor}{RGB}{240,250,240}
\definecolor{noteAccentColor}{RGB}{30,130,30}
\newenvironment{note}{
    \begin{tcolorbox}[
        enhanced,
        boxrule=0pt,
        frame hidden,
        sharp corners,
        colback=noteBackgroundColor,
        borderline west={4pt}{0pt}{noteAccentColor},
        breakable
    ]
    \color{noteAccentColor}\textbf{Note. }\color{black}
} {
    \end{tcolorbox}
}


\fancyhf{}
\setlength{\headheight}{24pt}

\date{\today}
\title{Physics 245 Lecture Notes, Fall 2024}

\begin{document}
\maketitle
\fig{quantum_computer_photo.jpg}

\tableofcontents

\section{9/26/2024 lecture}
\noindent\fbox{\fbox{\parbox{6.5in}{
            \begin{itemize}
                \item Course professor: Eric Hudson
                \item Grade is either (50\% homework, 25\% midterm, and 25\% final), or (50\% homework, 10\% midterm, and 40\% final).
                \item The midterm and final will ask questions from discussion section. ALso, the midterm will be during discussion section (on November 8th).
                \item Homeworks were going to be due every Tuesday before class (10:30 am), but this has been changed to Thursdays.
    \end{itemize}
}}}\bigskip\par
All information is stored in physical systems. For example, a book stores information via the arrangement of ink on paper, and a hard drive stores memory by magnetizing regions of some material. \textbf{Since information is physical, information processing is govered by physics.} The goal of quantum computing is to take full advantage of that, by manipulating information in ways that is not possible with a classical computer.
\bigskip
\begin{thm}
    The Hamiltonian operator on a quantum wavefunction $\Psi(x, t)$ is defined as
    \[ H := \left( V(x, t) - \frac{\hbar^2}{2m} \nabla^2 \right). \]
    The Schrödinger equation says that the eigenvalues of $H$ are energy values, and that
    \[ H\Psi(x, t) = i \hbar \frac{\partial}{\partial t} \Psi(x, t). \]
\end{thm}
\begin{example}
    The ``infinite square well" is a 1-dimensional system in which a particle is trapped in a potential
    \[ V(x) := \begin{cases}
        0 & x \in [0, L] \\
        +\infty & x \not\in [0, L]
    \end{cases}. \]
    Any solution to the Schrödinger equation is a linear combination of basis states $\Psi_n(x, t), n \in \N$:
    \[ \Psi_n(x, t) = \sqrt{ \frac{2}{L}} \sin \left( \frac{n \pi x}{L} \right) e^{-i E_n t / \hbar}. \]
    where the energy of the $n$th state is
    \[ E_n = \frac{\hbar^2 n^2 \pi^2}{2 m L^2}. \]
\end{example}
The infinite square well is pretty much the simplest example of a situation where we can solve the Schrödinger equation. For a slightly more interesting version of that, want to see what happens when we start with a wavefunction of the form
\[ \Psi(x,t) = \sum_{n \in \N} a_n(t) \Psi_n(x, t), a_n \in \C \]
and then suddenly turn on a potential
\[ V(x) = \begin{cases}
    \Omega \cos (\pi x /L) & x \in [0, L] \\
    +\infty & x \not\in [0, L]
\end{cases}. \]
Plugging in our wavefunction and Hamiltonian to the Schrödinger equation, we get
\begin{align*}
    H(x, t) \sum_{n \in \N} a_n \Psi_n(x, 0) e^{-i E_n t / \hbar} &= i \hbar \frac{\partial}{\partial t} \left( \sum_{n \in \N} a_n \Psi_n(x, 0) e^{-i E_n t / \hbar} \right) \\
\Omega \cos \left( \frac{\pi x}{L} \right)  \sum_{n \in \N} a_n \Psi_n(x, 0) e^{-i E_n t / \hbar} &= i \hbar \sum_{n \in \N} \dot{a_n} \Psi_n(x, 0) e^{-i E_n t / \hbar} \\
\end{align*}
FINISH THESE NOTES

\section{10/1/2024 lecture}
An ideal qubit is a perfectly isolated 2-level system, but in reality, the 2 states will always be couples to states of an external system. For example, if our qubit is an electron in a magnetic field whose available states are spin-up and spin-down, then any electric field nearby will cause Stark shifting of the energy levels. Also, unless we can force the electron to be in the ground state, its own spatial wavefunction will create electromagnetic effects that interact with spin.
\bigskip
\par
A vector space $V$ over a field $\F$ is a non-empty set $V$ with vector addition ($+ : V \times V \rightarrow V$) and scalar multiplication $\cdot : \F \times V \rightarrow V$) such that for any $u,v,w \in V$ and any $a,b \in \F$,
\begin{itemize}
    \item $u+(v+w)=(u+v)+w$
    \item $u+v=v+u$
    \item There exists $0 \in V$ such that $v+0=v$
    \item There exists $(-v) \in V$ such that $v+(-v)=0$
    \item $a(bv)=(ab)v$
    \item If $1 \in \F$ is the multiplicative identity, then $1v=v$
    \item $a(u+v)=au+av$
    \item $(a+b)v=av+bv$
\end{itemize}
A  normed vector space $V$ over some field $\F$ also has a norm $\norm{\cdot} : V \rightarrow \R$ such that for any $u,v \in V, a \in \F$,
\begin{itemize}
    \item $V$ satisfies the axioms of a vector space over $\F$
    \item $\norm{u} \geq 0$, and $\norm{u}=0$ iff $u=0$
    \item $\F$ is a subfield of $\C$, and $\norm{au}=\abs{a} \cdot \norm{u}$. Technically, there may be some other fields that would work, like the $p$-adic numbers, but we don't care about those. In fact, some definitions of a normed vector space would require that $\F$ is either $\R$ or $\C$
    \item $\norm{x+y}\leq \norm{x}+\norm{y}$
\end{itemize}
Note that a normed vector space is also a metric space, because if we define $d: V \times V \rightarrow V$ by $d(u,v)=\norm{u-v}$, then $(V, d)$ satisfies the metric space axioms. We call $V$ a Banach space $V$ iff it is complete in that metric.
\par
An inner product space $X$ is a vector space over $\F$, where $\F$ is either $\R$ or $\C$, with an inner product $\mean{\cdot, \cdot}: X \times X \rightarrow X$ such that for any $x,y,z \in X, a,b \in \F$,
\begin{itemize}
    \item $\mean{y,x} = \overline{\mean{x, y}}$
    \item $\mean{ax+by, z} = a \mean{x,z}+b\mean{y,z}$
    \item $\mean{x, x} \geq 0$, and $\mean{x,x}=0$ iff $x=0$
\end{itemize}
\begin{note}
    Mathematicians define the inner product to be linear in the first argument ($\mean{ax,y}=a\mean{x,y}$) and conjugate-linear in the second argument ($\mean{x,ay}=\overline{a}\mean{x,y}$). However, in Dirac's bra-ket notation, which physicsts use, the inner product is conjugate linear in the first argument ($ \braket{ax}{y} = a^* \braket{x}{y}$) and linear in the second argument ($ \braket{x}{ay} =a \braket{x}{y}$). We can clarify this by using different notation, and by assuming that if you are reading about quantum mechanics, you will see the physicists' notation. But if there is any ambiguity, specify which you are using.
\end{note}
\par
We call $V$ a Hilbert space iff it is a Banach space, and it is an inner product space, and $\norm{x}^2=\mean{x,x}$ for any $x \in V$.
\begin{thm}
    ``A Hilbert space is a safe space -- it's where physicists like to be" -- prof. Hudson.
\end{thm}

\section{10/3/2024 lecture}

\section{10/8/2024 discussion/review}
If the Hamiltonian does not depend on time, then the time evolution operator $U(t, t_0)$ can be defined as
\[ U(t, t_0) := \exp \left( - \frac{i (t - t_0) H}{\hbar} \right), \]
and it is useful because
\[ \ket{\Psi(t)} = U(t, t_0) \ket{\Psi(t_0)}. \]
If $H$ is diagonal then $U$ is as well.
\par
\begin{lem}
    The exponential of a skew-Hermitian matrix is unitary, and the determinant of the exponential of a traceless matrix is one.
\end{lem}
\begin{proof}
    If $A$ is skew-Hermitian, then $-A=A^\dag$, so
    \[ \exp(A)^{-1}=\exp(-A)=\exp(A^\dag)=\exp(A)^\dag, \]
    therefore $\exp(A)$ is unitary.
    \par
    If $A$ is traceless, then
    \[ \det(\exp(A)) = \exp(\Tr(A)) = \exp(0) = 1. \]
\end{proof}
Since $H$ is an $n \times n$ Hermitian matrix ($n$ could be infinity), then $i H t / \hbar \in \mathfrak{u}(n)$ is skew-Hermitian (also called anti-Hermitian, meaning its adjoint is its negative). Therefore its exponential, $\exp(iHt/\hbar) \in U(n) = (SU(n) \oplus U(1)) / \Z_n$, is unitary. By going into the interaction picture, we could also ensure $H$ is traceless, so $i H t /\hbar \in \mathfrak{su}(n)$ is traceless, so $\exp(iHt/\hbar) \in SU(n)$.

\section{10/10/2024 lecture}
The energy of a magnetic moment $\mu$ in a magnetic field is $E = - \mu \cdot B$. For most of this class, we will think of a 2-state qubit as an electron in a magnetic field, but in reality, any Two-Level System (TLS) is a qubit, and the physics is the same for any TLS.
\par
The spin state of the qubit can be written as
\[ \ket{\Psi} = c_0 \ket{\uparrow} + c_1 \ket{\downarrow}, \]
which can also be written as
\[ \ket{\Psi} = c_0 \ket{0} + c_1 \ket{1} = \begin{bmatrix}
    c_0 \\
    c_1
\end{bmatrix}, \]
where $c_0, c_1 \in \C$ satisfy $\norm{c_0}^2+\norm{c_1}^2 = 1$.
\begin{note}
    $\ket{0}$ represents the ``up"/``excited" state and $\ket{1}$ is the ``down"/``ground" state. This can be confusing because we are used to indexing states like $\ket{n}$ for the harmonic oscillator, where energy increases as $n$ increases. However, in this case, the $\ket{0}$ state has positive energy and the $\ket{1}$ state has negative energy.
\end{note}
The Hamiltonian for our electron is
\[ H = -\mu \cdot B = \frac{g \mu_B}{\hbar} S \cdot B, \]
where $g = -2.002319 \approx 2$ is the gyromagnetic ratio for the electron and
\[ \mu_B = \frac{e \hbar}{2 m_e} = 9.274 \times 10^{-24} J/T \]
is the ``Bohr magneton".
\par
If $B_x$ and $B_y$ are both zero, then
\[ H = \frac{g \mu_B B_z}{\hbar} S_z = \frac{g \mu_B B_z}{\hbar} \cdot \frac{\hbar \sigma_z}{2} \approx \mu_B B_z \sigma_z = \mu_B B_z \begin{bmatrix}
    1 & 0 \\
    0 & -1
\end{bmatrix}. \]


\section{10/15/2024 lecture}

\section{10/17/2024 lecture}
The \textbf{interaction picture} is an extremely useful tool for simplifying problems. The main idea is that we can factor a unitary matrix out of the time evolution operator, and this is sort of like changing to a moving reference frame.

\section{10/22/2024 lecture}

\section{10/24/2024 lecture}
This lecture was recorded, see Bruinlearn/pages

\section{10/29/2024 lecture}

\section{10/31/2024 lecture}

\section{11/5/2024 lecture}
If we drive a harmonic oscillator with a force $F=-F_0\sin(\omega t + \phi)$, the Hamiltonian is
\[ \frac{H}{\hbar} = \omega (a^\dag a + \frac{1}{2}) + \beta \sin (\omega t + \phi) (a + a^\dag). \]
The corresponding time evolution operator is
\[ D(\alpha) := U = \exp \left( \alpha a^\dag - \alpha^* a \right) \]
where $\alpha \propto \beta t$. We label the coherent states as
\[ \ket{\alpha} = D(\alpha) \ket{0}, \]
but this is in the interaction picture. If we leave the interaction picture, we have
\[ U_0 D(\alpha) \ket{0} = \ket{\alpha e^{-i \omega t}}. \]

\subsection{Physically implementing a QHO}
We have studied how to do quantum gates on a harmonic oscillator, but how do we actually implement a system like that? An LC circuit looks like a harmonic oscillator, where
\[ H = \frac{\Phi}{2L} + \frac{L \omega^2 Q}{2}. \]
We can also create a system of trapped ions by putting one ion or more in a line, then laser cooling them so that they remain on that line (instead of wiggling around).
\fig{LIT - Geometry.png}

\subsection{Composite systems}
Suppose we have 2 qubits,
\begin{align*}
    \ket{\Psi_1} &= a \ket{0} + b \ket{1} \\
    \ket{\Psi_2} &= c \ket{0} + d \ket{1}.
\end{align*}
Then the whole systems can be described by a tensor product,
\begin{align*}
    \ket{\Psi_{total}} &= \ket{\Psi_1} \otimes \ket{\Psi_2} \\
                       &= ac \ket{00} + ad \ket{01} + bc \ket{10} +bd \ket{11} \\
                       &= \begin{bmatrix}
                           ac \\
                           ad \\
                           bc \\
                           bd
                       \end{bmatrix}.
\end{align*}
To use this notation, we need to know how to treat operators on the whole system as matrices. For example,
\[ P_{01}=\ket{01}\bra{01}=P_0\otimes P_1 = \begin{bmatrix}
    1 & 0 \\
    0 & 0
\end{bmatrix} \otimes \begin{bmatrix}
    0 & 0 \\
    0 & 1
\end{bmatrix} = \begin{bmatrix}
    0 & 0 & 0 & 0 \\
    0 & 1 & 0 & 0 \\
    0 & 0 & 0 & 0 \\
    0 & 0 & 0 & 0
\end{bmatrix}. \]
In other words, to get the operator on the composite system, we take the Kronecker product of the operator on the first qubit and the operator on the second qubit.
\par
If we have a system of $n$ qubits, the state of the system lives in
\[ \mathcal{H}_{total} = \bigotimes_{i=1}^n \mathcal{H}^{(i)}. \]
The state of the system can be ``disentangled" (factored) as a tensor product of elements of $\mathcal{H}^{(1)}, \mathcal{H}^{(2)}, \dots$ iff the qubits are not entangled. For example, the state
\[ \frac{1}{\sqrt{2}} \left( \ket{00} + \ket{11} \right) \]
is entangled, because if you measure the state of one qubit, you know the other qubit must be in the same state. This cannot be factored as
\[ (a \ket{0} + b \ket{1}) \otimes (c \ket{0} + d \ket{1}) \]
because $ad$ and $bc$ would be zero, which implies either $ac$ or $bd$ is zero.
\section{11/7/2024 lecture}
If you have an atom inside a harmonic oscillator, and the atom can be either excited or unexcited (the excited state, $\ket{0}$, has $\omega_0$ more energy that the ground state, $\ket{1}$), the Hamiltonian for the system can be written as
\[ \frac{H}{\hbar} = \frac{\omega_0}{2} \sigma_z \otimes I + \omega I \otimes \left( N + \frac{1}{2} I \right) + V, \]
where the state of the atom is $\ket{\Psi_1}$, the state of the harmonic oscillator that the atom is trapped in (which has frequency $\omega$) is $\ket{\Psi_2}$, and the composite state is $\ket{\Psi_1} \otimes \ket{\Psi_2}$.
\par
The $V$ term is
\[ V := \frac{g}{2} \sigma_x \left( a^\dag + a \right) = \frac{g}{2} \left( \sigma_- + \sigma_+ \right) \left( a^\dag + a \right). \]
This term represets coupling between the atom and the harmonic oscillator. The motivation for this is that if the detuning $\delta = \omega - \omega_0$ is small, then the atom and the harmonic oscillator can exchange enrgy. I still don't understand exactly why we include it.
\par
To go to the interaction picture, we can define $H_0 = H - \hbar V$ and $U_0 = \exp \left( -i H_0 t/\hbar \right) $, so the Hamiltonian in the interaction picture is
\begin{align*}
    H_I &= U_0^\dag V U_0 \\
        &= \frac{g}{2} \left( e^{i \omega_0 \sigma_z t / 2} \sigma_x e^{-i \omega_0 \sigma_z t / 2} \right) \left( e^{i \omega (a^\dag a + 1/2)t} (a+a^\dag) e^{- i \omega (a^\dag a + 1/2)t} \right) \\
        &= \frac{g}{2} \left( e^{i \omega_0 t} \sigma_+ + e^{-i \omega_0 t} \sigma_- \right) \left( e^{-i \omega t} a + e^{i \omega t} a^\dag \right) \\
        &= \frac{g}{2} \left( e^{i (\omega_0-\omega) t} \sigma_+ \otimes a + e^{i (\omega_0+\omega) t} \sigma_+ \otimes a^\dag + e^{i (-\omega_0-\omega) t} \sigma_- \otimes a + e^{i (-\omega_0+\omega) t} \sigma_- \otimes a^\dag \right) \\
        &\approx \frac{g}{2} \left( e^{i (\omega_0-\omega) t} \sigma_+ \otimes a + e^{i (-\omega_0+\omega) t} \sigma_- \otimes a^\dag \right).
\end{align*}
That last step used the rotating wave approximation (RWA).
\par
Now if we go back from the interaction picture to the lab frame,
\[ H = H_0 + \frac{g\hbar}{2} \left( \sigma_+ a + \sigma_- a^\dag \right).  \]
It makes sense that we were able to ignore the terms $\sigma_+ \otimes a^\dag$ and $\sigma_- \otimes a$, because those operators represent the atom becoming excited while the harmonic oscillator gains energy, and the atom becoming unexcited while the harmonic oscillator loses energy, respectively. The whole point of the $V$ term was to represent the atom and the oscillator exchanging energy, but since energy is conserved, neither of those operators are meaningful.
\par
This new Hamiltonian is very important, so we call it the Jaynes-Cummings Hamiltonian:
\[ \frac{H}{\hbar} = \frac{\omega_0}{2} \sigma_z \otimes I + I \otimes \omega \left( N + \frac{1}{2} I \right) + \frac{g}{2} \left( \sigma_+ \otimes a + \sigma_- \otimes a^\dag \right). \]
The eigenstates of this Hamiltonian are denoted $\ket{i,n}$, where $i \in \left\{ 0,1 \right\}$ and $n \in \N$. Note that the coupling term $V$ will only ``link" each state to at most one other state:
\[ \bra{i,m} V \ket{j, n} = \begin{cases}
    \frac{g}{2} \sqrt{n} & i=1, j=0, n=m+1 \\
    \frac{g}{2} \sqrt{m} & i=0, j=1, n=m-1 \\
    0 & \text{otherwise}
\end{cases} \]
Therefore the Hamiltonian can be written in the basis $ \left( \ket{1,0}, \ket{1,0}, \ket{1,1}, \ket{2,0},\dots \right)$ as a block-diagonal matrix, where each block is a symmetric $2 \times 2$ matrix. Note that I left $\ket{0,0}$ out of the basis, because the Hamiltonian is simpler that way -- if we wanted to include it, we would add another entry at the top left (with zeroes below and to the right of it).
\[ H = \begin{bmatrix}
    H^{0} & 0 & 0 & \cdots \\
    0 & H^{1} & 0 & \cdots \\
    0 & 0 & H^{2} & \cdots \\
    \vdots & \vdots & \vdots & \ddots \\
\end{bmatrix}, H^{(n)} = \begin{bmatrix}
    \frac{\omega_0}{2} + \omega \left( n + \frac{1}{2} \right) & \frac{g}{2} \sqrt{n+1} \\
    \frac{g}{2} \sqrt{n+1} & - \frac{\omega_0}{2} + \omega \left( n + \frac{3}{2} \right)
\end{bmatrix} \]
I need to double check all my formulas for these lecture notes, then get notes for the remainder of the lecture, where we solved for the eigenvectors and eigenvalues of each block matrix in order to find the new energy levels (as a function of the detuning $\delta$ and the coupling strength $g$). Also get notes for relating this to the Stark effect.

\section{Magnus expansion}
\section{Rotation operator on Bloch sphere}
\section{Interaction picture}
\section{Baker-Campbell-Hausdorff formula}
\section{Displacement operator}
\[ D(\alpha) = \exp \left( \alpha a^\dag -\alpha^* a \right)  \]
Note that the displacement operator takes the vacuum state to a coherent state, but it does not map every coherent state to another coherent state.
\par
Give definition of displacement operator and prove it's basic properties, like $D(\alpha)^\dag = D(-\alpha)$ and $D^\dag(\alpha)D(\alpha)=I$. Also talk about Kermack-McCrae identities, optical phase space, Wigner quasiprobability distribution, Husimi Q representation, and squeezed coherent states. For problem 3 on HW5, figure out explicit formula for probability as a function of $t_w$. For that, refer to \url{https://physics.stackexchange.com/questions/553225/representation-of-the-displacement-operator-in-number-basis}

\end{document}
