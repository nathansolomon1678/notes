\documentclass{article}
\usepackage[margin=1in]{geometry}
\usepackage[linesnumbered,ruled,vlined]{algorithm2e}
\usepackage{amsfonts}
\usepackage{amsmath}
\usepackage{amssymb}
\usepackage{amsthm}
\usepackage{enumitem}
\usepackage{fancyhdr}
\usepackage{hyperref}
\usepackage{minted}
\usepackage{multicol}
\usepackage{pdfpages}
\usepackage{standalone}
\usepackage[many]{tcolorbox}
\usepackage{tikz-cd}
\usepackage{transparent}
\usepackage{xcolor}
% \tcbuselibrary{minted}

\author{Nathan Solomon}

\newcommand{\fig}[1]{
    \begin{center}
        \includegraphics[width=\textwidth]{#1}
    \end{center}
}

% Math commands
\renewcommand{\d}{\mathrm{d}}
\DeclareMathOperator{\id}{id}
\DeclareMathOperator{\im}{im}
\DeclareMathOperator{\proj}{proj}
\DeclareMathOperator{\Span}{span}
\DeclareMathOperator{\Tr}{Tr}
\DeclareMathOperator{\tr}{tr}
\DeclareMathOperator{\ad}{ad}
\DeclareMathOperator{\ord}{ord}
%%%%%%%%%%%%%%% \DeclareMathOperator{\sgn}{sgn}
\DeclareMathOperator{\Aut}{Aut}
\DeclareMathOperator{\Inn}{Inn}
\DeclareMathOperator{\Out}{Out}
\DeclareMathOperator{\stab}{stab}

\newcommand{\N}{\ensuremath{\mathbb{N}}}
\newcommand{\Z}{\ensuremath{\mathbb{Z}}}
\newcommand{\Q}{\ensuremath{\mathbb{Q}}}
\newcommand{\R}{\ensuremath{\mathbb{R}}}
\newcommand{\C}{\ensuremath{\mathbb{C}}}
\renewcommand{\H}{\ensuremath{\mathbb{H}}}
\newcommand{\F}{\ensuremath{\mathbb{F}}}

\newcommand{\E}{\ensuremath{\mathbb{E}}}
\renewcommand{\P}{\ensuremath{\mathbb{P}}}

\newcommand{\es}{\ensuremath{\varnothing}}
\newcommand{\inv}{\ensuremath{^{-1}}}
\newcommand{\eps}{\ensuremath{\varepsilon}}
\newcommand{\del}{\ensuremath{\partial}}
\renewcommand{\a}{\ensuremath{\alpha}}

\newcommand{\abs}[1]{\ensuremath{\left\lvert #1 \right\rvert}}
\newcommand{\norm}[1]{\ensuremath{\left\lVert #1\right\rVert}}
\newcommand{\mean}[1]{\ensuremath{\left\langle #1 \right\rangle}}
\newcommand{\floor}[1]{\ensuremath{\left\lfloor #1 \right\rfloor}}
\newcommand{\ceil}[1]{\ensuremath{\left\lceil #1 \right\rceil}}
\newcommand{\bra}[1]{\ensuremath{\left\langle #1 \right\rvert}}
\newcommand{\ket}[1]{\ensuremath{\left\lvert #1 \right\rangle}}
\newcommand{\braket}[2]{\ensuremath{\left.\left\langle #1\right\vert #2 \right\rangle}}

\newcommand{\catname}[1]{{\normalfont\textbf{#1}}}

\newcommand{\up}{\ensuremath{\uparrow}}
\newcommand{\down}{\ensuremath{\downarrow}}

% Custom environments
\newtheorem{thm}{Theorem}[section]

\definecolor{probBackgroundColor}{RGB}{250,240,240}
\definecolor{probAccentColor}{RGB}{140,40,0}
\newenvironment{prob}{
    \stepcounter{thm}
    \begin{tcolorbox}[
        boxrule=1pt,
        sharp corners,
        colback=probBackgroundColor,
        colframe=probAccentColor,
        borderline west={4pt}{0pt}{probAccentColor},
        breakable
    ]
    \color{probAccentColor}\textbf{Problem \thethm.} \color{black}
} {
    \end{tcolorbox}
}

\definecolor{exampleBackgroundColor}{RGB}{212,232,246}
\newenvironment{example}{
    \stepcounter{thm}
    \begin{tcolorbox}[
      boxrule=1pt,
      sharp corners,
      colback=exampleBackgroundColor,
      breakable
    ]
    \textbf{Example \thethm.}
} {
    \end{tcolorbox}
}

\definecolor{propBackgroundColor}{RGB}{255,245,220}
\definecolor{propAccentColor}{RGB}{150,100,0}
\newenvironment{prop}{
    \stepcounter{thm}
    \begin{tcolorbox}[
        boxrule=1pt,
        sharp corners,
        colback=propBackgroundColor,
        colframe=propAccentColor,
        breakable
    ]
    \color{propAccentColor}\textbf{Proposition \thethm. }\color{black}
} {
    \end{tcolorbox}
}

\definecolor{thmBackgroundColor}{RGB}{235,225,245}
\definecolor{thmAccentColor}{RGB}{50,0,100}
\renewenvironment{thm}{
    \stepcounter{thm}
    \begin{tcolorbox}[
        boxrule=1pt,
        sharp corners,
        colback=thmBackgroundColor,
        colframe=thmAccentColor,
        breakable
    ]
    \color{thmAccentColor}\textbf{Theorem \thethm. }\color{black}
} {
    \end{tcolorbox}
}

\definecolor{corBackgroundColor}{RGB}{240,250,250}
\definecolor{corAccentColor}{RGB}{50,100,100}
\newenvironment{cor}{
    \stepcounter{thm}
    \begin{tcolorbox}[
        enhanced,
        boxrule=0pt,
        frame hidden,
        sharp corners,
        colback=corBackgroundColor,
        borderline west={4pt}{0pt}{corAccentColor},
        breakable
    ]
    \color{corAccentColor}\textbf{Corollary \thethm. }\color{black}
} {
    \end{tcolorbox}
}

\definecolor{lemBackgroundColor}{RGB}{255,245,235}
\definecolor{lemAccentColor}{RGB}{250,125,0}
\newenvironment{lem}{
    \stepcounter{thm}
    \begin{tcolorbox}[
        enhanced,
        boxrule=0pt,
        frame hidden,
        sharp corners,
        colback=lemBackgroundColor,
        borderline west={4pt}{0pt}{lemAccentColor},
        breakable
    ]
    \color{lemAccentColor}\textbf{Lemma \thethm. }\color{black}
} {
    \end{tcolorbox}
}

\definecolor{proofBackgroundColor}{RGB}{255,255,255}
\definecolor{proofAccentColor}{RGB}{80,80,80}
\renewenvironment{proof}{
    \begin{tcolorbox}[
        enhanced,
        boxrule=1pt,
        sharp corners,
        colback=proofBackgroundColor,
        colframe=proofAccentColor,
        borderline west={4pt}{0pt}{proofAccentColor},
        breakable
    ]
    \color{proofAccentColor}\emph{\textbf{Proof. }}\color{black}
} {
    \qed \end{tcolorbox}
}

\definecolor{noteBackgroundColor}{RGB}{240,250,240}
\definecolor{noteAccentColor}{RGB}{30,130,30}
\newenvironment{note}{
    \begin{tcolorbox}[
        enhanced,
        boxrule=0pt,
        frame hidden,
        sharp corners,
        colback=noteBackgroundColor,
        borderline west={4pt}{0pt}{noteAccentColor},
        breakable
    ]
    \color{noteAccentColor}\textbf{Note. }\color{black}
} {
    \end{tcolorbox}
}


\fancyhf{}
\setlength{\headheight}{24pt}

\date{\today}
\title{Physics 245 Lecture Notes, Fall 2024}

\begin{document}
\maketitle
\fig{quantum_computer_photo.jpg}

\tableofcontents

\section{9/26/2024 lecture}
\noindent\fbox{\fbox{\parbox{6.5in}{
            \begin{itemize}
                \item Course professor: Eric Hudson
                \item Grade is either (50\% homework, 25\% midterm, and 25\% final), or (50\% homework, 10\% midterm, and 40\% final).
                \item The midterm and final will ask questions from discussion section. ALso, the midterm will be during discussion section (on November 8th).
                \item Homeworks were going to be due every Tuesday before class (10:30 am), but this has been changed to Thursdays.
    \end{itemize}
}}}\bigskip\par
All information is stored in physical systems. For example, a book stores information via the arrangement of ink on paper, and a hard drive stores memory by magnetizing regions of some material. Since information is physical, information processing is govered by physics. The goal of quantum computing is to take full advantage of that, by manipulating information in ways that is not possible with a classical computer.

\section{10/1/2024 lecture}
Remember to distinguish between Hilbert space and Banach space

\section{10/3/2024 lecture}

\section{10/8/2024 discussion/review}

\section{10/10/2024 lecture}

\section{10/15/2024 lecture}

\section{10/17/2024 lecture}
See Andrew's notes

\section{10/22/2024 lecture}

\section{10/24/2024 lecture}
This lecture was recorded, see Bruinlearn/pages

\section{10/29/2024 lecture}

\section{10/31/2024 lecture}

\section{11/5/2024 lecture}
If we drive a harmonic oscillator with a force $F=-F_0\sin(\omega t + \phi)$, the Hamiltonian is
\[ \frac{H}{\hbar} = \omega (a^\dag a + \frac{1}{2}) + \beta \sin (\omega t + \phi) (a + a^\dag). \]
The corresponding time evolution operator is
\[ D(\alpha) := U = \exp \left( \alpha a^\dag - \alpha^* a \right) \]
where $\alpha \propto \beta t$. We label the coherent states as
\[ \ket{\alpha} = D(\alpha) \ket{0}, \]
but this is in the interaction picture. If we leave the interaction picture, we have
\[ U_0 D(\alpha) \ket{0} = \ket{\alpha e^{-i \omega t}}. \]

\subsection{Physically implementing a QHO}
We have studied how to do quantum gates on a harmonic oscillator, but how do we actually implement a system like that? An LC circuit looks like a harmonic oscillator, where
\[ H = \frac{\Phi}{2L} + \frac{L \omega^2 Q}{2}. \]
We can also create a system of trapped ions by putting one ion or more in a line, then laser cooling them so that they remain on that line (instead of wiggling around).
\fig{LIT - Geometry.png}

\subsection{Composite systems}
Suppose we have 2 qubits,
\begin{align*}
    \ket{\Psi_1} &= a \ket{0} + b \ket{1} \\
    \ket{\Psi_2} &= c \ket{0} + d \ket{1}.
\end{align*}
Then the whole systems can be described by a tensor product,
\begin{align*}
    \ket{\Psi_{total}} &= \ket{\Psi_1} \otimes \ket{\Psi_2} \\
                       &= ac \ket{00} + ad \ket{01} + bc \ket{10} +bd \ket{11} \\
                       &= \begin{bmatrix}
                           ac \\
                           ad \\
                           bc \\
                           bd
                       \end{bmatrix}.
\end{align*}
To use this notation, we need to know how to treat operators on the whole system as matrices. For example,
\[ P_{01}=\ket{01}\bra{01}=P_0\otimes P_1 = \begin{bmatrix}
    1 & 0 \\
    0 & 0
\end{bmatrix} \otimes \begin{bmatrix}
    0 & 0 \\
    0 & 1
\end{bmatrix} = \begin{bmatrix}
    0 & 0 & 0 & 0 \\
    0 & 1 & 0 & 0 \\
    0 & 0 & 0 & 0 \\
    0 & 0 & 0 & 0
\end{bmatrix}. \]
In other words, to get the operator on the composite system, we take the Kronecker product of the operator on the first qubit and the operator on the second qubit.
\par
If we have a system of $n$ qubits, the state of the system lives in
\[ \mathcal{H}_{total} = \bigotimes_{i=1}^n \mathcal{H}^{(i)}. \]
The state of the system can be ``disentangled" (factored) as a tensor product of elements of $\mathcal{H}^{(1)}, \mathcal{H}^{(2)}, \dots$ iff the qubits are not entangled. For example, the state
\[ \frac{1}{\sqrt{2}} \left( \ket{00} + \ket{11} \right) \]
is entangled, because if you measure the state of one qubit, you know the other qubit must be in the same state. This cannot be factored as
\[ (a \ket{0} + b \ket{1}) \otimes (c \ket{0} + d \ket{1}) \]
because $ad$ and $bc$ would be zero, which implies either $ac$ or $bd$ is zero.
\section{11/7/2024 lecture}
If you have an atom inside a harmonic oscillator, and the atom can be either excited or unexcited (the excited state, $\ket{0}$, has $\omega_0$ more energy that the ground state, $\ket{1}$), the Hamiltonian for the system can be written as
\[ \frac{H}{\hbar} = \frac{\omega_0}{2} \sigma_z \otimes I + \omega I \otimes \left( N + \frac{1}{2} I \right) + V, \]
where the state of the atom is $\ket{\Psi_1}$, the state of the harmonic oscillator that the atom is trapped in (which has frequency $\omega$) is $\ket{\Psi_2}$, and the composite state is $\ket{\Psi_1} \otimes \ket{\Psi_2}$.
\par
The $V$ term is
\[ V := \frac{g}{2} \sigma_x \left( a^\dag + a \right) = \frac{g}{2} \left( \sigma_- + \sigma_+ \right) \left( a^\dag + a \right). \]
This term represets coupling between the atom and the harmonic oscillator. The motivation for this is that if the detuning $\delta = \omega - \omega_0$ is small, then the atom and the harmonic oscillator can exchange enrgy. I still don't understand exactly why we include it.
\par
To go to the interaction picture, we can define $H_0 = H - \hbar V$ and $U_0 = \exp \left( -i H_0 t/\hbar \right) $, so the Hamiltonian in the interaction picture is
\begin{align*}
    H_I &= U_0^\dag V U_0 \\
        &= \frac{g}{2} \left( e^{i \omega_0 \sigma_z t / 2} \sigma_x e^{-i \omega_0 \sigma_z t / 2} \right) \left( e^{i \omega (a^\dag a + 1/2)t} (a+a^\dag) e^{- i \omega (a^\dag a + 1/2)t} \right) \\
        &= \frac{g}{2} \left( e^{i \omega_0 t} \sigma_+ + e^{-i \omega_0 t} \sigma_- \right) \left( e^{-i \omega t} a + e^{i \omega t} a^\dag \right) \\
        &= \frac{g}{2} \left( e^{i (\omega_0-\omega) t} \sigma_+ \otimes a + e^{i (\omega_0+\omega) t} \sigma_+ \otimes a^\dag + e^{i (-\omega_0-\omega) t} \sigma_- \otimes a + e^{i (-\omega_0+\omega) t} \sigma_- \otimes a^\dag \right) \\
        &\approx \frac{g}{2} \left( e^{i (\omega_0-\omega) t} \sigma_+ \otimes a + e^{i (-\omega_0+\omega) t} \sigma_- \otimes a^\dag \right).
\end{align*}
That last step used the rotating wave approximation (RWA).
\par
Now if we go back from the interaction picture to the lab frame,
\[ H = H_0 + \frac{g\hbar}{2} \left( \sigma_+ a + \sigma_- a^\dag \right).  \]
It makes sense that we were able to ignore the terms $\sigma_+ \otimes a^\dag$ and $\sigma_- \otimes a$, because those operators represent the atom becoming excited while the harmonic oscillator gains energy, and the atom becoming unexcited while the harmonic oscillator loses energy, respectively. The whole point of the $V$ term was to represent the atom and the oscillator exchanging energy, but since energy is conserved, neither of those operators are meaningful.
\par
This new Hamiltonian is very important, so we call it the Jaynes-Cummings Hamiltonian:
\[ \frac{H}{\hbar} = \frac{\omega_0}{2} \sigma_z \otimes I + I \otimes \omega \left( N + \frac{1}{2} I \right) + \frac{g}{2} \left( \sigma_+ \otimes a + \sigma_- \otimes a^\dag \right). \]
The eigenstates of this Hamiltonian are denoted $\ket{i,n}$, where $i \in \left\{ 0,1 \right\}$ and $n \in \N$. Note that the coupling term $V$ will only ``link" each state to at most one other state:
\[ \bra{i,m} V \ket{j, n} = \begin{cases}
    \frac{g}{2} \sqrt{n} & i=1, j=0, n=m+1 \\
    \frac{g}{2} \sqrt{m} & i=0, j=1, n=m-1 \\
    0 & \text{otherwise}
\end{cases} \]
Therefore the Hamiltonian can be written in the basis $ \left( \ket{1,0}, \ket{1,0}, \ket{1,1}, \ket{2,0},\dots \right)$ as a block-diagonal matrix, where each block is a symmetric $2 \times 2$ matrix. Note that I left $\ket{0,0}$ out of the basis, because the Hamiltonian is simpler that way -- if we wanted to include it, we would add another entry at the top left (with zeroes below and to the right of it).
\[ H = \begin{bmatrix}
    H^{0} & 0 & 0 & \cdots \\
    0 & H^{1} & 0 & \cdots \\
    0 & 0 & H^{2} & \cdots \\
    \vdots & \vdots & \vdots & \ddots \\
\end{bmatrix}, H^{(n)} = \begin{bmatrix}
    \frac{\omega_0}{2} + \omega \left( n + \frac{1}{2} \right) & \frac{g}{2} \sqrt{n+1} \\
    \frac{g}{2} \sqrt{n+1} & - \frac{\omega_0}{2} + \omega \left( n + \frac{3}{2} \right)
\end{bmatrix} \]
I need to double check all my formulas for these lecture notes, then get notes for the remainder of the lecture, where we solved for the eigenvectors and eigenvalues of each block matrix in order to find the new energy levels (as a function of the detuning $\delta$ and the coupling strength $g$). Also get notes for relating this to the Stark effect.

\section{Magnus expansion}
\section{Rotation operator on Bloch sphere}
\section{Interaction picture}
\section{Baker-Campbell-Hausdorff formula}
\section{Displacement operator}
\[ D(\alpha) = \exp \left( \alpha a^\dag -\alpha^* a \right)  \]
Note that the displacement operator takes the vacuum state to a coherent state, but it does not map every coherent state to another coherent state.
\par
Give definition of displacement operator and prove it's basic properties, like $D(\alpha)^\dag = D(-\alpha)$ and $D^\dag(\alpha)D(\alpha)=I$. Also talk about Kermack-McCrae identities, optical phase space, Wigner quasiprobability distribution, Husimi Q representation, and squeezed coherent states. For problem 3 on HW5, figure out explicit formula for probability as a function of $t_w$. For that, refer to \url{https://physics.stackexchange.com/questions/553225/representation-of-the-displacement-operator-in-number-basis}

\end{document}
