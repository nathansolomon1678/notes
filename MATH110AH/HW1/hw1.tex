\documentclass[12pt]{article}
\usepackage[margin=1in]{geometry}
\usepackage{amsmath}
\usepackage{amsfonts}
\usepackage{tikz-cd}

\begin{document}

\title{Math 110AH Homework 1}
\author{Nathan Solomon}
\maketitle

\textbf{Assignment due October 11th at 11:59 pm}

\bigskip
\noindent\fbox{\fbox{\parbox{6.5in}{
\textbf{1.} Let $f : X \rightarrow Y$ and $g : Y \rightarrow Z$ be two maps. Prove that if $f$ and $g$ are injective (resp. surjective), then so is the composition $g \circ f$.
}}}\bigskip

If $f$ and $g$ are both injective, then for any distinct elements $x_1, x_2 \in X, f(x_1) \neq f(x_2)$ because $f$ is injective. Since $g$ is also injective, $g(f(x_1)) \neq g(f(x_2))$, therefore $g \circ f$ is injective.
\par
If $f$ and $g$ are both surjective, then for any element $z \in Z$, there exists an element $y \in Y$ such that $g(y) = z$, and there exists an element $x \in X$ such that $f(x) = y$. Since $g(f(x)) = z$, $g \circ f$ is surjective.

\bigskip
\noindent\fbox{\fbox{\parbox{6.5in}{
\textbf{2.} Prove that $(1 + 2 + \cdots + n)^2 = 1^3 + 2^2 + \cdots + n^3$.
}}}\bigskip

First, I'll prove that $1 + 2 + \cdots + n = (n^2 + n) / 2$. This is obvious in the base case ($n=1$). If it's true for some positive integer $n$, then it must also be true for the $n+1$ case, because
\begin{align*}
    1 + 2 + \cdots + n + (n + 1) &= \frac{n^2 + n}{2} + (n + 1) \\
                                 &= \frac{n^2}{2} + \frac{n}{2} + \frac{1}{2}  \\
                                 &= \frac{(n + 1)^2 + (n + 1)}{2}
\end{align*}
By induction, this implies the statement ``$1 + 2 + \cdots + n = (n^2 + n) / 2$" is true for any positive integer $n$.
\par
The statement ``$(1 + 2 + \cdots + n)^2 = 1^3 + 2^2 + \cdots + n^3$" is also obviously true in the base case ($n=1$). If that statement is true for some positive integer $n$, it must also be true for $n+1$, because

\begin{align*}
    (1 + 2 + \cdots + n + (n + 1))^2 &= \left( \frac{n^2 + n}{2} + (n + 1) \right)^2 \\
                                     &= \left( \frac{n^2 + n}{2} \right)^2 + 2 \cdot (n + 1) \cdot \left( \frac{n^2 + n}{2} \right) + (n + 1)^2 \\
                                     &= \left( \frac{n^4 + 2n^3 + n^2}{4} \right) + (n^3 + 2n^2 + n) + (n^2 + 2n + 1) \\
                                     &= \frac{n^4}{4} + \frac{3n^3}{2} + \frac{13n^2}{4} + 3n + 1 \\
                                     &= \left( \frac{n^4}{4} + \frac{n^3}{2} + \frac{n^2}{4} \right) + (n^3 + 3n^2 + 3n + 1) \\
                                     &= \left( \frac{n^2 + n}{2} \right)^2 + (n + 1)^3 \\
                                     &= 1^3 + 2^3 + \cdots + n^3 + (n + 1)^3
\end{align*}
So by induction, the statement $(1 + 2 + \cdots + n)^2 = 1^3 + 2^2 + \cdots + n^3$ must be true for any positive integer $n$.
\par
Note that this whole proof works just as well if we choose $n=0$ to be the base case instead of $n=1$. Although the notation ``$1 + 2 + \cdots + n$" implies $n \geq 3$, the formula works for any $n \geq 0$.
\bigskip

\noindent\fbox{\fbox{\parbox{6.5in}{
\textbf{3.} Prove that 13 divides $14^n - 1$ for any $n \in \mathbb{N}$.
}}}\bigskip

This is true in the base case ($n=1$), because $14^1 - 1 = 13$. If that statement is true for a natural number $n$, then there exists an integer $z$ such that $13z = 14^n - 1$. Since $14^{n+1} - 1 = 14 \cdot 14^n - 1 = (13 \cdot 14^n) + (14^n - 1) = 13 \cdot (14^n + z)$, 13 must also divide $14^{n+1} - 1$. By induction, 13 divides $14^n - 1$ for any $n \in \mathbb{N}$.
\par
Just like with the last question, this still works if we consider $\mathbb{N}$ to include zero.

\bigskip
\noindent\fbox{\fbox{\parbox{6.5in}{
\textbf{4.} Show that if $a^n-1$ is prime and $n > 1$, then $a=2$ and $n$ is prime. If $2^n + 1$ is prime, what can you say about $n$?
}}}\bigskip

For this question I will use $[x]$ to mean the equivalence class of $x$ in $\mathbb{Z}/(a-1)\mathbb{Z}$.
\par
Note that $a$ cannot be zero or one, because if it were, $a^n-1$ wouldn't be prime for any $n$. Since all prime numbers are positive, $a^n > 0$. If $n$ is odd, that would not work when $a$ is negative, and if $n$ is even, $a^n = (-a)^n$, so we can assume without loss of generality that $a$ is positive.
\par
First, note that since $a = 1 + (a - 1)$, $a = [1]$, which implies $a^n = [1]$, or equivalently, $a^n-1=[0]$.
\par
$a^n-1$ is prime, but now it also has to be divisible by $a-1$. The only factors of a prime are $\pm$ itself and $\pm 1$, so
\[ a-1 \in \{ a^n-1, 1-a^n, 1, -1 \} \]
We already ruled out the possibility that $a \leq 1$, which rules out the first option. If $a-1=1-a^n$, then $a^n-1=1-a$ is prime, but $a$ is positive, so we can rule out the second option as well. The fourth option would imply $a=0$, which we also already showed is not true, so we're left with the third option.
\[ a=2 \]
\par
Suppose there exists positive integers $x, y$ such that $xy=n$. Then
\begin{align*}
    (2^x-1) \times (1 + 2^x + 2^{2x} + \cdots + 2{x(y-1)}) &= \\
    (2^x + 2^{2y} + 2^{3x} + \cdots + 2^{xy}) - (1 + 2^x + 2^{2x} + \cdots + 2^{x(y-1)}) &= \\
                                                           2^{xy} - 1 &= 2^n-1
\end{align*}
Therefore, if $n$ is composite, then $2^n-1$ has to be composite as well. Since that's not the case, $n$ must be prime.
\par
We can use a similar method to show that if $2^n+1$ is prime, then $n$ has to be a power of two. Suppose $n$ is not a power of 2 -- then there exist positive integers $a$ and $b$ such that $b$ is odd, $b>1$, and $n=b \times 2^a$. Let $x = 2^{(2^a)}$. Then
\begin{align*}
    (1+x) \times \left( 1 + (-x) + (-x)^2 + \cdots + (-x)^{b-1} \right) &= \\
    \left( 1 + (-x) + (-x)^2 + \cdots + (-x)^{b-1} \right) - \left( (-x) + (-x)^2 + \cdots + (-x)^b \right) &= \\
    1 - (-x)^b &= \\
    1 + x^b &= \\
    1 + \left( 2^{(2^a)} \right)^b &= 2^n + 1
\end{align*}
Therefore, if $n$ is not a power of two, then $2^n+1$ has to be composite. Since that's not the case, $n$ must be a power of two.

\bigskip
\noindent\fbox{\fbox{\parbox{6.5in}{
\textbf{5.} Find all integer solutions of $93x + 39y = -6$.
}}}\bigskip

Let $a=93, b=39, c=-6, d := (a,b) = 3, x_0 = -3, y_0=7$. Then using the results from question 6, the general solution is
\[ (x, y) \in \left\{ \left( -3 + 13k, 7 - 31k \right): k \in \mathbb{Z} \right\} = \{ \dots, (-16, 38), (-3, 7), (10, -24), \dots \} \]

\bigskip
\noindent\fbox{\fbox{\parbox{6.5in}{
            \textbf{6.} Let $a, b, c$ be non-zero integers and let $d = \operatorname{gcd}(a, b)$. Prove that the equation $ax + by = c$ has a solution $x, y$ in integers if and only if $d|c$. Moreover, if $d|c$ and $x_0, y_0$ is a solution in integers then the general solution in integers is $x = x_0 + \frac{b}{d} k, y = y_0 - \frac{a}{d} k$ for all integers $k$.
}}}\bigskip

Since $ax+by$ is a linear combination of $a$ and $b$, which are both divisible by $d$, $ax+by$ must also be divisible by $d$, which is not possible unless $d$ divides $c$.
\par
We proved in class that $a$ and $b$ are coprime if and only if $ax + by = 1$ has a solution. Since $a/d$ and $b/d$ are coprime, we can let $x'$ and $y'$ be integer solutions to $ax'/d+by'/d=1$. Then $x:=x'cd$ and $y:=y'cd$ are solutions to $ax+by=c$.
\par
We have now proven that $ax+by=c$ has at least one solution $x,y \in \mathbb{Z}^2$ if and only if $d$ divides $c$.
\par
Suppose $(x_0, y_0)$ and $(x, y)$ are both solutions (not necessarily distinct). Then the difference between $ax_0+by_0$ and $ax+by$ has to be zero, meaning that $a(x-x_0)=-b(y-y_0)$. Conversely, if $ax_0+by_0=c$ and $a(x-x_0)=-b(y-y_0)$ than it is obvious that $ax+by=c$. If we let $k=b(x-x_0)/d$, then substitute and rearrange, we get the following equations:
\begin{align*}
    x &= x_0 + \frac{bk}{d} \\
    y &= y_0- \frac{ak}{d} \\
\end{align*}
However, the only way $x$ and $y$ can both be integers is if $k$ is an integer, so $(x, y)$ is an integer solution to $ax + by = c$ if and only if there is exists an integer $k$ that the two equations above are true for some pair of integers $x_0, y_0$ which already solve $ax_0+by_0=c$.

\bigskip
\noindent\fbox{\fbox{\parbox{6.5in}{
            \textbf{7.} Show that if $a, b \in \mathbb{N}$, $ab$ is the square of an integer, and $(a, b) = 1$, then $a$ and $b$ are squares.
}}}\bigskip

Let $p$ be any prime number that divides $a$, and let $d := p^n$ be the highest power of $p$ that divides $a$. Then $d$ is also the highest power of $p$ that divides $ab$, because if it weren't, $b$ would divide $p$, so the GCD of $a$ and $b$ would be at least $p$.
\par
Let $p^{n'}$ be the highest power of $p$ that divides $\sqrt{ab}$. Since $\left( p^{n'} \right)^2 = p^{2n'} = p^{n}$, we know that $n$ must be an even number.
\par
Let $a_1^{n_1} a_2^{n_2} \dots a_m^{n_m}$ be the prime factorization of $a$, where $a_1 < a_2 < \cdots < a_m$. Repeating the above process for $p=a_1, a_2, \dots, a_m$ will show that all of the exponents ($n_1, n_2, \dots, n_m$) are even.
\par
Let $\sqrt{a} := a_1^{n_1/2} a_2^{n_2/2} \dots a_m^{n_m/2}$. Then $\sqrt{a}$ is an integer and $a = \sqrt{a}^2$, so $a$ is a square.
\par
Repeating the entire process above but with $a$ replaced by $b$ shows that $b$ is also a square.

\bigskip
\noindent\fbox{\fbox{\parbox{6.5in}{
            \textbf{8.} Prove that if $(a, n) = 1$ and $(b, n) = 1$, then $(ab, n) = 1$.
}}}\bigskip

Suppose there is an integer $d > 1$ which divides both $ab$ and $n$. Then since $d$ divides $ab$, it must divide $a$ or $b$. That means $(ab, n) > 1$ (or equivalently, $(ab,n) \neq 1$, since the GCD is always a positive integer) implies that $(a,n) \neq 1$ or $(b,n) \neq 1$. Conversely, if $(a,n)=1$ and $(b,n)=1$, then $(ab,n)=1$.

\bigskip
\noindent\fbox{\fbox{\parbox{6.5in}{
            \textbf{9.} Is $2^{10} + 5^{12}$ a prime? (Hint: use the identity $4x^4 + y^4 = (2x^2 + y^2)^2 - (2xy)^2$.)
}}}\bigskip

Another way to see that $2^{10} + 5^{12}$ is not prime is to let $x=4$ and let $y=5^3$. Then
\begin{align*}
    2^{10} + 5^{12} &= 4x^4 + y^4 \\
                    &= (2x^2 + y^2)^2 - (2xy)^2 \\
                    &= (2x^2 + y^2 - 2xy) \cdot (2x^2 + y^2 + 2xy) \\
                    &= (32 + 15625 - 1000) \cdot (32 + 15625 + 1000) \\
                    &= 14657 \cdot 16657
\end{align*}
which is actually the prime factorization of $2^{10} + 5^{12}$.
\par
Question for the grader: If I had answered with just ``No, because $2^{10} + 5^{12} = 244141649 = 14657 \cdot 16657$", would I still get full points?

\bigskip
\noindent\fbox{\fbox{\parbox{6.5in}{
            \textbf{10.} Show that there are infinitely many primes $p \equiv 2 \pmod{3}$. (Hint: consider $3 p_1 p_2 \dots p_n - 1$.)
}}}\bigskip

For this question I will use $[n]$ to mean the equivalence class of $n$ in $\mathbb{Z}/3\mathbb{Z}$, and $\mathbb{P}$ to mean the set of all prime numbers.
\par
Suppose $P = \{ p_1, p_2, \dots, p_n \} = \mathbb{P} \cap [2]$ is a finite set of all the primes that are congruent to 2 (modulo 3). Then let $N = 3 p_1 p_2 \cdots p_n - 1$. For any $p_i \in P$, we know that $p_i$ and $N$ are coprime, because $p_i$ is greater than one and $N$ is one less than an integer multiple of $p_i$. Therefore $N$ is coprime to every element of $P$.
\par
Now consider the prime factorization of $N$. Every prime number in $[2]$ is in $P$, and $N$ is not divisible by any element of $P$. Therefore $N$ is the product of elements of $[0]$ and $[1]$, that is, there exists nonnegative integers $a$ and $b$ such that $[0]^a \times [1]^b = [2]$.
\par
However, $[0] \times [0] = [0]$, $[1] \times [1] = [1]$, and $[0] \times [1] = [0]$. We have reached a contradiction, so there must be infinitely many primes in $\mathbb{P} \cap [2]$.

\end{document}
