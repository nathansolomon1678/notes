\documentclass[12pt]{article}
\usepackage[margin=1in]{geometry}
\usepackage{amsmath}
\usepackage{amsfonts}
\usepackage{amsthm}
\newtheorem{thm}{Theorem}[section]
\newtheorem{cor}[thm]{Corollary}
\newtheorem{lem}[thm]{Lemma}
\usepackage{tikz-cd}
\renewcommand{\d}{\mathrm{d}}

\begin{document}

\title{Math 110AH Homework 9}
\author{Nathan Solomon}
\maketitle

\textbf{Assignment due December 6th at 11:59 pm.}

\section{}
\noindent\fbox{\fbox{\parbox{6.5in}{
            Let $H \subset G = \mathbb{Z} \times \mathbb{Z}$ be the cyclic group generated by $(2, 4)$. Is the quotient group $G/H$ isomorphic to $\mathbb{Z}$? (Hint: Consider elements of finite order of $G/H$.)
}}}\bigskip

No. Let $x = (1,2)$ and consider the coset $x+H$ in $G/H$. Since $x$ is not in $H$, $xH$ is not the identity in $G/H$. However, $2x$ is in $H$, so $2x+H$ is the identity in $G/H$. Therefore the coset $x+H$ has order 2 in $G/H$. We know that $ \mathbb{Z} $ has no elements of finite order, so it is not isomorphic to $G/H$.

\section{}
\noindent\fbox{\fbox{\parbox{6.5in}{
            Determine all subgroups of the alternating group $A_4$.
}}}\bigskip

By going through all elements methodically and seeing what subgroups they generate, we get the following list of subgroups:
\bigskip
\par
{\centering
\begin{tabular}{|c|c|}
    \hline
    Order & Subgroup of $A_4$ \\
    \hline
    \hline
    1 & $\{e\}$ \\
    \hline
    \hline
    2 & $\langle (12)(34) \rangle$ \\
    \hline
    2 & $\langle (13)(24) \rangle$ \\
    \hline
    2 & $\langle (14)(23) \rangle$ \\
    \hline
    \hline
    3 & $\langle (123) \rangle$ \\
    \hline
    3 & $\langle (124) \rangle$ \\
    \hline
    3 & $\langle (134) \rangle$ \\
    \hline
    3 & $\langle (234) \rangle$ \\
    \hline
    \hline
    4 & $\{ e, (12)(34), (13)(24), (14)(23) \}$ \\
    \hline
    \hline
    12 & $A_4$ \\
    \hline
\end{tabular}\par}

\section{}
\noindent\fbox{\fbox{\parbox{6.5in}{
            Let $g \in S_n$ be an odd element.
            \begin{itemize}
                \item (a) Show that the map $f_n: A_n \rightarrow A_n$ given by $f_n(x) = gxg^{-1}$, is an automorphism of $A_n$.
                \item (b) Prove that the automorphism $f_n$ of $A_n$ is not inner for $n \geq 3$.
            \end{itemize}
}}}\bigskip

In order to define the alternating group, we used the sign operator $\operatorname{sgn}$ from the symmetric group to the multiplicative group $\{-1, 1\}$. The sign operator is uniquely defined (and well defined) by the fact that it is a homomorphism and the fact that it maps every transposition to -1.
\begin{itemize}
    \item (a) We already know that conjugation by $g$ is an automorphism on $S_n$, so we just need to show that it maps elements of $A_n$ to elements of $A_n$. Using the fact that $\operatorname{sgn}$ is a homomorphism and the fact that $g$ is odd,
        \[ \operatorname{sgn}(f_n(x)) = \operatorname{sgn}(g) \operatorname{sgn}(x) \operatorname{sgn} (g^{-1}) = (-1) \operatorname{sgn}(x) (-1) = \operatorname{sgn}(x). \]
        This shows that if $x$ is an even permutation, then $f_n(x)$ is too, so $f_n$ is an automorphism on $A_n$.
    \item (b)
\end{itemize}

\section{}
\noindent\fbox{\fbox{\parbox{6.5in}{
            Show that the group $ \mathbb{Q} / \mathbb{Z} $ cannot be generated by a finite set of elements.
}}}\bigskip

Suppose $ \mathbb{Q} / \mathbb{Z} $ is generated by a finite set of elements. Call those elements $a_1, a_2, \dots a_n$. Each $a_i$ is rational, so it can be written as a fraction with denominator $b_i \in \mathbb{N}$. Then $b_i \cdot a_i$ is an integer, so every $a_i$ has finite order.
\par
For convenience, assume $b_i$ is the smallest natural number such that $b_i \cdot a_i$ is an integer. In other words, the fractions have been reduced so that the numerator and denominator are coprime, and the denominator $b_i$ is the order of $a_i$ in $ \mathbb{Q} / \mathbb{Z} $.
\par
Since $ \mathbb{Q} / \mathbb{Z} $ is an abelian group generated by $a_1, a_2 \dots a_n$, every element can be written as $a_1^{c_1} a_2^{c_2} \cdots a_n^{c_n}$ for some set of integers $c_1, c_2, \dots, c_n$. But because every $a_i$ has finite order $b_i$, we can write that element in a unique way, by assuming $0 \leq c_i < b_i$. Writing each element this way makes it clear that there are only a finite number of elements in $ \mathbb{Q} / \mathbb{Z} $.
\par
However, $ \mathbb{Q} / \mathbb{Z} $ has infinitely many elements, so this is a contradiction. Therefore $ \mathbb{Q} / \mathbb{Z} $ cannot be generated by finitely many elements.

\section{}
\noindent\fbox{\fbox{\parbox{6.5in}{
            Let $G$ be an (additively written) abelian group. An element $a \in G$ is called \textit{torsion} if $na = 0$ for some integer $n > 0$.
            \begin{itemize}
                \item (a) Prove that the set $G_{tors}$ of all torsion elements in $G$ is a subgroup of $G$.
                \item (b) Determine $( \mathbb{R} / \mathbb{Z} )_{tors}$.
                \item (c) Determine $( \mathbb{Q}^\times )_{tors}$.
            \end{itemize}
}}}\bigskip

\begin{itemize}
    \item (a) For any elements $a, b \in G_{tors}$, let $n_a$ and $n_b$ be positive integers such that $n_a a = 0 = n_b b$. Then $\operatorname{lcm}(n_a n_b) (a+b) = 0$, so $a+b$ is in the torsion group. If $n_a=1$ then $a=0$, so $-a=0$ is also in the torsion group. Otherwise, $n_a (n_a - 1) (-a) = n_a a = 0$, so $-a$ is in the torsion group. Because $G_{tors}$ is closed under multiplication and inversion, it is a subgroup of $G$.
    \item (b) If $x$ is a rational number in that group, let $a, b$ be integers such that $b > 0$ and $x=a/b$. Then $bx=0$, so $x$ is in the torsion group. If $x$ is an irrational number, there is integer $n$ such that $nx$ is an integer, and therefore $nx \neq 0$, so $x$ is not in the torsion group. Therefore
        \[ ( \mathbb{R} / \mathbb{Z})_{tors}  = ( \mathbb{Q} / \mathbb{Z}) \]
    \item (c) Suppose $a/b$ is an element of that group, where $0 \leq a < b$ and $b > 0$. Then there exists a natural number $n$ such that $a^n/b^n$ is an integer. This is only possible if $b^n$ divides $a^n$, meaning every number appears in the prime factorization of $a^n$ at least as many times as it appears in the prime factorization of $b^n$. Of course, that can only be true if every number appears in the prime factorization of $a$ at least as many times as it appears in the prime factorization of $b$. This is equivalent to saying $b$ divides $a$, so $a/b=\text{an integer}=0$. Therefore $( \mathbb{Q}^\times)_{tors}$ is the trivial subgroup.
\end{itemize}

\section{}
\noindent\fbox{\fbox{\parbox{6.5in}{
            Prove that $A_n$ is generated by all $n$-cycles if $n$ is odd.
}}}\bigskip

\section{}
\noindent\fbox{\fbox{\parbox{6.5in}{
            Describe all conjugacy classes in $A_4$ and $S_4$.
}}}\bigskip

\section{}
\noindent\fbox{\fbox{\parbox{6.5in}{
            A \textit{commutator} of $G$ is an element of the form $xyx^{-1}y^{-1}$ where $x,y \in G$. Let $G'$ be the subgroup of $G$ generated by all commutators. We call $G'$ the \textit{commutator subgroup} of $G$. Show all the following are true.
            \begin{itemize}
                \item (a) $G'$ is normal in $G$.
                \item (b) $G / G'$ is abelian.
                \item (c) If $N$ is a normal subgroup of $G$ and $G/N$ is abelian then $G' \subset N$.
            \end{itemize}
}}}\bigskip

\section{}
\noindent\fbox{\fbox{\parbox{6.5in}{
            A group $G$ is called \textit{perfect} if the commutator subgroup $G'$ coincides with $G$. Find all $n$ such that the alternating group $A_n$ is perfect.
}}}\bigskip

\section{}
\noindent\fbox{\fbox{\parbox{6.5in}{
            Let $N$ be a normal subgroup of $G$ and let $K$ be a subgroup of $G$ such that the restriction $K \rightarrow G/N$ of the canonical homomorphism $G \rightarrow G/N$ is an isomorphism. Prove that $G$ is a semidirect product of $N$ and $K$.
}}}\bigskip

\end{document}
