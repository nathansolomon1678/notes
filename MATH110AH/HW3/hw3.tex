\documentclass[12pt]{article}
\usepackage[margin=1in]{geometry}
\usepackage{amsmath}
\usepackage{amsfonts}
\usepackage{tikz-cd}

\begin{document}

\title{Math 110AH Homework 3}
\author{Nathan Solomon}
\maketitle

\textbf{Assignment due October 25th at 11:59 pm}

\bigskip
\noindent\fbox{\fbox{\parbox{6.5in}{
            \textbf{1.} Prove that for an element $a$ of a group, $a^n \cdot a^m = a^{n+m}$ and $(a^{-1})^n = (a^n)^{-1}$ for every $n,m \in \mathbb{Z}$.
}}}\bigskip

By definition,
\[ a^n = \begin{cases}
    \text{$a$ multiplied by itself $n$ times if $n > 0$} \\
    \text{the identity element if $n = 0$} \\
    \text{$a^{-1}$ multiplied by itself $-n$ times if $n < 0$}
\end{cases} \]

If $n=0$ then both parts of this question are obvious, and if $m=0$, the first part is obvious. Therefore we only need to consider the cases where both $n$ and $m$ are either positive or negative.

\begin{itemize}
    \item If $n$ and $m$ are both positive, then $a^n \cdot a^m$ is $a$ multiplied by itself $n+m$ times. If $n$ and $m$ are both negative, then $a^n \cdot a^m$ is $a^{-1}$ multiplied by itself $-n-m$ times, so we get $a^n \cdot a^m = a^{n+m}$ in this case too. If one of $(n, m)$ is positive but the other is negative, assume without loss of generality that $n$ is positive.
        \par
        If $n < -m$ then $a^n \cdot a^m = a^n \cdot (a^{-1})^{-m} = (a^{-1})^{-m-n} = a^{n+m}$. If $n > -m$ then $a^n \cdot a^m = a^n \cdot (a^{-1})^{-m} = a^{n-(-m)} = a^{n+m}$.
        \par
        We have proven that in all cases, $a^n \cdot a^m = a^{n+m}$.
    \item If $n$ is positive, $(a^{-1})^n \cdot a^n$ is $a^{-1}$ multiplied by itself $a$ times, times $a$ multiplied by itself $n$ times, which is clearly 1. If $n$ is negative, it's 1 again, for the exact same reason (except we use the property that $(a^{-1})^{-1} = a$). In either case, we get that $a^n$ is the inverse of $(a^{-1})^n$.
\end{itemize}

\bigskip
\noindent\fbox{\fbox{\parbox{6.5in}{
            \textbf{2.} Show that $((ab)c)d = a(b(cd))$ for all elements $a,b,c,d$ of a group.
}}}\bigskip

Repeatedly applying the associative rule, we get
\begin{align*}
    ((ab)c)d &= (ab)(cd) \\
             &= a(b(cd)).
\end{align*}
In the first line, we use the rule $(xy)z=x(yz)$ where $x=ab, y=c, z=d$, and to get to the second line, we use the same rule, except with $x=a, y=b, z=cd$.

\bigskip
\noindent\fbox{\fbox{\parbox{6.5in}{
            \textbf{3.} Show that if $G$ is a group in which $(ab)^2 = a^2b^2$ for all $a,b, \in G$, then $G$ is abelian.
}}}\bigskip

For any two elements $a,b \in G$, the product $ab$ can be rewritten as
\[ ab = a^{-1} aabb b^{-1}. \]
But if we know that $(ab)^2 = a^2b^2$, then that's equivalent to
\begin{align*}
    ab &= a^{-1} a^2 b^2 b^{-1} \\
       &= a^{-1} (ab)^2 b^{-1} \\
       &= a^{-1} abab b^{-1} \\
       &= ba.
\end{align*}
We have proven that $ab=ba$ for any elements $a,b \in G$, so $G$ is abelian.

\bigskip
\noindent\fbox{\fbox{\parbox{6.5in}{
            \textbf{4.} Find all elements of order 3 in $\mathbb{Z}/18\mathbb{Z}$.
}}}\bigskip

Suppose $x$ is an element that satisfies that property. Then $3x = 18m$ for some integer $m$. That's equivalent to $x=6m$, so $x \in \{ \dots, -6, 0, 6, 12, 18, \dots \}$. But in $\mathbb{Z}/18\mathbb{Z}$, that's equivalent to $\{ 0, 6, 12 \}$. Now there are only 3 possible solutions, so we check them manually and see that the only elements of order 3 in $\mathbb{Z}/18\mathbb{Z}$ are 6 and 12.

\bigskip
\noindent\fbox{\fbox{\parbox{6.5in}{
            \textbf{5.} Prove that the composite of two homomorphisms (resp. isomorphisms) is also a homomorphism (resp. isomorphism).
}}}\bigskip

Suppose $f$ and $g$ are both group homomorphisms, and the domain of $f$ is the codomain of $g$. Then for any $a, b$ in the domain of $g$,
\[ f(g(a)) \times f(g(b)) = f(g(a) \times g(b)) = f(g(a \times b)) \]
so $f \circ g$ is a group homomorphism.
\par
Suppose $f$ and $g$ are isomorphisms. Then in addition to being group homomorphisms (which implies $f \circ g$ is a group homomorphism), they are also both injective and surjective. According to a result from an earlier homework, that implies their composition is also injective and surjective. Since $f \circ g$ is a bijective group homomorphism, it is also an isomorphism.

\bigskip
\noindent\fbox{\fbox{\parbox{6.5in}{
            \textbf{6.} Prove that the group $(\mathbb{Z}/9\mathbb{Z})^\times$ is isomorphic to $\mathbb{Z}/6\mathbb{Z}$.
}}}\bigskip

\textit{Alternate proof: $(\mathbb{Z}/9\mathbb{Z})^\times$ is a group with 6 elements and which has a generator (either 2 or 5), and $\mathbb{Z}/6\mathbb{Z}$ is a group of 6 elements which has a generator (either 1 or 5). This implies they are both isomorphic to the cylcic group of order 6 ($C_6$), and therefore isomorphic to each other. If this proof is rigorous enough for you, no need to read the rest of my answer.}
\bigskip
\par
The group $(\mathbb{Z}/n\mathbb{Z})^\times$ is defined as the multiplicative group of integers in $[0,n-1]$ which are coprime to $n$, so
\[ \operatorname{Forget}((\mathbb{Z}/9\mathbb{Z})^\times) = \{ 1,2,4,5,7,8 \}. \]
Now consider the function $f: \mathbb{Z}/6\mathbb{Z} \rightarrow (\mathbb{Z}/9\mathbb{Z})^\times$, defined as
\[ f(x) = 2^x \hspace{1cm} \forall x \in \mathbb{Z}/6\mathbb{Z}. \]
By the properties of exponentials,
\[ f(a) \cdot f(b) = 2^a \cdot 2^b = 2^{a+b} = f(a + b), \]
so $f$ is a homomorphism. Also, from the table below, we clearly see $f$ is bijective:
\begin{table}[h]
    \centering
    \begin{tabular}{|c||c|c|c|c|c|c|}
        \hline
        x & 0 & 1 & 2 & 3 & 4 & 5 \\
        \hline
        f(x) & 1 & 2 & 4 & 8 & 7 & 5 \\
        \hline
    \end{tabular}
\end{table}
\par
Therefore $f$ is an isomorphism from $\mathbb{Z}/6\mathbb{Z}$ to $(\mathbb{Z}/9\mathbb{Z})^\times$.

\bigskip
\noindent\fbox{\fbox{\parbox{6.5in}{
            \textbf{7.} Let $G$ be an abelian group an let $a,b \in G$ have finite order $n$ and $m$ respectively. Suppose that $n$ and $m$ are relatively prime. Show that $ab$ has order $nm$.
}}}\bigskip

\textit{Proof outline: First, I'll show that the order of $ab$ is at most $nm$. Then if the order of $ab$ is less than $nm$, I'll consider the case where the order is divisible by $n$, divisible by $m$, or divisible by both, and show that all of those cases lead to a contradiction.}
\bigskip
\par
Since $G$ is abelian,
\[ (ab)^{nm} = a^{nm} b^{nm} = (a^n)^m (b^m)^n = 1^m 1^n = 1. \]
Let $x$ be the order of $ab$. Since $x$ is the smallest positive integer such that $(ab)^x=1$, and $nm$ is a positive integer, $x$ cannot be larger than $nm$.
\par
Since $n$ and $m$ are coprime, their greatest common divisor is $nm$. That means that if $x$ is a positive integer and $x < nm$, then either $n$ or $m$ will not divide $x$. Without loss of generality, we can suppose $x$ is not divisible by $n$ (and $m$ may or may not divide $x$; we still need to check both cases).
\begin{itemize}
    \item If $m$ divides $x$, then $(ab)^x = a^x b^x = a^x \neq 1$, which contradicts our earlier statement that $n$ does not divide $x$.
    \item If $x$ isn't divisible by $n$ or by $m$, then let $c = a^x$. Since $(ab)^x = a^x b^x = 1$, $b^x$ must be equal to $c^{-1}$. We have shown that the subgroup generated by $a$ and the subgroup generated by $b$ both contain $c$. That means $c^n = 1$ and $c^m = 1$. Since $n$ and $m$ are coprime (and positive), that can only be true if $c$ is the identity.
        \par
        However, $c$ was defined as $a^x$, and the subgroup generated by $a$ has order $n$, and $n$ does not divide $x$, so $c$ cannot be the identity. This is also a contradiction.
\end{itemize}
We have shown that the order $x$ of $ab$ cannot be larger than $nm$, but also that if $x < nm$, then we get a contradiction in all cases. Therefore the order of $ab$ is $nm$.

\bigskip
\noindent\fbox{\fbox{\parbox{6.5in}{
            \textbf{8.} a) Prove that for every positive integer $n$ the set of all complex $n$-th roots of unity is a cyclic group of order $n$ with respect to complex multiplication.
            \par b) Prove that if $G$ is a cyclic group of order $n$ and $k$ divides $n$, then $G$ has exactly one subgroup of order $k$.
}}}\bigskip

\begin{itemize}
    \item (a) A number $z \in \mathbb{Z}$ is an $n^{th}$ root of unity if and only if it satisfies $z^n = 1$. By breaking $z$ into polar form (that is, $|z| \times \frac{z}{|z|}$), we see that $z$ must have magnitude 1 and an argument which, when multiplied by $n$, gives an integer multiple of $2 \pi$. In other words, the $n^{th}$ roots of unity are
        \[ \{ \exp{(2 \pi i k / n)} : k \in \mathbb{Z} \}. \]
        But by the division theorem, there exist integers $a,b$ such that $k=an+b$ and $0 \leq b < n$. Since $\exp{(2 \pi i k)} = \exp{(2 \pi i a n)} \exp{(2 \pi i b / n)}$, that set is equivalent to
        \[ \left\{ \exp{\left( \frac{2 \pi i k}{n} \right) } : k \in \{0, 1, 2, \dots, n-1 \} \right\}. \]
        That set contains $n$ distinct elements, one of which is the multiplicative identity. Also, every element has a unique multiplicative inverse (which is its complex conjugate) and multiplication is associative. One can easily prove that all those properties hold, and that it's closed under multiplication, so it's a group. Specifically, it's the cyclic group of order $n$, because
        \[ \exp{(2 \pi i / n)} \]
        is a generator.
    \item (b) Let $g$ be a generator of $G$ and let $H$ be the set of all elements $a \in G$ such that $a^k=1$. Then for each of those elements, there exists some $j$ such that $g^j = a$, which implies $g^{jk} = 1$. But since $g$ generates the whole group $G$, which has order $n$, $jk$ must be an integer multiple of $n$, and so $j$ is an integer multiple of $n/k$. Therefore $a$ is an element of
        \[ H = \{ 1, g^{n/k}, g^{2n/k}, \dots, g^{(k-1)n/k} \} \]
        which is a subgroup of order $k$. But if there's any other subgroup $H'$ of $G$ which also has order $k$, then every element $a'$ of that subgroup would also satisfy $(a')^k=1$. According to the logic above, that would imply $a$ is in $H$, so we can conclude that $H$ is the only subgroup of order $k$.

\end{itemize}

\bigskip
\noindent\fbox{\fbox{\parbox{6.5in}{
            \textbf{9.} Prove that if $G$ is a finite group of even order, then $G$ contains an element of order 2. (Hint: Consider the set of pairs $(a, a^{-1})$.)
}}}\bigskip

Consider the set of ordered pairs $(a, a^{-1})$ for every element $a \in G$. Since there is one unique such pair for each element $a \in G$, there are an even number of those pairs.
\par
Additionally, there are an even number of those pairs for which $a \neq a^{-1}$, and since an even number minus an even number is an even number, there must also be an even number of those pairs for which $a = a^{-1}$. We already know that there is one pair which satisfies that property: $(e, e)$. Therefore there must be at least one such pair in addition to $(e, e)$.
\par
Call the first number of that other pair $a$. Then $a$ is not the identity, but it does satisfy $a = a^{-1}$ (which implies $a^2 = 1$), so it has order 2.

\bigskip
\noindent\fbox{\fbox{\parbox{6.5in}{
            \textbf{10.} Find the order of $GL_n(\mathbb{Z}/p\mathbb{Z})$ for a prime integer $p$.
}}}\bigskip

First, note that $\mathbb{Z}/p\mathbb{Z}$ is a ring with $p$ elements.

The $n^{th}$ general linear group over a ring is the set of $n$ by $n$ matrices over that ring for which all the columns are linearly independent. Any such matrix can be constructed by the following process: choose the first (leftmost) column to be any nonzero module (over that ring) with $n$ elements, then choose each column after that to be any module (over that ring) with $n$ elements that is linearly independent from all the other columns which have been chosen.
\par
The first column can be anything except the zero vector, so there are $p^n - 1$ options. The $j^{th}$ column can be anything outside the span of the first $j-1$ columns. That span must have dimension $j-1$, meaning it contains $p^{j-1}$ distinct elements, so there are $p^n - p^{j-1}$ options for the $j^{th}$ column.
\par
Therefore when choosing elements for the entire $n$ by $n$ matrix, there are
\[ \left[ p^n - p^0 \right] \times \left[ p^n - p^1 \right] \times \left[ p^n - p^2 \right] \times \cdots \times \left[ p^n - p^{n-1} \right] \]
distinct options. That expression can't really be simplified, so we conclude that the order of $GL_n(\mathbb{Z}/p\mathbb{Z})$ is
\[ \prod_{i=0}^{n-1} [ p^n - p^i ]. \]

\end{document}
