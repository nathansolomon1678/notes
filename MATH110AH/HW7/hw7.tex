\documentclass[12pt]{article}
\usepackage[margin=1in]{geometry}
\usepackage{amsmath}
\usepackage{amsfonts}
\usepackage{amsthm}
\newtheorem{thm}{Theorem}[section]
\newtheorem{cor}[thm]{Corollary}
\newtheorem{lem}[thm]{Lemma}
\usepackage{tikz-cd}
\renewcommand{\d}{\mathrm{d}}

\begin{document}

\title{Math 110AH Homework 7}
\author{Nathan Solomon}
\maketitle

\textbf{Assignment due November 22nd at 11:59 pm.}
\par
\textbf{Problems 1, 3, 5, 7, and 8 are graded.}

\section{}
\noindent\fbox{\fbox{\parbox{6.5in}{
            Write out the disjoint cycle decomposition of $\sigma = (1234)(456)(145)$. Determine the order of $\sigma$.
}}}\bigskip

\[ \sigma = (152346) \]
Since $\sigma$ is a 6-cycle, it has order 6.

\section{}
\noindent\fbox{\fbox{\parbox{6.5in}{
            Prove that $S_7$ contains a cyclic subgroup of order 10.
}}}\bigskip

\begin{lem}\label{lcm_cycles}
    The order of a permutation is equal to the LCM of the cycle type.
\end{lem}
\begin{proof}
    For any cycle $a$, write out the disjoint cycle decomposition, and call those disjoint cycles $a_1, a_2, \dots, a_n$. For each disjoint cycle $a_i$, the order of $a_i$ is equal to the length of the cycle, so the cycle type of $a$ is the multiset of the orders of each $a_i$.
    \par
    We know that disjoint cycles commute, so for any natural number $x$,
    \[ a^x = (a_1 a_2 \cdots a_n)^x = a_1^x a_2^x \cdots a_n^x. \]
    If $a_i$ and $a_j$ are disjoint, then $a_i^x$ and $a_j^x$ are also disjoint, which implies $a^x$ is the identity if and only if for every index $i$, $a_i^x$ is the identity. Therefore the order of $a$ is the least common multiple of the orders of each $a_i$, or equivalently, the LCM of the cycle type.
\end{proof}

Define $\sigma \in S_7$ as $\sigma = (12)(34567)$. According to lemma~\ref{lcm_cycles}, $\sigma$ has order 10, so $\langle \sigma \rangle$ is a cyclic subgroup with order 10.

\section{}
\noindent\fbox{\fbox{\parbox{6.5in}{
            Find the largest order of an element in $S_5$.
}}}\bigskip

For any element $x \in S_5$, the sum of the cycle type is at most five (by ``cycle type", I mean the multiset consisting of the length of all the disjoint cycles in $x$, ignoring 1-cycles). Therefore there are only a few possibilities for the cycle type, and using lemma~\ref{lcm_cycles}, we can calculate the order of $x$ in each of those cases.

\bigskip
{\centering
\begin{tabular}{|c|c|}
    \hline
    Cycle type & order \\
    \hline
    \hline
    $\{\}$ & 1 \\
    $\{2\}$ & 2 \\
    $\{3\}$ & 3 \\
    $\{4\}$ & 4 \\
    $\{5\}$ & 5 \\
    $\{2,2\}$ & 2 \\
    $\{2,3\}$ & 6 \\
    \hline
\end{tabular}\par}
\bigskip

From this table, we see that the order of $x$ can be 6, but cannot be larger than 6.
\par
\textit{Note: you can generalize this from $S_5$ to $S_n$ by using the Landau function, which is given by sequence A000793 in the OEIS.}

\section{}
\noindent\fbox{\fbox{\parbox{6.5in}{
            Find all elements in $S_5$ that commute with the cycle $(123)$.
}}}\bigskip

\begin{lem}\label{conjugation}
    If $x$ is a cycle of the form $(1,2,\dots,n)$ and $y$ is another permutation, then $yxy^{-1} = (y(1),y(2),\dots,y(n))$. 
\end{lem}
\begin{proof}
    We proved this in class.
\end{proof}

The cycle $(123)$ commutes with $y$ if and only if $y(123)y^{-1}=(123)$, which, according to lemma~\ref{conjugation}, is true if and only if one of the following is true:
\begin{itemize}
    \item $y(1)=1$ and $y(2)=2$, and $y(3)=3$
    \item $y(1)=2$ and $y(2)=3$, and $y(3)=1$
    \item $y(1)=3$ and $y(2)=1$, and $y(3)=2$
\end{itemize}
One of those conditions will be satisfied if and only if
\[ y \in \{ e, (123), (321), (45), (123)(45), (321)(45) \}. \]

\section{}
\noindent\fbox{\fbox{\parbox{6.5in}{
            How many conjugacy classes are there in $S_5$?
}}}\bigskip

\begin{lem}\label{conj_classes}
    For any permutation $x \in S_n$, the conjugacy class of $x$ is the set of all permutations $y \in S_n$ which have the same cycle type as $x$.
\end{lem}
\begin{proof}
    By applying lemma~\ref{conjugation} to the disjoint cycles of $x$, we can easily see that the conjugate of $x$ by any permutation will have the same cycle type as $x$.
    \par
    If $x$ and $y$ have the same cycle type, write their disjoint cycle decompositions in order from shortest cycle to longest. Then the written expressions match up perfectly and you can easily use lemma~\ref{conjugation} to defined a permutation $p$ such that $pxp^{-1}=y$.
\end{proof}

According to the table in problem 3, there are 7 possibilities for the cycle type of an element in $S_5$, so by lemma~\ref{conj_classes}, $S_5$ has 7 conjugacy classes.

\section{}
\noindent\fbox{\fbox{\parbox{6.5in}{
            \begin{itemize}
                \item (a) Prove that $S_n$ is generated by $(1,2),(1,3),\dots, (1,n)$. (Hint: Use $(1,j)(1,i)(1,j)=(i,j)$.)
                \item (b) Prove that $S_n$ is generated by $(1,2), (2,3), \dots, (n-1,n)$.
                \item (c) Prove that $S_n$ is generated by the two cycles $(1,2)$ and $(1,2,\dots,n)$. (Hint: Use $(1,2,\dots,n)(i-1,i)(1,2,\dots,n)^{-1} = (i, i+1)$.)
            \end{itemize}
}}}\bigskip

\begin{lem}\label{transpositions_generate_s}
    The symmetric group $S_n$ is generated by the set of transpositions $(i,j)$.
\end{lem}
\begin{proof}
    We know that any permutation can be written as a product of disjoint cycles, so we just need to show that every disjoint cycle is a product of transpositions. For any disjoint cycle of the form $(1,2,\dots,n)$, we can rewrite it as
    \[ (1,2,\dots,n) = (2,3) (3,4) \cdots (n-1,n) (n,1). \]
    So by expanding a permutation into a product of disjoint cycles and then expanding every cycle into a product of transpositions, we see that transpositions generate the symmetric group.
\end{proof}

\begin{itemize}
    \item (a) Let $x \in S_n$ be any permutation on $n$ elements. Then by lemma~\ref{transpositions_generate_s}, we can rewrite $x$ as a product of transpositions. Each transposition has the form $(i,j)$, which is equal to $(1,j)(1,i)(1,j)$. Therefore $x$ can be expanded as a product of transpositions which all have the form $(1,i)$ (for some index $i$).
    \item (b) Using the result of part (a), any permutation $x \in S_n$ can be expanded as a product of transpositions of the form $(1,i)$. Then each of those can be rewritten as
        \[ (1,i) = (1,2)(2,3)(3,4)\cdots(i-2,i-1)(i-1,i)(i-2,i-1)(i-3,i-2)\cdots(2,3)(1,2). \]
        Therefore $x$ can be expanded as a product of adjacent swaps (that is, transpositions of the form $(i-1,i)$.
    \item (c) Using the result from part (b), any permutation $x \in S_n$ can be written as a product of ``adjacent swaps" $(i, i+1)$. Each of those can be written as
        \[ (1,2,\dots,n)^{i-1}(1,2)(1,2,\dots,n)^{1-i}. \]
        Therefore $x$ can be written as a product of $(1,2)$ and $(1,2,\dots,n)$, meaning $S_n$ is generated by those two elements alone.
\end{itemize}

\section{}
\noindent\fbox{\fbox{\parbox{6.5in}{
            Show that the alternating groups $A_n$ ($n \geq 4$) have trivial center.
}}}\bigskip

Let $p$ be any permutation in $A_n$ other than the identity.
\begin{itemize}
    \item If $p$ contains only 2-cycles, we can choose symbols $a,b,c,d$ such that $p(a)=b$ and $p(b)=a$. In this case, $p$ does not commute with the 3-cycle $(abc)$, because $p \circ (abc)$ maps $a$ to $a$ and $(abc) \circ p$ maps $a$ to $c$.
    \item If $p$ does not contain only 2-cycles, then since $p \neq e$, p must contain a cycle whose length is at least 3, meaning we can choose symbols $a,b,c,d$ such that $p(a)=b$ and $p(b)=c$. Then $p$ does not commute with $(ab)(cd)$, because $(ab)(cd)\circ p$ maps $a$ to $a$, and $p \circ (ab)(cd)$ maps $a$ to $c$.
\end{itemize}
In either case, we have found an even permutation that $p$ does not commute with. Therefore the center of $A_n$ is trivial (when $n \geq 4$).

\section{}
\noindent\fbox{\fbox{\parbox{6.5in}{
            Show that $Aut(S_3)$ is isomorphic to $S_3$.
}}}\bigskip

Any permutation $p \in S_3$ can be written as a product of $r:=(123)$ and $s:=(12)$. Specifically, $p$ can be written as $r^as^b$ where $a \in \{0,1,2\}$ and $b \in \{0,1\}$. This comes from the result from problem 6 part (c), which states $r$ and $s$ generate $S_3$, as well as the relations $srs=r^{-1}$ and $s^2=e$.
\par
Define $\varphi: S_3 \rightarrow Aut(S_3)$ by $\varphi(p) = f$, where $f: S_3 \rightarrow S_3$ is the isomorphism which maps $r$ to $r^a$ and $s$ to $s^b$. One can easily check that $f$ and $\varphi$ are both well-defined, they're both homomorphisms, and they're both bijective, so $\varphi$ is an isomorphism from $S_3$ to $Aut(S_3)$.

\section{}
\noindent\fbox{\fbox{\parbox{6.5in}{
            Let $N=\{e, (12)(34), (13)(24), (14)(23)\} \subset S_4$. Show that $S_4/N$ is isomorphic to $S_3$.
}}}\bigskip

Let $f: S_4 \rightarrow S_4$ be a homomorphism defined by the following process: for any permutation $x \in S_4$, let $y$ be the permutation consisting of two disjoint transpositions, such that $x(1)=y(1)$. Given $x$, there is a unique $y$ that satisfies that property. Then we can define $f$ by $f(x)=xy$, and from that definition, we see that $f$ is indeed a well-defined homomorphism.
\par
A permutation $x$ is in the kernel of $f$ if an only the $y$ that corresponds to $x$ is the inverse of $x$. Since $y$ is made of 2-cycles, it's equal to its own inverse, so $Ker(f)=N$.
\par
Applying the first isomorphism theorem, $S_4/N$ is isomorphic to $Im(f)$. By Lagrange's theorem, $Im(f)$ has $24/4=6$ elements, so if we find 6 distinct elements of $Im(f)$, those are the only elements of $Im(f)$. For any permutation $x$ which acts only on the elements 2, 3, and 4, the corresponding $y$ will be the identity, so $f(x)=x$, meaning $S_3$ is a subset of $Im(f)$, and $S_3$ also has 6 elements, so $S_3 \cong S_4/N$.
\par

\section{}
\noindent\fbox{\fbox{\parbox{6.5in}{
            Prove that the alternating group $A_4$ does not have a subgroup of order 6.
}}}\bigskip

Every group of order 6 is either $C_6$ (which is isomorphic to $C_2 \times C_3$) or $D_6$ (which is isomorphic to $S_3$). Repeating the process that was used in problem 3, we see that no element of $A_4$ can have order greater than 4, so $A_4$ cannot have a subgroup isomorphic to $C_6$.
\par
This implies that if $A_4$ has a subgroup of order 6, then that subgroup is isomorphic to $D_6$, which means there exist elements $r, s \in A_4$ such that $srs=r^{-1}$ and $s^2=e$ and $r^3=e$. Every element of $A_4$ is either the identity, a 3-cycle, or the product of two disjoint transpositions. Since the identity commutes with everything $r \neq e \neq s$, so we infer that $r$ is a 3-cycle and $s$ is either a transposition or two disjoint transpositions.
\par
Without loss of generality, we can call one of the transpositions in $s$ $(ab)$, and call $r$ either $(abc)$ or $(bcd)$. In the first case, $srs$ maps $a$ to either $c$ or $d$ (depending on whether $s$ is one transposition or two), so $srs \neq r^{-1}$. In the second case, $srs$ maps $b$ to $b$, which also implies $srs \neq r^{-1}$. We have reached a contradiction in either case, so no subgroup of $A_4$ can be isomorphic to $D_6$.
\par
But since every group of order 6 is isomorphic to either $C_6$ or $D_6$, and we have shown no subgroup of $A_4$ can be isomorphic to either of those, there cannot be any subgroup of $A_4$ which has order 6.

\end{document}
