\documentclass[12pt]{article}
\usepackage[margin=1in]{geometry}
\usepackage{amsmath}
\usepackage{amsfonts}
\usepackage{tikz-cd}

\begin{document}

\title{Math 110AH Homework 6}
\author{Nathan Solomon}
\maketitle

\textbf{Assignment due November 16th at 11:59 pm.}
\par
\textbf{Problems 1, 8, and 10 are graded.}

\section{}
\noindent\fbox{\fbox{\parbox{6.5in}{
            Assume that a subset $S \subset G$ of a group $G$ satisfies $gSg^{-1} \subset S$ for all $g \in G$. Prove that the subgroup $\langle S \rangle$ generated by $S$ is normal in $G$.
}}}\bigskip

Remember that $\langle S \rangle$ can be defined as
\[ \langle S \rangle = \{ s_1^{p_1} s_2^{p_2} \dots s_n^{p_n}: n \in \mathbb{N}_0, \text{ and for every index $j \leq n$, } s_j \in S \text{ and } p_j \in \mathbb{N} \}. \]
Now take an arbitrary element $s_1^{p_1} s_2^{p_2} \dots s_n^{p_n}$ of that subgroup and conjugate it by an arbitrary element $g \in G$. We can now show that the new element is also in $\langle S \rangle$:
\begin{align*}
    g (s_1^{p_1} s_2^{p_2} \dots s_n^{p_n}) g^{-1} &= \\
    (g s_1^{p_1} g^{-1}) (g s_2^{p_2} g^{-1}) \dots (g s_n^{p_n} g^{-1}) &= \\
    (g s_1 g^{-1})^{p_1} (g s_2 g^{-1})^{p_2} \dots (g s_n g^{-1})^{p_n} &\in \langle S \rangle.
\end{align*}
That last step is valid because for any index $j \leq n$, the fact that $s_j \in S$ implies $g s_j g^{-1} \in S$, and by the definition of $\langle S \rangle$, the finite product of elements of $S$ is also in $\langle S \rangle$.

\section{}
\noindent\fbox{\fbox{\parbox{6.5in}{
            Prove that for every integer $n > 0$ there exists a unique cyclic subgroup $H_n \subset \mathbb{Q}/ \mathbb{Z}$ of order $n$.
}}}\bigskip

For any $n$, let $h_n = \frac{1}{n}$ and let $H_n = \langle h_n \rangle$. For any integer $a$, $a \cdot h_n$ is an integer if and only if $a$ is an integer multiple of $n$. Therefore $h_n$ and $H_n$ both have order $n$. Now we just need to show that $H_n$ is unique.
\par
Suppose there exists some other cyclic subgroup $G$ of $ \mathbb{Q} / \mathbb{Z}$ which has order $n$. Then since it's cyclic, there exists a generator $g \in G$ such that $g$ has order $n$. In other words, $n \cdot g$ is an integer, which means $g$ is an integer multiple of $ \frac{1}{n} $. Since $G$ is generated by $g$, every element of $G$ is an integer multiple of $ \frac{1}{n} $. $H_n$ is the set of all integer multiples of $ \frac{1}{n} $, and we know that $H_n$ and $G$ both have $n$ elements, so $H_n = G$. Therefore $H_n$ is unique.

\section{}
\noindent\fbox{\fbox{\parbox{6.5in}{
            Let $H_n \subset G = \mathbb{Q} / \mathbb{Z}$ be a cyclic subgroup of order $n$. Prove that $G/H_n$ is isomorphic to $G$. (Hint: consider the homomorphism $f:G \rightarrow G, f(x) = nx$.)
}}}\bigskip

In the previous problem, we found the unique group $H_n$ which satisfies that property. Now let $f: G/H_n \rightarrow G$ be the function defined by $f([x]) = n[x]$. We need to show that $f$ is a homomorphism, it's well-defined, and it's bijective.
\begin{itemize}
    \item \textbf{Homomorphism:} $f(x_1) + f(x_2) = nx_1 + nx_2 = n(x_1+x_2) = f(x_1+x_2)$.
    \item \textbf{Well-defined:} For any element $[x] \in G / H_n$, let $x_1$ and $x_2$ be any two representative elements of $[x]$. Then $x_2-x_1$ is an integer multiple of $1/n$, since $x_2-x_1 \in H_n$, which means
        \[ f(x_1) = nx_1 + \mathbb{Z} = n(x_1) + n(x_2-x_1) + \mathbb{Z} = nx_2 + \mathbb{Z} = f(x_2). \]
    \item \textbf{Injective:} Let $x_1, x_2$ be distinct elements of $G/H_n$, meaning $x_1-x_2$ is not an integer multiple of $1/n$. Then
        \[ f(x_1) - f(x_2) = n(x_1-x_2) \]
        which is not an integer, so $f(x_1) \neq f(x_2)$.
    \item \textbf{Surjective:} For any element $x \in G$, $x/n$ is in $G/H_n$ and $f(x/n) = n(x/n) = x$.
\end{itemize}
Therefore $f$ is an isomorphism from $G/H_n$ to $G$.

\section{}
\noindent\fbox{\fbox{\parbox{6.5in}{
            Find all elements of finite order in $ \mathbb{R} / \mathbb{Z}$.
}}}\bigskip

\textit{First, I will show rational numbers have finite order, then I will show that irrational elements do not have finite order.}
\par
Let $[x] \in \mathbb{Q} / \mathbb{Z}$ be any rational number in that group. Then there exist integers $a$ and $b$ such that $[x] = [\frac{a}{b}] $. Since $|b| \cdot [x] = [\frac{a \cdot |b|}{b}] = [0]$, the order of $[x]$ is at most $|b|$, so it's finite. Therefore every rational number in $ \mathbb{R} / \mathbb{Z}$ has finite order.
\par
Now let $[x] \in (\mathbb{R} / \mathbb{Z}) \backslash ( \mathbb{Q} / \mathbb{Z})$ be an irrational number, and suppose $[x]$ has finite order $b$ in $ \mathbb{R} / \mathbb{Z}$. That means $[b \cdot x] = 0$, so let $x$ be any representative element of $[x]$, and $b \cdot x$ will be an integer, which we'll call $a$. Then $x = \frac{a}{b} \in \mathbb{Q}$, so $[x] \in \mathbb{Q} / \mathbb{Z}$, which is a contradiction. Therefore any element $[x] \in \mathbb{R} / \mathbb{Z}$ that is not in $ \mathbb{Q} / \mathbb{Z}$ cannot have finite order.
\par
The set of elements in $ \mathbb{R} / \mathbb{Z}$ with finite order is $ \mathbb{Q} / \mathbb{Z}$.

\section{}
\noindent\fbox{\fbox{\parbox{6.5in}{
            Show that $Aut( \mathbb{Z} )$ is a cyclic group of order 2. What is $Inn( \mathbb{Z} )$?
}}}\bigskip

Let $f$ be an automorphism on $ \mathbb{Z}$. For any $n \in \mathbb{Z}, n \cdot 1$, so 1 is a generator of $\mathbb{Z}$. Isomorphisms map generators to generators, so $f(1)$ must also be a generator (of $\mathbb{Z}$). If $|f(1)| = n$, then the group generated by $f(n)$ is all the integer multiples of $n$, which is equal to $ \mathbb{Z}$ if and only if $n= \pm 1$. Let $f$ be the identity map on $ \mathbb{Z}$ and let $g$ be the negation map, so $Aut( \mathbb{Z} ) = \{f, g\}$. Since $f^2=g^2=f$, this is the cyclic group of order 2.
\par
The group of integers is abelian, so $ \mathbb{Z} = Z( \mathbb{Z})$. Using the fact that $G/Z(G) \cong Inn(G)$ for any group $G$ and the fact that $G/G$ is the trivial group, we see that $Inn( \mathbb{Z})$ is the trivial group.

\section{}
\noindent\fbox{\fbox{\parbox{6.5in}{
            Prove that every automorphism of $S_3$ is inner.
}}}\bigskip

Let $\varphi: S_3 \rightarrow Aut(S_3)$ be the homomorphism defined by $\varphi(g)(h) = ghg^{-1}$. As shown in question 8, the center of $S_3$ is trivial, and since the kernel of $\varphi$ is equal to the center of $S_3$, $\varphi$ has to be injective. $Inn(S_3)$ is equal to the image of $\varphi$, so $Inn(S_3)$ has 6 elements -- the same as $|S_3|$.
\par
We know that the elements of $S_3$ are
\[ \{ e, (12), (13), (23), (123), (132) \}. \]
Let $f$ be any automorphism on $S_3$. Then $f$ must map $(12)$ to some order 2 element. There are 3 options for $f((12))$, since $S_3$ has 3 elements with order 2. Then $f$ must map $(23)$ to a different order 2 element, which gives 2 options. Since $S_3$ is generated by $(12)$ and $(23)$, these two choices fully determine $f$, so there are 6 possible values of $f$.
\par
We know that $Inn(S_3) \subset Aut(S_3)$, but we have just shown that those groups both have 6 elements, so they must be equal.

\section{}
\noindent\fbox{\fbox{\parbox{6.5in}{
            Prove that $Aut( \mathbb{Z} / 2 \mathbb{Z} \times \mathbb{Z} / 2 \mathbb{Z})$ is isomorphic to the symmetric group $S_3$. (Hint: Notice that every automorphism of $ \mathbb{Z} / 2 \mathbb{Z} \times \mathbb{Z} / 2 \mathbb{Z}$ permutes all three nonzero elements of the group.)
}}}\bigskip

Label the elements of $K_4$ as $a=(0,0), b=(1,0), c=(0,1), d=(1,1)$. Let $\varphi: S_3 \rightarrow Aut(K_4)$ be the function defined by the following table. One can check that all of the functions $\varphi(p): K_4 \rightarrow K_4$ are indeed isomorphisms. Also, one can check that $\varphi$ is an isomorphism by comparing the multiplication table for $S_3$ to the multiplication table for $Aut(K_4)$, or just by noticing that the right hand side of the table resembles the permutation $p$. The last thing to check is that we have listed all 6 elements of $S_3$ and of $Aut(K_4)$ -- we know $|S_3| = 3! = 6$, and we can confirm that there are no more automorphisms on $K_4$, because automorphisms fix the identity ($a$) and are bijections.
\bigskip
\par
{\centering
\begin{tabular}{|c|c|}
    \hline
    $p \in S_3$ & $(\varphi(p)(a), \varphi(p)(b), \varphi(p)(c), \varphi(p)(d))$ \\
    \hline
    \hline
    $e$ & (a, b, c, d) \\
    $(12)$ & (a, c, b, d) \\
    $(13)$ & (a, d, c, b) \\
    $(23)$ & (a, b, d, c) \\
    $(123)$ & (a, c, d, b) \\
    $(132)$ & (a, d, b, c) \\
    \hline
\end{tabular}\par}

\section{}
\noindent\fbox{\fbox{\parbox{6.5in}{
            Show that the center of $S_n$ is trivial if $n \geq 3$.
}}}\bigskip

\textit{In this proof, I'll use $L$ to mean any set with $n$ elements, and $S_n$ to mean the set of bijections from $L$ to $L$.}
\par
Let $p \in S_n$ (where $n \geq 3$) be a nontrivial permutation of a set of $n$ letters, which I'll call $L$. Then there exists elements $a, b \in L, a \neq b,$ such that $p(a) = b$. Since $L$ has at least 3 elements, we can let $c \in L$ be an element other than $a$ or $b$, and let $q \in S_n$ be the transposition which switches $b$ and $c$, but does not change any other elements of $L$.
\par
One can check that $q \circ p$ maps $a$ to $c$, but $p \circ q$ maps $a$ to $b$, so $q \circ p \neq p \circ q$. We have shown that for any element $p \in S_n$, if $n \geq 3$, there exists an element $q \in S_n$ which does not commute with $p$, so the center of $S_n$ is trivial.

\section{}
\noindent\fbox{\fbox{\parbox{6.5in}{
            \begin{itemize}
                \item (a) A subgroup $H$ of $G$ is called \textit{characteristic}, if $f(H) = H$ for any automorphism $f$ of $G$. Show that a characteristic subgroup of $H$ is normal in $G$.
                \item (b) Prove that if $K$ is a characteristic subgroup of $H$ and $H$ is a characteristic subgroup of $G$, then $K$ is characteristic subgroup of $G$.
                \item (c) Prove that if $K$ is a characteristic subgroup of $H$ and $H$ is normal in $G$, then $K$ is normal in $G$.
            \end{itemize}
}}}\bigskip

\begin{itemize}
    \item (a) For any element $g \in G$, let $f$ be the automorphism on $G$ which maps $h$ to $ghg^{-1}$. If $H$ is a characteristic subgroup of $G$, then $f(h) \in H$ for any $h \in H$, so $H \trianglelefteq G$.
    \item (b) Let $f$ be any automorphism of $G$. Then $f(H)=H$, so the restriction $f|_H$ is an automorphism of $H$. Since $K$ is a characteristic subgroup of $H$, $f|_H(K)=K$, meaning $f(K)=K$, so $K$ is a characteristic subgroup of $G$.
    \item (c) For any $k \in K$ and any $g \in G$, let $f: G \rightarrow G$ be the automorphism which maps any element $h \in G$ to $ghg^{-1}$. Since $H$ is normal in $G$, $f(H)=H$, so the restriction $f|_H$ is an automorphism of $H$. Then $K$ is a characteristic subgroup of $H$, so $f(K)=K$, meaning $gkg^{-1}\in K$, so $K \trianglelefteq G$.
\end{itemize}

\section{}
\noindent\fbox{\fbox{\parbox{6.5in}{
            Let $N$ be an abelian normal subgroup of a finite group $G$. Assume that the orders $|G/N|$ and $|Aut(N)|$ are relatively prime. Prove that $N$ is contained in the center of $G$. (Hint: Consider the conjugation homomorphism $f: G \rightarrow Aut(N), f(g)(n) = gng^{-1}$.)
}}}\bigskip

Let $f$ be the function defined in the hint. Then $Ker(f)$ is the set of elements $g \in G$ such that $f(g)$ is the identity map on $N$ -- that is, such that $g$ commutes with every element of $N$. Since $N$ is abelian, it is clearly a subgroup of $Ker(f)$. Also, by the first isomorphism theorem, $Ker(f)$ is a normal subgroup of $G$, so we have
\[ N \subset Ker(f) \trianglelefteq G. \]
By the the third isomorphism theorem, that implies
\[ \frac{G/N}{Ker(f)/N} \cong G/Ker(f). \]
Now we can apply the first isomorphism theorem to replace $G/Ker(f)$ with $Im(f)$.
\[ \frac{G/N}{Ker(f)/N} \cong Im(f) \subset Aut(N). \]
Applying Lagrange's theorem to that, we get
\[ \frac{|G/N|}{|Ker(f)/N|} < |Aut(N)|. \]
Since $|G/N|$ and $|Aut(N)$ are coprime, and $|Ker(f)/N|$ is an integer, we know the two sides of that inequality are coprime, but since the left hand side is smaller, the left hand side is one. By Lagrange's theorem again, that implies $Ker(f)=G$, meaning every element of $N$ commutes with every element of $G$. Therefore $N \subset Z(G)$.

\end{document}
