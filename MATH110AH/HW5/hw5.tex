\documentclass[12pt]{article}
\usepackage[margin=1in]{geometry}
\usepackage{amsmath}
\usepackage{amsfonts}
\usepackage{tikz-cd}

\begin{document}

\title{Math 110AH Homework 5}
\author{Nathan Solomon}
\maketitle

\textbf{Assignment due November 9th at 11:59 pm}
\par
\textbf{Questions 2, 5(a), 6, and 10 will be graded.}

\section{}
\noindent\fbox{\fbox{\parbox{6.5in}{
            \begin{itemize}
                \item (a) Let $H \subset G$ be a subgroup. Show that $H$ is the image of a homomorphism from some group to $G$.
                \item (b) Let $N \subset G$ be a normal subgroup. Show that $N$ is the kernel of a homomorphism from $G$ to some group.
            \end{itemize}
}}}\bigskip

\begin{itemize}
    \item (a) Let $f: H \rightarrow G$ be an inclusion map (aka ``the canonical homomorphism"), defined by $f(h) = h$ for any $h \in H$. Then $f$ is a homomorphism, and its image is $H$.
    \item (b)
\end{itemize}

\section{}
\noindent\fbox{\fbox{\parbox{6.5in}{
            Let $n$ be a positive integer. Show that the map
            \[ f: \mathbb{Q}/\mathbb{Z} \rightarrow \mathbb{Q}/\mathbb{Z}, \hspace{1cm} f(a+\mathbb{Z}) = na + \mathbb{Z} \]
            is a well defined homomorphism. Find $Ker(f)$ and $Im(f)$.
}}}\bigskip

Let $z_1, z_2$ be any integers. Then
\[ f(a + z_1) = n(a+z_1) + \mathbb{Z} = n(a+z_2) + (nz_1-nz_2 + \mathbb{Z}) = n(a+z_2) + \mathbb{Z} = f(a+z_2). \]
This proves that $f(a+ \mathbb{Z})$ is the same no matter which representative element of $ \mathbb{Z}$ we use, so $f$ is well defined.
\par
Also, $f$ is a homomorphism, because
\[ f(a + \mathbb{Z}) + f(b + \mathbb{Z}) = na + nb + \mathbb{Z} = f(a+b + \mathbb{Z}). \]
\bigskip
\par
The coset $a + \mathbb{Z}$ is in the kernel of $f$ if an only if $na + \mathbb{Z} = \mathbb{Z}$. In other words, $Ker(f)$ is the set of integer multiples of $1/n$.
\par
Also, $f$ is a surjective map because for any $b + \mathbb{Z}$ in the codomain, you can let $a = b/n$, so then $f(a + \mathbb{Z}) = na + \mathbb{Z} = b+ \mathbb{Z}$. Since $f$ is surjective, its image is the same as its codomain.
\bigskip
\par
We have shown that $f$ is a well defined homomorphism with
\[ Ker(f) = \frac{1}{n} \cdot \mathbb{Z} \]
and
\[ Im(f) = \mathbb{Q} / \mathbb{Z}. \]

\section{}
\noindent\fbox{\fbox{\parbox{6.5in}{
            Let $K \subset H \subset G$ be subgroups. Show that if $K$ has finite index in $G$ then $[G:K] = [G:H] [H:K]$.
}}}\bigskip

\textit{Note: everyone's first thought when they see this is to use Lagrange's theorem, but that's only applicable if $G$ is a finite group.}

\section{}
\noindent\fbox{\fbox{\parbox{6.5in}{
            Let $H \subset G$ be a subgroup. Show that the correspondence $Ha \mapsto a^{-1} H$ is a bijection between the sets of right and left cosets of $H$ in $G$.
}}}\bigskip

\textit{Proof outline: first, I'll prove that that mapping is injective, then I'll prove it's surjective.}

\begin{itemize}
    \item Let $a_1^{-1}H$ and $a_2^{-1}H$ be any two left cosets in the image of that map. If those two cosets are equal, then $a_2 a_1^{-1} \in H$, which means
        \[ Ha_1 = (Ha_2a_1^{-1})a_1 = Ha_2. \]
        Since $a_1^{-1}H=a_2^{-1}H$ implies $Ha_1=Ha_2$, that map is injective.
    \item Let $bH$ be any left coset of $H$ in $G$. Then $Hb^{-1}$ is a right coset of $H$ which that correspondence maps to $bH$. Therefore that correspondence is surjective.
\end{itemize}

\textit{Note: whenever I say something like ``let $a^{-1}H$ be a left coset of H", what I really mean is ``take any left coset of H, and call one element of that coset $a^{-1}$. Then that coset can be written as $a^{-1}H$."}
\par
Since that correspondence is both injective and surjective, it is a bijection (from the set of right cosets of $H$ to the set of left cosets of $H$).

\section{}
\noindent\fbox{\fbox{\parbox{6.5in}{
            Let $f: G \rightarrow H$ be a surjective group homomorphism.
            \begin{itemize}
                \item (a) Let $H'$ be a subgroup of $H$. Show that $G' = f^{-1}(H')$ is a subgroup of $G$. Prove that the correspondence $H' \mapsto G'$ is a bijection between the set of all subgroups of $H$ and the set of all subgroups of $G$ containing $Ker(f)$.
                \item (b) Let $H'$ be a normal subgroup of $H$. Show that $G' = f^{-1}(H')$ is a normal subgroup of $G$. Prove that $G/G' \simeq H/H'$ and the correspondence $H' \mapsto G'$ is a bijection between the set of all normal subgroups of $H$ and the set of all normal subgroups of $G$ containing $Ker(f)$.
            \end{itemize}
}}}\bigskip

\section{}
\noindent\fbox{\fbox{\parbox{6.5in}{
            Show that every subgroup of index 2 is normal.
}}}\bigskip

Let $H$ be a subgroup of index 2 in $G$. We want to show that for any $h \in H$ and any $g \in G$, $g^{-1}hg \in H$. There are two cases to consider: when $g \in H$, and when $g \in G \backslash H$.

\begin{itemize}
    \item If $g \in H$, then $g^{-1}hg$ is the product of three elements which are all in $H$, so $g^{-1}hg \in H$.
    \item If $g \not\in H$, then $hg$ is also not in $H$, because if it were, that would imply $h^{-1}(hg) = g$ is in $H$. Since $H$ is index two, there are only two left cosets of $H$, which are $H$ and $gH$, and the union of those two cosets is $G$. Since $hg$ is not in $H$, it must be in $gH$, which means $g^{-1}hg \in g^{-1}gH = H$.
\end{itemize}

We have shown that for any $h \in H$ and any $g \in G$ (regardless of whether $g \in H$), $g^{-1}hg \in H$, so $H$ is normal. The only restriction we placed on $H$ is that it is a subgroup of index 2, so this proves that any subgroup of index 2 is normal.

\section{}
\noindent\fbox{\fbox{\parbox{6.5in}{
            Let $H \subset G$ be a subgroup. Suppose that for any $a \in G$ there exists $b \in G$ such that $aH=Hb$. Show that $H$ is normal in $G$.
}}}\bigskip

\section{}
\noindent\fbox{\fbox{\parbox{6.5in}{
            Show that the group $\mathbb{Z}/m\mathbb{Z} \times \mathbb{Z}/m\mathbb{Z} \times \mathbb{Z}/m\mathbb{Z}, m>1$, can be generated by three elements and cannot be generated by two elements.
}}}\bigskip

Let $g_1 = (1,0,0), g_2 = (0,1,0), g_3 = (0,0,1)$. Then any element $(x_1,x_2,x_3)$ of that group is equal to $x_1 g_1 + x_2 g_2 + x_3 g_3$, so the three $g_i$'s together generate $( \mathbb{Z}/m \mathbb{Z})^3$.
\bigskip
\par
Now consider any two elements $x_1, x_2 \in (\mathbb{Z}/m\mathbb{Z})^3$. Since $\langle x_1, x_2 \rangle$ is abelian, any element of $\langle x_1, x_2 \rangle$ can be written as $a \cdot x_1 + b \cdot x_2$ for some integers $a$ and $b$. However, $m \cdot x_1 = 0 = m \cdot x_2$, so there are only $m$ possible values each for $a \cdot x_1$ and $b \cdot x_2$. Therefore the order of $\langle x_1, x_2 \rangle$ is at most $m^2$, which is less than the order of $( \mathbb{Z}/m \mathbb{Z})^3$, so those two groups are not equal. That means $( \mathbb{Z}/m \mathbb{Z})^3$ cannot be generated by two elements.

\section{}
\noindent\fbox{\fbox{\parbox{6.5in}{
            Let $p$ be an odd prime. Prove that the congruence $x^2 \equiv -1 \pmod{p}$ has an integer solution if and only if $p \equiv 1 \pmod{4}$. (Hint: use Fermat's Little Theorem assuming we know that the group $(\mathbb{Z}/p\mathbb{Z})^\times$ is cyclic.)
}}}\bigskip

\begin{itemize}
    \item If $p \equiv 1 \pmod{4}$ then let $g$ be a generator of $(\mathbb{Z}/p\mathbb{Z})^\times$ and let $n$ be the smallest natural number such that $g^n = -1$. We know the order of $g$ is $p-1$, so $n < p-1$. Also, $g^{2n}=(-1)^2=1$, so the order of $g$ (which is $p-1$) must divide $2n$. This is enough for us to determine that $n=(p-1)/2$. Since $p-1$ is a multiple of 4, we can define $x := g^{(p-1)/4}$ which satisfies the property $x^2 \equiv -1 \pmod{p}$.
    \item If $p \equiv 3 \pmod{4}$ then suppose there exists some integer $x$ such that $x^2 \equiv -1 \pmod{p}$. It's pretty clear that $x$ is not $\pm 1$, so $n=4$ is the smallest natural number such that $x^n=4$. We know the order of every element in a group must divide the order of the group, but $x$ has order $4$, and $(\mathbb{Z}/p\mathbb{Z})^\times$ has order $p-1$, which is not divisible by 4. This is a contradiction, so there cannot be any integers $x$ such that $x^2 \equiv 1 \pmod{p}$.
\end{itemize}

Since $p$ is an odd prime, it must be congruent to either 1 or 3 (mod 4). By considering those two cases, we have proven that there exists a solution $x \in \mathbb{Z}$ to the congruence $x^2 \equiv -1 \pmod{p}$ if and only if $p \equiv 1 \pmod{4}$.

\section{}
\noindent\fbox{\fbox{\parbox{6.5in}{
            Prove that if a group $G$ contains a subgroup $H$ of finite index, then $G$ contains a normal subgroup $N$ of finite index such that $N \subset H$. (Hint: Consider the homomorphism of $G$ to the symmetric group of all left cosets of $H$ in $G$ taking any $x \in G$ to $f_x$ defined by $f_x(aH)=xaH$.)
}}}\bigskip

\end{document}
