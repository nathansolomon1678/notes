\documentclass[12pt]{article}
\usepackage[margin=1in]{geometry}
\usepackage{amsmath}
\usepackage{amsfonts}
\usepackage{tikz-cd}
\renewcommand{\d}{\mathrm{d}}

\begin{document}

\title{Math 110AH Reference Sheet}
\author{Nathan Solomon}
\maketitle

Still need to write notes on the following:
\begin{itemize}
    \item review Charlie's notes
    \item Sylow's theorems
    \item solvable, normalizer, derived series, dicyclic group (specail case with 8 elements is called the quaternion group), semidirect product (internal and external), construction of dihedral group as a semidirect product, isotropy group, effective action, nilpotent group, characteristic subgroup
    \item center of dihedral group. Cauchy's theorem
\end{itemize}

The \textbf{division theorem} says that for any integers $a, b$ such that $b \neq 0$, there exist unique integers $q$ and $r$ (called the ``quotient" and ``remainder", respectively) such that $a=qb+r$ and $0 \leq r < |b|$.
\bigskip
\par
\textbf{Bézout's identity} says that if $a, b, c, x_0, y_0 \in \mathbb{Z}$ and $d = \operatorname{gcd}(a, b)$ and $a x_0 + b y_0 = c$, then the pair of integers $(x, y)$ is a solution to $ax+by=c$ if and only if
\[ (x, y) = \left( x_0 + \frac{bk}{d}, y_0 - \frac{ak}{d} \right) \]
for some integer $k$.
\bigskip
\par
An \textbf{isomorphism} can be defined as a bijective homomorphism. This is not always true, but in the context of groups and the material we cover in this class, it works.
\bigskip
\par
The \textbf{order of a group} $G$ is just the cardinality $|G|$, but the \textbf{order of an element of a group} is the order of the subgroup generated by that one element. For example, the cyclic group of 4 elements ($C_4$) and the Klein 4-group ($K_4$) both have order 4, but one way to tell they're not isomorphic is to note that $C_4$ contains two elements of order 4, but in $K_4$, all elements except the identity have order 2.
\bigskip
\par
\textbf{Cauchy's theorem} says that for any finite group $G$ whose order is divisible by a prime number $p$, there must be an element $g \in G$ with order $p$.
\bigskip
\par
The \textbf{Chinese remainder theorem} says that if $n_1, n_2, \dots, n_k$ are pairwise coprime and $N=n_1n_2 \cdots n_k$, then
\[ \mathbb{Z}/N\mathbb{Z} \cong \mathbb{Z}/n_1\mathbb{Z} \times \mathbb{Z}/n_2\mathbb{Z} \times \cdots \times \mathbb{Z}/n_k\mathbb{Z}. \]
In other words, if we know the remainder when some integer $a$ is divided by $n_1$, when it's divided by $n_2$, and so on all the way to $n_k$, we know what the remainder is when $a$ is divided by $N$.
\bigskip
\par
\textbf{Euler's totient function} is a function $\varphi: \mathbb{N} \rightarrow \mathbb{N}$ defined as follows:
\[ \varphi(n) = \left| \left\{ k \in \mathbb{N} : k \leq n, \operatorname{gcd}(n, k) = 1 \right\} \right|. \]
If $n \neq 1$ then $\varphi(n)$ is equal to the number of generators of the cyclic group of order $n$, and it's also equal to the order of the multiplicative group $(\mathbb{Z}/n\mathbb{Z})^\times$. Once you know the prime factors of $n$, you can calculate $\varphi(n)$ very quickly using the following rules, which come from the Chinese remainder theorem:
\begin{itemize}
    \item If $p$ is prime and $k \geq 1$, then $\varphi(p^k) = p^{k-1}(p-1)$.
    \item If $a$ and $b$ are coprime, then $\varphi(ab) = \varphi(a) \cdot \varphi(b)$.
\end{itemize}
\bigskip
\par
\textbf{Euler's theorem} says that if $a$ and $n$ are coprime positive integers, then
\[ a^{\varphi(n)} \equiv 1 \pmod{n}. \]
Also, if that congruence is true for some positive integers $a$ and $n$, then $a$ and $n$ are coprime.
\bigskip
\par
\textbf{Lagrange's theorem} says that if $H$ is a subgroup of a finite group $G$, then
\[ |G| = [G:H] \cdot |H|. \]
\bigskip
\par
\textbf{Fermat's little theorem} says that if $p$ is prime and $a$ is coprime to $p$, then
\[ a^{p-1} \equiv 1 \pmod{p}, \]
which is easy to get from Euler's theorem. But more generally, if $p$ is prime and $a$ is any integer, then Fermat's little theorem also says
\[ a^p \equiv a \pmod{p}. \]
\bigskip
\par
The \textbf{Euclidean algorithm} calculates the greatest common divisor of two natural numbers in logarithmic time. If you ever forget how it works, just picture that one gif from wikipedia where a rectangle is being tiled by squares.
\bigskip
\par
If $S$ is a subset (not necessarily a subgroup) of a group $G$, then the \textbf{free group} generated by $S$ is written as $\langle S \rangle$ and is defined as the intersection of all subgroups of $G$ which contain all the elements of $S$.
\bigskip
\par
If $H$ is a subgroup of $G$, then the \textbf{quotient}, $G/H:=\{gH:g\in G\}$, is the set of all left cosets of $H$ (in $G$). The quotient is a group if and only if $H \trianglelefteq G$ ($H$ is normal in $G$).
\bigskip
\par
The \textbf{first isomorphism theorem} says that if $f: G \rightarrow H$ is a group homomorphism, then $Ker(f)$ is normal in $G$, $Im(f)$ is a subgroup of $G$, and $Im(f) \cong G / Ker(f)$.
\bigskip
\par
The \textbf{second isomorphism theorem} says that if $S$ is a subgroup of $G$ and $N$ is a normal subgroup of $G$, then $(SN)/N \cong S(S \cap N)$.
\bigskip
\par
The \textbf{third isomorphism theorem} says that if $N$ and $K$ are both normal subgroups of $G$ and $N \subset K$, then $(G/N)/(K/N) \cong G/K$.
\bigskip
\par
The \textbf{universal property of quotient groups} is that if $G$ and $H$ are groups, $N$ is a normal subgroup of $G$, and $f: G \rightarrow H$ is a group homomorphism whose kernel conttains $N$, then there exists a unique group homomorphism $\overline{f}: G/N \rightarrow H$ such that $f = \overline{f} \circ \pi$, where $\pi$ is the canonical homomorphism from $G$ to $G/N$.
\bigskip
\par
The \textbf{center} of a group $G$, written as $Z(G)$, is the set of elements which commute with all other elements of $G$.
\bigskip
\par
An \textbf{inner automorphism on $G$} is an isomorphism from $G$ to $G$ that can be defined as conjugation by some element of $G$. The set of all inner automorphism of $G$ is called $Inn(G)$, and by the first isomorphism theorem,
\[ G/Z(G) \cong Inn(G). \]
\bigskip
\par
The \textbf{outer morphism group} is defined as $Out(G) = Aut(G)/Inn(G)$.
\bigskip
\par
The \textbf{conjugacy class} of an element $x \in G$ is the set of all elements of the form $gxg^{-1}$ for some $g \in G$. One interesting example is the alternating group $A_n$ when $n \geq 5$. We proved in class that when $n \geq 5$, $A_n$ is \textbf{simple}, meaning it has no normal subgroups, so the conjugacy classes of $A_n$ are IDK.
\bigskip
\par
A \textbf{perfect} group is equal to its own commutator subgroup.
\bigskip
\par
The \textbf{orbit-stabilizer theorem} says that if the group $G$ acts on $X$, then for any $x \in X$, there is a bijection between $G / Stab(x)$ and $Orb(x)$, so $|G \cdot x| = |Orb(x)| = [G : Stab(x)] = [G : G_x] = |G|/|G_x|$.
\bigskip
\par
A group action is \textbf{faithful} if the only element which behaves like the empty permutation is the identity element of the group. It is called \textbf{transitive} if there is only one orbit.
\bigskip
\par
Here is a \textbf{list of all groups of order below 16}:
\bigskip
\par
{\centering
\begin{tabular}{|c|c|c|}
    \hline
    Order & Abelian groups & Non-Abelian groups \\
    \hline
    \hline
    1 & $C_1 \cong \{e\}$ & \\
    \hline
    2 & $C_2$ & \\
    \hline
    3 & $C_3$ & \\
    \hline
    4 & $C_4, K_4$ & \\
    \hline
    5 & $C_5$ & \\
    \hline
    6 & $C_6$ &  $D_3 \cong S_3$ \\
    \hline
    7 & $C_7$ & \\
    \hline
    8 & $C_8, C_2 \oplus C_4, C_2 \oplus C_2 \oplus C_2$ & $D_4, Dic_2$\\
    \hline
    9 & $C_9, C_3 \oplus C_3$ & \\
    \hline
    10 & $C_{10}$ & $D_5$ \\
    \hline
    11 & $C_{11}$ & \\
    \hline
    12 & $C_{12}, C_2 \oplus C_6$ & $D_6, A_4, Dic_3$ \\
    \hline
    13 & $C_{13}$ & \\
    \hline
    14 & $C_{14}$ & $D_7$ \\
    \hline
    15 & $C_{15}$ & \\
    \hline
    $\cdots$ & $\cdots$ & $\cdots$ \\
    \hline
\end{tabular}\par}
\bigskip

\end{document}
