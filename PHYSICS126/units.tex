\documentclass[12pt]{article}
\usepackage[margin=1in]{geometry}
\usepackage{amsmath}
\usepackage{amssymb}
\usepackage{amsfonts}
\usepackage{amsthm}
\newtheorem{thm}{Theorem}[section]
\newtheorem{cor}[thm]{Corollary}
\newtheorem{lem}[thm]{Lemma}
\usepackage{tikz-cd}
\renewcommand{\d}{\mathrm{d}}

\begin{document}

\title{Conversion from SI to natural units}
\author{Nathan Solomon}
\maketitle

The dimensions for any quantity can be written as a column vector in $ \mathbb{Z}^3$ by listing the powers of mass, length, and time (in that order) as a column vector.
\[ [M] = \begin{bmatrix}
    1 \\
    0 \\
    0
\end{bmatrix} \hspace{1cm}
[L] = \begin{bmatrix}
    0 \\
    1 \\
    0
\end{bmatrix} \hspace{1cm}
[T] = \begin{bmatrix}
    0 \\
    0 \\
    1
\end{bmatrix}\]
Then energy, the reduced Planck constant, and the speed of light have dimensions represented by the following vectors:
\[ [E] = \begin{bmatrix}
    1 \\
    2 \\
    -2
\end{bmatrix} \hspace{1cm}
[\hbar] = \begin{bmatrix}
    1 \\
    2 \\
    -1
\end{bmatrix} \hspace{1cm}
[c] = \begin{bmatrix}
    0 \\
    1 \\
    -1
\end{bmatrix}\]
However, we could also write the dimensions for any quantity in another basis. When using natural units, it's convenient to work in the basis where
\[ [E] = \begin{bmatrix}
    1 \\
    0 \\
    0
\end{bmatrix} \hspace{1cm}
[\hbar] = \begin{bmatrix}
    0 \\
    1 \\
    0
\end{bmatrix} \hspace{1cm}
[c] = \begin{bmatrix}
    0 \\
    0 \\
    1
\end{bmatrix}\]
To convert from the $([E], [\hbar], [c])$ basis to the $([M], [L], [T])$ basis, all we need to do is left multiply by a change-of-basis matrix, which I'll call $A$.
\[A := \begin{bmatrix}
    1 & 1 & 0 \\
    2 & 2 & 1 \\
    -2 & -1 & -1
\end{bmatrix} \]
The inverse of that is
\[ A^{-1} = \begin{bmatrix}
    1 & -1 & -1 \\
    0 & 1 & 1 \\
    -2 & 1 & 0
\end{bmatrix} \]
so to convert from the $([M], [L], [T])$ basis to the $([E], [\hbar], [c])$ basis, multiply by $A^{-1}$.
\par
By looking at the entries of $A^{-1}$, we can read off the following conversions:
\begin{align*}
    [M] &= [E/c^2] \\
    [L] &= [\hbar c / E] \\
    [T] &= [\hbar / E]
\end{align*}

\end{document}
