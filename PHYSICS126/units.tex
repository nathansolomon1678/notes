\documentclass[12pt]{article}
\usepackage[margin=1in]{geometry}
\usepackage{amsmath}
\usepackage{amssymb}
\usepackage{amsfonts}
\usepackage{amsthm}
\newtheorem{thm}{Theorem}[section]
\newtheorem{cor}[thm]{Corollary}
\newtheorem{lem}[thm]{Lemma}
\usepackage{tikz-cd}
\renewcommand{\d}{\mathrm{d}}

\begin{document}

\title{Conversion from SI to natural units}
\author{Nathan Solomon}
\maketitle

The dimensions for any quantity can be written as a column vector in $ \mathbb{Z}^3$ by listing the powers of mass, length, and time (in that order) as a column vector.
\[ [M] = \begin{bmatrix}
    1 \\
    0 \\
    0
\end{bmatrix} \hspace{1cm}
[L] = \begin{bmatrix}
    0 \\
    1 \\
    0
\end{bmatrix} \hspace{1cm}
[T] = \begin{bmatrix}
    0 \\
    0 \\
    1
\end{bmatrix}\]
Then energy, the reduced Planck constant, and the speed of light have dimensions represented by the following vectors:
\[ [E] = \begin{bmatrix}
    1 \\
    2 \\
    -2
\end{bmatrix} \hspace{1cm}
[\hbar] = \begin{bmatrix}
    1 \\
    2 \\
    -1
\end{bmatrix} \hspace{1cm}
[c] = \begin{bmatrix}
    0 \\
    1 \\
    -1
\end{bmatrix}\]
However, we could also write the dimensions for any quantity in another basis. When using natural units, it's convenient to work in the basis where
\[ [E] = \begin{bmatrix}
    1 \\
    0 \\
    0
\end{bmatrix} \hspace{1cm}
[\hbar] = \begin{bmatrix}
    0 \\
    1 \\
    0
\end{bmatrix} \hspace{1cm}
[c] = \begin{bmatrix}
    0 \\
    0 \\
    1
\end{bmatrix}\]
To convert from the $([E], [\hbar], [c])$ basis to the $([M], [L], [T])$ basis, all we need to do is left-multiply by a change-of-basis matrix, which I'll call $A$.
\[A := \begin{bmatrix}
    1 & 1 & 0 \\
    2 & 2 & 1 \\
    -2 & -1 & -1
\end{bmatrix} \]
The inverse of that is
\[ A^{-1} = \begin{bmatrix}
    1 & -1 & -1 \\
    0 & 1 & 1 \\
    -2 & 1 & 0
\end{bmatrix} \]
so to convert from the $([M], [L], [T])$ basis to the $([E], [\hbar], [c])$ basis, multiply by $A^{-1}$.
\par
By looking at the entries of $A^{-1}$, we can read off the following conversions:
\begin{align*}
    [M] &= [E/c^2] \\
    [L] &= [\hbar c / E] \\
    [T] &= [\hbar / E]
\end{align*}
\par
This is super cool because it immediately tells us how many times we need to multiply or divide by $\hbar$ and $c$ to get a quantity that's a power of energy. For example, viscosity ($\mu$) has units of
\[ [\mu] = [M] \cdot [L]^{-1} \cdot [T]^{-1} = [E/c^2] \cdot [\hbar c / E]^{-1} \cdot [\hbar / E]^{-1} = [E]^3 \cdot [\hbar]^{-2} \cdot [c]^{-3} \]
so to get $\mu$ from SI units to natural units, we need to multiply by $\hbar^2c^3$.
\par
Suppose the quantity we want to convert to natural units is $\mu = 3 \, \mathrm{kg} \, \mathrm{m}^{-1} \, \mathrm{s}^{-1}$. Multiplying by $\hbar^2 c^3$ is annoying because we would still need to do a lot of math, but there is another way to do this.
\par
First, take the logarithm of the quantity we want to convert. \textit{Note that we're using the base 10 log here instead of the natural log, since this will make it easier to convert between $eV$, $keV$, $MeV$, and $GeV$ later on.}
\[ \log (\mu) = \log(3) + \log (\mathrm{kg}) - \log(\mathrm{m}) - \log(\mathrm{s}) \in \operatorname{span}_\mathbb{R} \left( \left\{ 1, \log(\mathrm{kg}), \log(\mathrm{m}), \log(\mathrm{s}) \right\} \right) \]
can be written in the $(1, \log(\mathrm{kg}), \log(\mathrm{m}), \log(\mathrm{s}))$ basis as
\[ \begin{bmatrix}
    \log 3 \\
    1 \\
    -1 \\
    -1
\end{bmatrix} = \begin{bmatrix}
    0.477121 \\
    1 \\
    -1 \\
    -1
\end{bmatrix} \]
and we can convert that to the $(1, \log(\mathrm{eV}), \log(\mathrm{\hbar}), \log(\mathrm{c}))$ basis by left-multiplying by the inverse of some matrix $B$, where $B$ is the matrix we left-multiply by to convert from the $(1, \log(\mathrm{eV}), \log(\hbar), \log(c))$ basis to the $(1, \log(\mathrm{kg}), \log(\mathrm{m}), \log(\mathrm{s}))$ basis.
\begin{align*}
    B &= \begin{bmatrix}
    1 & -18.795290 & -33.976924 & 8.476821 \\
    0 & 1 & 1 & 0 \\
    0 & 2 & 2 & 1 \\
    0 & -2 & -1 & -1
\end{bmatrix} \\
    B^{-1} &= \begin{bmatrix}
        1 & 35.748931 & 6.704814 & 15.181634 \\
        0 & 1 & -1 & -1 \\
        0 & 0 & 1 & 1 \\
        0 & -2 & 1 & 0
    \end{bmatrix}
\end{align*}
The matrix $B$ was derived by writing out 1, $\mathrm{eV}$, $\hbar$, and $c$ in SI units and taking the base 10 log of both sides to obtain a system of equations. For example,
\begin{align*}
    c &= 299792458 \, \mathrm{m} / \mathrm{s} \\
    \log(c) &= 8.476821 + 0 \cdot \log(\mathrm{kg}) + 1 \cdot \log(\mathrm{m}) + (-1) \cdot \log(\mathrm{s})
\end{align*}
Note that if we leave off the ``real component" (in that case, the 8.476821 term), this process is equivalent to deriving the matrix $A$ that we used earlier, which is why the bottom-right $3 \times 3$ submatrix of $B$ is $A$, and the bottom-right $3 \times 3$ submatrix of $B^{-1}$ is $A^{-1}$.
\par
If we left-multiply the vector we wrote for $\log(\mu)$ (in the $(1, \log(\mathrm{kg}), \log(\mathrm{m}), \log(\mathrm{s}))$ basis) by $B^{-1}$, we get
\[ \begin{bmatrix}
        1 & 35.748931 & 6.704814 & 15.181634 \\
        0 & 1 & -1 & -1 \\
        0 & 0 & 1 & 1 \\
        0 & -2 & 1 & 0
    \end{bmatrix} \cdot \begin{bmatrix}
    0.477121 \\
    1 \\
    -1 \\
    -1
\end{bmatrix} = \begin{bmatrix}
    14.339605 \\
    3 \\
    -2 \\
    -3
\end{bmatrix} \]
which means that
\[ \mu = 3 \, \mathrm{kg} \, \mathrm{m}^{-1} \, \mathrm{s}^{-1} = 10^{14.339605} \, \mathrm{eV}^3 \, \hbar^{-2} \, c^{-3} \]
so in natural units (the unit system where $\hbar=1=c$), $\mu$ is
\[ 10^{14.339605} \, \mathrm{eV}^3 = 2.185770247 \times 10^{14} \, \mathrm{eV}^3. \]

\end{document}
