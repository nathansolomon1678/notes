\documentclass[class=article, crop=false]{standalone}
\usepackage[margin=1in]{geometry}
\usepackage[linesnumbered,ruled,vlined]{algorithm2e}
\usepackage{amsfonts}
\usepackage{amsmath}
\usepackage{amssymb}
\usepackage{amsthm}
\usepackage{enumitem}
\usepackage{fancyhdr}
\usepackage{hyperref}
\usepackage{minted}
\usepackage{multicol}
\usepackage{pdfpages}
\usepackage{standalone}
\usepackage[many]{tcolorbox}
\usepackage{tikz-cd}
\usepackage{transparent}
\usepackage{xcolor}
% \tcbuselibrary{minted}

\author{Nathan Solomon}

\newcommand{\fig}[1]{
    \begin{center}
        \includegraphics[width=\textwidth]{#1}
    \end{center}
}

% Math commands
\renewcommand{\d}{\mathrm{d}}
\DeclareMathOperator{\id}{id}
\DeclareMathOperator{\im}{im}
\DeclareMathOperator{\proj}{proj}
\DeclareMathOperator{\Span}{span}
\DeclareMathOperator{\Tr}{Tr}
\DeclareMathOperator{\tr}{tr}
\DeclareMathOperator{\ad}{ad}
\DeclareMathOperator{\ord}{ord}
%%%%%%%%%%%%%%% \DeclareMathOperator{\sgn}{sgn}
\DeclareMathOperator{\Aut}{Aut}
\DeclareMathOperator{\Inn}{Inn}
\DeclareMathOperator{\Out}{Out}
\DeclareMathOperator{\stab}{stab}

\newcommand{\N}{\ensuremath{\mathbb{N}}}
\newcommand{\Z}{\ensuremath{\mathbb{Z}}}
\newcommand{\Q}{\ensuremath{\mathbb{Q}}}
\newcommand{\R}{\ensuremath{\mathbb{R}}}
\newcommand{\C}{\ensuremath{\mathbb{C}}}
\renewcommand{\H}{\ensuremath{\mathbb{H}}}
\newcommand{\F}{\ensuremath{\mathbb{F}}}

\newcommand{\E}{\ensuremath{\mathbb{E}}}
\renewcommand{\P}{\ensuremath{\mathbb{P}}}

\newcommand{\es}{\ensuremath{\varnothing}}
\newcommand{\inv}{\ensuremath{^{-1}}}
\newcommand{\eps}{\ensuremath{\varepsilon}}
\newcommand{\del}{\ensuremath{\partial}}
\renewcommand{\a}{\ensuremath{\alpha}}

\newcommand{\abs}[1]{\ensuremath{\left\lvert #1 \right\rvert}}
\newcommand{\norm}[1]{\ensuremath{\left\lVert #1\right\rVert}}
\newcommand{\mean}[1]{\ensuremath{\left\langle #1 \right\rangle}}
\newcommand{\floor}[1]{\ensuremath{\left\lfloor #1 \right\rfloor}}
\newcommand{\ceil}[1]{\ensuremath{\left\lceil #1 \right\rceil}}
\newcommand{\bra}[1]{\ensuremath{\left\langle #1 \right\rvert}}
\newcommand{\ket}[1]{\ensuremath{\left\lvert #1 \right\rangle}}
\newcommand{\braket}[2]{\ensuremath{\left.\left\langle #1\right\vert #2 \right\rangle}}

\newcommand{\catname}[1]{{\normalfont\textbf{#1}}}

\newcommand{\up}{\ensuremath{\uparrow}}
\newcommand{\down}{\ensuremath{\downarrow}}

% Custom environments
\newtheorem{thm}{Theorem}[section]

\definecolor{probBackgroundColor}{RGB}{250,240,240}
\definecolor{probAccentColor}{RGB}{140,40,0}
\newenvironment{prob}{
    \stepcounter{thm}
    \begin{tcolorbox}[
        boxrule=1pt,
        sharp corners,
        colback=probBackgroundColor,
        colframe=probAccentColor,
        borderline west={4pt}{0pt}{probAccentColor},
        breakable
    ]
    \color{probAccentColor}\textbf{Problem \thethm.} \color{black}
} {
    \end{tcolorbox}
}

\definecolor{exampleBackgroundColor}{RGB}{212,232,246}
\newenvironment{example}{
    \stepcounter{thm}
    \begin{tcolorbox}[
      boxrule=1pt,
      sharp corners,
      colback=exampleBackgroundColor,
      breakable
    ]
    \textbf{Example \thethm.}
} {
    \end{tcolorbox}
}

\definecolor{propBackgroundColor}{RGB}{255,245,220}
\definecolor{propAccentColor}{RGB}{150,100,0}
\newenvironment{prop}{
    \stepcounter{thm}
    \begin{tcolorbox}[
        boxrule=1pt,
        sharp corners,
        colback=propBackgroundColor,
        colframe=propAccentColor,
        breakable
    ]
    \color{propAccentColor}\textbf{Proposition \thethm. }\color{black}
} {
    \end{tcolorbox}
}

\definecolor{thmBackgroundColor}{RGB}{235,225,245}
\definecolor{thmAccentColor}{RGB}{50,0,100}
\renewenvironment{thm}{
    \stepcounter{thm}
    \begin{tcolorbox}[
        boxrule=1pt,
        sharp corners,
        colback=thmBackgroundColor,
        colframe=thmAccentColor,
        breakable
    ]
    \color{thmAccentColor}\textbf{Theorem \thethm. }\color{black}
} {
    \end{tcolorbox}
}

\definecolor{corBackgroundColor}{RGB}{240,250,250}
\definecolor{corAccentColor}{RGB}{50,100,100}
\newenvironment{cor}{
    \stepcounter{thm}
    \begin{tcolorbox}[
        enhanced,
        boxrule=0pt,
        frame hidden,
        sharp corners,
        colback=corBackgroundColor,
        borderline west={4pt}{0pt}{corAccentColor},
        breakable
    ]
    \color{corAccentColor}\textbf{Corollary \thethm. }\color{black}
} {
    \end{tcolorbox}
}

\definecolor{lemBackgroundColor}{RGB}{255,245,235}
\definecolor{lemAccentColor}{RGB}{250,125,0}
\newenvironment{lem}{
    \stepcounter{thm}
    \begin{tcolorbox}[
        enhanced,
        boxrule=0pt,
        frame hidden,
        sharp corners,
        colback=lemBackgroundColor,
        borderline west={4pt}{0pt}{lemAccentColor},
        breakable
    ]
    \color{lemAccentColor}\textbf{Lemma \thethm. }\color{black}
} {
    \end{tcolorbox}
}

\definecolor{proofBackgroundColor}{RGB}{255,255,255}
\definecolor{proofAccentColor}{RGB}{80,80,80}
\renewenvironment{proof}{
    \begin{tcolorbox}[
        enhanced,
        boxrule=1pt,
        sharp corners,
        colback=proofBackgroundColor,
        colframe=proofAccentColor,
        borderline west={4pt}{0pt}{proofAccentColor},
        breakable
    ]
    \color{proofAccentColor}\emph{\textbf{Proof. }}\color{black}
} {
    \qed \end{tcolorbox}
}

\definecolor{noteBackgroundColor}{RGB}{240,250,240}
\definecolor{noteAccentColor}{RGB}{30,130,30}
\newenvironment{note}{
    \begin{tcolorbox}[
        enhanced,
        boxrule=0pt,
        frame hidden,
        sharp corners,
        colback=noteBackgroundColor,
        borderline west={4pt}{0pt}{noteAccentColor},
        breakable
    ]
    \color{noteAccentColor}\textbf{Note. }\color{black}
} {
    \end{tcolorbox}
}



\fancyhf{}
\lhead{Nathan Solomon}
\rhead{Page \thepage}
\pagestyle{fancy}

\begin{document}
\section{4/3/2024 lecture}

\subsection{Syllabus}
The instructor's email is \url{ryan.thorngren@physics.ucla.edu} and his office hours will probably be Fridays from 1-2pm. The TA, Yanyan Li, can be reached at \url{yanyanli@ucla.edu}
\par
Grading is based on weekly problem sets, which will be due Fridays. To get an A, you need to put in a good effort on all of them.
\par
There course topics and textbooks are:
\begin{itemize}
    \item Group theory and representation theory of finite group (Artin's \emph{Algebra})
    \item Crystallographic groups (Sternberg's \emph{Group Theory and Physics})
    \item Lie groups (Hall's \emph{Lie Groups, Lie Algebras, and Representations} and Georgi's \emph{Lie Algebras in Particle Physics})
    \item If we have spare time, we can cover bector bundles and characteristic classes (Nakahara's \emph{Geometry, Topology, and Physics})
\end{itemize}

\subsection{Intro to groups}
A \emph{group} is a pair $(G, \cdot)$ where $G$ is a set and $\cdot: G \times G \rightarrow G$ is a binary operator, such that the following are all true.
\begin{itemize}
    \item (Unital axiom) There is some $e \in G$, called the \emph{identity element}, such that for any $g \in G$, $eg=g=ge$.
    \item (Associative axiom) For any 3 elements $a,b,c \in G$, $(ab)c=a(bc)$.
    \item (Invertible axiom) For any $g \in G$, there exists some element $g^{-1} \in G$ such that $g g^{-1}=e=g^{-1}g$.
\end{itemize}
\par
A group is called \emph{abelian} iff $ab=ba$ for any two elements $a,b$ in the group. It is called \emph{nonabelian} iff it is not abelian.
\begin{note}
    Depending on context, the operator $\cdot$ can be called ``addition", ``multiplication", ``composition", or just ``the group operation". It is often written as $+$, $\times$, $*$, or $\circ$. Repeated multiplication is denoted by $g^n$, where $g \in G$ and $n \in \N$, and repeated addition is denoted by $ng=g+g+\cdots +g$ ($n$ times). Additive notation is pretty much only used for abelian groups, but not every abelian group is written additively.
\end{note}
Physicist generally care about groups because they represent symmetries. For example, if you multiply every electric charge in the universe by a complex number with magnitude 1 (that is, by an element of $U(1)$), the Lagrangian doesn't change.
\begin{example}
    The set of transformations of a regular polyhedron forms a group, when you define the group operation to be composition of transformations. Note that no matter which of the 5 regular polyhedra we choose, this is not a commutative group.
\end{example}
\begin{prop}
    The rotational group of a tetrahedron is $A_4$, the rotational group of a cube or octahedron is $S_4$, and the rotational group of a dodecahedron or icosahedron is $A_5$. We will not prove this, for now.
\end{prop}
For any integer $n > 0$, we define the \emph{symmetric group} $S_n$ to be the set of permutations of $n$ elements, with the group operation being composition. Alternatively, we can define $S_n$ to be the set of bijections $\sigma: \left\{ 1, \dots, n \right\} \rightarrow \left\{ 1, \dots, n \right\}$.
\par
\begin{lem}
    Every permutation is a product of disjoint cyclic permutations.
\end{lem}
\begin{cor}
    Since every cyclic permutation is a product of transpositions, we can also say that every permutation is a product of transpositions.
\end{cor}
\par
The \emph{order} of a finite group $G$, denoted $|G|$, is the number of elements in $G$. For example, the order of $S_n$ is $|S_n|=n!$. The \emph{order} of an element $g \in G$ is the smallest integer $n>0$ such that $g^n=e$.
\begin{note}
    It's easy to get confused between the terms ``symmetry group" and ``symmetric group. The symmetry group of something is the set of transformations that preserve some property of it. The symmetric group of a set is the group of permutations of elements in that set.
\end{note}

\subsection{Group homomorphisms}
Let $G,H$ be groups. A map $\varphi: G \rightarrow H$ is called a \emph{(group) homomorphism} iff for every $g_1, g_2 \in G, \varphi(g_1g_2)=\varphi(g_1)\varphi(g_2)$. Furthermore, a group homomorphism $\varphi$ is called a \emph{(group) isomorphism} iff it is also bijective.
\par
Group isomorphism is an equivalence relation, which we denote with $\cong$.
\begin{example}
    The map $\exp: \R \rightarrow \R_{\neq 0}$ is a group isomorphism (from $(\R, +)$ to $(\R_{\neq 0}, \times)$), because $e^{a+b}=e^ae^b$ for any $a,b \in \R$ and it is bijective.
\end{example}
One helpful trick for proving that a homomorphism is an isomorphism is to note that a group homomorphism $\varphi: G \rightarrow H$ is injective iff $\varphi(g)=e_H$ implies $g=e_G$. This is fairly easy to prove.
\begin{thm}
    Let $T$ be the set of symmetries of a regular tetrahedron. Then $T \cong S_4$.
\end{thm}
\begin{proof}
    Label the edges of a regular tetrahedron as $ \left\{ 1,2,3,4 \right\}$ and let $\varphi: T \rightarrow S_4$ be the map which takes a transformation of the tetrahedron to the corresponding permutation of its vertices. This is clearly an injective homomorphism, so we just need to show that it's also surjective.
    \bigskip
    \par
    For any two vertices of a regular tetrahedron, consider the pependicular bisector of the edge that joins them. Reflection across that plane is an element of $T$, and $\varphi$ takes that element to a transposition in $S_4$. We can obtain every transposition in $S_4$ this way, and we know that every element of $S_4$ is a product of transpositions, so $\varphi$ is surjective.
\end{proof}
\begin{prop}
    More generally, the symmetry group of an $n$-simplex is $S_{n+1}$.
\end{prop}
\begin{proof}
    The proof of this is exactly the same as it is for a 3-simplex (tetrahedron), except the perpendicular bisector is now an $(n-1)$ dimensional hyperplane instead of a plane, and the number of vertices to be permuted is now $n+1$ instead of 4.
\end{proof}

\subsection{Generators and relations}
A subset $S \subset G$ is said to \emph{generate} $G$ iff every element of $G$ can be written as a finite product of the elements of $S$ and the inverses of the elements of $S$. Elements of the generating set $S$ are called \emph{generators}.
\par
A \emph{word} in a group $G$ is an expression of the form $g_{i_1}^{\pm 1} g_{i_2}^{\pm 1} \cdots g_{i_n}^{\pm 1}$. Two words are considered different if they are written differently, even if the expressions they represent are equal. A word which represents an expression equal to the identity is called a \emph{relation}.
\begin{example}
    The symmetric group $S_n$ is generated by the transpositions $ \left\{ s_1, s_2, \dots, s_{n-1} \right\}$, where $s_i = (i, i+1)$. Since every transposition has order 2, we have the relation $s_i^2=e$. We also have the relation $s_is_j=s_js_i$ whenever $|i-j|>1$, because $s_i$ and $s_j$ are disjoint -- this can also be written as $s_is_js_i^{-1}s_j^{-1}$. Lastly, we have the relation that $(s_is_{i+1})^3=e$ (whenever $i<n-1$), because $s_is_{i+1}$ is a 3-cycle.
\end{example}
Given a set $S$ of symbols and a set $R$ of relations in $S$, we define the \emph{group presentation} $ \braket{S}{R}$ to be the group of equivalence classes of words in $S$ (with the equivalence relation making every element of $R$ equal to the identity, and with the group operation being concatenation).
\par
The \emph{cyclic group (of $n$ elements)} is the group $C_n := \braket{s}{s^n}$. $C_n$ is an abelian group of order $n$, in which $s$ has order $n$.
\begin{prop}
    \[ S_n \cong \braket{s_1, \dots, s_{n-1}}{ \left\{ s_i^2 \right\} \cup \left\{ s_is_js_i^{-1}s_j^{-1} : |i-j|>1 \right\} \cup \left\{ (s_is_{i+1})^3 \right\}} \]
\end{prop}
\begin{proof}
    The proof of this is a painful exercise. You would need to show that there is a map from the right hand side to $S_n$ which is an injective and surjective homomorphism.
\end{proof}
\begin{thm}
    Good news: every group has a presentation. Bad news: given a finite presentation, there is no algorithm to determine any of the following.
    \begin{itemize}
        \item Whether two words are equal
        \item Whether the group is finite
        \item Whether the group is \emph{trivial} (meaning isomorphic to the \emph{trivial group} $\{e\}=1$)
    \end{itemize}
    \par
    This is because simplifying a word often requires expanding it before simplifying it, and there is no upper bound on how much you need to expand it before you can start simplifying. Therefore, even a brute force algorithm for any of those three problems could take arbitrarily long to run.
\end{thm}

\end{document}
