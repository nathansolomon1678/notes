\documentclass[class=article, crop=false]{standalone}
\usepackage[margin=1in]{geometry}
\usepackage[linesnumbered,ruled,vlined]{algorithm2e}
\usepackage{amsfonts}
\usepackage{amsmath}
\usepackage{amssymb}
\usepackage{amsthm}
\usepackage{chemformula}
\usepackage{enumitem}
\usepackage{fancyhdr}
\usepackage{graphicx}
\usepackage{hyperref}
\usepackage{listings}
\usepackage{minted}
\usepackage{multicol}
\usepackage{pdfpages}
\usepackage{siunitx}
\usepackage{standalone}
\usepackage{svg}
\usepackage[many]{tcolorbox}
\usepackage{tikz-cd}
\usepackage{transparent}
\usepackage{xcolor}
% \tcbuselibrary{minted}

\author{Nathan Solomon}

\newcommand{\fig}[1]{
    \begin{center}
        \includegraphics[width=\textwidth]{#1}
    \end{center}
}

% Math commands
\renewcommand{\d}{\mathrm{d}}
\DeclareMathOperator{\id}{id}
\DeclareMathOperator{\im}{im}
\DeclareMathOperator{\proj}{proj}
\DeclareMathOperator{\Span}{span}
\DeclareMathOperator{\Tr}{Tr}
\DeclareMathOperator{\tr}{tr}
\DeclareMathOperator{\ad}{ad}
\DeclareMathOperator{\ord}{ord}
%%%%%%%%%%%%%%% \DeclareMathOperator{\sgn}{sgn}
\DeclareMathOperator{\Aut}{Aut}
\DeclareMathOperator{\Inn}{Inn}
\DeclareMathOperator{\Out}{Out}
\DeclareMathOperator{\stab}{stab}

\newcommand{\N}{\ensuremath{\mathbb{N}}}
\newcommand{\Z}{\ensuremath{\mathbb{Z}}}
\newcommand{\Q}{\ensuremath{\mathbb{Q}}}
\newcommand{\R}{\ensuremath{\mathbb{R}}}
\newcommand{\C}{\ensuremath{\mathbb{C}}}
\renewcommand{\H}{\ensuremath{\mathbb{H}}}
\newcommand{\F}{\ensuremath{\mathbb{F}}}

\newcommand{\E}{\ensuremath{\mathbb{E}}}
\renewcommand{\P}{\ensuremath{\mathbb{P}}}

\newcommand{\es}{\ensuremath{\varnothing}}
\newcommand{\inv}{\ensuremath{^{-1}}}
\newcommand{\eps}{\ensuremath{\varepsilon}}
\newcommand{\del}{\ensuremath{\partial}}
\renewcommand{\a}{\ensuremath{\alpha}}

\newcommand{\abs}[1]{\ensuremath{\left\lvert #1 \right\rvert}}
\newcommand{\norm}[1]{\ensuremath{\left\lVert #1\right\rVert}}
\newcommand{\mean}[1]{\ensuremath{\left\langle #1 \right\rangle}}
\newcommand{\floor}[1]{\ensuremath{\left\lfloor #1 \right\rfloor}}
\newcommand{\ceil}[1]{\ensuremath{\left\lceil #1 \right\rceil}}
\newcommand{\bra}[1]{\ensuremath{\left\langle #1 \right\rvert}}
\newcommand{\ket}[1]{\ensuremath{\left\lvert #1 \right\rangle}}
\newcommand{\braket}[2]{\ensuremath{\left.\left\langle #1\right\vert #2 \right\rangle}}

\newcommand{\catname}[1]{{\normalfont\textbf{#1}}}

\newcommand{\up}{\ensuremath{\uparrow}}
\newcommand{\down}{\ensuremath{\downarrow}}

% Custom environments
\newtheorem{thm}{Theorem}[section]

\definecolor{probBackgroundColor}{RGB}{250,240,240}
\definecolor{probAccentColor}{RGB}{140,40,0}
\newenvironment{prob}{
    \stepcounter{thm}
    \begin{tcolorbox}[
        boxrule=1pt,
        sharp corners,
        colback=probBackgroundColor,
        colframe=probAccentColor,
        borderline west={4pt}{0pt}{probAccentColor},
        breakable
    ]
    \color{probAccentColor}\textbf{Problem \thethm.} \color{black}
} {
    \end{tcolorbox}
}

\definecolor{exampleBackgroundColor}{RGB}{212,232,246}
\newenvironment{example}{
    \stepcounter{thm}
    \begin{tcolorbox}[
      boxrule=1pt,
      sharp corners,
      colback=exampleBackgroundColor,
      breakable
    ]
    \textbf{Example \thethm.}
} {
    \end{tcolorbox}
}

\definecolor{propBackgroundColor}{RGB}{255,245,220}
\definecolor{propAccentColor}{RGB}{150,100,0}
\newenvironment{prop}{
    \stepcounter{thm}
    \begin{tcolorbox}[
        boxrule=1pt,
        sharp corners,
        colback=propBackgroundColor,
        colframe=propAccentColor,
        breakable
    ]
    \color{propAccentColor}\textbf{Proposition \thethm. }\color{black}
} {
    \end{tcolorbox}
}

\definecolor{thmBackgroundColor}{RGB}{235,225,245}
\definecolor{thmAccentColor}{RGB}{50,0,100}
\renewenvironment{thm}{
    \stepcounter{thm}
    \begin{tcolorbox}[
        boxrule=1pt,
        sharp corners,
        colback=thmBackgroundColor,
        colframe=thmAccentColor,
        breakable
    ]
    \color{thmAccentColor}\textbf{Theorem \thethm. }\color{black}
} {
    \end{tcolorbox}
}

\definecolor{corBackgroundColor}{RGB}{240,250,250}
\definecolor{corAccentColor}{RGB}{50,100,100}
\newenvironment{cor}{
    \stepcounter{thm}
    \begin{tcolorbox}[
        enhanced,
        boxrule=0pt,
        frame hidden,
        sharp corners,
        colback=corBackgroundColor,
        borderline west={4pt}{0pt}{corAccentColor},
        breakable
    ]
    \color{corAccentColor}\textbf{Corollary \thethm. }\color{black}
} {
    \end{tcolorbox}
}

\definecolor{lemBackgroundColor}{RGB}{255,245,235}
\definecolor{lemAccentColor}{RGB}{250,125,0}
\newenvironment{lem}{
    \stepcounter{thm}
    \begin{tcolorbox}[
        enhanced,
        boxrule=0pt,
        frame hidden,
        sharp corners,
        colback=lemBackgroundColor,
        borderline west={4pt}{0pt}{lemAccentColor},
        breakable
    ]
    \color{lemAccentColor}\textbf{Lemma \thethm. }\color{black}
} {
    \end{tcolorbox}
}

\definecolor{proofBackgroundColor}{RGB}{255,255,255}
\definecolor{proofAccentColor}{RGB}{80,80,80}
\renewenvironment{proof}{
    \begin{tcolorbox}[
        enhanced,
        boxrule=1pt,
        sharp corners,
        colback=proofBackgroundColor,
        colframe=proofAccentColor,
        borderline west={4pt}{0pt}{proofAccentColor},
        breakable
    ]
    \color{proofAccentColor}\emph{\textbf{Proof. }}\color{black}
} {
    \qed \end{tcolorbox}
}

\definecolor{noteBackgroundColor}{RGB}{240,250,240}
\definecolor{noteAccentColor}{RGB}{30,130,30}
\newenvironment{note}{
    \begin{tcolorbox}[
        enhanced,
        boxrule=0pt,
        frame hidden,
        sharp corners,
        colback=noteBackgroundColor,
        borderline west={4pt}{0pt}{noteAccentColor},
        breakable
    ]
    \color{noteAccentColor}\textbf{Note. }\color{black}
} {
    \end{tcolorbox}
}



\fancyhf{}
\lhead{Nathan Solomon}
\rhead{Page \thepage}
\pagestyle{fancy}

\begin{document}
\section{4/19/2024 lecture}

\subsection{Affine transformations of $\R^d$}

Recall that there are 32 finite subgroups of $O(3)$ which act on a lattice. Since a lattice in 3 dimensions can be defined by 3 basis vectors, the image of any representation in that basis will have only integer entries.
\par
\begin{note}
    Any repeating pattern repeats according to translations which define a lattice group.
\end{note}
Let $G$ be the group of all translations and orthogonal transformations in $\R^d$, with the group operation being composition. The subgroup of translations (the Lie group $\R^d$) is normal in $G$.
\par
Let $\R^d \rtimes O(d) = \left\{ (u, M) \in \R^d \times O(d) \right\}$ be the set with the group operation $(u_1, M_1) \circ (u_2, M_2) = (u_1 + M_1 u_2, M_1 M_2)$.
\begin{prob}
    Verify that $\R^d \rtimes O(d)$ is a group.
\end{prob}
\begin{prop}
    There is an isomorphism $\varphi: \R^d \times O(d) \rightarrow G$.
\end{prop}
An \emph{automorphism} of $G$ is an isomorphism $\varphi: G \rightarrow G$. The \emph{automorphism group} $\Aut (G)$ is the set of automorphisms of $G$ with the group operation being composition.
\par
An \emph{action} of a group $H$ on a group $G$ is a homomorphism from $H$ to $\Aut(G)$.
\par
Given two groups $G, H$ and an action $\alpha: H \rightarrow \Aut (G)$, define the \emph{semi-direct product} $\rtimes_\alpha$ so that $G \rtimes_\alpha H$ is the set $G \times H$ with the group operation $(g_1, h_1) \circ (g_2, h_2) = (g_1 \alpha(h_1)(g_2), h_1 h_2)$.
\begin{prob}
    Show that for any two groups $G$ and $H$ and any action $\alpha: H \rightarrow \Aut(G)$, the semi-direct product $G \rtimes_\alpha H$ is a group. Also show that $G \trianglelefteq G \rtimes_\alpha H$.
\end{prob}
\begin{prob}
    Show that $\Z_3 \rtimes \Z_2 \cong S_3$.
\end{prob}

\subsection{Crystallographic groups}
A group $G$ is called a \emph{$d$-dimensional crystallographic group} iff $G <( \R^d \rtimes O(d))$ is discrete and $G \cap ( \R^d \times \left\{ I \right\})$ is a $d$-dimensional lattice group $L = \mean{v_1, v_2, \dots, v_d} < \R^d$.
\begin{prop}
    $L \trianglelefteq G$. The proof of this is an exercise.
\end{prop}
\begin{prob}
    Show that $G/L$ is a finite subgroup of $O(d)$, and if $d=3$, then $G/L$ is one of the 32 groups we classified.
\end{prob}
We sometimes call $G/L$ the ``point group", and denote it by $G_{pt}$. There is an exact sequence
\begin{center}
\begin{tikzcd}[column sep=small]
    L \arrow[hookrightarrow]{r} & G \arrow[twoheadrightarrow]{r} & G/L.
\end{tikzcd}
\end{center}
Alternatively, given $G_{pt}$ and the lattice group $L$ on which it acts, we can define $G$ as $L \rtimes G_{pt}$. In $G$, the stabilizer of the origin is $G_{pt}$, and the stabilizer of any other point in $L$ is conjugate to $G_{pt}$.
\par
If $d=2$, these groups are called ``wallpaper groups".
\par
TALK ABOUT NON-SYMMORPHIC GROUPS AND GLIDE/SCREW GROUPS, ALSO EXPLAIN WHAT THE 32 POINT GROUPS (IN 3D) ARE

\subsection{Recognizing semi-direct products}
Given an extension
\begin{center}
\begin{tikzcd}[column sep=small]
    0 \arrow[hookrightarrow]{r} & L \arrow[hookrightarrow]{r} & G \arrow[twoheadrightarrow, "\pi"]{r} & G_{pt} \arrow[twoheadrightarrow]{r} & 0,
\end{tikzcd}
\end{center}
a \emph{section} is a map $s: G_{pt} \rightarrow G$ (not necessarily a homomorphism) such that $s(e_{G_{pt}}) = e_G$ and $\pi \circ s = \id_G$.
\begin{thm}
    Thee exists a section $s$ which is a homomorphism iff $G = L \rtimes G_{pt}$.
\end{thm}
\begin{proof}
    REDO THIS PROOF
\end{proof}
\begin{prop}
    A $d$-dimensional space group is symmorphic iff it is a semi-direct product. JUSTIFY WHY THIS IS TRUE AND DEFINE WHAT IT MEANS TO BE SYMMORPHIC
\end{prop}

\subsection{Group cohomology}
SEE PAGE 91 OF BROWN'S ``COHOMOLOGY OF GROUPS"
\par
There is a technique that can be used to measure how far a group is from being a semidirect product. GET NOTES FROM CLASSMATES ABOUT THIS, AND INCLUDE THE 2-COCYCLE EQUATION. THE LINEARITY OF THAT EQUATION CAN BE USED TO CLASSIFY AND ENUMERATE THE 230 SPACE GROUPS.
\par
WHAT IS THE DIFFERENCE BETWEEN A CRYSTALLOGRAPHIC GROUP AND A SPACE GROUP?

\end{document}
