\documentclass[class=article, crop=false]{standalone}
\usepackage[margin=1in]{geometry}
\usepackage[linesnumbered,ruled,vlined]{algorithm2e}
\usepackage{amsfonts}
\usepackage{amsmath}
\usepackage{amssymb}
\usepackage{amsthm}
\usepackage{enumitem}
\usepackage{fancyhdr}
\usepackage{hyperref}
\usepackage{minted}
\usepackage{multicol}
\usepackage{pdfpages}
\usepackage{standalone}
\usepackage[many]{tcolorbox}
\usepackage{tikz-cd}
\usepackage{transparent}
\usepackage{xcolor}
% \tcbuselibrary{minted}

\author{Nathan Solomon}

\newcommand{\fig}[1]{
    \begin{center}
        \includegraphics[width=\textwidth]{#1}
    \end{center}
}

% Math commands
\renewcommand{\d}{\mathrm{d}}
\DeclareMathOperator{\id}{id}
\DeclareMathOperator{\im}{im}
\DeclareMathOperator{\proj}{proj}
\DeclareMathOperator{\Span}{span}
\DeclareMathOperator{\Tr}{Tr}
\DeclareMathOperator{\tr}{tr}
\DeclareMathOperator{\ad}{ad}
\DeclareMathOperator{\ord}{ord}
%%%%%%%%%%%%%%% \DeclareMathOperator{\sgn}{sgn}
\DeclareMathOperator{\Aut}{Aut}
\DeclareMathOperator{\Inn}{Inn}
\DeclareMathOperator{\Out}{Out}
\DeclareMathOperator{\stab}{stab}

\newcommand{\N}{\ensuremath{\mathbb{N}}}
\newcommand{\Z}{\ensuremath{\mathbb{Z}}}
\newcommand{\Q}{\ensuremath{\mathbb{Q}}}
\newcommand{\R}{\ensuremath{\mathbb{R}}}
\newcommand{\C}{\ensuremath{\mathbb{C}}}
\renewcommand{\H}{\ensuremath{\mathbb{H}}}
\newcommand{\F}{\ensuremath{\mathbb{F}}}

\newcommand{\E}{\ensuremath{\mathbb{E}}}
\renewcommand{\P}{\ensuremath{\mathbb{P}}}

\newcommand{\es}{\ensuremath{\varnothing}}
\newcommand{\inv}{\ensuremath{^{-1}}}
\newcommand{\eps}{\ensuremath{\varepsilon}}
\newcommand{\del}{\ensuremath{\partial}}
\renewcommand{\a}{\ensuremath{\alpha}}

\newcommand{\abs}[1]{\ensuremath{\left\lvert #1 \right\rvert}}
\newcommand{\norm}[1]{\ensuremath{\left\lVert #1\right\rVert}}
\newcommand{\mean}[1]{\ensuremath{\left\langle #1 \right\rangle}}
\newcommand{\floor}[1]{\ensuremath{\left\lfloor #1 \right\rfloor}}
\newcommand{\ceil}[1]{\ensuremath{\left\lceil #1 \right\rceil}}
\newcommand{\bra}[1]{\ensuremath{\left\langle #1 \right\rvert}}
\newcommand{\ket}[1]{\ensuremath{\left\lvert #1 \right\rangle}}
\newcommand{\braket}[2]{\ensuremath{\left.\left\langle #1\right\vert #2 \right\rangle}}

\newcommand{\catname}[1]{{\normalfont\textbf{#1}}}

\newcommand{\up}{\ensuremath{\uparrow}}
\newcommand{\down}{\ensuremath{\downarrow}}

% Custom environments
\newtheorem{thm}{Theorem}[section]

\definecolor{probBackgroundColor}{RGB}{250,240,240}
\definecolor{probAccentColor}{RGB}{140,40,0}
\newenvironment{prob}{
    \stepcounter{thm}
    \begin{tcolorbox}[
        boxrule=1pt,
        sharp corners,
        colback=probBackgroundColor,
        colframe=probAccentColor,
        borderline west={4pt}{0pt}{probAccentColor},
        breakable
    ]
    \color{probAccentColor}\textbf{Problem \thethm.} \color{black}
} {
    \end{tcolorbox}
}

\definecolor{exampleBackgroundColor}{RGB}{212,232,246}
\newenvironment{example}{
    \stepcounter{thm}
    \begin{tcolorbox}[
      boxrule=1pt,
      sharp corners,
      colback=exampleBackgroundColor,
      breakable
    ]
    \textbf{Example \thethm.}
} {
    \end{tcolorbox}
}

\definecolor{propBackgroundColor}{RGB}{255,245,220}
\definecolor{propAccentColor}{RGB}{150,100,0}
\newenvironment{prop}{
    \stepcounter{thm}
    \begin{tcolorbox}[
        boxrule=1pt,
        sharp corners,
        colback=propBackgroundColor,
        colframe=propAccentColor,
        breakable
    ]
    \color{propAccentColor}\textbf{Proposition \thethm. }\color{black}
} {
    \end{tcolorbox}
}

\definecolor{thmBackgroundColor}{RGB}{235,225,245}
\definecolor{thmAccentColor}{RGB}{50,0,100}
\renewenvironment{thm}{
    \stepcounter{thm}
    \begin{tcolorbox}[
        boxrule=1pt,
        sharp corners,
        colback=thmBackgroundColor,
        colframe=thmAccentColor,
        breakable
    ]
    \color{thmAccentColor}\textbf{Theorem \thethm. }\color{black}
} {
    \end{tcolorbox}
}

\definecolor{corBackgroundColor}{RGB}{240,250,250}
\definecolor{corAccentColor}{RGB}{50,100,100}
\newenvironment{cor}{
    \stepcounter{thm}
    \begin{tcolorbox}[
        enhanced,
        boxrule=0pt,
        frame hidden,
        sharp corners,
        colback=corBackgroundColor,
        borderline west={4pt}{0pt}{corAccentColor},
        breakable
    ]
    \color{corAccentColor}\textbf{Corollary \thethm. }\color{black}
} {
    \end{tcolorbox}
}

\definecolor{lemBackgroundColor}{RGB}{255,245,235}
\definecolor{lemAccentColor}{RGB}{250,125,0}
\newenvironment{lem}{
    \stepcounter{thm}
    \begin{tcolorbox}[
        enhanced,
        boxrule=0pt,
        frame hidden,
        sharp corners,
        colback=lemBackgroundColor,
        borderline west={4pt}{0pt}{lemAccentColor},
        breakable
    ]
    \color{lemAccentColor}\textbf{Lemma \thethm. }\color{black}
} {
    \end{tcolorbox}
}

\definecolor{proofBackgroundColor}{RGB}{255,255,255}
\definecolor{proofAccentColor}{RGB}{80,80,80}
\renewenvironment{proof}{
    \begin{tcolorbox}[
        enhanced,
        boxrule=1pt,
        sharp corners,
        colback=proofBackgroundColor,
        colframe=proofAccentColor,
        borderline west={4pt}{0pt}{proofAccentColor},
        breakable
    ]
    \color{proofAccentColor}\emph{\textbf{Proof. }}\color{black}
} {
    \qed \end{tcolorbox}
}

\definecolor{noteBackgroundColor}{RGB}{240,250,240}
\definecolor{noteAccentColor}{RGB}{30,130,30}
\newenvironment{note}{
    \begin{tcolorbox}[
        enhanced,
        boxrule=0pt,
        frame hidden,
        sharp corners,
        colback=noteBackgroundColor,
        borderline west={4pt}{0pt}{noteAccentColor},
        breakable
    ]
    \color{noteAccentColor}\textbf{Note. }\color{black}
} {
    \end{tcolorbox}
}



\fancyhf{}
\lhead{Nathan Solomon}
\rhead{Page \thepage}
\pagestyle{fancy}

\begin{document}
\section{5/22/2024 lecture}
A Lie group representation is a smooth group homomorphism $\rho: G \rightarrow GL(n, \F)$. This induces a map $\rho_*: \mathfrak{g} \rightarrow \mathfrak{gl}(n, \F)$. Lets define a \textit{Lie algebra representation} as a Lie algebra homomorphism $R: \mathfrak{g} \rightarrow \mathfrak{gl}(n,\F)$.
\par
Note that in order for $R$ to be a Lie algebra homomorphism, it must satisfy the criterion that $R([A,B])=R(A)R(B)-R(B)R(A)$.
\begin{prop}
    If $G$ is connected and simply connected, then a Lie group representation of $G$ is equivalent to a Lie algebra representation of $\mathfrak{g}$.
\end{prop}
\begin{proof}
    Use the structure theorems.
\end{proof}
In general, if $G$ is a finite-dimensional matrix Lie group, we can consider $G$ as a subgroup of $GL(n, \F)$ (where $\F$ is either $\R$ or $\C$) and define its Lie group $\mathfrak{g}$ as
\[ \mathfrak{g} = \left\{ A \in \mathfrak{gl}(n, \F) : \forall t \in \F, e^{tA} \in G \right\}. \]
\begin{example}
    $\mathfrak{sl}(n, \F)$ is the set of $n \times n$ matrices over $\F$ with zero trace.
\end{example}
\begin{example}
    $\mathfrak{sl}(2,\C)$ is the set of matrices of the form $ \begin{bmatrix}
        a & b \\
        c & -a
    \end{bmatrix}$, which has dimension 6 over $\R$. We can define the following basis vectors:
    \[ h=\begin{bmatrix}
        1 & 0 \\
        0 & -1
    \end{bmatrix}, e=\begin{bmatrix}
        0 & 1 \\
        0 & 0
    \end{bmatrix}, f=\begin{bmatrix}
        0 & 0 \\
        1 & 0
    \end{bmatrix} \]
    so that $[h,e]=2e$ and $[e,f]=h$ and $[f,h]=2f$. You can think of this as a two state system where the lower state has $h$-eigenvalue $-1$ and the higher state has $h$-eigenvalue $1$. Then $e$ behaves as a raising operator and $f$ behaves as a lowering operator.
\end{example}
In general, a Lie algebra can be defined by basis vectors $ \left\{ A_i \right\}$ and ``structure constants" $t_{ij}$ satisfying $t_{ij}=-t_{ji}$ and $[A_i,A_j]=\sum_k t_{ij}^k A_k$
\begin{thm}
    In general, a Lie group and its corresponding Lie algebra will have the same dimension. PROVE THIS STATEMENT, ALSO FIND ANY EXCEPTIONS
\end{thm}
\begin{example}
    $\mathfrak{u}(n)$ is the set of complex-valued matrices $A$ such that $ \left( \exp(sA) \right)^\dag \left( \exp(sA) \right)=I_n$ for all $s \in \R$ (COULD WE HAVE SAID $s \in \C$ INSTEAD??? WHY NOT, WOULD IT IMPLY THAT $\mathfrak{u}(n)$ IS A COMPLEX MANIFORLD, WHICH IT ISNT??) Then we can write this as a formal power series in $s$ and use that to find that $A^\dag=-A$ ($A$ is ``anti-Hermitian"). In other words, $A$ is $i$ times some Hermitian matrix. Counting the degrees of freedom for a Hermitian matrix, we see that $\dim_\R \mathfrak{u}(n) = \dim_\R \left( U(n) \right) =n^2$.
    \par
    $\mathfrak{su}(n)$ is the Lie algebra of traceless anti-Hermitian matrices, so it has dimension $n^2-1$ over $\R$. Every $A \in \mathfrak{u}(n)$ can be written as $B+ \frac{\Tr(A)}{n} \cdot I$ for some $B \in \mathfrak{su}(n)$, which implies $\mathfrak{u}(n) = \mathfrak{su}(n) \oplus \R$.
\end{example}
A Lie subalgebra is a subalgebra of a Lie algebra which is closed under the Lie bracket (e.g. $\mathfrak{su}(n) \subset \mathfrak{u}(n)$).
\par
The set of $2 \times 2$ Hermitian matrices is a real vector space with basis vectors $ \left\{ I, \sigma_x, \sigma_y, \sigma_z \right\}$, where the Pauli spin matrices are
\[ \sigma_x = \begin{bmatrix}
    0 & 1 \\
    1 & 0
\end{bmatrix}, \sigma_y = \begin{bmatrix}
    0 & -i \\
    i & 0
\end{bmatrix}, \sigma_z = \begin{bmatrix}
    1 & 0 \\
    0 & -1
\end{bmatrix}. \]
Then the set of anti-Hermitian matrices is $\mathfrak{u}(2)=\Span_\R \left\{ iI, i \sigma_x, i\sigma_y, i \sigma_z \right\}$ and the set of traceless anti-Hermitian matrices is $\mathfrak{su}(2)=\Span_\R \left\{ i \sigma_x, i \sigma_y, i \sigma_z \right\}$.
\par
Define the Levi-Cevita symbol $\varepsilon^{ijk}$ to be $+1$ if $ijk$ is an even permutation of $123$ (or $xyz$, or $abc$), $-1$ if $ijk$ is an odd permutation of $123$, and $0$ if any indices are repeated (e.g. if $i=j$). It is implicitly summed over all $i,j,k$. Then
\[ \sigma_a \sigma_b = i \varepsilon^{abc} \sigma_c \]
and
\[ [i \sigma_a, i \sigma_b] = -2i \varepsilon^{abc} \sigma_c. \]
\subsection{Complexification}
The \textit{complexification} of a real Lie algebra $\mathfrak{g}$ is a complex Lie algebra $\mathfrak{g}^\C$ of elements of the form $A+iB$, where $A,B \in \mathfrak{g}$. The Lie bracket for $\mathfrak{g}^\C$ is then defined such that
\[ [A+iB,C+iD]=[A,C]+i[A,D]+i[B,C]-[B,D]. \]
\begin{prop}
    $\mathfrak{su}(n)^\C \cong \mathfrak{sl}(n,\C)$
\end{prop}
\begin{proof}
    Define $\varphi: \mathfrak{su}(n)^\C \rightarrow \mathfrak{sl}(n,\C)$ to be the Lie algebra homomorphism which takes $A+iB$ to FINISH THIS PROOF, AND USE THE SAME TRICK TO SHOW THAT $\mathfrak{u}(n)^\C \cong \mathfrak{gl}(n,\R)$ AND THAT $\mathfrak{gl}(n,\R)^\C = \mathfrak{gl}(n, \C)$.
\end{proof}

GET NOTES FROM RYAN ABOUT RELATING COMPLEX REPRESENTATIONS OF SU2 TO COMPLEX REPRESENTATIONS OF SL(2,C).

\subsection{Highest weight vectors}


\end{document}
