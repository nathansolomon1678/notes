\documentclass[class=article, crop=false]{standalone}
\usepackage[margin=1in]{geometry}
\usepackage[linesnumbered,ruled,vlined]{algorithm2e}
\usepackage{amsfonts}
\usepackage{amsmath}
\usepackage{amssymb}
\usepackage{amsthm}
\usepackage{enumitem}
\usepackage{fancyhdr}
\usepackage{hyperref}
\usepackage{minted}
\usepackage{multicol}
\usepackage{pdfpages}
\usepackage{standalone}
\usepackage[many]{tcolorbox}
\usepackage{tikz-cd}
\usepackage{transparent}
\usepackage{xcolor}
% \tcbuselibrary{minted}

\author{Nathan Solomon}

\newcommand{\fig}[1]{
    \begin{center}
        \includegraphics[width=\textwidth]{#1}
    \end{center}
}

% Math commands
\renewcommand{\d}{\mathrm{d}}
\DeclareMathOperator{\id}{id}
\DeclareMathOperator{\im}{im}
\DeclareMathOperator{\proj}{proj}
\DeclareMathOperator{\Span}{span}
\DeclareMathOperator{\Tr}{Tr}
\DeclareMathOperator{\tr}{tr}
\DeclareMathOperator{\ad}{ad}
\DeclareMathOperator{\ord}{ord}
%%%%%%%%%%%%%%% \DeclareMathOperator{\sgn}{sgn}
\DeclareMathOperator{\Aut}{Aut}
\DeclareMathOperator{\Inn}{Inn}
\DeclareMathOperator{\Out}{Out}
\DeclareMathOperator{\stab}{stab}

\newcommand{\N}{\ensuremath{\mathbb{N}}}
\newcommand{\Z}{\ensuremath{\mathbb{Z}}}
\newcommand{\Q}{\ensuremath{\mathbb{Q}}}
\newcommand{\R}{\ensuremath{\mathbb{R}}}
\newcommand{\C}{\ensuremath{\mathbb{C}}}
\renewcommand{\H}{\ensuremath{\mathbb{H}}}
\newcommand{\F}{\ensuremath{\mathbb{F}}}

\newcommand{\E}{\ensuremath{\mathbb{E}}}
\renewcommand{\P}{\ensuremath{\mathbb{P}}}

\newcommand{\es}{\ensuremath{\varnothing}}
\newcommand{\inv}{\ensuremath{^{-1}}}
\newcommand{\eps}{\ensuremath{\varepsilon}}
\newcommand{\del}{\ensuremath{\partial}}
\renewcommand{\a}{\ensuremath{\alpha}}

\newcommand{\abs}[1]{\ensuremath{\left\lvert #1 \right\rvert}}
\newcommand{\norm}[1]{\ensuremath{\left\lVert #1\right\rVert}}
\newcommand{\mean}[1]{\ensuremath{\left\langle #1 \right\rangle}}
\newcommand{\floor}[1]{\ensuremath{\left\lfloor #1 \right\rfloor}}
\newcommand{\ceil}[1]{\ensuremath{\left\lceil #1 \right\rceil}}
\newcommand{\bra}[1]{\ensuremath{\left\langle #1 \right\rvert}}
\newcommand{\ket}[1]{\ensuremath{\left\lvert #1 \right\rangle}}
\newcommand{\braket}[2]{\ensuremath{\left.\left\langle #1\right\vert #2 \right\rangle}}

\newcommand{\catname}[1]{{\normalfont\textbf{#1}}}

\newcommand{\up}{\ensuremath{\uparrow}}
\newcommand{\down}{\ensuremath{\downarrow}}

% Custom environments
\newtheorem{thm}{Theorem}[section]

\definecolor{probBackgroundColor}{RGB}{250,240,240}
\definecolor{probAccentColor}{RGB}{140,40,0}
\newenvironment{prob}{
    \stepcounter{thm}
    \begin{tcolorbox}[
        boxrule=1pt,
        sharp corners,
        colback=probBackgroundColor,
        colframe=probAccentColor,
        borderline west={4pt}{0pt}{probAccentColor},
        breakable
    ]
    \color{probAccentColor}\textbf{Problem \thethm.} \color{black}
} {
    \end{tcolorbox}
}

\definecolor{exampleBackgroundColor}{RGB}{212,232,246}
\newenvironment{example}{
    \stepcounter{thm}
    \begin{tcolorbox}[
      boxrule=1pt,
      sharp corners,
      colback=exampleBackgroundColor,
      breakable
    ]
    \textbf{Example \thethm.}
} {
    \end{tcolorbox}
}

\definecolor{propBackgroundColor}{RGB}{255,245,220}
\definecolor{propAccentColor}{RGB}{150,100,0}
\newenvironment{prop}{
    \stepcounter{thm}
    \begin{tcolorbox}[
        boxrule=1pt,
        sharp corners,
        colback=propBackgroundColor,
        colframe=propAccentColor,
        breakable
    ]
    \color{propAccentColor}\textbf{Proposition \thethm. }\color{black}
} {
    \end{tcolorbox}
}

\definecolor{thmBackgroundColor}{RGB}{235,225,245}
\definecolor{thmAccentColor}{RGB}{50,0,100}
\renewenvironment{thm}{
    \stepcounter{thm}
    \begin{tcolorbox}[
        boxrule=1pt,
        sharp corners,
        colback=thmBackgroundColor,
        colframe=thmAccentColor,
        breakable
    ]
    \color{thmAccentColor}\textbf{Theorem \thethm. }\color{black}
} {
    \end{tcolorbox}
}

\definecolor{corBackgroundColor}{RGB}{240,250,250}
\definecolor{corAccentColor}{RGB}{50,100,100}
\newenvironment{cor}{
    \stepcounter{thm}
    \begin{tcolorbox}[
        enhanced,
        boxrule=0pt,
        frame hidden,
        sharp corners,
        colback=corBackgroundColor,
        borderline west={4pt}{0pt}{corAccentColor},
        breakable
    ]
    \color{corAccentColor}\textbf{Corollary \thethm. }\color{black}
} {
    \end{tcolorbox}
}

\definecolor{lemBackgroundColor}{RGB}{255,245,235}
\definecolor{lemAccentColor}{RGB}{250,125,0}
\newenvironment{lem}{
    \stepcounter{thm}
    \begin{tcolorbox}[
        enhanced,
        boxrule=0pt,
        frame hidden,
        sharp corners,
        colback=lemBackgroundColor,
        borderline west={4pt}{0pt}{lemAccentColor},
        breakable
    ]
    \color{lemAccentColor}\textbf{Lemma \thethm. }\color{black}
} {
    \end{tcolorbox}
}

\definecolor{proofBackgroundColor}{RGB}{255,255,255}
\definecolor{proofAccentColor}{RGB}{80,80,80}
\renewenvironment{proof}{
    \begin{tcolorbox}[
        enhanced,
        boxrule=1pt,
        sharp corners,
        colback=proofBackgroundColor,
        colframe=proofAccentColor,
        borderline west={4pt}{0pt}{proofAccentColor},
        breakable
    ]
    \color{proofAccentColor}\emph{\textbf{Proof. }}\color{black}
} {
    \qed \end{tcolorbox}
}

\definecolor{noteBackgroundColor}{RGB}{240,250,240}
\definecolor{noteAccentColor}{RGB}{30,130,30}
\newenvironment{note}{
    \begin{tcolorbox}[
        enhanced,
        boxrule=0pt,
        frame hidden,
        sharp corners,
        colback=noteBackgroundColor,
        borderline west={4pt}{0pt}{noteAccentColor},
        breakable
    ]
    \color{noteAccentColor}\textbf{Note. }\color{black}
} {
    \end{tcolorbox}
}



\fancyhf{}
\lhead{Nathan Solomon}
\rhead{Page \thepage}
\pagestyle{fancy}

\begin{document}
\section{5/15/2024 lecture}
Last lecture, we discussed how any open neighborhood of the identity in a connected Lie group generates the whole group.

\subsection{The exponential map}
The exponential function $\exp: \C \rightarrow \C$ is defined by
\[ \exp(x) = \sum_{n=0}^\infty \frac{x^n}{n!}. \]
Since it's defined as a power series, it is an \textit{entire} function.
\par
Alternatively, the exponential function on $\C$ can be defined as the unique solution to
\[ f'(x)=x, \hspace{1cm} f(0)=1. \]
For any $n \times n$ matrix over either $\R$ or $\C$, we can define $\exp$ with the same power series.
\begin{prop}
    For any $A \in \C^{n \times n}$, the following power series is absolutely convergent:
    \[ \exp(A) := \sum_{k=0}^\infty \frac{A^k}{k!}. \]
\end{prop}
\begin{proof}
    \begin{align*}
        \abs{\exp(A)_{i,j}} &= \abs{\left( \sum_{k=0}^\infty \frac{A^k}{k!} \right)_{i,j}} \\
                            &\leq \sum_{k=0}^\infty \frac{1}{k!} n^{k-1} \max_{p,q} \abs{A_{p,q}}^k
    \end{align*}
    So every entry of $\exp(A)$ is bounded by an absolutely convergent series.
\end{proof}
\begin{example}
    Let $A$ be any 2 by 2 complex matrix such that $A^2=I$. Then $\exp(tA)= \left( \cosh t \right) I + \left( \sinh t \right) A$. Notably, if $t=i \theta$, then $\exp(i \theta A) = I \cdot \cos \theta + A \cdot i \sin \theta$.
\end{example}
\subsection{The matrix logarithm}
The matrix logarithm of $A$ can be defined by the following formula:
\[ \log (I+A) = \sum_{k=1}^\infty \frac{A^k}{k} (-1)^{k+1}. \]
We say that the log of a matrix exists iff that series is absolutely convergent -- for example, if the largest eigenvalue of $A$ has magnitude less than 1, then the matrix log exists.
\begin{thm}
    \begin{enumerate}
    \item If the log of a (real or complex) matrix exists (meaning $\log A$ is an absolutely convergent power series), then
    \begin{align*}
        \log(\exp(A))&=A \text{    and} \\
        \exp(\log(A))&=A.
    \end{align*}
\item $\exp(0)=I$ and $\log(I)=0$.
\item The following statements are equivalent:
    \begin{itemize}
        \item $AB=BA$
        \item $\exp(A+B)=\exp(A)\exp(B)$
        \item $\log(AB)=\log(A)+\log(B)$
    \end{itemize}
    DOES THAT STILL WORK EVEN IF THE LOG DOESNT CONVERGE??? ARE THOSE THREE STATEMENTS REALLY ALL EQUIVALENT???
    \end{enumerate}
\end{thm}
\begin{proof}
    The proofs of these are an exercise.
    \par
    \[ \exp(A+B)=\sum_{k=0}^\infty \frac{1}{k!} \sum_{\ell=o}^\infty \binom{k}{\ell} A^\ell B^{k-\ell} \]
    compare that to $\exp(A)\exp(B)$.
\end{proof}
\begin{thm}
    The exponential of any $A \in \C^{n \times n}$ is invertible, and its inverse is the exponential of $-A$.
\end{thm}
\begin{proof}
    \[ \exp(A)\exp(-A)=\exp(0)=I \]
\end{proof}
\begin{prob}
    Is the matrix exponential from the set of complex $n \times n$ matrices a surjection onto the set of invertible complex $n \times n$ matrices?
\end{prob}
\begin{prop}
    For any complex $n \times n$ matrices $A$ and $B$, there exists a matrix $C$ such that
    \[ \exp(A) \exp(B) = \exp(C). \]
\end{prop}
\begin{proof}
    For some very small $t \in \C$,
    \[ \exp(tA)\exp(tB)-\exp(t(A+B))= \frac{t^2}{2} [A,B]+O(t^3). \]
\end{proof}
INCLUDE THE BAKER CAMPBELL HAUSDORFF FORMULA. It gives a formal power series in $t$ (not necessarily convergent) for $\log(\exp(tA)\exp(tB))$ in terms of iterated commutators of $A$ and $B$
SEE MCGREEVYS NOTES ABOUT HOW TO DERIVE THAT USING A DIFFERENTIAL EQUATION AND AN ANSATZ
\subsection{Suzuki-Trotter formula}
\begin{thm}
    \[ e^{A+B}=\lim_{n \rightarrow \infty} \left( e^{A/n}e^{B/n} \right)^n. \]
\end{thm}
\begin{proof}
    \begin{align*}
        e^{tA/n}e^{tB/n}&=\exp \left( \frac{t}{n} (A+B)+ \frac{t^2}{2n^2} [A,B] + O\left( \frac{t^3}{n^3}\right) \right)
    \end{align*}
    FINISH THIS PROOF
\end{proof}
\subsection{Lie algebras}
If $G$ is a topologically closed subgroup of $GL(n,\F)$ (where $\F$ is either $\R$ or $\C$), let $\mathfrak{G}= \left\{ A \in \F^{n \times n}: e^{tA} \in G \forall t \right\}$. Then
\begin{enumerate}
    \item $\mathfrak{G}$ is a vector space
    \item $\mathfrak{G}$ is closed under the Lie bracket (which for matrix Lie algebras is always the commutator, I think??? IS THAT TRUE?)
    \item $\mathfrak{G}$ is the tangent space to $G$ at the identity
    \item $G$ is a Lie group
\end{enumerate}
To prove statements (1), you can use the Suzuki-Trotter formula. THE REST OF THESE STATEMENTS ARE EXERCISES

\end{document}
