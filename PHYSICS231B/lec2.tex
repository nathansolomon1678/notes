\documentclass[class=article, crop=false]{standalone}
\usepackage[margin=1in]{geometry}
\usepackage[linesnumbered,ruled,vlined]{algorithm2e}
\usepackage{amsfonts}
\usepackage{amsmath}
\usepackage{amssymb}
\usepackage{amsthm}
\usepackage{chemformula}
\usepackage{enumitem}
\usepackage{fancyhdr}
\usepackage{graphicx}
\usepackage{hyperref}
\usepackage{listings}
\usepackage{minted}
\usepackage{multicol}
\usepackage{pdfpages}
\usepackage{siunitx}
\usepackage{standalone}
\usepackage{svg}
\usepackage[many]{tcolorbox}
\usepackage{tikz-cd}
\usepackage{transparent}
\usepackage{xcolor}
% \tcbuselibrary{minted}

\author{Nathan Solomon}

\newcommand{\fig}[1]{
    \begin{center}
        \includegraphics[width=\textwidth]{#1}
    \end{center}
}

% Math commands
\renewcommand{\d}{\mathrm{d}}
\DeclareMathOperator{\id}{id}
\DeclareMathOperator{\im}{im}
\DeclareMathOperator{\proj}{proj}
\DeclareMathOperator{\Span}{span}
\DeclareMathOperator{\Tr}{Tr}
\DeclareMathOperator{\tr}{tr}
\DeclareMathOperator{\ad}{ad}
\DeclareMathOperator{\ord}{ord}
%%%%%%%%%%%%%%% \DeclareMathOperator{\sgn}{sgn}
\DeclareMathOperator{\Aut}{Aut}
\DeclareMathOperator{\Inn}{Inn}
\DeclareMathOperator{\Out}{Out}
\DeclareMathOperator{\stab}{stab}

\newcommand{\N}{\ensuremath{\mathbb{N}}}
\newcommand{\Z}{\ensuremath{\mathbb{Z}}}
\newcommand{\Q}{\ensuremath{\mathbb{Q}}}
\newcommand{\R}{\ensuremath{\mathbb{R}}}
\newcommand{\C}{\ensuremath{\mathbb{C}}}
\renewcommand{\H}{\ensuremath{\mathbb{H}}}
\newcommand{\F}{\ensuremath{\mathbb{F}}}

\newcommand{\E}{\ensuremath{\mathbb{E}}}
\renewcommand{\P}{\ensuremath{\mathbb{P}}}

\newcommand{\es}{\ensuremath{\varnothing}}
\newcommand{\inv}{\ensuremath{^{-1}}}
\newcommand{\eps}{\ensuremath{\varepsilon}}
\newcommand{\del}{\ensuremath{\partial}}
\renewcommand{\a}{\ensuremath{\alpha}}

\newcommand{\abs}[1]{\ensuremath{\left\lvert #1 \right\rvert}}
\newcommand{\norm}[1]{\ensuremath{\left\lVert #1\right\rVert}}
\newcommand{\mean}[1]{\ensuremath{\left\langle #1 \right\rangle}}
\newcommand{\floor}[1]{\ensuremath{\left\lfloor #1 \right\rfloor}}
\newcommand{\ceil}[1]{\ensuremath{\left\lceil #1 \right\rceil}}
\newcommand{\bra}[1]{\ensuremath{\left\langle #1 \right\rvert}}
\newcommand{\ket}[1]{\ensuremath{\left\lvert #1 \right\rangle}}
\newcommand{\braket}[2]{\ensuremath{\left.\left\langle #1\right\vert #2 \right\rangle}}

\newcommand{\catname}[1]{{\normalfont\textbf{#1}}}

\newcommand{\up}{\ensuremath{\uparrow}}
\newcommand{\down}{\ensuremath{\downarrow}}

% Custom environments
\newtheorem{thm}{Theorem}[section]

\definecolor{probBackgroundColor}{RGB}{250,240,240}
\definecolor{probAccentColor}{RGB}{140,40,0}
\newenvironment{prob}{
    \stepcounter{thm}
    \begin{tcolorbox}[
        boxrule=1pt,
        sharp corners,
        colback=probBackgroundColor,
        colframe=probAccentColor,
        borderline west={4pt}{0pt}{probAccentColor},
        breakable
    ]
    \color{probAccentColor}\textbf{Problem \thethm.} \color{black}
} {
    \end{tcolorbox}
}

\definecolor{exampleBackgroundColor}{RGB}{212,232,246}
\newenvironment{example}{
    \stepcounter{thm}
    \begin{tcolorbox}[
      boxrule=1pt,
      sharp corners,
      colback=exampleBackgroundColor,
      breakable
    ]
    \textbf{Example \thethm.}
} {
    \end{tcolorbox}
}

\definecolor{propBackgroundColor}{RGB}{255,245,220}
\definecolor{propAccentColor}{RGB}{150,100,0}
\newenvironment{prop}{
    \stepcounter{thm}
    \begin{tcolorbox}[
        boxrule=1pt,
        sharp corners,
        colback=propBackgroundColor,
        colframe=propAccentColor,
        breakable
    ]
    \color{propAccentColor}\textbf{Proposition \thethm. }\color{black}
} {
    \end{tcolorbox}
}

\definecolor{thmBackgroundColor}{RGB}{235,225,245}
\definecolor{thmAccentColor}{RGB}{50,0,100}
\renewenvironment{thm}{
    \stepcounter{thm}
    \begin{tcolorbox}[
        boxrule=1pt,
        sharp corners,
        colback=thmBackgroundColor,
        colframe=thmAccentColor,
        breakable
    ]
    \color{thmAccentColor}\textbf{Theorem \thethm. }\color{black}
} {
    \end{tcolorbox}
}

\definecolor{corBackgroundColor}{RGB}{240,250,250}
\definecolor{corAccentColor}{RGB}{50,100,100}
\newenvironment{cor}{
    \stepcounter{thm}
    \begin{tcolorbox}[
        enhanced,
        boxrule=0pt,
        frame hidden,
        sharp corners,
        colback=corBackgroundColor,
        borderline west={4pt}{0pt}{corAccentColor},
        breakable
    ]
    \color{corAccentColor}\textbf{Corollary \thethm. }\color{black}
} {
    \end{tcolorbox}
}

\definecolor{lemBackgroundColor}{RGB}{255,245,235}
\definecolor{lemAccentColor}{RGB}{250,125,0}
\newenvironment{lem}{
    \stepcounter{thm}
    \begin{tcolorbox}[
        enhanced,
        boxrule=0pt,
        frame hidden,
        sharp corners,
        colback=lemBackgroundColor,
        borderline west={4pt}{0pt}{lemAccentColor},
        breakable
    ]
    \color{lemAccentColor}\textbf{Lemma \thethm. }\color{black}
} {
    \end{tcolorbox}
}

\definecolor{proofBackgroundColor}{RGB}{255,255,255}
\definecolor{proofAccentColor}{RGB}{80,80,80}
\renewenvironment{proof}{
    \begin{tcolorbox}[
        enhanced,
        boxrule=1pt,
        sharp corners,
        colback=proofBackgroundColor,
        colframe=proofAccentColor,
        borderline west={4pt}{0pt}{proofAccentColor},
        breakable
    ]
    \color{proofAccentColor}\emph{\textbf{Proof. }}\color{black}
} {
    \qed \end{tcolorbox}
}

\definecolor{noteBackgroundColor}{RGB}{240,250,240}
\definecolor{noteAccentColor}{RGB}{30,130,30}
\newenvironment{note}{
    \begin{tcolorbox}[
        enhanced,
        boxrule=0pt,
        frame hidden,
        sharp corners,
        colback=noteBackgroundColor,
        borderline west={4pt}{0pt}{noteAccentColor},
        breakable
    ]
    \color{noteAccentColor}\textbf{Note. }\color{black}
} {
    \end{tcolorbox}
}



\fancyhf{}
\lhead{Nathan Solomon}
\rhead{Page \thepage}
\pagestyle{fancy}

\begin{document}
\section{4/5/2024 lecture}

\subsection{Matrix groups}
Matrix groups are groups whose elements are matrices. For any field $\F$ and any $n \in \N$, we define the \emph{general linear group $GL(n,\F)$} to be the multiplicative group of invertible (nonsingular) $n \times n$ matrices whose entries are elements of $\F$. $GL(n, \F)$ is a subset of $M(n,\F)$ (the set of $n \times n$ matrices over $\F$). $M(n,\F)$ is an algebra over $\F$ but is not a group.
\begin{note}
    In this class, we will always let $\F$ be either $\R$ or $\C$.
\end{note}
The determinant is a homomorphism $\det: GL(n,\F) \rightarrow \F^\times$. Here, $\F^\times$ is the group of invertible elements in $\F$. More generally, we can define the multiplicative group $R^\times$ of any ring $R$ to be the group of invertible elements (AKA units), but if $R$ is a field, then every element except 0 is invertible, so $\F^\times = \F \backslash \left\{ 0 \right\}$.

\subsection{Subgroup}
A \emph{subgroup} $H$ of a group $G$ is a subset of the elements of $G$ which contains the identity, is closed under inversion, and is closed under composition. In other words, $H$ is a subgroup of $G$ if and only if for any $h_1, h_2 \in H$,
\begin{itemize}
    \item $e_G \in H$
    \item $h_1^{-1} \in H$
    \item $h_1 h_2 \in H$
\end{itemize}
Every subgroup is a group. The notation for ``$H$ is a subgroup of $G$" is $H \leq G$. If $H$ is a \emph{proper subgroup} of $G$, meaning $H \leq G$ and $H \neq G$, then we write $H < G$.
\par
If $H$ is a group, we can say that $H \leq G$ if and only if there is an injective homomorphism $i: H \rightarrow G$.

\subsection{Subgroups of the general linear group}
We define the \emph{orthogonal group $O(n)$} to be the set of orthogonal matrices in $GL(n,\R)$. Since every orthogonal matrix $A$ satisfies $A^TA=1$, the determinant of $A$ must be $\pm 1$. Therefore we define the \emph{special orthogonal group $SO(n)$} to be the set of matrices in $O(n)$ with determinant 1.
\[ SO(n) < O(n) < GL(n,\R) \]
The \emph{special linear group $SL(n,\F)$} is the group of matrices in $GL(n,\F)$ with determinant 1.
\par
Similarly, we define the \emph{unitary group $U(n)$} to be the subgroup of unitary matrices in $GL(n,\C)$, and we define the \emph{special unitary group $SU(n,\C)$} to be $U(n) \cap SL(n,\C)$ -- that is, the set of $n \times n$ complex unitary matrices with determinant 1.
\begin{example}
    Let $T_3$ be the group of rotational symmetries of a tetrahedron, and let $\widetilde{T_3}$ be the group of all symmetries (rotations plus reflections) of a tetrahedron. Then $T_3 < \widetilde{T_3} < O(3)$, and $T_3 < SO(3)$.
\end{example}
\begin{prop}
    $GL(n,\C)$ is a proper subgroup of $GL(2n,\R)$.
\end{prop}
\begin{proof}
    The map which takes every entry $a+bi$ to the $2 \times 2$ block $\begin{bmatrix}
        a & -b \\
        b & a
    \end{bmatrix}$ is an isomorphism between $GL(n,\C)$ and a proper subgroup of $GL(2n,\R)$.
\end{proof}

\subsection{Representations}
A \emph{representation} of $G$ is a homomorphism from $G$ to $GL(n,\F)$. This allows us to make groups slightly more intuitive, since we can see them as linear transformations on $\F^n$. A representation is called \emph{faithful} iff it is injective.

\subsection{Cosets and quotient groups}
If $H \leq G$, then for any $g \in G$, define the \emph{left coset} $gH$ to be $\left\{ gh: h \in H \right\}$, and let the \emph{right coset} $Hg$ be $ \left\{ hg: h \in H \right\}$. Note that cosets are not necessariy subgroups -- for example, the set of odd integers is not a subgroup of $\Z^+$ (the additive group of integers).
\begin{example}
    There are only two distinct cosets of $SO(n)$ in $O(n)$: those with determinant 1, and those with determinant $-1$.
\end{example}
If $H$ is a subgroup of $G$, we define the \emph{quotient $G/H$} to be the set of left $H$-cosets.
\par
The \emph{index of $H$ in $G$}, denoted $[G:H]$, is defined as $|G/H|$ (the number of left cosets of $H$ in $G$).
\begin{thm}
    \textbf{Lagrange's theorem:} If $H$ is a subgroup of a finite group $G$, then
    \[ |G/H| = |G|/|H|. \]
\end{thm}

\begin{proof}
    Every element of $G$ is in at least one coset, because $g \in gH$. For any $g$, $|gH|=|H|$. Therefore the number of cosets ($|G/H|$) times the number of elements per coset ($|H|$) is equal to the total number of elements in the group $G$.
\end{proof}

\begin{cor}
    The order of a finite group is divisible by the order of any element.
\end{cor}
A subgroup $H \leq G$ is called a \emph{normal subgroup}, and denoted by $H \trianglelefteq G$ iff $g^{-1}hg \in H$ for any $g \in G, h \in H$ -- that is, iff $H$ is invariant under \emph{conjugation} by any $g \in G$.
\begin{thm}\label{permutationconjugation}
    For any cycle $(x_1\ \ x_2 \ \ \dots \ \ x_k) \in S_n$, the conjugate of that cycle by some permutation $\sigma \in S_n$ is
    \[ \sigma^{-1} (x_1\ \ x_2 \ \ \dots \ \ x_k) \sigma = (\sigma(x_1)\ \ \sigma(x_2) \ \ \dots \ \ \sigma(x_k)). \]
\end{thm}
\begin{thm}
    $G/H$ is a group iff $H \trianglelefteq G$. In this case, $G/H$ is called the \emph{quotient group}.
\end{thm}
\begin{proof}
    If $H \trianglelefteq G$, then we can define the product of two cosets $g_1H, g_2H \in G/H$ to be $g_1 g_2 H$. If $H$ were not a normal subgroup of $H$, than $G/H$ would not be closed under multiplication.
\end{proof}

\subsection{First isomorphism theorem}
If $\varphi:G \rightarrow K$ is a homomorphism, then define the \emph{kernel of $\varphi$}, denote $\ker \varphi$, to be the set of elements in $G$ which $\varphi$ maps to $e_K$. The kernel of any homomorphism forms a normal subgroup of the codomain. REMEMBE TO PROVE THIS TOO. ALSO DEFINE THE IMAGE AND USE A COMMUTATIVE DIAGRAM TO PROVE THE FIRST ISOMORPHISM THEOREM
\bigskip
\par
\begin{center}
\begin{tikzcd}[column sep=small]
    G/H \arrow[dr, ""] \arrow[hookrightarrow, "i"] & H \\
        & \arrow[u, "\varphi"]
        & G
\end{tikzcd}
    
\end{center}

\subsection{Product of groups}
If $G,H$ are groups, then the group $G \times H$ is the group of pairs $(g,h)$ where composition is given by componentwise composition. If $G$ and $H$ are finite, then $|G \times H|=|G| \times |H|$. ISNT THIS CALLED THE DIRECT SUM??
\par
ALSO INCLUDE NOTES ON SHORT EXACT SEQUENCES, SIMPLE GROUPS (note that integers modulo a prime are a simple group), AND EXTENSIONS

\end{document}
