\documentclass{article}
\usepackage[margin=1in]{geometry}
\usepackage[linesnumbered,ruled,vlined]{algorithm2e}
\usepackage{amsfonts}
\usepackage{amsmath}
\usepackage{amssymb}
\usepackage{amsthm}
\usepackage{enumitem}
\usepackage{fancyhdr}
\usepackage{hyperref}
\usepackage{minted}
\usepackage{multicol}
\usepackage{pdfpages}
\usepackage{standalone}
\usepackage[many]{tcolorbox}
\usepackage{tikz-cd}
\usepackage{transparent}
\usepackage{xcolor}
% \tcbuselibrary{minted}

\author{Nathan Solomon}

\newcommand{\fig}[1]{
    \begin{center}
        \includegraphics[width=\textwidth]{#1}
    \end{center}
}

% Math commands
\renewcommand{\d}{\mathrm{d}}
\DeclareMathOperator{\id}{id}
\DeclareMathOperator{\im}{im}
\DeclareMathOperator{\proj}{proj}
\DeclareMathOperator{\Span}{span}
\DeclareMathOperator{\Tr}{Tr}
\DeclareMathOperator{\tr}{tr}
\DeclareMathOperator{\ad}{ad}
\DeclareMathOperator{\ord}{ord}
%%%%%%%%%%%%%%% \DeclareMathOperator{\sgn}{sgn}
\DeclareMathOperator{\Aut}{Aut}
\DeclareMathOperator{\Inn}{Inn}
\DeclareMathOperator{\Out}{Out}
\DeclareMathOperator{\stab}{stab}

\newcommand{\N}{\ensuremath{\mathbb{N}}}
\newcommand{\Z}{\ensuremath{\mathbb{Z}}}
\newcommand{\Q}{\ensuremath{\mathbb{Q}}}
\newcommand{\R}{\ensuremath{\mathbb{R}}}
\newcommand{\C}{\ensuremath{\mathbb{C}}}
\renewcommand{\H}{\ensuremath{\mathbb{H}}}
\newcommand{\F}{\ensuremath{\mathbb{F}}}

\newcommand{\E}{\ensuremath{\mathbb{E}}}
\renewcommand{\P}{\ensuremath{\mathbb{P}}}

\newcommand{\es}{\ensuremath{\varnothing}}
\newcommand{\inv}{\ensuremath{^{-1}}}
\newcommand{\eps}{\ensuremath{\varepsilon}}
\newcommand{\del}{\ensuremath{\partial}}
\renewcommand{\a}{\ensuremath{\alpha}}

\newcommand{\abs}[1]{\ensuremath{\left\lvert #1 \right\rvert}}
\newcommand{\norm}[1]{\ensuremath{\left\lVert #1\right\rVert}}
\newcommand{\mean}[1]{\ensuremath{\left\langle #1 \right\rangle}}
\newcommand{\floor}[1]{\ensuremath{\left\lfloor #1 \right\rfloor}}
\newcommand{\ceil}[1]{\ensuremath{\left\lceil #1 \right\rceil}}
\newcommand{\bra}[1]{\ensuremath{\left\langle #1 \right\rvert}}
\newcommand{\ket}[1]{\ensuremath{\left\lvert #1 \right\rangle}}
\newcommand{\braket}[2]{\ensuremath{\left.\left\langle #1\right\vert #2 \right\rangle}}

\newcommand{\catname}[1]{{\normalfont\textbf{#1}}}

\newcommand{\up}{\ensuremath{\uparrow}}
\newcommand{\down}{\ensuremath{\downarrow}}

% Custom environments
\newtheorem{thm}{Theorem}[section]

\definecolor{probBackgroundColor}{RGB}{250,240,240}
\definecolor{probAccentColor}{RGB}{140,40,0}
\newenvironment{prob}{
    \stepcounter{thm}
    \begin{tcolorbox}[
        boxrule=1pt,
        sharp corners,
        colback=probBackgroundColor,
        colframe=probAccentColor,
        borderline west={4pt}{0pt}{probAccentColor},
        breakable
    ]
    \color{probAccentColor}\textbf{Problem \thethm.} \color{black}
} {
    \end{tcolorbox}
}

\definecolor{exampleBackgroundColor}{RGB}{212,232,246}
\newenvironment{example}{
    \stepcounter{thm}
    \begin{tcolorbox}[
      boxrule=1pt,
      sharp corners,
      colback=exampleBackgroundColor,
      breakable
    ]
    \textbf{Example \thethm.}
} {
    \end{tcolorbox}
}

\definecolor{propBackgroundColor}{RGB}{255,245,220}
\definecolor{propAccentColor}{RGB}{150,100,0}
\newenvironment{prop}{
    \stepcounter{thm}
    \begin{tcolorbox}[
        boxrule=1pt,
        sharp corners,
        colback=propBackgroundColor,
        colframe=propAccentColor,
        breakable
    ]
    \color{propAccentColor}\textbf{Proposition \thethm. }\color{black}
} {
    \end{tcolorbox}
}

\definecolor{thmBackgroundColor}{RGB}{235,225,245}
\definecolor{thmAccentColor}{RGB}{50,0,100}
\renewenvironment{thm}{
    \stepcounter{thm}
    \begin{tcolorbox}[
        boxrule=1pt,
        sharp corners,
        colback=thmBackgroundColor,
        colframe=thmAccentColor,
        breakable
    ]
    \color{thmAccentColor}\textbf{Theorem \thethm. }\color{black}
} {
    \end{tcolorbox}
}

\definecolor{corBackgroundColor}{RGB}{240,250,250}
\definecolor{corAccentColor}{RGB}{50,100,100}
\newenvironment{cor}{
    \stepcounter{thm}
    \begin{tcolorbox}[
        enhanced,
        boxrule=0pt,
        frame hidden,
        sharp corners,
        colback=corBackgroundColor,
        borderline west={4pt}{0pt}{corAccentColor},
        breakable
    ]
    \color{corAccentColor}\textbf{Corollary \thethm. }\color{black}
} {
    \end{tcolorbox}
}

\definecolor{lemBackgroundColor}{RGB}{255,245,235}
\definecolor{lemAccentColor}{RGB}{250,125,0}
\newenvironment{lem}{
    \stepcounter{thm}
    \begin{tcolorbox}[
        enhanced,
        boxrule=0pt,
        frame hidden,
        sharp corners,
        colback=lemBackgroundColor,
        borderline west={4pt}{0pt}{lemAccentColor},
        breakable
    ]
    \color{lemAccentColor}\textbf{Lemma \thethm. }\color{black}
} {
    \end{tcolorbox}
}

\definecolor{proofBackgroundColor}{RGB}{255,255,255}
\definecolor{proofAccentColor}{RGB}{80,80,80}
\renewenvironment{proof}{
    \begin{tcolorbox}[
        enhanced,
        boxrule=1pt,
        sharp corners,
        colback=proofBackgroundColor,
        colframe=proofAccentColor,
        borderline west={4pt}{0pt}{proofAccentColor},
        breakable
    ]
    \color{proofAccentColor}\emph{\textbf{Proof. }}\color{black}
} {
    \qed \end{tcolorbox}
}

\definecolor{noteBackgroundColor}{RGB}{240,250,240}
\definecolor{noteAccentColor}{RGB}{30,130,30}
\newenvironment{note}{
    \begin{tcolorbox}[
        enhanced,
        boxrule=0pt,
        frame hidden,
        sharp corners,
        colback=noteBackgroundColor,
        borderline west={4pt}{0pt}{noteAccentColor},
        breakable
    ]
    \color{noteAccentColor}\textbf{Note. }\color{black}
} {
    \end{tcolorbox}
}


\fancyhf{}
\setlength{\headheight}{24pt}

\date{\today}
\title{Physics 231B Homework \#7}

\begin{document}
\maketitle

Consider the $n$-fold tensor product
\[ \C^{2^n} = \bigotimes_{i=1}^n \C^2 \]
and define the Pauli matrices $X_i, Y_i, Z_i$ as acting on the $i$th factor via
\begin{align*}
    X_1 &= X \otimes I \otimes \cdots \otimes I, \\
    X_2 &= I \otimes X \otimes I \cdots \otimes I, \\
        & \cdots \\
    X_n &= I \otimes I \otimes \cdots X,
\end{align*}
and so on (using the shorthand $\sigma^x = X, \sigma^y=Y, \sigma^z=Z$). With these, we define $2n$ ``Majorana" operators via $\gamma_1=X_1, \gamma_2=Y_1$, and for $1 \leq k < n$:
\begin{align*}
    \gamma_{2k+1} &= \left( \prod_{i=1}^k Z_i \right) X_i \\
    \gamma_{2k+2} &= \left( \prod_{i=1}^k Z_i \right) Y_i. \\
\end{align*}
This is called the Jordan-Wigner transformation. The next few problems will develop spinor representations from these variables.

\bigskip
\begin{prob}
    Prove that these satisfy the Clifford algebra relations:
    \[ \begin{cases}
        \gamma_i^2=1 & \text{for any $i$} \\
        \gamma_i\gamma_j = -\gamma_j\gamma_i & \text{whenever $i \neq j$}
    \end{cases} \]
\end{prob}

\bigskip
\begin{prob}
    Recall $\mathfrak{so}(2)$ is the Lie algebra of anti-symmetric real matrices. Prove that
    \[ A_{\mu \nu} \mapsto \frac{1}{4} A_{\mu \nu} \gamma_\mu \gamma_\nu \]
    defines a Lie algebra representation of $\mathfrak{so}(2n)$ on $\C^{2^n}$. In other words, prove
    \[ \frac{1}{16} A_{ij}B_{kl} [\gamma_i\gamma_j,\gamma_k,\gamma_l] = \frac{1}{4} [A,B]_{ab} \gamma_a \gamma_b. \]
\end{prob}

\bigskip
\begin{prob}
    We define the group $Spin(2n)$ by exponentiating these operators inside $U(2^n)$:
    \[ \exp \left( \frac{1}{2} A_{\mu \nu} \gamma_\mu \gamma_\nu \right). \]
    Show that the conjugation action of these operators on the Clifford algebra acts on $\gamma_i$ according to the $2n$-dimensional representation on the index $i$, thus giving a surjective map $Spin(2n) \rightarrow SO(2n)$.
\end{prob}

\bigskip
\begin{prob}
    Show that the representation of $Spin(2n)$ on $\C^{2^n}$ we have obtained from this construction does not define an $SO(2n)$ representation. Decompose this representations into irreducibles of $Spin(4) = SU(2) \times SU(2)$ in the case $n=2$.
\end{prob}

\bigskip
\begin{prob}
    The action of $Spin(2n)$ on $\C^{2^n}$ is reducible because of the commuting element $\gamma_c := \prod_{i=1}^n Z_i$. Prove that this element satisfies
    \begin{align*}
        \gamma_c^2 &= 1 \\
        \gamma_i \gamma_c = -\gamma_c \gamma_i.
    \end{align*}
    Thus we get a $2^n$ dimensional representation of $\mathfrak{so}(2n+1)$ and a corresponding double cover group $Spin(2n+1)$. Verify for $n=1$ we get the identification $Spin(3)=SU(2)$.
    
\end{prob}

\end{document}
