\documentclass{article}
\usepackage[margin=1in]{geometry}
\usepackage[linesnumbered,ruled,vlined]{algorithm2e}
\usepackage{amsfonts}
\usepackage{amsmath}
\usepackage{amssymb}
\usepackage{amsthm}
\usepackage{enumitem}
\usepackage{fancyhdr}
\usepackage{hyperref}
\usepackage{minted}
\usepackage{multicol}
\usepackage{pdfpages}
\usepackage{standalone}
\usepackage[many]{tcolorbox}
\usepackage{tikz-cd}
\usepackage{transparent}
\usepackage{xcolor}
% \tcbuselibrary{minted}

\author{Nathan Solomon}

\newcommand{\fig}[1]{
    \begin{center}
        \includegraphics[width=\textwidth]{#1}
    \end{center}
}

% Math commands
\renewcommand{\d}{\mathrm{d}}
\DeclareMathOperator{\id}{id}
\DeclareMathOperator{\im}{im}
\DeclareMathOperator{\proj}{proj}
\DeclareMathOperator{\Span}{span}
\DeclareMathOperator{\Tr}{Tr}
\DeclareMathOperator{\tr}{tr}
\DeclareMathOperator{\ad}{ad}
\DeclareMathOperator{\ord}{ord}
%%%%%%%%%%%%%%% \DeclareMathOperator{\sgn}{sgn}
\DeclareMathOperator{\Aut}{Aut}
\DeclareMathOperator{\Inn}{Inn}
\DeclareMathOperator{\Out}{Out}
\DeclareMathOperator{\stab}{stab}

\newcommand{\N}{\ensuremath{\mathbb{N}}}
\newcommand{\Z}{\ensuremath{\mathbb{Z}}}
\newcommand{\Q}{\ensuremath{\mathbb{Q}}}
\newcommand{\R}{\ensuremath{\mathbb{R}}}
\newcommand{\C}{\ensuremath{\mathbb{C}}}
\renewcommand{\H}{\ensuremath{\mathbb{H}}}
\newcommand{\F}{\ensuremath{\mathbb{F}}}

\newcommand{\E}{\ensuremath{\mathbb{E}}}
\renewcommand{\P}{\ensuremath{\mathbb{P}}}

\newcommand{\es}{\ensuremath{\varnothing}}
\newcommand{\inv}{\ensuremath{^{-1}}}
\newcommand{\eps}{\ensuremath{\varepsilon}}
\newcommand{\del}{\ensuremath{\partial}}
\renewcommand{\a}{\ensuremath{\alpha}}

\newcommand{\abs}[1]{\ensuremath{\left\lvert #1 \right\rvert}}
\newcommand{\norm}[1]{\ensuremath{\left\lVert #1\right\rVert}}
\newcommand{\mean}[1]{\ensuremath{\left\langle #1 \right\rangle}}
\newcommand{\floor}[1]{\ensuremath{\left\lfloor #1 \right\rfloor}}
\newcommand{\ceil}[1]{\ensuremath{\left\lceil #1 \right\rceil}}
\newcommand{\bra}[1]{\ensuremath{\left\langle #1 \right\rvert}}
\newcommand{\ket}[1]{\ensuremath{\left\lvert #1 \right\rangle}}
\newcommand{\braket}[2]{\ensuremath{\left.\left\langle #1\right\vert #2 \right\rangle}}

\newcommand{\catname}[1]{{\normalfont\textbf{#1}}}

\newcommand{\up}{\ensuremath{\uparrow}}
\newcommand{\down}{\ensuremath{\downarrow}}

% Custom environments
\newtheorem{thm}{Theorem}[section]

\definecolor{probBackgroundColor}{RGB}{250,240,240}
\definecolor{probAccentColor}{RGB}{140,40,0}
\newenvironment{prob}{
    \stepcounter{thm}
    \begin{tcolorbox}[
        boxrule=1pt,
        sharp corners,
        colback=probBackgroundColor,
        colframe=probAccentColor,
        borderline west={4pt}{0pt}{probAccentColor},
        breakable
    ]
    \color{probAccentColor}\textbf{Problem \thethm.} \color{black}
} {
    \end{tcolorbox}
}

\definecolor{exampleBackgroundColor}{RGB}{212,232,246}
\newenvironment{example}{
    \stepcounter{thm}
    \begin{tcolorbox}[
      boxrule=1pt,
      sharp corners,
      colback=exampleBackgroundColor,
      breakable
    ]
    \textbf{Example \thethm.}
} {
    \end{tcolorbox}
}

\definecolor{propBackgroundColor}{RGB}{255,245,220}
\definecolor{propAccentColor}{RGB}{150,100,0}
\newenvironment{prop}{
    \stepcounter{thm}
    \begin{tcolorbox}[
        boxrule=1pt,
        sharp corners,
        colback=propBackgroundColor,
        colframe=propAccentColor,
        breakable
    ]
    \color{propAccentColor}\textbf{Proposition \thethm. }\color{black}
} {
    \end{tcolorbox}
}

\definecolor{thmBackgroundColor}{RGB}{235,225,245}
\definecolor{thmAccentColor}{RGB}{50,0,100}
\renewenvironment{thm}{
    \stepcounter{thm}
    \begin{tcolorbox}[
        boxrule=1pt,
        sharp corners,
        colback=thmBackgroundColor,
        colframe=thmAccentColor,
        breakable
    ]
    \color{thmAccentColor}\textbf{Theorem \thethm. }\color{black}
} {
    \end{tcolorbox}
}

\definecolor{corBackgroundColor}{RGB}{240,250,250}
\definecolor{corAccentColor}{RGB}{50,100,100}
\newenvironment{cor}{
    \stepcounter{thm}
    \begin{tcolorbox}[
        enhanced,
        boxrule=0pt,
        frame hidden,
        sharp corners,
        colback=corBackgroundColor,
        borderline west={4pt}{0pt}{corAccentColor},
        breakable
    ]
    \color{corAccentColor}\textbf{Corollary \thethm. }\color{black}
} {
    \end{tcolorbox}
}

\definecolor{lemBackgroundColor}{RGB}{255,245,235}
\definecolor{lemAccentColor}{RGB}{250,125,0}
\newenvironment{lem}{
    \stepcounter{thm}
    \begin{tcolorbox}[
        enhanced,
        boxrule=0pt,
        frame hidden,
        sharp corners,
        colback=lemBackgroundColor,
        borderline west={4pt}{0pt}{lemAccentColor},
        breakable
    ]
    \color{lemAccentColor}\textbf{Lemma \thethm. }\color{black}
} {
    \end{tcolorbox}
}

\definecolor{proofBackgroundColor}{RGB}{255,255,255}
\definecolor{proofAccentColor}{RGB}{80,80,80}
\renewenvironment{proof}{
    \begin{tcolorbox}[
        enhanced,
        boxrule=1pt,
        sharp corners,
        colback=proofBackgroundColor,
        colframe=proofAccentColor,
        borderline west={4pt}{0pt}{proofAccentColor},
        breakable
    ]
    \color{proofAccentColor}\emph{\textbf{Proof. }}\color{black}
} {
    \qed \end{tcolorbox}
}

\definecolor{noteBackgroundColor}{RGB}{240,250,240}
\definecolor{noteAccentColor}{RGB}{30,130,30}
\newenvironment{note}{
    \begin{tcolorbox}[
        enhanced,
        boxrule=0pt,
        frame hidden,
        sharp corners,
        colback=noteBackgroundColor,
        borderline west={4pt}{0pt}{noteAccentColor},
        breakable
    ]
    \color{noteAccentColor}\textbf{Note. }\color{black}
} {
    \end{tcolorbox}
}


\fancyhf{}
\setlength{\headheight}{24pt}

\date{\today}
\title{Physics 231B Homework \#1}

\begin{document}
\maketitle

\begin{prob}
    Let $G$ be a group, and consider the map $\alpha: G \rightarrow G$ defined by $\alpha(g)=f^{-1}$. Show that $\alpha$ is a homomorphism if and only if $g$ is abelian.
\end{prob}
\par
If $G$ is abelian, then let $a,b$ be any two elements of $G$. Then $a^{-1}b^{-1} = \left( ba \right)^{-1} = \alpha \left( ba \right)$ and
\[ a^{-1}b^{-1}=b^{-1}a^{-1}=\alpha(b) \alpha(a) \]
so $\alpha(ba)=\alpha(b)\alpha(a)$, meaning $\alpha$ is a homomorphism.
\par
If $\alpha$ is a homomorphism, then for any $a,b \in G$,
\[ ab=\alpha \left( (ab)^{-1} \right) = \alpha(b^{-1}a^{-1}) = \alpha(b^{-1}) \alpha(a^{-1}) = ba, \]
so $G$ is abelian.

\bigskip
\begin{prob}
    Suppose $n$ and $m$ are coprime, i.e. $\gcd(n,m)=1$. Show that the groups $C_n \times C_m$ and $C_{nm}$ are isomorphic.
\end{prob}
To make the notation easier, for any integer $x>1$, write $C_x$ as the additive group $\Z/x\Z$. Let $f: C_n \times C_m \rightarrow C_{nm}$ be the map defined by $f((a,b)) = ma+b$ (for any $a \in C_n, b \in C_m$). This is a homomorphism because
\[ f\left( (a,b) + (a', b') \right) = f\left( (a+a',b+b') \right) =m(a+a')+b+b'=ma+b + ma'+b' = f\left( (a,b) \right) + f\left( (a',b') \right). \]
Also, $f$ is surjective, because for any $c \in C_{nm}$ we can let $a=\floor{c/m}$ and $b=c-ma$, so that $f((a,b))=c$. Since $C_n \times C_m$ and $C_{nm}$ have the same finite order, $f$ being surjective implies $f$ is also injective.

\bigskip
\begin{prob}
    \textbf{Artin Chapter 2 Problem 4.8, page 71.}
    \begin{itemize}
        \item Prove that the elementary matrices of the first and third types (1.2.4) generate $GL_n(\R)$.
        \item Prove that the elementary matrices of the first type generate $SL_n(\R)$. Do the $2 \times 2$ case first.
    \end{itemize}
\end{prob}
\begin{itemize}
    \item First, you can easily turn the identity matrix into any matrix in Jordan normal form (with no zeros on the main diagonal) by using type (i) and type (iii) moves to get the first column to be what you want, then the second column, and so on.
        \par
        Next, note that all elementary matrices of the second type can be obtained from elementary matrices of the first and thirds types, because whenever you want to swap rows $i$ and $j$ of a matrix, you can do so by multiplying row $i$ by $-1$, adding row $i$ to row $j$, multiplying row $j$ by $-1$, adding row $j$ to row $i$, then adding row $i$ to row $j$.
        \par
        Since every matrix is similar to a Jordan form matrix, every invertible matrix $A$ is similar to a Jordan form matrix with no zeros on the main diagonal (if there were zeros on the main diagonal, $A$ would have determinant zero). By multiplying the identity matrix by type (i) and (iii) matrices, we can turn it first into a Jordan form matrix similar to $A$, then into $A$.
    \item $SL_n(\R)$ is the kernel of $\det: GL_n(\R) \rightarrow \R^\times$, so every matrix that can be written as a product of type (i) matrices is in $SL_n(\R)$. However, an arbitrary element of $SL_n(\R)$ is a product of type (i) and (iii) matrices, and since diagonal matrices commute with everything, it can be rewritten as some diagonal matrix (with determinant 1) times a product of type (i) matrices. Now we just need to prove that any diagonal matrix with determinant 1 is a product of type (i) matrices.
        \par
        In the $2 \times 2$ case, this is true because
        \[ \begin{bmatrix}
            a & 0 \\
            0 & \frac{1}{a}
        \end{bmatrix} = \begin{bmatrix}
            1 & 0 \\
            a & 1
        \end{bmatrix} \cdot \begin{bmatrix}
            1 & 0 \\
            a & 1
        \end{bmatrix}. \]
\end{itemize}

\bigskip
\begin{prob}
    \textbf{Artin Chapter 2 Problem 5.5, page 71.} Prove that the $n \times n$ matrices that have the block form $M=\begin{bmatrix}
        A & B \\
        0 & D
    \end{bmatrix}$, with $A$ in $GL_r(\R)$ and $D$ in $GL_{n-r}(\R)$, form a subgroup $H$ of $GL_n(\R)$, and that the map $H \rightarrow GL_r(\R)$ that sends $M$ to $A$ is a homomorphism. What is its kernel?
\end{prob}
To be a subgroup, $H$ needs to be closed under multiplication. It is, because the product of any two matrices in $H$ can be written as
\[ \begin{bmatrix}
    A & B \\
    0 & D
\end{bmatrix} \cdot \begin{bmatrix}
    A' & B' \\
    0 & D'
\end{bmatrix} = \begin{bmatrix}
    AA' & AB'+BD' \\
    0 & DD'
\end{bmatrix} \in H \]
where $AA'$ is an $r \times r$ matrix block, $AB'+BD'$ is an $r \times (n-r)$ block, and $DD'$ is an $(n-r) \times (n-r)$ block.
\par
The kernel of the given map from $M$ to $A$ is the set of matrices $ \begin{bmatrix}
    A & B \\
    0 & D
\end{bmatrix}$ where the top left block is $A = I_r$.

\bigskip
\begin{prob}
    \textbf{Artin Chapter 2 Problem 8.7, page 73.} A group $G$ of order 22 contains elements $x$ and $y$, where $x \neq 1$ and $y$ is not a power of $x$. Prove that the subgroup generated by these elements is the whole group $G$.
\end{prob}
Lagrange's theorem says that for any subgroup $H$ of a finite group $G$, $|G|=|H| \cdot [G:H]$. A corollary of that is that the order of $H$ must divide the order of $G$. In particular, for any $x \in G$, the order of $x$ must divide the order of $G$, because the subgroup generated by $x$ has the same order as $x$.
\par
This means the order of $x$ and $y$ must both divide 22. The prime factorization of 22 is $2 \times 11$, so the order of $x$ could be either 1, 2, 11, or 22. However, we're given that $x$ is not the identity, and if the order of $x$ is 22, $x$ would generate $G$ without even needing $y$. Therefore, let's assume $x$ has order 2 or 11.
\par
Let $H := \mean{x,y}$ be the subgroup of $G$ generated by $x$ and $y$. By Lagrange's theorem, the order of $H$ must be a multiple of the order of $x$ (because $\mean{x}$ is a subgroup of $\mean{x,y}$). Since $H$ contains all powers of $x$ as well as $y$, which is not a power of $x$, $H$ must have at least $\ord(x)+1$ elements. Also by Lagrange's theorem, the order of $H$ divides 22.
\par
If $x$ has order 2, this would imply the order of $H$ is a multiple of 2 which is larger than 2 and which divides 22. The smallest such possibility is 22. Similarly, if $x$ has order 11, then $H$ would have to contain more than 11 elements, but the order of $H$ must still divide 22, so it could only be 22.
\par
Since $H$ is a subgroup of $G$ with the same finite order, the groups must be equal.

\bigskip
\begin{prob}
    \textbf{Artin Chapter 2 Problem 12.2, page 75.} In the general linear group $GL_3(\R)$, consider the subsets
    \[ H=\begin{bmatrix}
        1 & * & * \\
        0 & 1 & * \\
        0 & 0 & 1
    \end{bmatrix}\text{, and } K = \begin{bmatrix}
        1 & 0 & * \\
        0 & 1 & 0 \\
        0 & 0 & 1
    \end{bmatrix}, \]
    where $*$ represents an arbitrary real number. Show that $H$ is a subgroup of $GL_3(\R)$, that $K$ is a normal subgroup of $H$, and identify the quotient group $H/K$. Determine the center of $H$.
\end{prob}

The product of any two elements in $H$ can be written as
\[ \begin{bmatrix}
    1 & a & b \\
    0 & 1 & c \\
    0 & 0 & 1
\end{bmatrix} \cdot \begin{bmatrix}
    1 & a' & b' \\
    0 & 1 & c' \\
    0 & 0 & 1
\end{bmatrix} = \begin{bmatrix}
    1 & a'+a & b'+ac'+b \\
    0 & 1 & c'+c \\
    0 & 0 & 1
\end{bmatrix} \]
which is in $H$, so $H$ is closed under multiplication (and therefore, is a subgroup). Also, from that equation, we see that the two matrices on the left hand side commute iff $ac'=a'c$. The only way that can be true for any $a',c' \in \R$ is if $a'=c'=0$, so the center of $H$ is $K$. Therefore $K \trianglelefteq H$.
\par
The quotient group is the coset space

\bigskip
\begin{prob}
    \textbf{Optional problem.} Let $G$ be the group generated by $s_1, \dots, s_{n-1}$ with relations
    \begin{align*}
        s_i^2=e &\hspace{1cm} i\in\{1, \dots, n-1\} \\
        (s_is_{i+1})^3=e &\hspace{1cm} i\in\{1, \dots, n-2\} \\
        s_i s_j s_i s_j=e &\hspace{1cm} i,j\in\{1, \dots, n-2\} \text{ and } |i-j|>1
    \end{align*}
    We showed in class that there is a homomorphism $\tau: G \rightarrow S_n$ which sends these generators to the transpositions
    \[ \tau(s_i) = (i\ \ i+1) .\]
    We also showed that this map is surjective. Show that is it also injective and thus $G$ is isomorphic to $S_n$.
\end{prob}

We know that the symmetric group can be generated from the set of adjacent transpositions, and we also know that the image of any homomorphism is closed under multiplication, so $\tau$ must be surjective. The hard part of this problem is to show that $\tau$ is also injective. One way to do this would be to show that $\abs{G}=n!$, but I'm not sure how to do that.

\end{document}
