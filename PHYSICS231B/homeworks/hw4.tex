\documentclass{article}
\usepackage[margin=1in]{geometry}
\usepackage[linesnumbered,ruled,vlined]{algorithm2e}
\usepackage{amsfonts}
\usepackage{amsmath}
\usepackage{amssymb}
\usepackage{amsthm}
\usepackage{enumitem}
\usepackage{fancyhdr}
\usepackage{hyperref}
\usepackage{minted}
\usepackage{multicol}
\usepackage{pdfpages}
\usepackage{standalone}
\usepackage[many]{tcolorbox}
\usepackage{tikz-cd}
\usepackage{transparent}
\usepackage{xcolor}
% \tcbuselibrary{minted}

\author{Nathan Solomon}

\newcommand{\fig}[1]{
    \begin{center}
        \includegraphics[width=\textwidth]{#1}
    \end{center}
}

% Math commands
\renewcommand{\d}{\mathrm{d}}
\DeclareMathOperator{\id}{id}
\DeclareMathOperator{\im}{im}
\DeclareMathOperator{\proj}{proj}
\DeclareMathOperator{\Span}{span}
\DeclareMathOperator{\Tr}{Tr}
\DeclareMathOperator{\tr}{tr}
\DeclareMathOperator{\ad}{ad}
\DeclareMathOperator{\ord}{ord}
%%%%%%%%%%%%%%% \DeclareMathOperator{\sgn}{sgn}
\DeclareMathOperator{\Aut}{Aut}
\DeclareMathOperator{\Inn}{Inn}
\DeclareMathOperator{\Out}{Out}
\DeclareMathOperator{\stab}{stab}

\newcommand{\N}{\ensuremath{\mathbb{N}}}
\newcommand{\Z}{\ensuremath{\mathbb{Z}}}
\newcommand{\Q}{\ensuremath{\mathbb{Q}}}
\newcommand{\R}{\ensuremath{\mathbb{R}}}
\newcommand{\C}{\ensuremath{\mathbb{C}}}
\renewcommand{\H}{\ensuremath{\mathbb{H}}}
\newcommand{\F}{\ensuremath{\mathbb{F}}}

\newcommand{\E}{\ensuremath{\mathbb{E}}}
\renewcommand{\P}{\ensuremath{\mathbb{P}}}

\newcommand{\es}{\ensuremath{\varnothing}}
\newcommand{\inv}{\ensuremath{^{-1}}}
\newcommand{\eps}{\ensuremath{\varepsilon}}
\newcommand{\del}{\ensuremath{\partial}}
\renewcommand{\a}{\ensuremath{\alpha}}

\newcommand{\abs}[1]{\ensuremath{\left\lvert #1 \right\rvert}}
\newcommand{\norm}[1]{\ensuremath{\left\lVert #1\right\rVert}}
\newcommand{\mean}[1]{\ensuremath{\left\langle #1 \right\rangle}}
\newcommand{\floor}[1]{\ensuremath{\left\lfloor #1 \right\rfloor}}
\newcommand{\ceil}[1]{\ensuremath{\left\lceil #1 \right\rceil}}
\newcommand{\bra}[1]{\ensuremath{\left\langle #1 \right\rvert}}
\newcommand{\ket}[1]{\ensuremath{\left\lvert #1 \right\rangle}}
\newcommand{\braket}[2]{\ensuremath{\left.\left\langle #1\right\vert #2 \right\rangle}}

\newcommand{\catname}[1]{{\normalfont\textbf{#1}}}

\newcommand{\up}{\ensuremath{\uparrow}}
\newcommand{\down}{\ensuremath{\downarrow}}

% Custom environments
\newtheorem{thm}{Theorem}[section]

\definecolor{probBackgroundColor}{RGB}{250,240,240}
\definecolor{probAccentColor}{RGB}{140,40,0}
\newenvironment{prob}{
    \stepcounter{thm}
    \begin{tcolorbox}[
        boxrule=1pt,
        sharp corners,
        colback=probBackgroundColor,
        colframe=probAccentColor,
        borderline west={4pt}{0pt}{probAccentColor},
        breakable
    ]
    \color{probAccentColor}\textbf{Problem \thethm.} \color{black}
} {
    \end{tcolorbox}
}

\definecolor{exampleBackgroundColor}{RGB}{212,232,246}
\newenvironment{example}{
    \stepcounter{thm}
    \begin{tcolorbox}[
      boxrule=1pt,
      sharp corners,
      colback=exampleBackgroundColor,
      breakable
    ]
    \textbf{Example \thethm.}
} {
    \end{tcolorbox}
}

\definecolor{propBackgroundColor}{RGB}{255,245,220}
\definecolor{propAccentColor}{RGB}{150,100,0}
\newenvironment{prop}{
    \stepcounter{thm}
    \begin{tcolorbox}[
        boxrule=1pt,
        sharp corners,
        colback=propBackgroundColor,
        colframe=propAccentColor,
        breakable
    ]
    \color{propAccentColor}\textbf{Proposition \thethm. }\color{black}
} {
    \end{tcolorbox}
}

\definecolor{thmBackgroundColor}{RGB}{235,225,245}
\definecolor{thmAccentColor}{RGB}{50,0,100}
\renewenvironment{thm}{
    \stepcounter{thm}
    \begin{tcolorbox}[
        boxrule=1pt,
        sharp corners,
        colback=thmBackgroundColor,
        colframe=thmAccentColor,
        breakable
    ]
    \color{thmAccentColor}\textbf{Theorem \thethm. }\color{black}
} {
    \end{tcolorbox}
}

\definecolor{corBackgroundColor}{RGB}{240,250,250}
\definecolor{corAccentColor}{RGB}{50,100,100}
\newenvironment{cor}{
    \stepcounter{thm}
    \begin{tcolorbox}[
        enhanced,
        boxrule=0pt,
        frame hidden,
        sharp corners,
        colback=corBackgroundColor,
        borderline west={4pt}{0pt}{corAccentColor},
        breakable
    ]
    \color{corAccentColor}\textbf{Corollary \thethm. }\color{black}
} {
    \end{tcolorbox}
}

\definecolor{lemBackgroundColor}{RGB}{255,245,235}
\definecolor{lemAccentColor}{RGB}{250,125,0}
\newenvironment{lem}{
    \stepcounter{thm}
    \begin{tcolorbox}[
        enhanced,
        boxrule=0pt,
        frame hidden,
        sharp corners,
        colback=lemBackgroundColor,
        borderline west={4pt}{0pt}{lemAccentColor},
        breakable
    ]
    \color{lemAccentColor}\textbf{Lemma \thethm. }\color{black}
} {
    \end{tcolorbox}
}

\definecolor{proofBackgroundColor}{RGB}{255,255,255}
\definecolor{proofAccentColor}{RGB}{80,80,80}
\renewenvironment{proof}{
    \begin{tcolorbox}[
        enhanced,
        boxrule=1pt,
        sharp corners,
        colback=proofBackgroundColor,
        colframe=proofAccentColor,
        borderline west={4pt}{0pt}{proofAccentColor},
        breakable
    ]
    \color{proofAccentColor}\emph{\textbf{Proof. }}\color{black}
} {
    \qed \end{tcolorbox}
}

\definecolor{noteBackgroundColor}{RGB}{240,250,240}
\definecolor{noteAccentColor}{RGB}{30,130,30}
\newenvironment{note}{
    \begin{tcolorbox}[
        enhanced,
        boxrule=0pt,
        frame hidden,
        sharp corners,
        colback=noteBackgroundColor,
        borderline west={4pt}{0pt}{noteAccentColor},
        breakable
    ]
    \color{noteAccentColor}\textbf{Note. }\color{black}
} {
    \end{tcolorbox}
}


\fancyhf{}
\setlength{\headheight}{24pt}

\date{\today}
\title{Physics 231B Homework \#4}

\begin{document}
\maketitle

\bigskip
\begin{prob}
    \textbf{Artin Chapter 10 Problem 2.1, page 314.} Prove that the standard three-dimensional representation of the tetrahedral group $T$ is irreducible as a complex representation.
\end{prob}
The best way to approach this problem is to imagine a regular tetrahedron whose vertices are in a unit sphere, and think of $T$ is the set of rotations preserving that tetrahedron.
\par
Suppose $T$ is reducible, meaning there is a nontrivial subspace of $\R^3$ which is invariant under the action of $T$. Such a subspace must be either a line through the origin or a plane through the origin. Since a plane is the orthogonal complement of a line in $\R^3$, finding a $T$-invariant plane and finding a $T$-invariant line are the same problem. Therefore I will only consider a $T$-invariant line.
\par
A $T$-invariant line could not exist because the group of rotations which fixes that line is $O(2)$. Every finite subgroup of $O(2)$ is the trivial group, a cyclic group, or a dihedral group. $T$ is none of those, so the line is not $T$-invariant.
\par
This is a contradiction, so $T$ is irreducible.

\bigskip
\begin{prob}
    \textbf{Artin Chapter 10 Problem 3.1, page 315.} Let $G$ be a cyclic group of order 3. The matrix $A= \begin{bmatrix}
        -1 & -1 \\
        1 & 0
    \end{bmatrix}$ has order 3, so it defines a matrix representation of $G$. Use the averaging process to produce a $G$-invariant form from the standard Hermitian product $X * Y$ on $\C^2$.
\end{prob}
\begin{note}
    ``$*$" denotes elementwise matrix multiplication, just like in Python.
\end{note}
The elements of $G$ are
\[ G = \left\{ \begin{bmatrix}
        1 & 0 \\
        0 & 1
\end{bmatrix}, \begin{bmatrix}
        -1 & -1 \\
        1 & 0
\end{bmatrix}, \begin{bmatrix}
        0 & 1 \\
        -1 & -1
\end{bmatrix} \right\}. \]
The average of the Hermitian norm squared of each element of $G$ is
\begin{align*}
    \frac{1}{3} \sum_{g \in G} g^\dag g &= \frac{1}{3} \left( I_2^\dag I_2 + A^\dag A + (A^2)^\dag (A^2) \right) \\
                                       &= \frac{1}{3} \left( \begin{bmatrix}
                                               1 & 0 \\
                                               0 & 1
                                       \end{bmatrix} \begin{bmatrix}
                                               1 & 0 \\
                                               0 & 1
                                       \end{bmatrix} + \begin{bmatrix}
                                               -1 & 1 \\
                                               -1 & 0
                                       \end{bmatrix} \begin{bmatrix}
                                               -1 & -1 \\
                                               1 & 0
                                       \end{bmatrix} + \begin{bmatrix}
                                               0 & -1 \\
                                               1 & -1
                                       \end{bmatrix} \begin{bmatrix}
                                               0 & 1 \\
                                               -1 & -1
                                       \end{bmatrix} \right) \\
                                       &= \frac{1}{3} \left( \begin{bmatrix}
                                               1 & 0 \\
                                               0 & 1
                                       \end{bmatrix} + \begin{bmatrix}
                                               2 & 1 \\
                                               1 & 1
                                       \end{bmatrix} + \begin{bmatrix}
                                               1 & 1 \\
                                               1 & 2
                                       \end{bmatrix} \right) \\
                                       &= \frac{2}{3} \left( \begin{bmatrix}
                                               2 & 1 \\
                                               1 & 2
                                       \end{bmatrix} \right) 
\end{align*}
Therefore we can define an inner product form which maps $v \in \C^2$ to $v^\dag \left( \frac{2}{3} \begin{bmatrix}
        2 & 1 \\
        1 & 2
\end{bmatrix} \right) v$. This is $G$ invariant because
\begin{align*}
    \braket{Au}{Av} &= (Au)^\dag \left( \frac{2}{3} \begin{bmatrix}
        2 & 1 \\
        1 & 2
\end{bmatrix} \right) (Av) \\
&= u^\dag A^\dag \left( \frac{2}{3} \begin{bmatrix}
        2 & 1 \\
        1 & 2
\end{bmatrix} \right) Av \\
&= u^\dag \left( \frac{2}{3} \begin{bmatrix}
        -1 & 1 \\
        -1 & 0
\end{bmatrix} \begin{bmatrix}
        2 & 1 \\
        1 & 2
\end{bmatrix} \begin{bmatrix}
        -1 & -1 \\
        1 & 0
\end{bmatrix} \right) v \\
&= u^\dag \left( \frac{2}{3} \begin{bmatrix}
        2 & 1 \\
        1 & 2
\end{bmatrix} \right) v \\
&= \braket{u}{v},
\end{align*}
meaning the action of $I_2$, $A$, or $A^2$ on any vectors $u, v \in \C^2$ will not change the inner product $ \braket{u}{v} $.
\bigskip
\begin{prob}
    \textbf{Artin Chapter 10 Problem 3.2, page 315.} Let $\rho: G \rightarrow GL(V)$ be a representation of a finite group on a real vector space $V$. Prove the following:
    \begin{itemize}
        \item (a) There exists a $G$-invariant, positive-definite symmetric form $\mean{,}$ on $V$.
        \item (b) $\rho$ is a direct sum of irreducible representations.
        \item (c) Every finite subgroup of $GL_n(\R)$ is conjugate to a subgroup of $O(n)$.
    \end{itemize}
\end{prob}
\begin{itemize}
    \item (a) We can use the same averaging process as in the above problem. Define a matrix $M$ to be
        \[ M := \sum_{g \in G} \rho(g)^T \rho(g) \]
        and define a symmetric form $ \braket{x}{y} = x^TMy$. This is bilinear, symmetric, positive-definite, and $G$-invariant.
    \item (b) Assume $V$ is finite dimensional. If there is some nontrivial subspace $U$ of $V$, then $V = U \oplus U^\perp$. Every action $\rho(g)$ will map elements of $U$ to elements of $U$ and elements of $U^\perp$ to elements of $U^\perp$. Therefore the image of $\rho$ is contained in $GL(U) \oplus GL(U^\perp)$.
        \par
        This shows that any reducible finite-dimensional representation of $G$ can be rewritten as a direct sum of representations with strictly smaller dimensions. Every one-dimensional representation is trivially irreducible, so by induction, $\rho$ is a direct sum of irreducible representations.
    \item (c)
\end{itemize}

\bigskip
\begin{prob}
    Given a representation $R = \bigoplus_i n_iR_i$ decomposed into irreps $R_i$, prove that the operator
    \[ P_i = \frac{\dim(R_i)}{|G|} \sum_{g \in G} \chi_i(g)^*R_g \]
    is a projector onto the subspace $n_iR_i$ of $R$.
\end{prob}
There are a few ways to show that $P_i$ is a projection operator -- one method is to show that both $P_i^2=P_i$ and $\Tr(P_i) = \dim(P_i)$.
\par
\begin{align*}
    \Tr(P_i) &= \frac{\dim(P_i)}{|G|} \sum{g \in G} \chi_i^*(g) \chi_i(g) \\
             &= \dim(R_i).
\end{align*}
Showing that $P_i^2=P_i$ is a bit harder because it requires re-indexing:
\begin{align*}
    P_i^2 &= \frac{\dim(R_i)^2}{\abs{G}^2} \sum_{g_1 \in G} \sum_{g_2 \in G} \chi_i^*(g_1) \chi_i^*(g_2) R_{g_1} R_{g_2} \\
          &= \frac{\dim(R_i)}{\abs{G}^2} \sum_{g_1 \in G} \sum_{g_1g_2 \in G} \chi_i^*(g_1) \chi_i^*(g_1g_2) R_{g_1g_2} \\
          &= \frac{\dim(R_i)}{\abs{G}^2} \sum_{g_1 \in G} \chi_i^{reg}(g_1) \sum_{g_1g_2 \in G} \chi_i^*(g_1g_2) R_{g_1g_2} \\
          &= \frac{\dim(R_i)}{\abs{G}} \sum_{(g_1g_2) \in G} \chi_i^*(g_1g_2) R_{(g_1g_2)} \\
          &= P_i.
\end{align*}

\bigskip
\begin{prob}
    \begin{itemize}
        \item (a) Given a representation $R$, prove that the following defines a representation:
            \[ R_g^\vee = \left( R_g^{-1} \right)^T. \]
            ($R^\vee$ is called the dual representation.)
        \item (b) Show there is a $G$-equivariant linear map $R \otimes R^\vee \rightarrow \C$.
        \item (c) Prove that if $M$ is the vector space of matrices on $R$, conjugation of these matrices endows $M$ with a $G$ representation isomorphic to $R \otimes R^\vee$.
    \end{itemize}
\end{prob}
\begin{itemize}
    \item (a) Let $R$ be a representation which maps $g \in G$ to $R_g \in GL(V)$. Then let $\varphi$ be the function which maps $R_g$ to $R_g^\vee := (R_g^{-1})^T$. Then $\varphi \circ R$ is also a representation, because
        \[ (\varphi \circ R)(g) (\varphi \circ R) (g') = (R_g^{-1})^T (R_{g'}^{-1})^T = (R_{g'}^{-1}R_g^{-1})^T = ((R_gR_{g'})^{-1})^T = (\varphi \circ R)(gg'). \]
    \item (b) Let $R'$ be shorthand for $R \otimes R^\vee$. We want to find some linear map $A: R' \rightarrow \C$ such that 
    \item (c)
\end{itemize}

\bigskip
\begin{prob}
    For each pair $R_i$, $R_j$ of irreps of $S_3$, compute the decomposition of $R_i \otimes R_j$ into irreps. (Together with direct sum, this gives irreps of $G$ the structure of a ring called the ``representation ring".)
\end{prob}
The 3 irreps of $S_3$ are the trivial representation $R_1$, the sign representation $R_2$, and the faithful representation $R_3$.
\begin{center}
    \begin{tabular}{c|ccc}
        Conjugacy class & Trivial permutation & 2-cycle & 3-cycle \\
        \hline
        Character in $R_1$ & 1 & 1 & 1 \\
        Character in $R_2$ & 1 & -1 & 1 \\
        Character in $R_3$ & 2 & 0 & -1 \\
        \hline
        \# of elements & 1 & 3 & 2
    \end{tabular}
\end{center}
Now that we have the character table, we can simply use the rules $\chi(R_i \otimes R_j)=\chi(R_i)\chi(R_j)$ and $\chi(R_i \oplus R_j) = \chi(R_i) + \chi(R_j)$ to decompose all the pairs $R_i \otimes R_j$ into irreps.
\begin{align*}
    R_1 \otimes R_1 &= R_1 \\
    R_1 \otimes R_2 &= R_2 \\
    R_1 \otimes R_3 &= R_3 \\
    R_2 \otimes R_2 &= R_1 \\
    R_2 \otimes R_3 &= R_3 \\
    R_3 \otimes R_3 &= R_1 \oplus R_2 \oplus R_3 \\
\end{align*}

\end{document}
