\documentclass{article}
\usepackage[margin=1in]{geometry}
\usepackage[linesnumbered,ruled,vlined]{algorithm2e}
\usepackage{amsfonts}
\usepackage{amsmath}
\usepackage{amssymb}
\usepackage{amsthm}
\usepackage{chemformula}
\usepackage{enumitem}
\usepackage{fancyhdr}
\usepackage{graphicx}
\usepackage{hyperref}
\usepackage{listings}
\usepackage{minted}
\usepackage{multicol}
\usepackage{pdfpages}
\usepackage{siunitx}
\usepackage{standalone}
\usepackage{svg}
\usepackage[many]{tcolorbox}
\usepackage{tikz-cd}
\usepackage{transparent}
\usepackage{xcolor}
% \tcbuselibrary{minted}

\author{Nathan Solomon}

\newcommand{\fig}[1]{
    \begin{center}
        \includegraphics[width=\textwidth]{#1}
    \end{center}
}

% Math commands
\renewcommand{\d}{\mathrm{d}}
\DeclareMathOperator{\id}{id}
\DeclareMathOperator{\im}{im}
\DeclareMathOperator{\proj}{proj}
\DeclareMathOperator{\Span}{span}
\DeclareMathOperator{\Tr}{Tr}
\DeclareMathOperator{\tr}{tr}
\DeclareMathOperator{\ad}{ad}
\DeclareMathOperator{\ord}{ord}
%%%%%%%%%%%%%%% \DeclareMathOperator{\sgn}{sgn}
\DeclareMathOperator{\Aut}{Aut}
\DeclareMathOperator{\Inn}{Inn}
\DeclareMathOperator{\Out}{Out}
\DeclareMathOperator{\stab}{stab}

\newcommand{\N}{\ensuremath{\mathbb{N}}}
\newcommand{\Z}{\ensuremath{\mathbb{Z}}}
\newcommand{\Q}{\ensuremath{\mathbb{Q}}}
\newcommand{\R}{\ensuremath{\mathbb{R}}}
\newcommand{\C}{\ensuremath{\mathbb{C}}}
\renewcommand{\H}{\ensuremath{\mathbb{H}}}
\newcommand{\F}{\ensuremath{\mathbb{F}}}

\newcommand{\E}{\ensuremath{\mathbb{E}}}
\renewcommand{\P}{\ensuremath{\mathbb{P}}}

\newcommand{\es}{\ensuremath{\varnothing}}
\newcommand{\inv}{\ensuremath{^{-1}}}
\newcommand{\eps}{\ensuremath{\varepsilon}}
\newcommand{\del}{\ensuremath{\partial}}
\renewcommand{\a}{\ensuremath{\alpha}}

\newcommand{\abs}[1]{\ensuremath{\left\lvert #1 \right\rvert}}
\newcommand{\norm}[1]{\ensuremath{\left\lVert #1\right\rVert}}
\newcommand{\mean}[1]{\ensuremath{\left\langle #1 \right\rangle}}
\newcommand{\floor}[1]{\ensuremath{\left\lfloor #1 \right\rfloor}}
\newcommand{\ceil}[1]{\ensuremath{\left\lceil #1 \right\rceil}}
\newcommand{\bra}[1]{\ensuremath{\left\langle #1 \right\rvert}}
\newcommand{\ket}[1]{\ensuremath{\left\lvert #1 \right\rangle}}
\newcommand{\braket}[2]{\ensuremath{\left.\left\langle #1\right\vert #2 \right\rangle}}

\newcommand{\catname}[1]{{\normalfont\textbf{#1}}}

\newcommand{\up}{\ensuremath{\uparrow}}
\newcommand{\down}{\ensuremath{\downarrow}}

% Custom environments
\newtheorem{thm}{Theorem}[section]

\definecolor{probBackgroundColor}{RGB}{250,240,240}
\definecolor{probAccentColor}{RGB}{140,40,0}
\newenvironment{prob}{
    \stepcounter{thm}
    \begin{tcolorbox}[
        boxrule=1pt,
        sharp corners,
        colback=probBackgroundColor,
        colframe=probAccentColor,
        borderline west={4pt}{0pt}{probAccentColor},
        breakable
    ]
    \color{probAccentColor}\textbf{Problem \thethm.} \color{black}
} {
    \end{tcolorbox}
}

\definecolor{exampleBackgroundColor}{RGB}{212,232,246}
\newenvironment{example}{
    \stepcounter{thm}
    \begin{tcolorbox}[
      boxrule=1pt,
      sharp corners,
      colback=exampleBackgroundColor,
      breakable
    ]
    \textbf{Example \thethm.}
} {
    \end{tcolorbox}
}

\definecolor{propBackgroundColor}{RGB}{255,245,220}
\definecolor{propAccentColor}{RGB}{150,100,0}
\newenvironment{prop}{
    \stepcounter{thm}
    \begin{tcolorbox}[
        boxrule=1pt,
        sharp corners,
        colback=propBackgroundColor,
        colframe=propAccentColor,
        breakable
    ]
    \color{propAccentColor}\textbf{Proposition \thethm. }\color{black}
} {
    \end{tcolorbox}
}

\definecolor{thmBackgroundColor}{RGB}{235,225,245}
\definecolor{thmAccentColor}{RGB}{50,0,100}
\renewenvironment{thm}{
    \stepcounter{thm}
    \begin{tcolorbox}[
        boxrule=1pt,
        sharp corners,
        colback=thmBackgroundColor,
        colframe=thmAccentColor,
        breakable
    ]
    \color{thmAccentColor}\textbf{Theorem \thethm. }\color{black}
} {
    \end{tcolorbox}
}

\definecolor{corBackgroundColor}{RGB}{240,250,250}
\definecolor{corAccentColor}{RGB}{50,100,100}
\newenvironment{cor}{
    \stepcounter{thm}
    \begin{tcolorbox}[
        enhanced,
        boxrule=0pt,
        frame hidden,
        sharp corners,
        colback=corBackgroundColor,
        borderline west={4pt}{0pt}{corAccentColor},
        breakable
    ]
    \color{corAccentColor}\textbf{Corollary \thethm. }\color{black}
} {
    \end{tcolorbox}
}

\definecolor{lemBackgroundColor}{RGB}{255,245,235}
\definecolor{lemAccentColor}{RGB}{250,125,0}
\newenvironment{lem}{
    \stepcounter{thm}
    \begin{tcolorbox}[
        enhanced,
        boxrule=0pt,
        frame hidden,
        sharp corners,
        colback=lemBackgroundColor,
        borderline west={4pt}{0pt}{lemAccentColor},
        breakable
    ]
    \color{lemAccentColor}\textbf{Lemma \thethm. }\color{black}
} {
    \end{tcolorbox}
}

\definecolor{proofBackgroundColor}{RGB}{255,255,255}
\definecolor{proofAccentColor}{RGB}{80,80,80}
\renewenvironment{proof}{
    \begin{tcolorbox}[
        enhanced,
        boxrule=1pt,
        sharp corners,
        colback=proofBackgroundColor,
        colframe=proofAccentColor,
        borderline west={4pt}{0pt}{proofAccentColor},
        breakable
    ]
    \color{proofAccentColor}\emph{\textbf{Proof. }}\color{black}
} {
    \qed \end{tcolorbox}
}

\definecolor{noteBackgroundColor}{RGB}{240,250,240}
\definecolor{noteAccentColor}{RGB}{30,130,30}
\newenvironment{note}{
    \begin{tcolorbox}[
        enhanced,
        boxrule=0pt,
        frame hidden,
        sharp corners,
        colback=noteBackgroundColor,
        borderline west={4pt}{0pt}{noteAccentColor},
        breakable
    ]
    \color{noteAccentColor}\textbf{Note. }\color{black}
} {
    \end{tcolorbox}
}


\fancyhf{}
\setlength{\headheight}{24pt}

\date{\today}
\title{Physics 231B Homework \#2}

\begin{document}
\maketitle

\bigskip
\begin{prob}
    \textbf{Artin Chapter 6 Problem 4.2, page 188.}
    \begin{itemize}
        \item (a) List all subgroups of the dihedral group $D_4$, and decide which ones are normal.
        \item (b) List all the proper subgroups $N$ of the dihedral group $D_{15}$, and identify the quotient groups $D_{15}/N$.
        \item (c) List the subgroups of $D_6$ that do not contain $x^3$.
    \end{itemize}
\end{prob}
\begin{itemize}
    \item (a) Using the notation from Artin,
        \[ D_4 = \braket{x,y}{x^4=y^2=xyxy=e}. \]
        Since this group has order 8, by Lagrange's theorem, every subgroup must have order 1, 2, 4, or 8. Knowing that, we can methodically list all subgroups. The trivial group $\{e\}$ and the entire group $D_4$ are both normal subgroups of $D_4$, so we only need to think about the proper subgroups, which have 2 or 8 elements.
        \par
        If a subgroup of $D_4$ has 2 elements, it must be generated by one element of order two; therefore it must be either $ \left\{ e, x^2 \right\}$ or $ \left\{ e, y \right\}$. $ \left\{ e, x^2 \right\}$ is normal, but $ \left\{ e, y \right\}$ is not, because $x^{-1}yx=yx^2 \not\in \left\{ e, y \right\}$. 
        \par
        If a subgroup of $D_4$ has 4 elements, it is isomorphic to either $C_4$ or $K_4$. If it is isomorphic to $C_4$, it is generated by an element of order 4, so it must be $ \left\{ e, x, x^2, x^3 \right\}$, which is normal in $D_4$. If the subgroup is isomorphic to $K_4$, it is generated by the only two elements in $D_4$ which have order 2, so it is $ \left\{ e, x^2, y, x^2y \right\}$, which is also normal in $D_4$
    \item (b) Since $D_{15}$ has order 30, every proper subgroup of it must have order 2, 3, 5, 6, 10, or 15. From here, we can guess all of those subgroups:
        \begin{itemize}
            \item $\mean{y}$
            \item $\mean{x^5}$
            \item $\mean{x^3}$
            \item $\mean{y, x^5}$
            \item $\mean{y, x^3}$
            \item $\mean{x^3, x^5} = \mean{x}$
        \end{itemize}
        To find out the quotient by each of those subgroups, consider some homomorphism from $D_{15}$ to $N$ which takes an element $y$, $x^5$, or $x^3$ to $e$ iff that element is in $N$. Then the image of that homomorphism is the quotient group, and is generated by the intersection of $N$ with $ \left\{ y, x^5, x^3 \right\}$.
        \begin{itemize}
            \item $D_{15} / \mean{y} = \mean{x} \cong C_{15}$
            \item $D_{15} / \mean{x^5} = \mean{y, x^3} \cong D_5$
            \item $D_{15} / \mean{x^3} = \mean{y, x^5} \cong D_3$
            \item $D_{15} / \mean{y, x^5} = \mean{x^3} \cong C_5$
            \item $D_{15} / \mean{y, x^3} = \mean{x^5} \cong C_3$
            \item $D_{15} / \mean{x} = \mean{y} \cong C_2$
        \end{itemize}
    \item (c) Since $D_6$ is generated by $y$, $x^3$, and $x^2$, the subgroups of $D_6$ which do not contain $x^3$ are
        \begin{itemize}
            \item $\{e\}$
            \item $\mean{y} = \left\{ e, y \right\} \cong C_2$
            \item $\mean{x^2} = \left\{ e, x^2, x^4 \right\} \cong C_3$
            \item $\mean{y, x^2} = \left\{ e, x^2, x^4, y, yx^2, yx^4 \right\} \cong D_3$
        \end{itemize}
        
\end{itemize}

\bigskip
\begin{prob}
    \textbf{Artin Chapter 6 Problem 4.3, page 188.}
    \begin{itemize}
        \item (a) Compute the left cosets of the subgroup $H = \left\{ 1, x^5 \right\}$ in the dihedral group $D_{10}$.
        \item (b) Prove that $H$ is normal and that $D_{10}/H$ is isomorphic to $D_5$.
        \item (c) Is $D_{10}$ isomorphic to $D_5 \times H$?
    \end{itemize}
\end{prob}
\begin{itemize}
    \item (a) By Lagrange's theorem, there are 10 left cosets of $H$. Those are $\left\{1,x^5\right\}$, $\left\{x,x^6\right\}$, $\left\{x^2,x^7\right\}$, $\left\{x^3,x^8\right\}$, $\left\{x^4,x^9\right\}$, $\left\{y,yx^5\right\}$, $\left\{yx,yx^6\right\}$, $\left\{yx^2,yx^7\right\}$, $\left\{yx^3,yx^8\right\}$, and $\left\{yx^4,yx^9\right\}$.
    \item (b)
        \[ x^{-1}Hx = \left\{ x^{-1}1x, x^{-1}x^5x \right\} = H \]
        \[ y^{-1}Hy = \left\{ yy, yx^5y \right\} = H \]
        Since $H$ is closed under conjugation by $x$ and by $y$, and since $D_{10} = \mean{x, y}$, $H \trianglelefteq D_{10}$.
        \par
        If you label the coset $ \left\{ y, yx^5 \right\}$ as $y'$ and $ \left\{ x, x^6 \right\}$ as $x'$, then we can observe that $ \mean{x', y'}$ has 10 elements, and that $x'^5=y'^2=x'y'x'y'=H$, so $D_{10}/H\cong D_5$. 
    \item (c) Yes, because $D_{10}$ is generated by $x^2$, $x^5$, and $y$. Since $x^5$ commutes with both $x^2$ and $y$, we can say
        \[ D_{10} = \mean{x^2, x^5, y} \cong \mean{x^2, y} \times \mean{x^5} = D_{5} \times H. \]
        
\end{itemize}

\bigskip
\begin{prob}
    \textbf{Artin Chapter 6 Problem 5.1, page 188.} Let $\ell_1$ and $\ell_2$ be lines through the origin in $\R^2$ that intersect in an angle $\pi/n$, and let $r_i$ be the reflection about $\ell_i$. Prove that $r_1$ and $r_2$ generate a dihedral group $D_n$.
\end{prob}
Let $q=r_1r_2$. Since $q$ is the product of 2 reflections, it must be a rotation, but we want to know by how much. By considering a point (other than the origin) on $\ell_1$, we can see that $q$ is a rotation about the origin by $2\pi/n$. Therefore $q$ has order $n$, so $r_1$ and $q$ together generate all $2n$ elements of $D_n$.

\bigskip
\begin{prob}
    \textbf{Artin Chapter 6 Problem 5.10, page 189.} Let $f$ and $g$ be rotations of the plane about distinct points, with arbitrary nonzero angles of rotation $\theta$ and $\phi$. Prove that the group generated by $f$ and $g$ contains a translation.
\end{prob}
There is a homomorphism from the group of rotation and reflections of the plane to $U(1)$, defined by mapping translations to $1$ and mapping a rotation by an angle $x$ to $e^{ix}$. The kernel of this map is the group of translations of the plane. Using the definition of a homomorphism, $f^{-1}g^{-1}fg$ is in that kernel of that map, since it gets mapped to $\exp(i\cdot(-\theta - \phi + \theta + \phi)) = 1$. In other words, the total angle of rotation is zero.
\par
This means that to prove $f^{-1}g^{-1}fg$ is a translation, we just need to show that it is not the identity map. $f$ and $g$ do not commute because if $x$ is the fixed point of $f$, then $gfx=gx$, and since $gx\neq x$, $fgx\neq gx$. Since $gf \neq fg$, $f^{-1}g^{-1}fg$ is a non-identity element of the group of translations of the plane.

\bigskip
\begin{prob}
    \textbf{Artin Chapter 6 Problem 7.7, page 191.} Let $G = GL_n(\R)$ operate on the set $V = \R^n$ by left multiplication.
    \begin{itemize}
        \item (a) Describe the decomposition of $V$ into orbits for this operation.
        \item (b) What is the stabilizer of $e_1$?
    \end{itemize}
\end{prob}
\begin{itemize}
    \item (a) There are only two orbits: one contains the zero vector, and the other contains everything else. That's because every matrix maps the zero vector to itself, and for any nonzero vectors $x, y \in V$, there is an invertible matrix which maps $x$ to $y$.
    \item (b) The stabilizer of $e_1$ is the set of matrices $A$ such that $Ae_1=e_1$. That is true iff the top left entry of $A$ is 1 and every other entry in the leftmost column of $A$ is 0.
\end{itemize}

\bigskip
\begin{prob}
    \textbf{Artin Chapter 6 Problem 9.1, page 191.} Use the counting formula to determine the orders of the groups of rotational symmetries of a cube and of a tetrahedron.
\end{prob}
For both the cube and the tetrahedron, consider a point $x$ in the center of a face. In the case of a cube, the stabilizer of $x$ has order 4, because the face it's on can be rotated 4 ways, and the orbit of $x$ has 6 distinct points (because there are 6 faces, and a rotation could map $x$ to the center of any face. By the counting formula, the group of rotational symmetries of a cube has order $4 \cdot 6 = 24$.
\par
Similarly, if $x$ is in the center of a face of a tetrahedron, then its orbit has 4 points (the centers of each face), and the stabilizer of $x$ has order 3 (there are 3 rotational symmetries of the face which $x$ lies on). Therefore the group of rotational symmetries of a tetrahedron has order $3 \cdot 4 = 12$.

\end{document}
