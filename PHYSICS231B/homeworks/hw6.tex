\documentclass{article}
\usepackage[margin=1in]{geometry}
\usepackage[linesnumbered,ruled,vlined]{algorithm2e}
\usepackage{amsfonts}
\usepackage{amsmath}
\usepackage{amssymb}
\usepackage{amsthm}
\usepackage{chemformula}
\usepackage{enumitem}
\usepackage{fancyhdr}
\usepackage{graphicx}
\usepackage{hyperref}
\usepackage{listings}
\usepackage{minted}
\usepackage{multicol}
\usepackage{pdfpages}
\usepackage{siunitx}
\usepackage{standalone}
\usepackage{svg}
\usepackage[many]{tcolorbox}
\usepackage{tikz-cd}
\usepackage{transparent}
\usepackage{xcolor}
% \tcbuselibrary{minted}

\author{Nathan Solomon}

\newcommand{\fig}[1]{
    \begin{center}
        \includegraphics[width=\textwidth]{#1}
    \end{center}
}

% Math commands
\renewcommand{\d}{\mathrm{d}}
\DeclareMathOperator{\id}{id}
\DeclareMathOperator{\im}{im}
\DeclareMathOperator{\proj}{proj}
\DeclareMathOperator{\Span}{span}
\DeclareMathOperator{\Tr}{Tr}
\DeclareMathOperator{\tr}{tr}
\DeclareMathOperator{\ad}{ad}
\DeclareMathOperator{\ord}{ord}
%%%%%%%%%%%%%%% \DeclareMathOperator{\sgn}{sgn}
\DeclareMathOperator{\Aut}{Aut}
\DeclareMathOperator{\Inn}{Inn}
\DeclareMathOperator{\Out}{Out}
\DeclareMathOperator{\stab}{stab}

\newcommand{\N}{\ensuremath{\mathbb{N}}}
\newcommand{\Z}{\ensuremath{\mathbb{Z}}}
\newcommand{\Q}{\ensuremath{\mathbb{Q}}}
\newcommand{\R}{\ensuremath{\mathbb{R}}}
\newcommand{\C}{\ensuremath{\mathbb{C}}}
\renewcommand{\H}{\ensuremath{\mathbb{H}}}
\newcommand{\F}{\ensuremath{\mathbb{F}}}

\newcommand{\E}{\ensuremath{\mathbb{E}}}
\renewcommand{\P}{\ensuremath{\mathbb{P}}}

\newcommand{\es}{\ensuremath{\varnothing}}
\newcommand{\inv}{\ensuremath{^{-1}}}
\newcommand{\eps}{\ensuremath{\varepsilon}}
\newcommand{\del}{\ensuremath{\partial}}
\renewcommand{\a}{\ensuremath{\alpha}}

\newcommand{\abs}[1]{\ensuremath{\left\lvert #1 \right\rvert}}
\newcommand{\norm}[1]{\ensuremath{\left\lVert #1\right\rVert}}
\newcommand{\mean}[1]{\ensuremath{\left\langle #1 \right\rangle}}
\newcommand{\floor}[1]{\ensuremath{\left\lfloor #1 \right\rfloor}}
\newcommand{\ceil}[1]{\ensuremath{\left\lceil #1 \right\rceil}}
\newcommand{\bra}[1]{\ensuremath{\left\langle #1 \right\rvert}}
\newcommand{\ket}[1]{\ensuremath{\left\lvert #1 \right\rangle}}
\newcommand{\braket}[2]{\ensuremath{\left.\left\langle #1\right\vert #2 \right\rangle}}

\newcommand{\catname}[1]{{\normalfont\textbf{#1}}}

\newcommand{\up}{\ensuremath{\uparrow}}
\newcommand{\down}{\ensuremath{\downarrow}}

% Custom environments
\newtheorem{thm}{Theorem}[section]

\definecolor{probBackgroundColor}{RGB}{250,240,240}
\definecolor{probAccentColor}{RGB}{140,40,0}
\newenvironment{prob}{
    \stepcounter{thm}
    \begin{tcolorbox}[
        boxrule=1pt,
        sharp corners,
        colback=probBackgroundColor,
        colframe=probAccentColor,
        borderline west={4pt}{0pt}{probAccentColor},
        breakable
    ]
    \color{probAccentColor}\textbf{Problem \thethm.} \color{black}
} {
    \end{tcolorbox}
}

\definecolor{exampleBackgroundColor}{RGB}{212,232,246}
\newenvironment{example}{
    \stepcounter{thm}
    \begin{tcolorbox}[
      boxrule=1pt,
      sharp corners,
      colback=exampleBackgroundColor,
      breakable
    ]
    \textbf{Example \thethm.}
} {
    \end{tcolorbox}
}

\definecolor{propBackgroundColor}{RGB}{255,245,220}
\definecolor{propAccentColor}{RGB}{150,100,0}
\newenvironment{prop}{
    \stepcounter{thm}
    \begin{tcolorbox}[
        boxrule=1pt,
        sharp corners,
        colback=propBackgroundColor,
        colframe=propAccentColor,
        breakable
    ]
    \color{propAccentColor}\textbf{Proposition \thethm. }\color{black}
} {
    \end{tcolorbox}
}

\definecolor{thmBackgroundColor}{RGB}{235,225,245}
\definecolor{thmAccentColor}{RGB}{50,0,100}
\renewenvironment{thm}{
    \stepcounter{thm}
    \begin{tcolorbox}[
        boxrule=1pt,
        sharp corners,
        colback=thmBackgroundColor,
        colframe=thmAccentColor,
        breakable
    ]
    \color{thmAccentColor}\textbf{Theorem \thethm. }\color{black}
} {
    \end{tcolorbox}
}

\definecolor{corBackgroundColor}{RGB}{240,250,250}
\definecolor{corAccentColor}{RGB}{50,100,100}
\newenvironment{cor}{
    \stepcounter{thm}
    \begin{tcolorbox}[
        enhanced,
        boxrule=0pt,
        frame hidden,
        sharp corners,
        colback=corBackgroundColor,
        borderline west={4pt}{0pt}{corAccentColor},
        breakable
    ]
    \color{corAccentColor}\textbf{Corollary \thethm. }\color{black}
} {
    \end{tcolorbox}
}

\definecolor{lemBackgroundColor}{RGB}{255,245,235}
\definecolor{lemAccentColor}{RGB}{250,125,0}
\newenvironment{lem}{
    \stepcounter{thm}
    \begin{tcolorbox}[
        enhanced,
        boxrule=0pt,
        frame hidden,
        sharp corners,
        colback=lemBackgroundColor,
        borderline west={4pt}{0pt}{lemAccentColor},
        breakable
    ]
    \color{lemAccentColor}\textbf{Lemma \thethm. }\color{black}
} {
    \end{tcolorbox}
}

\definecolor{proofBackgroundColor}{RGB}{255,255,255}
\definecolor{proofAccentColor}{RGB}{80,80,80}
\renewenvironment{proof}{
    \begin{tcolorbox}[
        enhanced,
        boxrule=1pt,
        sharp corners,
        colback=proofBackgroundColor,
        colframe=proofAccentColor,
        borderline west={4pt}{0pt}{proofAccentColor},
        breakable
    ]
    \color{proofAccentColor}\emph{\textbf{Proof. }}\color{black}
} {
    \qed \end{tcolorbox}
}

\definecolor{noteBackgroundColor}{RGB}{240,250,240}
\definecolor{noteAccentColor}{RGB}{30,130,30}
\newenvironment{note}{
    \begin{tcolorbox}[
        enhanced,
        boxrule=0pt,
        frame hidden,
        sharp corners,
        colback=noteBackgroundColor,
        borderline west={4pt}{0pt}{noteAccentColor},
        breakable
    ]
    \color{noteAccentColor}\textbf{Note. }\color{black}
} {
    \end{tcolorbox}
}


\fancyhf{}
\setlength{\headheight}{24pt}

\date{\today}
\title{Physics 231B Homework \#6}

\begin{document}
\maketitle
Most of the facts about manifolds we discussed and used, and things about the fundamental group can be found in Nakahara's book ``Geometry, Topology, and Physics". The structure theorems on Lie groups and algebras can be found in Kirillov Ch 3 and Hall Ch 5.

\bigskip
\begin{prob}
    Suppose $G$ is a closed subgroup of $GL(n,\R)$, and $\gamma(t):(-1,1) \rightarrow G$ is a smooth map with $\gamma(0)=I$. Show that
    \[ \frac{d}{dt}\gamma(t) |_{t=0} \in \mathfrak{g}, \]
    that is, that if $A= \frac{d}{dt}\gamma(t) |_{t=0}$, then $\exp(tA) \in G$ for all $t$. (This gives an identification of the tangent space of $G$ at $I$ with its Lie algebra.)
\end{prob}

\bigskip
\begin{prob}
    On the last homework you constructed a map $SU(2) \times SU(2) \rightarrow SO(4)$. Now show explicitly a Lie algebra isomorphism between $\mathfrak{su}(2) \oplus \mathfrak{su}(2)$ and $\mathfrak{so}(4)$.
\end{prob}

\bigskip
\begin{prob}
    \begin{itemize}
        \item (a) Classify the representations of $SO(4)$ in terms of representations of $SU(2) \times SU(2)$.
        \item (b) What $SU(2) \times SU(2)$ representation corresponds to the vector representation of $SO(4)$?
        \item (c) (bonus) What $SO(4)$ representation corresponds to the $SU(2) \times SU(2)$ representation $(1,0)$?
    \end{itemize}
\end{prob}

\bigskip
\begin{prob}
    There is an invertible $\R$-linear map $F: \C^n \rightarrow \R^{2n}$ given by
    \[ f(x_1+iy_1, \dots, x_n + iy_n) = (x_1, \dots, x_n, y_1, \dots, y_n). \]
    \begin{itemize}
        \item (a) Prove that $F(iz)=JF(z)$ where $J$ is the matrix
            \[ \begin{pmatrix}
                0 & I_n \\
                -I_n & 0
            \end{pmatrix}. \]
        \item (b) Prove that a $2n \times 2n$ real matrix $N$ is the image of an $n \times n$ complex matrix $M$ if and only if $NJ=JN$.
        \item (c) Given such a pair of matrices $(M,N)$ as above, show that there is a basis of $\C^n$ where $M$ has real entries if and only if there is a $2n \times 2n$ real matrix $T$ (called a real structure) satisfying $T^2=I$, $TJ=-JT$, and $TN=NT$.
    \end{itemize}
\end{prob}

\bigskip
\begin{prob}
    A representation (of a group of Lie algebra) on $\C^n$ has a real structure if there is a matrix $T$ as above commuting with each of the representation matrices. In this case, problem 3 implies that these matrices are real in a certain basis, and define also a real representation.
    \begin{itemize}
        \item (a) Show that the spin 1 representation of $SU(2)$ has a real structure but the spin 1/2 representation does not.
        \item (b) Show that the spin-1/2 representation has a ``pseudoreal structure", meaning a matrix $T$ as above satisfying $T^2=-1$ instead of $T^2=1$.
    \end{itemize}
\end{prob}

\bigskip
\begin{prob}
    Any real Lie algebra $\mathfrak{g}$ has an associated complex Lie algebra $\mathfrak{g}^\C$ (called its complexification) whose elements are complex linear combinations of elements of $\mathfrak{g}$. We extend the Lie bracket on $\mathfrak{g}$ to $\mathfrak{g}^\C$ by linearity: $[iA,B]:= i[A,B]$, $[iA,iB]:=-[A,B]$ for $A,B \in \mathfrak{g}$.
    \begin{itemize}
        \item (a) Show that for each $n$, the real Lie algebras $\mathfrak{so}(p,q)$ for $p+q=n$ all have isomorphic complexifications.
        \item (b) For $n=4$, use this to construct the Lorentz algebra $\mathfrak{so}(3,1)$ matrices acting on a 4-vector.
    \end{itemize}
\end{prob}

\end{document}
