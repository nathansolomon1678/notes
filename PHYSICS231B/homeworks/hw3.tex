\documentclass{article}
\usepackage[margin=1in]{geometry}
\usepackage[linesnumbered,ruled,vlined]{algorithm2e}
\usepackage{amsfonts}
\usepackage{amsmath}
\usepackage{amssymb}
\usepackage{amsthm}
\usepackage{chemformula}
\usepackage{enumitem}
\usepackage{fancyhdr}
\usepackage{graphicx}
\usepackage{hyperref}
\usepackage{listings}
\usepackage{minted}
\usepackage{multicol}
\usepackage{pdfpages}
\usepackage{siunitx}
\usepackage{standalone}
\usepackage{svg}
\usepackage[many]{tcolorbox}
\usepackage{tikz-cd}
\usepackage{transparent}
\usepackage{xcolor}
% \tcbuselibrary{minted}

\author{Nathan Solomon}

\newcommand{\fig}[1]{
    \begin{center}
        \includegraphics[width=\textwidth]{#1}
    \end{center}
}

% Math commands
\renewcommand{\d}{\mathrm{d}}
\DeclareMathOperator{\id}{id}
\DeclareMathOperator{\im}{im}
\DeclareMathOperator{\proj}{proj}
\DeclareMathOperator{\Span}{span}
\DeclareMathOperator{\Tr}{Tr}
\DeclareMathOperator{\tr}{tr}
\DeclareMathOperator{\ad}{ad}
\DeclareMathOperator{\ord}{ord}
%%%%%%%%%%%%%%% \DeclareMathOperator{\sgn}{sgn}
\DeclareMathOperator{\Aut}{Aut}
\DeclareMathOperator{\Inn}{Inn}
\DeclareMathOperator{\Out}{Out}
\DeclareMathOperator{\stab}{stab}

\newcommand{\N}{\ensuremath{\mathbb{N}}}
\newcommand{\Z}{\ensuremath{\mathbb{Z}}}
\newcommand{\Q}{\ensuremath{\mathbb{Q}}}
\newcommand{\R}{\ensuremath{\mathbb{R}}}
\newcommand{\C}{\ensuremath{\mathbb{C}}}
\renewcommand{\H}{\ensuremath{\mathbb{H}}}
\newcommand{\F}{\ensuremath{\mathbb{F}}}

\newcommand{\E}{\ensuremath{\mathbb{E}}}
\renewcommand{\P}{\ensuremath{\mathbb{P}}}

\newcommand{\es}{\ensuremath{\varnothing}}
\newcommand{\inv}{\ensuremath{^{-1}}}
\newcommand{\eps}{\ensuremath{\varepsilon}}
\newcommand{\del}{\ensuremath{\partial}}
\renewcommand{\a}{\ensuremath{\alpha}}

\newcommand{\abs}[1]{\ensuremath{\left\lvert #1 \right\rvert}}
\newcommand{\norm}[1]{\ensuremath{\left\lVert #1\right\rVert}}
\newcommand{\mean}[1]{\ensuremath{\left\langle #1 \right\rangle}}
\newcommand{\floor}[1]{\ensuremath{\left\lfloor #1 \right\rfloor}}
\newcommand{\ceil}[1]{\ensuremath{\left\lceil #1 \right\rceil}}
\newcommand{\bra}[1]{\ensuremath{\left\langle #1 \right\rvert}}
\newcommand{\ket}[1]{\ensuremath{\left\lvert #1 \right\rangle}}
\newcommand{\braket}[2]{\ensuremath{\left.\left\langle #1\right\vert #2 \right\rangle}}

\newcommand{\catname}[1]{{\normalfont\textbf{#1}}}

\newcommand{\up}{\ensuremath{\uparrow}}
\newcommand{\down}{\ensuremath{\downarrow}}

% Custom environments
\newtheorem{thm}{Theorem}[section]

\definecolor{probBackgroundColor}{RGB}{250,240,240}
\definecolor{probAccentColor}{RGB}{140,40,0}
\newenvironment{prob}{
    \stepcounter{thm}
    \begin{tcolorbox}[
        boxrule=1pt,
        sharp corners,
        colback=probBackgroundColor,
        colframe=probAccentColor,
        borderline west={4pt}{0pt}{probAccentColor},
        breakable
    ]
    \color{probAccentColor}\textbf{Problem \thethm.} \color{black}
} {
    \end{tcolorbox}
}

\definecolor{exampleBackgroundColor}{RGB}{212,232,246}
\newenvironment{example}{
    \stepcounter{thm}
    \begin{tcolorbox}[
      boxrule=1pt,
      sharp corners,
      colback=exampleBackgroundColor,
      breakable
    ]
    \textbf{Example \thethm.}
} {
    \end{tcolorbox}
}

\definecolor{propBackgroundColor}{RGB}{255,245,220}
\definecolor{propAccentColor}{RGB}{150,100,0}
\newenvironment{prop}{
    \stepcounter{thm}
    \begin{tcolorbox}[
        boxrule=1pt,
        sharp corners,
        colback=propBackgroundColor,
        colframe=propAccentColor,
        breakable
    ]
    \color{propAccentColor}\textbf{Proposition \thethm. }\color{black}
} {
    \end{tcolorbox}
}

\definecolor{thmBackgroundColor}{RGB}{235,225,245}
\definecolor{thmAccentColor}{RGB}{50,0,100}
\renewenvironment{thm}{
    \stepcounter{thm}
    \begin{tcolorbox}[
        boxrule=1pt,
        sharp corners,
        colback=thmBackgroundColor,
        colframe=thmAccentColor,
        breakable
    ]
    \color{thmAccentColor}\textbf{Theorem \thethm. }\color{black}
} {
    \end{tcolorbox}
}

\definecolor{corBackgroundColor}{RGB}{240,250,250}
\definecolor{corAccentColor}{RGB}{50,100,100}
\newenvironment{cor}{
    \stepcounter{thm}
    \begin{tcolorbox}[
        enhanced,
        boxrule=0pt,
        frame hidden,
        sharp corners,
        colback=corBackgroundColor,
        borderline west={4pt}{0pt}{corAccentColor},
        breakable
    ]
    \color{corAccentColor}\textbf{Corollary \thethm. }\color{black}
} {
    \end{tcolorbox}
}

\definecolor{lemBackgroundColor}{RGB}{255,245,235}
\definecolor{lemAccentColor}{RGB}{250,125,0}
\newenvironment{lem}{
    \stepcounter{thm}
    \begin{tcolorbox}[
        enhanced,
        boxrule=0pt,
        frame hidden,
        sharp corners,
        colback=lemBackgroundColor,
        borderline west={4pt}{0pt}{lemAccentColor},
        breakable
    ]
    \color{lemAccentColor}\textbf{Lemma \thethm. }\color{black}
} {
    \end{tcolorbox}
}

\definecolor{proofBackgroundColor}{RGB}{255,255,255}
\definecolor{proofAccentColor}{RGB}{80,80,80}
\renewenvironment{proof}{
    \begin{tcolorbox}[
        enhanced,
        boxrule=1pt,
        sharp corners,
        colback=proofBackgroundColor,
        colframe=proofAccentColor,
        borderline west={4pt}{0pt}{proofAccentColor},
        breakable
    ]
    \color{proofAccentColor}\emph{\textbf{Proof. }}\color{black}
} {
    \qed \end{tcolorbox}
}

\definecolor{noteBackgroundColor}{RGB}{240,250,240}
\definecolor{noteAccentColor}{RGB}{30,130,30}
\newenvironment{note}{
    \begin{tcolorbox}[
        enhanced,
        boxrule=0pt,
        frame hidden,
        sharp corners,
        colback=noteBackgroundColor,
        borderline west={4pt}{0pt}{noteAccentColor},
        breakable
    ]
    \color{noteAccentColor}\textbf{Note. }\color{black}
} {
    \end{tcolorbox}
}


\fancyhf{}
\setlength{\headheight}{24pt}

\date{\today}
\title{Physics 231B Homework \#3}

\begin{document}
\maketitle

\bigskip
\begin{prob}
    Prove that $A_4 \cong (\Z_2 \times \Z_2) \rtimes \Z_3$.
\end{prob}
Since $S_4$ is generated by transpositions (12), (23), and (34), and $A_4$ is the subgroup of $S_4$ equal to the product of an even number of transpositions, $A_4$ is generated by pairs of transpositions. In particular, the permutations (12)(34) and (13)(24) commute with each other, because
\[ (12)(34) \circ (13)(24) = (14)(23) = (13)(24) \circ (12)(34). \]
If we let $g_1 = (12)(34)$ and $g_2 = (13)(24)$, then $g_1^2=e=g_2^2$ and $g_1g_2=g_2g_1$, so $\mean{g_1, g_2} \cong \Z_2 \times \Z_2$.
\par
Now define an action $\alpha: \Z_3 \rightarrow \Aut(A_4)$ such that $\alpha(1)$ is conjugation by (123).
\begin{align*}
\alpha(1) &= \left( x \mapsto (321)x(123) \right) \\
\alpha(1)(g_1) &= (13)(24) = g_2 \\
\alpha(1)(g_2) &= (14)(23) = g_1g_2 \\
\alpha(1)(g_1g_2) &= (12)(34) = g_1.
\end{align*}
Now consider a homomorphism $f$ from $A_4$ to $ \mean{g_1, g_2} \rtimes_\alpha \Z_3$. To define any homomorphism, we only need to say how it acts on a set of generators, which I will choose to be $ \left\{ g_1, g_2, (123) \right\}$.
\begin{align*}
    f(g_1) &= ((g_1, e), 0) \\
    f(g_2) &= ((e, g_2), 0) \\
    f((123)) &= ((e,e), 1).
\end{align*}
By Lagrange's theorem, there are 12 elements in the image of $f$, which means $f$ is injective and surjective (because $|A_4|$ is also 12). Since we have found a bijective homomorphism from $A_4$ to $\mean{g_1,g_2} \rtimes_\alpha \Z_3 \cong (\Z_2 \times \Z_2) \rtimes \Z_3$, we can conclude those groups are isomorphic.

\bigskip
\begin{prob}
    Prove that $A_4$ has no subgroup of size 6.
\end{prob}
Using the notation from my answer to problem 1, any element of $A_4 \cong \mean{g_1, g_2} \rtimes_\alpha \Z_3$ can be written as $((a,b),c)$, where $a$ is either $e$ or $g_1$, $b$ is either $e$ or $g_2$, and $c$ is either 0, 1, or 2.
\par
Now consider a subgroup of order 6. If every element in that subgroup has $c=0$, then the subgroup would have order 1, 2, or 4, because it would be a subgroup of $\mean{g_1, g_2}$. If every element in that subgroup has $a=e=b$, then that subgroup would either be the trivial group or it would be $\Z_3$. Therefore, to contain at least 6 elements, our subgroup must contain one element where $c \neq 0$ and at least one element where either $a \neq e$ or $b \neq e$. However, those/that element(s) alone are enough to generate a group with 12 elements. Therefore there is no subgroup of $A_4$ with exactly 6 elements.

\bigskip
\begin{prob}
    \textbf{Artin Chapter 6 Problem 12.2, page 193.} Describe the orbits of poles for the group of rotations of an octahedron.
\end{prob}
Poles are defined on page 184 of Artin as points which are fixed by a non-identity element of the group. Artin also notes that every pole is collinear with the center of the octahedron and either (1) a vertex of the octahedron, (2) the center of a face of the octahedron, or (3) the center of an edge of the octahedron.
\par
In the first case, the orbit is 6 points in the shape of an octahedron, because the orbit of any vertex is the set of all vertices of the octahedron.
\par
In the second case, the orbit is 8 points in the shape of the cube, because the orbit of any point in the center of a face is the set of points in the center of any other face.
\par
In the third case, the orbit is 12 points in the shape of a cuboctahedron, since the orbit of any point in the center of an edge is the set of points in the centers of all other edges.

\bigskip
\begin{prob}
    \textbf{Artin Chapter 6 Problem M.1, page 193.} Let $G$ be a two-dimensional crystallographic group such that no element $g \neq 1$ fixes any point in the plane. Prove that $G$ is generated by two translations, or else by one translation and one glide.
\end{prob}
Suppose $B$ is a basis for $G$. $B$ cannot contain either a reflection or a rotation, because every rotation fixes one point, and every reflection fixes an entire line. Since every isometry of the plane is either a translation, a glide, a reflection, or a rotation, we can conclude that the elements of $B$ are all translations or glides.
\par
In order for $G$ to be two dimensional, $B$ must have exatly two elements. If $B$ contains two glides, we can compose the two glides to form a translation. Replacing one of those two glides in $B$ with their composition would not change the group generated by $B$.
\par
Therefore $B$ can be either be chosen to have two translations, or one glide and one translation.

\bigskip
\begin{prob}
    Show that $\R^d \rtimes O(d)$ is isomorphic to the subgroup of $GL(d+1, \R)$ consisting of matrices of the form
    \[ \begin{pmatrix}
        M & v \\
        0 \cdots 0 & 1
    \end{pmatrix}, \]
    where $v \in \R^d$ is a column vector and $M \in O(d)$.
\end{prob}
Let $A$ be the set of matrices of that form, and let $f: \R^d \rtimes O(d) \rightarrow A$ be the function which takes $(v, M)$ to
\[ \begin{pmatrix}
    M & v \\
    0 \cdots 0 & 1
\end{pmatrix}. \]
Then $f$ is clearly bijective, so we just need to show that it's a homomorphism.
\par
For any $v, v' \in \R^d$ and $M, M' \in O(d)$,
\begin{align*}
    f((v, M)\cdot (v', M')) &= f((v+Mv', MM')) \\
                            &= \begin{pmatrix}
                                MM' & Mv'+v \\
                                0 \cdots 0 & 1
                            \end{pmatrix} \\
                            &= \begin{pmatrix}
                                M & v \\
                                0 \cdots 0 & 1
                            \end{pmatrix} \cdot \begin{pmatrix}
                                M' & v' \\
                                0 \cdots 0 & 1
                            \end{pmatrix} \\
                            &= f((v,M))\cdot f((v',M')),
\end{align*}
so $f$ is an isomorphism.

\bigskip
\begin{prob}
    Consider the wallpaper group $\Z^2 \rtimes C_4$ we discussed in class acting on $\R^2$. What are the points in $\R^2$ which have non-trivial stabilizers? What are their stabilizer groups?
\end{prob}
Every point in $\Z^2$ has a non-trivial stabilizer, because for each of those points, the stabilizer is the set of rotations about that point. Additionally, a point $(x,y) \in \R^2$ has a non-trivial stabilizer iff a rotation by 90°, 180°, or 270° followed by a translation by $(a,b) \in \Z^2$ brings it back to $(x,y)$ -- that is, iff one of the following is true:
\begin{itemize}
    \item $x=a-y$ and $y=b+x$
    \item $x=a-x$ and $y=b-y$
    \item $x=a+y$ and $y=b-x$
\end{itemize}
If either the first of those criteria is true, then we would also have $x=(a-b)-x$ and $y=(b+a)-y$, and if the third criterion is true, then we would also have $x=(a+b)-x$ and $y=(b-a)-y$. Putting those together, we see that $(x,y)$ has a non-trivial stabilizer if and only if both $2x$ and $2y$ are integers.

\end{document}
