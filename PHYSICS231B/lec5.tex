\documentclass[class=article, crop=false]{standalone}
\usepackage[margin=1in]{geometry}
\usepackage[linesnumbered,ruled,vlined]{algorithm2e}
\usepackage{amsfonts}
\usepackage{amsmath}
\usepackage{amssymb}
\usepackage{amsthm}
\usepackage{enumitem}
\usepackage{fancyhdr}
\usepackage{hyperref}
\usepackage{minted}
\usepackage{multicol}
\usepackage{pdfpages}
\usepackage{standalone}
\usepackage[many]{tcolorbox}
\usepackage{tikz-cd}
\usepackage{transparent}
\usepackage{xcolor}
% \tcbuselibrary{minted}

\author{Nathan Solomon}

\newcommand{\fig}[1]{
    \begin{center}
        \includegraphics[width=\textwidth]{#1}
    \end{center}
}

% Math commands
\renewcommand{\d}{\mathrm{d}}
\DeclareMathOperator{\id}{id}
\DeclareMathOperator{\im}{im}
\DeclareMathOperator{\proj}{proj}
\DeclareMathOperator{\Span}{span}
\DeclareMathOperator{\Tr}{Tr}
\DeclareMathOperator{\tr}{tr}
\DeclareMathOperator{\ad}{ad}
\DeclareMathOperator{\ord}{ord}
%%%%%%%%%%%%%%% \DeclareMathOperator{\sgn}{sgn}
\DeclareMathOperator{\Aut}{Aut}
\DeclareMathOperator{\Inn}{Inn}
\DeclareMathOperator{\Out}{Out}
\DeclareMathOperator{\stab}{stab}

\newcommand{\N}{\ensuremath{\mathbb{N}}}
\newcommand{\Z}{\ensuremath{\mathbb{Z}}}
\newcommand{\Q}{\ensuremath{\mathbb{Q}}}
\newcommand{\R}{\ensuremath{\mathbb{R}}}
\newcommand{\C}{\ensuremath{\mathbb{C}}}
\renewcommand{\H}{\ensuremath{\mathbb{H}}}
\newcommand{\F}{\ensuremath{\mathbb{F}}}

\newcommand{\E}{\ensuremath{\mathbb{E}}}
\renewcommand{\P}{\ensuremath{\mathbb{P}}}

\newcommand{\es}{\ensuremath{\varnothing}}
\newcommand{\inv}{\ensuremath{^{-1}}}
\newcommand{\eps}{\ensuremath{\varepsilon}}
\newcommand{\del}{\ensuremath{\partial}}
\renewcommand{\a}{\ensuremath{\alpha}}

\newcommand{\abs}[1]{\ensuremath{\left\lvert #1 \right\rvert}}
\newcommand{\norm}[1]{\ensuremath{\left\lVert #1\right\rVert}}
\newcommand{\mean}[1]{\ensuremath{\left\langle #1 \right\rangle}}
\newcommand{\floor}[1]{\ensuremath{\left\lfloor #1 \right\rfloor}}
\newcommand{\ceil}[1]{\ensuremath{\left\lceil #1 \right\rceil}}
\newcommand{\bra}[1]{\ensuremath{\left\langle #1 \right\rvert}}
\newcommand{\ket}[1]{\ensuremath{\left\lvert #1 \right\rangle}}
\newcommand{\braket}[2]{\ensuremath{\left.\left\langle #1\right\vert #2 \right\rangle}}

\newcommand{\catname}[1]{{\normalfont\textbf{#1}}}

\newcommand{\up}{\ensuremath{\uparrow}}
\newcommand{\down}{\ensuremath{\downarrow}}

% Custom environments
\newtheorem{thm}{Theorem}[section]

\definecolor{probBackgroundColor}{RGB}{250,240,240}
\definecolor{probAccentColor}{RGB}{140,40,0}
\newenvironment{prob}{
    \stepcounter{thm}
    \begin{tcolorbox}[
        boxrule=1pt,
        sharp corners,
        colback=probBackgroundColor,
        colframe=probAccentColor,
        borderline west={4pt}{0pt}{probAccentColor},
        breakable
    ]
    \color{probAccentColor}\textbf{Problem \thethm.} \color{black}
} {
    \end{tcolorbox}
}

\definecolor{exampleBackgroundColor}{RGB}{212,232,246}
\newenvironment{example}{
    \stepcounter{thm}
    \begin{tcolorbox}[
      boxrule=1pt,
      sharp corners,
      colback=exampleBackgroundColor,
      breakable
    ]
    \textbf{Example \thethm.}
} {
    \end{tcolorbox}
}

\definecolor{propBackgroundColor}{RGB}{255,245,220}
\definecolor{propAccentColor}{RGB}{150,100,0}
\newenvironment{prop}{
    \stepcounter{thm}
    \begin{tcolorbox}[
        boxrule=1pt,
        sharp corners,
        colback=propBackgroundColor,
        colframe=propAccentColor,
        breakable
    ]
    \color{propAccentColor}\textbf{Proposition \thethm. }\color{black}
} {
    \end{tcolorbox}
}

\definecolor{thmBackgroundColor}{RGB}{235,225,245}
\definecolor{thmAccentColor}{RGB}{50,0,100}
\renewenvironment{thm}{
    \stepcounter{thm}
    \begin{tcolorbox}[
        boxrule=1pt,
        sharp corners,
        colback=thmBackgroundColor,
        colframe=thmAccentColor,
        breakable
    ]
    \color{thmAccentColor}\textbf{Theorem \thethm. }\color{black}
} {
    \end{tcolorbox}
}

\definecolor{corBackgroundColor}{RGB}{240,250,250}
\definecolor{corAccentColor}{RGB}{50,100,100}
\newenvironment{cor}{
    \stepcounter{thm}
    \begin{tcolorbox}[
        enhanced,
        boxrule=0pt,
        frame hidden,
        sharp corners,
        colback=corBackgroundColor,
        borderline west={4pt}{0pt}{corAccentColor},
        breakable
    ]
    \color{corAccentColor}\textbf{Corollary \thethm. }\color{black}
} {
    \end{tcolorbox}
}

\definecolor{lemBackgroundColor}{RGB}{255,245,235}
\definecolor{lemAccentColor}{RGB}{250,125,0}
\newenvironment{lem}{
    \stepcounter{thm}
    \begin{tcolorbox}[
        enhanced,
        boxrule=0pt,
        frame hidden,
        sharp corners,
        colback=lemBackgroundColor,
        borderline west={4pt}{0pt}{lemAccentColor},
        breakable
    ]
    \color{lemAccentColor}\textbf{Lemma \thethm. }\color{black}
} {
    \end{tcolorbox}
}

\definecolor{proofBackgroundColor}{RGB}{255,255,255}
\definecolor{proofAccentColor}{RGB}{80,80,80}
\renewenvironment{proof}{
    \begin{tcolorbox}[
        enhanced,
        boxrule=1pt,
        sharp corners,
        colback=proofBackgroundColor,
        colframe=proofAccentColor,
        borderline west={4pt}{0pt}{proofAccentColor},
        breakable
    ]
    \color{proofAccentColor}\emph{\textbf{Proof. }}\color{black}
} {
    \qed \end{tcolorbox}
}

\definecolor{noteBackgroundColor}{RGB}{240,250,240}
\definecolor{noteAccentColor}{RGB}{30,130,30}
\newenvironment{note}{
    \begin{tcolorbox}[
        enhanced,
        boxrule=0pt,
        frame hidden,
        sharp corners,
        colback=noteBackgroundColor,
        borderline west={4pt}{0pt}{noteAccentColor},
        breakable
    ]
    \color{noteAccentColor}\textbf{Note. }\color{black}
} {
    \end{tcolorbox}
}



\fancyhf{}
\lhead{Nathan Solomon}
\rhead{Page \thepage}
\pagestyle{fancy}

\begin{document}
\section{4/17/2024 lecture}

\subsection{Finite subgroups of $SO(3)$}
Let $G$ be any finite subgroup of $SO(3)$. Let $G$ act on the unit sphere $S^2$ and let $P$ be the set of points with a nontrivial stabilizer -- that is, the set of points on $S^2$ which are fixed by some nontrivial element of $G$. Then by the fixed-point theorem (SEE WHAT ARTIN SAYS ABOUT THIS THEOREM), the number of $G$-orbits in $P$ is $r\leq 3$.
\par
\begin{example}
    Consider the subgroup of $SO(3)$ generated by a rotation by $2 \pi / n$ around the $z$-axis and a rotation by $\pi$ about the $x$-axis, where $n$ is some odd number. This group turns out to be isomorphic to the dihedral group $D_n$. INCLUDE DRAWING OF THE AXES OF ROTATION FOR THIS GROUP.
\end{example}
\par
Now consider other finite groups of $SO(3)$. These must satisfy the equation
\[ \frac{1}{2} + \frac{1}{3} + \frac{1}{n_3} = 1 + \frac{2}{n}. \]
HOW DID WE GET THIS EQUATION, WHAT ARE $n$ and $n_3$???
The only solutions are $n \in \left\{ 12,24,60 \right\}$, which corresponds to the rotational symmetry groups of the 5 platonic solids.
\par
IS EVERY FINITE SUBGROUP OF $SO(3)$ ISOMORPHIC TO EITHER THE DIHEDRAL GROUP OR A POLYHEDRAL GROUP???

\subsection{Lattice groups}
A \emph{$d$-dimensional lattice group} is a discrete subgroup $L$ of $\R^d$ which is isomorphic to $\Z^d$. Equivalently, if we have a basis $ \left\{ v_1, v_2, \dots, v_d \right\}$ of $\R^d$, then $L$ is the additive group generated by (translations by) those vectors.
\par
Suppose the rotation $R \in SO(3)$ preserves the lattice $L := \mean{v_1, \dots, v_d}$. Let $M$ be the change-of-basis matrix which maps $e_i$ to $v_i$. Then $n_{ij} := M^{-1}RM$ is a matrix with integer entries which satisfies
\[ R \cdot v_i = \sum_{j=1}^d n_{ij} v_j. \]
That means the trace of $n_{ij}$ is an integer, but since $\tr R = \tr n$, the trace of $R$ also has to be an integer. For example, if $R$ is the rotation in the $xy$-plane
\[ R = \begin{bmatrix}
    \cos \varphi & - \sin \varphi & 0 \\
    \sin \varphi & \cos \varphi & 0 \\
    0 & 0 & 1
\end{bmatrix}, \]
then we have $\tr R = 2 \cos \varphi +1 \in \Z$, which means $\varphi$ is an integer multiple of $2 \pi$ divided by 1, 2, 3, 4, or 6.

\subsection{Finite subgroups of $O(3)$}
The center of $O(3)$ is $ \left\{ \pm I \right\}^ \times$.
\par
Let $\widetilde{O}$ be the set of symmetries of the cube in $O(3)$. This just happens to be isomorphic to $S_4$. HOW DO WE PROVE THIS???
\par
More generally, consider a finite subgroup $G$ of $O(3)$ which is not a subgroup of $SO(3)$. Then there are two cases of interest:
\begin{enumerate}
    \item If $-I \in G$, then $G = G_+ \cup (-I) \cdot G_+$, where $G_+ := G \cap SO(3)$ is the subgroup of $G$ with positive determinant and $G_- = G \backslash G_+ = (-I) \cdot G_+$ is the subset of $G$ with negative determinant. This is the simpler, easier case.
    \item If $-I \not\in G$, then $G_- = G \backslash G_+ = \left\{ g \in G : \det (g)=-1 \right\}$. Define a new set $G' = G_+ \cup (-I) \cdot G_-$, which we can verify is a closed, finite subgroup of $SO(3)$. Also, we know that $G_+$ is a normal subgroup of $G'$ (with index 2 in $G'$), because the conjugate of any element $R \in G_+$ by some $-S \in G' \backslash G_+$ is $(-S)^{-1}R(-S)=S^{-1}RS$ which is in $G$ and has determinant 1, so it's in $G_+$.
\end{enumerate}
\begin{thm}
    The set of symmetries of a cube is $S_4$.
\end{thm}
\begin{proof}
    The set of rotational symmetries of a tetrahedron is $A_4$. We can embed the vertices of a tetrahedron in the vertices of a cube in two different ways, and a reflection of the cube will convert between those two tetrahedra.
\end{proof}
\begin{cor}
    Given a cube, choose 4 vertices such that none of the vertices are adjacent to each other (there are two ways to do this) and label those vertices. Then any symmetry of the cube is uniquely determined by a permutation of those 4 vertices.
\end{cor}
\begin{prob}
    SHow that there are no subgroups of $A_4$ of size 6 (i.e. of index 2).
\end{prob}

\subsection{Crystallographic groups}
A group $G$ is called a \emph{crystallographic group} iff it is a discrete subgroup of the group of affine transformations of $\R^d$ (that is, the group generated by translations, rotations, and reflections) and it contains a $d$-dimensional lattice subgroup $L$.
\par
It follows that $L$ is normal, and $G/L$ is a finite subgroup of $O(3)$.

\end{document}
