\documentclass[class=article, crop=false]{standalone}
\usepackage[margin=1in]{geometry}
\usepackage[linesnumbered,ruled,vlined]{algorithm2e}
\usepackage{amsfonts}
\usepackage{amsmath}
\usepackage{amssymb}
\usepackage{amsthm}
\usepackage{enumitem}
\usepackage{fancyhdr}
\usepackage{hyperref}
\usepackage{minted}
\usepackage{multicol}
\usepackage{pdfpages}
\usepackage{standalone}
\usepackage[many]{tcolorbox}
\usepackage{tikz-cd}
\usepackage{transparent}
\usepackage{xcolor}
% \tcbuselibrary{minted}

\author{Nathan Solomon}

\newcommand{\fig}[1]{
    \begin{center}
        \includegraphics[width=\textwidth]{#1}
    \end{center}
}

% Math commands
\renewcommand{\d}{\mathrm{d}}
\DeclareMathOperator{\id}{id}
\DeclareMathOperator{\im}{im}
\DeclareMathOperator{\proj}{proj}
\DeclareMathOperator{\Span}{span}
\DeclareMathOperator{\Tr}{Tr}
\DeclareMathOperator{\tr}{tr}
\DeclareMathOperator{\ad}{ad}
\DeclareMathOperator{\ord}{ord}
%%%%%%%%%%%%%%% \DeclareMathOperator{\sgn}{sgn}
\DeclareMathOperator{\Aut}{Aut}
\DeclareMathOperator{\Inn}{Inn}
\DeclareMathOperator{\Out}{Out}
\DeclareMathOperator{\stab}{stab}

\newcommand{\N}{\ensuremath{\mathbb{N}}}
\newcommand{\Z}{\ensuremath{\mathbb{Z}}}
\newcommand{\Q}{\ensuremath{\mathbb{Q}}}
\newcommand{\R}{\ensuremath{\mathbb{R}}}
\newcommand{\C}{\ensuremath{\mathbb{C}}}
\renewcommand{\H}{\ensuremath{\mathbb{H}}}
\newcommand{\F}{\ensuremath{\mathbb{F}}}

\newcommand{\E}{\ensuremath{\mathbb{E}}}
\renewcommand{\P}{\ensuremath{\mathbb{P}}}

\newcommand{\es}{\ensuremath{\varnothing}}
\newcommand{\inv}{\ensuremath{^{-1}}}
\newcommand{\eps}{\ensuremath{\varepsilon}}
\newcommand{\del}{\ensuremath{\partial}}
\renewcommand{\a}{\ensuremath{\alpha}}

\newcommand{\abs}[1]{\ensuremath{\left\lvert #1 \right\rvert}}
\newcommand{\norm}[1]{\ensuremath{\left\lVert #1\right\rVert}}
\newcommand{\mean}[1]{\ensuremath{\left\langle #1 \right\rangle}}
\newcommand{\floor}[1]{\ensuremath{\left\lfloor #1 \right\rfloor}}
\newcommand{\ceil}[1]{\ensuremath{\left\lceil #1 \right\rceil}}
\newcommand{\bra}[1]{\ensuremath{\left\langle #1 \right\rvert}}
\newcommand{\ket}[1]{\ensuremath{\left\lvert #1 \right\rangle}}
\newcommand{\braket}[2]{\ensuremath{\left.\left\langle #1\right\vert #2 \right\rangle}}

\newcommand{\catname}[1]{{\normalfont\textbf{#1}}}

\newcommand{\up}{\ensuremath{\uparrow}}
\newcommand{\down}{\ensuremath{\downarrow}}

% Custom environments
\newtheorem{thm}{Theorem}[section]

\definecolor{probBackgroundColor}{RGB}{250,240,240}
\definecolor{probAccentColor}{RGB}{140,40,0}
\newenvironment{prob}{
    \stepcounter{thm}
    \begin{tcolorbox}[
        boxrule=1pt,
        sharp corners,
        colback=probBackgroundColor,
        colframe=probAccentColor,
        borderline west={4pt}{0pt}{probAccentColor},
        breakable
    ]
    \color{probAccentColor}\textbf{Problem \thethm.} \color{black}
} {
    \end{tcolorbox}
}

\definecolor{exampleBackgroundColor}{RGB}{212,232,246}
\newenvironment{example}{
    \stepcounter{thm}
    \begin{tcolorbox}[
      boxrule=1pt,
      sharp corners,
      colback=exampleBackgroundColor,
      breakable
    ]
    \textbf{Example \thethm.}
} {
    \end{tcolorbox}
}

\definecolor{propBackgroundColor}{RGB}{255,245,220}
\definecolor{propAccentColor}{RGB}{150,100,0}
\newenvironment{prop}{
    \stepcounter{thm}
    \begin{tcolorbox}[
        boxrule=1pt,
        sharp corners,
        colback=propBackgroundColor,
        colframe=propAccentColor,
        breakable
    ]
    \color{propAccentColor}\textbf{Proposition \thethm. }\color{black}
} {
    \end{tcolorbox}
}

\definecolor{thmBackgroundColor}{RGB}{235,225,245}
\definecolor{thmAccentColor}{RGB}{50,0,100}
\renewenvironment{thm}{
    \stepcounter{thm}
    \begin{tcolorbox}[
        boxrule=1pt,
        sharp corners,
        colback=thmBackgroundColor,
        colframe=thmAccentColor,
        breakable
    ]
    \color{thmAccentColor}\textbf{Theorem \thethm. }\color{black}
} {
    \end{tcolorbox}
}

\definecolor{corBackgroundColor}{RGB}{240,250,250}
\definecolor{corAccentColor}{RGB}{50,100,100}
\newenvironment{cor}{
    \stepcounter{thm}
    \begin{tcolorbox}[
        enhanced,
        boxrule=0pt,
        frame hidden,
        sharp corners,
        colback=corBackgroundColor,
        borderline west={4pt}{0pt}{corAccentColor},
        breakable
    ]
    \color{corAccentColor}\textbf{Corollary \thethm. }\color{black}
} {
    \end{tcolorbox}
}

\definecolor{lemBackgroundColor}{RGB}{255,245,235}
\definecolor{lemAccentColor}{RGB}{250,125,0}
\newenvironment{lem}{
    \stepcounter{thm}
    \begin{tcolorbox}[
        enhanced,
        boxrule=0pt,
        frame hidden,
        sharp corners,
        colback=lemBackgroundColor,
        borderline west={4pt}{0pt}{lemAccentColor},
        breakable
    ]
    \color{lemAccentColor}\textbf{Lemma \thethm. }\color{black}
} {
    \end{tcolorbox}
}

\definecolor{proofBackgroundColor}{RGB}{255,255,255}
\definecolor{proofAccentColor}{RGB}{80,80,80}
\renewenvironment{proof}{
    \begin{tcolorbox}[
        enhanced,
        boxrule=1pt,
        sharp corners,
        colback=proofBackgroundColor,
        colframe=proofAccentColor,
        borderline west={4pt}{0pt}{proofAccentColor},
        breakable
    ]
    \color{proofAccentColor}\emph{\textbf{Proof. }}\color{black}
} {
    \qed \end{tcolorbox}
}

\definecolor{noteBackgroundColor}{RGB}{240,250,240}
\definecolor{noteAccentColor}{RGB}{30,130,30}
\newenvironment{note}{
    \begin{tcolorbox}[
        enhanced,
        boxrule=0pt,
        frame hidden,
        sharp corners,
        colback=noteBackgroundColor,
        borderline west={4pt}{0pt}{noteAccentColor},
        breakable
    ]
    \color{noteAccentColor}\textbf{Note. }\color{black}
} {
    \end{tcolorbox}
}



\fancyhf{}
\lhead{Nathan Solomon}
\rhead{Page \thepage}
\pagestyle{fancy}

\begin{document}
\section{4/10/2024 lecture}

\subsection{Quotient groups}
\begin{example}
    $SO(n)$ is a normal subgroup of $O(n)$, so we can define the quotient group $O(n)/SO(n)$, which is isomorphic to $C_2 := \braket{x}{x^2=e} \cong \left\{ \pm 1 \right\}^\times$.
\end{example}

Let $n \Z$ be the subgroup $ \left\{ nm:m \in \Z \right\}$. Since $\Z$ is an additive group, be sure not to confuse $n\Z$ with the coset $n+\Z$. We know that $n\Z$ is a normal subgroup of $\Z$ -- it's easy to prove that every subgroup of an abelian group is normal.
\par
Now we can define the \emph{group of integers modulo $n$} to be $\Z/n\Z$. Some people write this as $\Z_n$, because that's shorter.
\begin{thm}
    For any $N \in \N$, the cyclic group $C_n := \braket{x}{x^n=e}$ is isomorphic to $\Z/n\Z$. Therefore, we can use $C_n$ and $\Z_n$ interchangeably.
\end{thm}
\begin{proof}
    Let $\varphi: C_n \rightarrow \Z/n\Z$ be the homomorphism which maps $x$ to the coset $1+n\Z$ (and thus, also maps $x^m$ to $m+n\Z$). You can easily shows that $\varphi$ is an injective and surjective homomorphism.
\end{proof}

\subsection{Exact sequences and extensions}
A path in a commutative diagram is called an \emph{exact sequence} iff the kernel of each morphism (except the first one) is equal to the image of the previous one. Right now, we only care about the category $\catname{Grp}$, in which morphisms are group homomorphism. For example, if $H \trianglelefteq G$, then
\begin{center}
\begin{tikzcd}[column sep=small]
    0 \arrow[hookrightarrow]{r} & H \arrow[hookrightarrow, "i"]{r} & G \arrow[twoheadrightarrow, "\pi"]{r} & G/H \arrow[twoheadrightarrow]{r} & 0
\end{tikzcd}
\end{center}
is an exact sequence because $\ker \pi = \im i$.
\par
A group $G$ is called an \emph{extension of $Q$ by $K$} iff there is an exact sequence
\begin{center}
\begin{tikzcd}[column sep=small]
    0 \arrow[hookrightarrow]{r} & K \arrow[hookrightarrow, "i"]{r} & G \arrow[twoheadrightarrow, "\pi"]{r} & Q \arrow[twoheadrightarrow]{r} & 0.
\end{tikzcd}
\end{center}
\begin{example}
    The \emph{Klein 4-group $K_4 := \Z_2 \times \Z_2$} and the group $\Z_4$ are distinct extensions of $\Z_2$ by $\Z_2$.
\end{example}

\subsection{Conjugacy classes}
Two elements $g_1, g_2 \in G$ are called \emph{conjugate} iff there exists some $h \in G$ such that $hg_1h^{-1}=g_2$. Conjugacy is an equaivalence relation, and the equivalence classes of that relation are called the \emph{conjugacy classes}. The conjugacy class of $g \in G$ is written as
\[ C(g) := \left\{ h \in G: \text{ $h$ and $g$ are conjugate} \right\}. \]
\par
For matrices, conjugacy is the same as similarity, meaning two matrices are conjugate iff they represent the same linear transformation in different bases.
\par
Since every permutation $\sigma \in S_n$, $\sigma$ can be written as the product of disjoint cycles (by lemma \ref{disjointcycleslemma} WHY IS THIS NUMBER OFF?), we can define the \emph{cycle type} of a permutation to be the multiset of the lengths of those cycles (CHECK THAT THIS IS UNIQUELY DEFINED).
\begin{thm}
    The conjugacy class of some permutation $\sigma \in S_n$ is the set of permutations in $S_n$ with the same cycle type as $\sigma$.
\end{thm}
\begin{proof}
    By \ref{permutationconjugation}. FINISH THIS PROOF.
\end{proof}
\begin{prob}
    How many conjugacy classes does $S_n$ have? If this is too hard, just consider the $n=4$ case.
\end{prob}
Any permutation which is conjugate to $\sigma \in S_n$ must have the same number of 1-cycles as $\sigma$, the same number of 2-cycles, etc. Therefore each conjugacy class of $S_n$ can be uniquely determined by a partition of $n$ of the form $n=a_1+a_2+\cdots+a_m$, where $a_1>a_2>\cdots>a_m$. So if $n=4$, there are 5 conjugacy classes of $S_n$:
\begin{itemize}
    \item $4=4$
    \item $4=3+1$
    \item $4=2+2$
    \item $4=2+1+1$
    \item $4=1+1+1+1$
\end{itemize}
IS THERE A GENERAL FORMULA FOR THE NUMBER OF CONJUGACY CLASSES OF THE SYMMETRIC GROUP

\subsection{The alternating group}
The \emph{sign of a permutation} is 1 if it can be written as a product of an even number of permutations, and $-1$ otherwise.
\par
Let \emph{the permutation matrix $P(\sigma)$} of some permutation $\sigma \in S_n$ be the orthogonal matrix which permutes the basis vectors $e_i \in \R^n$. Then we can define the sign of $\sigma$ to be $\det (P(\sigma))$. Note that the sign of any transposition is $-1$.
\begin{center}
\begin{tikzcd}[column sep=small]
    S_n \arrow[hookrightarrow, "P"]{r} & O(n) \arrow[twoheadrightarrow, "\det"]{r} & \Z_2.
\end{tikzcd}
\end{center}
Now we can define \emph{the alternating group $A_n$} to be the kernel of $\det \circ P$. By Lagrange's theorem (CITE THAT), $\abs{A_n}=n!/2$.
\begin{prop}
    For $n \geq 5$, $A_5$ is simple. In fact, every group of order less than 60 is \emph{solvable}. This is not really relevant to us, but in Galois theory, this is used to prove the Abel-Ruffini theorem.
\end{prop}

TO PROVE THAT $A_5$ IS SIMPLE, FIND THE SIZES OF ALL CONJUGACY CLASSES SINCE EVERY SUBGROUP OF $A_5$ CONTAINS EITHER AN ENTIRE CONJUGACY CLASS OF ITS ELEMENTS, THE SIZE OF ANY SUBGROUP OF $A_5$ IS THE SUM OF SOME SUBSET OF $( 1, 15, 20,12,12) $ BUT THAT SUM CAN ONLY DIVIDE 60 IF IT IS EITHER 1 OR 60.
\par
ALSO TALK ABOUT THE SYLOW THEOREMS

\end{document}
